\documentclass{article}
\usepackage{ae,aecompl}
\usepackage{todonotes}
\usepackage{chngcntr}
\usepackage{tikz-cd}
\usepackage{graphicx}
\graphicspath{ {./images/}}
\usepackage[all,cmtip]{xy}
\usepackage{amsmath, amscd}
\usepackage{amsthm}
\usepackage{amssymb}
\usepackage{amsfonts}
\usepackage{bm}
\usepackage{qsymbols}
\usepackage{latexsym}
\usepackage{mathrsfs}
\usepackage{mathtools}
\usepackage{cite}
\usepackage{color}
\usepackage{url}
\usepackage{enumerate}
\usepackage{verbatim}
\usepackage[draft=false, colorlinks=true]{hyperref}
\usepackage{pdfpages}
\usepackage[margin=1.2in]{geometry}
\usepackage{IEEEtrantools}

\usepackage{fancyhdr}


\usepackage[nameinlink]{cleveref}


\DeclareMathOperator*{\ac}{accept}
\DeclareMathOperator*{\amax}{argmax}
\DeclareMathOperator*{\amin}{argmin}
\DeclareMathOperator*{\Aut}{Aut}
\newcommand {\al}{{\alpha}}
\newcommand {\abs}[1]{{\left\lvert#1\right\rvert}}
\newcommand {\A}{{\mathcal{A}}}
\newcommand {\AM}{{\mathrm{AM}}}
\newcommand {\AMp}{{\AM_{p}^{X}\!(\Ri_\w)}}
\newcommand {\B}{{\mathcal{B}}}
\DeclareMathOperator*{\Be}{Bern}
\newcommand {\Br}{{\dot{B}}}
\newcommand {\Ba}{{\mathfrak{B}}}
\newcommand {\C}{{\mathbb C}}
\newcommand {\ce}{\mathrm{c}}
\newcommand {\Ce}{\mathrm{C}}
\newcommand {\Cc}{\mathrm{C_{c}}}
\newcommand {\Ccinf}{\mathrm{C_{c}^{\infty}}}
\DeclareMathOperator{\cov}{Cov}
\DeclareMathOperator{\DEV}{DEV}
\newcommand {\Di}{{\mathbb D}}
\newcommand {\dom}{\mathrm{dom}}
\newcommand{\dist}{\stackrel{\mathrm{dist}}{=}}
\newcommand {\ud}{\mathrm{d}}
\newcommand {\ue}{\mathrm{e}}
\newcommand {\eps}{\varepsilon}
\newcommand {\veps}{\varepsilon}
\newcommand {\vrho}{{\varrho}}
\newcommand {\E}{{\mathbb{E}}}
\newcommand {\Ec}{{\mathcal{E}}}
\newcommand {\Ell}{L}
\newcommand {\Ellp}{{L_{p}[0,1]}}
\newcommand {\Ellpprime}{{L_{p'}([0,1])}}
\newcommand {\Ellq}{{L_{q}([0,1])}}
\newcommand {\Ellqprime}{{L_{q'}([0,1])}}
\newcommand {\Ellr}{L^{r}}
\newcommand {\Ellone}{{L_{1}([0,1])}}
\newcommand{\Elltwo}{{L_{2}([0,1])}}
\newcommand{\Ellinfty}{L^{\infty}}
\newcommand{\Ellinftyc}{L_{\mathrm{c}}^{\infty}}
\newcommand{\exb}[1]{\exp\left\{#1\right\}}
\DeclareMathOperator*{\Ext}{Ext}
\newcommand{\F}{{\mathcal{F}}}
\newcommand{\Fe}{{\mathbb{F}}}
\newcommand{\G}{{\mathcal{G}}}
\newcommand{\HF}{\mathcal{H}_{\text{FIO}}^{1}(\Rd)}
\newcommand{\Hr}{H}
\newcommand{\HT}{\mathcal{H}}
\newcommand{\ui}{\mathrm{i}}
\newcommand{\I}{{I}}
\newcommand{\J}{{\mathcal{J}}}
\newcommand{\id}{{\mathrm{id}}}
\newcommand{\iid}{\stackrel{\mathclap{\normalfont\mbox{iid}}}{\sim}}
\newcommand{\im}{{\text{im }}}
\newcommand{\ind}{{\perp\!\!\!\perp}}
\DeclareMathOperator*{\Int}{int}
\newcommand{\intx}{{\overline{\int_{X}}}}
\newcommand{\inte}{{\overline{\int_{\E}}}}
\newcommand{\la}{\lambda}
\newcommand{\rb}{\rangle}
\newcommand{\lb}{{\langle}}
\newcommand{\La}{\Lambda}
\newcommand{\calL}{{\mathcal{L}}}
\newcommand{\lp}{{\mathcal{L}}^{p}}
\newcommand{\lpo}{{\overline{\mathcal{L}}^{p}\!}}
\newcommand{\Lpo}{{\overline{\Ell}^{p}\!}}
\newcommand{\M}{{\mathbf{M}}}
\newcommand{\Ma}{{\mathcal{M}}}
\newcommand{\N}{{{\mathbb N}}}
\newcommand{\Na}{{{\mathcal{N}}}}
\newcommand{\norm}[1]{\left\|#1\right\|}
\newcommand{\normm}[1]{{\left\vert\kern-0.25ex\left\vert\kern-0.25ex\left\vert #1 
    \right\vert\kern-0.25ex\right\vert\kern-0.25ex\right\vert}}
\newcommand{\Om}{{{\Omega}}}
\newcommand{\one}{{{\bf 1}}}
\newcommand{\pic}{\text{Pic }}
\newcommand{\ph}{{\varphi}}
\newcommand{\Pa}{{\mathbb{P}}}
\newcommand{\Po}{{\mathcal{P}}}
\newcommand{\Q}{{\mathbb{Q}}}
\newcommand{\R}{{\mathbb R}}
\newcommand{\Rd}{{\mathbb{R}^{d}}}
\DeclareMathOperator{\rej}{reject }
\newcommand{\Rn}{{\mathbb{R}^{n}}}
\newcommand{\cR}{{\mathcal{R}}}
\newcommand{\Rad}{{\mathrm{Rad}}}
\newcommand{\ran}{{\mathrm{ran}}}
\newcommand{\Ri}{{\mathrm{R}}}
\newcommand{\supp}{{\mathrm{supp}}}
\newcommand{\Se}{\mathrm{S}}
\newcommand{\Sp}{S^{*}(\Rn)}
\newcommand{\St}{{\mathrm{St}}}
\newcommand{\Sw}{\mathcal{S}}
\newcommand{\T}{{\mathcal{T}}}
\newcommand{\ta}{{\theta}}
\newcommand{\Ta}{{\Theta}}
\newcommand{\topp}{\stackrel{p}{\to}}
\newcommand{\todd}{\stackrel{d}{\to}}
\newcommand{\toL}[1]{\stackrel{L^{#1}}{\to}} 
\newcommand{\toas}{\stackrel{a.s.}{\to}}
\DeclareMathOperator{\V}{Var}
\newcommand {\w}{{\omega}}
\newcommand {\W}{{\mathrm{W}}}
\newcommand {\Wnp}{\text{$\mathrm{W}$\textsuperscript{$n,\!p$}}}
\newcommand {\Wnpeq}{\text{$\mathrm{W}$\textsuperscript{$n\!,\!p$}}}
\newcommand {\Wonep}{\text{$\mathrm{W}$\textsuperscript{$1,\!p$}}}
\newcommand {\Wonepeq}{\text{$\mathrm{W}$\textsuperscript{$1\!,\!p$}}}
\newcommand {\X}{{\mathcal{X}}}
\newcommand {\Z}{{{\mathbb Z}}}
\newcommand {\Za}{{\mathcal{Z}}}
\newcommand {\Zd}{{\Z[\sqrt{d}]}}
\newcommand {\vanish}[1]{\relax}

\newcommand {\wh}{\widehat}
\newcommand {\wt}{\widetilde}
\newcommand {\red}{\color{red}}

% Distributions
\newcommand{\normal}{\mathsf{N}}
\newcommand{\poi}{\mathsf{Poisson}}
\newcommand{\bern}{\mathsf{Bernoulli}}
\newcommand{\bin}{\mathsf{Binomal}}
\newcommand{\multi}{\mathsf{Multinomial}}
\newcommand{\Exp}{\mathsf{Exp}}



% put your command and environment definitions here




% some theorem environments
% remove "[theorem]" if you do not want them to use the same number sequence


  \newtheorem{thrm}{Theorem}
  \newtheorem{lemma}{Lemma}
  \newtheorem{prop}{Proposition}
  \newtheorem{cor}{Corollary}

  \newtheorem{conj}{Conjecture}
  \renewcommand{\theconj}{\Alph{conj}}  % numbered A, B, C etc

  \theoremstyle{definition}
  \newtheorem{defn}{Definition}
  \newtheorem{ex}{Example}
  \newtheorem{exs}{Examples}
  \newtheorem{question}{Question}
  \newtheorem{remark}{Remark}
  \newtheorem{notn}{Notation}
  \newtheorem{exer}{Exercise}




\title{STATS305B -- Lecture 7}
\author{Jonathon Taylor\\ Scribed by Michael Howes}
\date{01/26/22}

\pagestyle{fancy}
\fancyhf{}
\rhead{STATS305B -- Lecture 7}
\lhead{01/26/22}
\rfoot{Page \thepage}

\begin{document}
\maketitle
\tableofcontents
\section{Binary GLMs}
\subsection{Fitting}
Let $F$ be a CDF with density $f$. The deviance for a binary GLM with link function $g=F^{-1}$ is,
\[\DEV(\beta|Y) = 2\sum_{i=1}^n - Y_i \frac{F(X_i^T\beta)}{F(X_i^T\beta)(1-F(X_i^T\beta))} - \log\left(1-F(X_i^T\beta)\right). \]
Thus, 
\[\nabla \DEV(\beta|Y) = 2\sum_{i=1}^n X_i \frac{f(X_i^T\beta)^2}{F(X_i^T\beta)(1-F(X_i^T\beta))}\left[\frac{Y_i-F(X_i^T\beta)}{f(X_i^T\beta)}\right]. \]
Since $X_i^T\beta = F^{-1}(\E_\beta[Y_i])$, we have that $\E_\beta [F(X_i^T\beta)-Y_i] = 0$. Thus, we have
\[\E[\nabla^2 \DEV(\beta|Y)] = 2\sum_{i=1}^n X_iX_i^T\frac{f(X_i^T\beta)^2}{F(X_i^T\beta)(F(X_i^T\beta)-1)} = 2X^TW_\beta X, \]
where $W_\beta = \diag\left(\frac{f(X_i)^2}{F(X_i^T\beta)(1-F(X_i^T\beta))}\right).$ We can also rewrite $\nabla \DEV(\beta|Y)$ in terms of the $W_\beta$,
\[\nabla \DEV(\beta|Y) = 2X^TW_\beta \left(\frac{Y-F(X\beta)}{f(X\beta)}\right). \]
To fit a binary glm, we can use Fisher scoring. Fisher scoring is an iterative quasi-Newton method given by{
\begin{align*}
    \widehat{\beta}^{(k+1)}& = \wh{\beta}^{(k)} - \E_{\beta^{(k)}}[\nabla^2 \DEV(\beta^{(k)}|Y)]^{-1}\nabla \DEV(\wh{\beta}^{(k)}|Y) \\
    &= (X^TW_{\wh{\beta}^{(k)}}X)^{-1}X^TW_{\wh{\beta}^{(k)}}\left(X\wh{\beta}^{(k)}+\frac{Y-F(X\wh{\beta}^{(k)})}{f(X\wh{\beta}^{(k)})}\right).
\end{align*}
Note that the above method is a form of iterative re-weighted least squares. This was important for historical reasons since least squares was one of the main algorithms implemented on early computers. When we view Fisher scoring as iterative re-weighted least squares, the response at step $k+1$ is,
\[Z^{(k+1)} = X\wh{\beta}^{(k)} + \frac{Y-F(X\wh{\beta}^{(k)})}{f(X\wh{\beta}^{(k)})} = g(\E_{\beta^{(k)}}(Y)) + g'(\E_{\wh{\beta}^{(k)}}(Y))(Y-\E_{\wh{\beta}^{(k)}}[Y]). \]
This is because if $\mu = F(X^T\beta)$, then $g(\mu) = F^{-1}(\mu) = X^T\beta$ and $g'(\mu) = \frac{1}{F'(F^{-1}(\mu))} = \frac{1}{f(X\beta)}$. The weight matrix at time $k+1$ is,
\begin{align*}
    W^{(k+1)} &= \diag\left(\frac{f(X_i^T\wh{\beta}^{(k)})^2}{F(X_i^T\wh{\beta}^{(k)})(1-F(X_i^T\wh{\beta}^{(k)}))}\right)\\
    &=\diag\left(\frac{F(X_i^T\wh{\beta}^{(k)})(1-F(X_i^T\wh{\beta}^{(k)}))}{f(X_i^T\wh{\beta}^{(k)})^2}\right)^{-1}\\
    &=\frac{1}{f(X\wh{\beta}^{(k)})^2}\V_{\wh{\beta}^{(k)}}\left(Y\right)^{-1}\\
    &=g'(\wh{\mu}^{(k)})^2 V(\wh{\mu}^{(k)})^{-1},
\end{align*}
where $\wh{\mu}^{(k)} = \E_{\wh{\beta}}[Y]$. This shows how Fisher scoring can be generalized to other glms. Instead of minimizing the deviance, we simply run iterative re-weighted least squares with features $X$, iterative response \[Z^{(k)} = g(\wh{\mu}^{(k)})+g'(\wh{\mu}^{(k)})(Y-\wh{\mu}^{(k)})\] and iterative weights $W^{(k)} = g'(\wh{\mu}^{(k)})^2V(\wh{\mu}^{(k)})^{-1}$. This is important because in general for glms we do not have a full model from which we can calculate and optimize a likelihood. We just have the functions $g$ and $V$. Note that, in the binary case, if the model is true, then 
\[\wh{\beta} \approx \normal(\beta^*, (X^TW_{\wh{\beta}}X)^{-1}). \]
If the model is not true, then we have the sandwich form
\[\wh{\beta}-\beta^* \approx \normal(0, Q^{-1}_{\beta^*}\Sigma_{\beta^*}Q^{-1}_{\beta^*}), \]
where $Q_{\beta^*} = \E_{\beta^*}[X^TW_{\beta^*}X]$ and $\Sigma_{\beta^*} = \V(X^T(Y-\pi_{\beta^*}(X)))$. In practice, we can estimate $Q_{\beta^*}$ with $X^TW_{\wh{\beta}}X$ and bootstrap for $\Sigma_{\beta^*}$ or we can bootstrap directly for $\V(\wh{\beta})$. 
\subsection{Inference}
The difference of deviance can be used as a likelihood ratio test. If $M_R \subseteq M_F$ are two models, then if $M_R$ contains the true model 
\[\DEV(M_R)-\DEV(M_F) \stackrel{n \to \infty}{\sim} \chi^2_{df_R-df_F}. \]
Unlike in linear regression, this test is different to the Wald test for a single predictor (i.e. when $M_F$ is $M_R$ plus one additional predictor).
\section{Poisson data}
Suppose $Y \sim \poi(\la)$, then for $y=0,1,2,\ldots,$ we have
\[\Pa_\la(Y=y) = \frac{\la^y}{y!}\exp(-\la) = \exp(y \log(\la)-\la)\frac{1}{y!}. \]
The canonical link for this family is $g = \log$, giving the model $\log(\la_i) = X_i^T\beta$ where $\la_i = \E[Y_i|X_i]$. This model is called the \emph{log-linear model}. The variance function for this model is $\V(Y_i|X_i) = \la_i$. The deviance is 
\[\DEV(\beta|Y) = 2\sum_{i=1}^n -Y_iX_i^T\beta+e^{X_i^T\beta}+Y_i\log(Y_i)-Y_i,  \]
the terms $Y_i\log(Y_i)-Y_i$ come from the saturated model where we have $\la_i = Y_i$. The gradient and Hessian of the deviance are thus,
\[\nabla \DEV(\beta|Y) = 2X^T(\exp(X\beta)-Y) = 2X^T(\E_{\beta}[Y]-Y), \]
and
\[\nabla^2 \DEV(\beta|Y) = 2X^TW_{\beta}X, \]
where $W_{\beta} = \V_\beta(Y) = \exp(X\beta)$. Thus, we can fit this model by using Newton--Raphson. This gives us the iterative rule,
\begin{align*}
    \wh{\beta}^{(k+1)} &= \wh{\beta}^{(k)} -  \nabla^2 \DEV(\wh{\beta}^{(k)}|Y)^{-1}\left(\nabla\DEV(\wh{{\beta}^{(k)}}|Y)\right)\\
    &=\wh{\beta}^{(k)} - \left(X^TW_{\beta}X\right)^{-1}X^T(\E_{\wh{\beta}^{(k)}}[Y]-Y)\\
    &=\wh{\beta}^{(k)} + \left(X^T\exp(X\beta)X\right)^{-1}X^T(Y-\exp(X\beta)).
\end{align*}
This iterative algorithm once again corresponds to a form of iterative re-weighted least squares, with response,
\begin{align*}
    Z^{(k+1)} &= X\wh{\beta}^{(k)} +  (Y-\exp(X\wh{\beta}^{(k)}))/\exp(X\wh{\beta}^{(k)})\\
    &=g(\wh{\la}^{(k)}) + g'(\wh{\la}^{(k)})(Y-\wh{\la}^{(k)})
\end{align*}
where $\wh{\la}^{k} = \exp(X\wh{\beta}^{(k)}) = \E_{\wh{\beta}^{(k)}}[Y]$.
and weight matrix,
\begin{align*}
    W^{(k+1)} &= \exp(X\wh{\beta}^{(k)})\\
    &=\V_{\wh{\beta}^{(k)}}(Y).
\end{align*}
\subsection{Residuals}
Like the binary models, there are two types of residuals for the log-linear model. We have the Pearson residuals,
\[e_i = \frac{Y_i-\wh{\la}_i}{\sqrt{\wh{\la}_i}} = \frac{Y_i - \E_{\wh{\la}_i}[Y]}{\sqrt{\V_{\wh{\la}_i}(Y)}}. \]
We also have the deviance residuals. Note that
\[\DEV(\wh{\la}|Y) = \sum_{i=1}^n \DEV(\wh{\la}_i|Y_i). \]
Thus, we can define the deviance residuals as 
\[d_i = \text{sign}(Y_i - \wh{\la}_i)\sqrt{\DEV(\wh{\la}_i|Y_i)}, \]
recall that $\DEV(\wh{\la}_i|Y_i) = 2\left(\wh{\la}_i-Y_i\log(\wh{\la}_i)-Y_i+Y_i\log(Y_i)\right)$. We also have a hat matrix for the log-linear model. It is given by
\[H_{\wh{\beta}} = W_{\wh{\beta}}^{1/2}X(X^TW_{\wh{\beta}}X)^{-1}X^TW_{\wh{\beta}}^{1/2}, \]
where $W_{\wh{\beta}}=\exp(X\wh{\beta})$.
\subsection{``Unnatural'' models for Poisson data}
We can get other models for Poisson data by changing the link function $g$. The link function satisfies $g(\la_i) = X_i^T\beta$ and hence $\la_i = g^{-1}(X_i^T\beta)$. The natural choice if $g = \log$ but two other choices are \texttt{identity}: $g(\la) = \la$ and \texttt{inverse}: $g(\la) = 1/\la$. These can be used in \texttt{glm()} in R by specifying the link function. For a link function $g$, the deviance is
\[\DEV(\beta|Y) = 2\sum_{i=1}^n \left[g^{-1}(X_i^T\beta)-Y_i\log\left(g^{-1}(X_i^T\beta)\right)-Y_i + Y_i\log(Y_i)\right].\]
We can fit a Poisson glm with Fisher scoring which we can present as an IRLS algorithm with
\[Z^{(k+1)} = X\wh{\beta}^{(k)}+g'(\wh{\la}^{(k)})(Y-\wh{\la}^{(k)}), \]
and 
\[W^{(k+1)} = g'(\wh{\la}^{(k)})^2 V(\wh{\la}^{(k)})^{-1}, \]
where $\wh{\la}^{(k)} = g^{-1}(X\wh{\beta}^{(k)})$ and $V(\la) = \la$. Note that 
\[\nabla \DEV(\beta|Y) = 2X^T\left(\frac{\E_{\beta}[Y]-Y}{g'(\E_{\beta}[Y])\E_{\beta}[Y]}\right),\]
and 
\begin{align*}
    \E_\beta[\nabla^2 \DEV(\beta|Y)]&=2X^T\left(\frac{1}{g'(\E_{\beta}[Y])^2\E_{\beta}[Y]}\right)
    2X^TW_\beta X,
\end{align*}
where $W_\beta = \diag\left(\frac{1}{g'(\la_i)^2\V_{\la_i}(Y)}\right)=\diag\left(\frac{1}{g'(\la_i)^2\la_i}\right)$, where $\la_i = g^{-1}(X_i^T\beta)$. We again have a sandwich estimator for the variance of $\wh{\beta}^{(k)}$,
\[\wh{\beta}-\beta^* \approx \normal(0, Q^{-1}\Sigma Q^{-1}),\]
where $Q = X^TW_{\beta^*} X$ and $\Sigma = \V(X^TW^{(\infty)}Z^{(\infty)})$. When the model is correct, $X^TW_{\beta^*}X \approx \Sigma$. 
\section{Over-dispersion}
The Poisson model requires that $\E[Y_i]=\V(Y_i)$ but this may be far from true. In simple clustering models we in fact have $\V(Y_i) = \phi \E[Y_i]$ where $\phi$ has to be estimated from the data. In a Poisson glm we can estimate $\phi$ with
\[\wh{\phi} = \frac{1}{n-p} \sum_{i}e_i^2, \]
where $p$ is the number of parameters and $e_i$ are the Pearson residuals. Another way to incorporate over-dispersion is to work with a negative binomial distribution. Consider the following non-standard parametrization of the negative binomial distribution, $Y \sim \mathsf{Negative binomial}(\mu,k)$
\[\Pa(Y= y) = \frac{\Gamma(y+k)}{\Gamma(k)\Gamma(y+1)}\left(\frac{k}{\mu+k}\right)^k \left(1-\frac{k}{\mu+k}\right)^y. \]
For a fixed $k$, this is a one-dimensional exponential family with $\E[Y]=\mu$ and $\V(Y) = \mu + \frac{\mu^2}{k}$. Thus, by varying $k$, we get different amounts of over-dispersion in our model. The parameter $k$ can be estimated from the data.
\end{document}
\documentclass{article}
\usepackage{ae,aecompl}
\usepackage{todonotes}
\usepackage{chngcntr}
\usepackage{tikz-cd}
\usepackage{graphicx}
\graphicspath{ {./images/}}
\usepackage[all,cmtip]{xy}
\usepackage{amsmath, amscd}
\usepackage{amsthm}
\usepackage{amssymb}
\usepackage{amsfonts}
\usepackage{bm}
\usepackage{qsymbols}
\usepackage{latexsym}
\usepackage{mathrsfs}
\usepackage{mathtools}
\usepackage{cite}
\usepackage{color}
\usepackage{url}
\usepackage{enumerate}
\usepackage{verbatim}
\usepackage[draft=false, colorlinks=true]{hyperref}
\usepackage{pdfpages}
\usepackage[margin=1.2in]{geometry}
\usepackage{IEEEtrantools}

\usepackage{fancyhdr}


\usepackage[nameinlink]{cleveref}


\DeclareMathOperator*{\ac}{accept}
\DeclareMathOperator*{\amax}{argmax}
\DeclareMathOperator*{\amin}{argmin}
\DeclareMathOperator*{\Aut}{Aut}
\newcommand {\al}{{\alpha}}
\newcommand {\abs}[1]{{\left\lvert#1\right\rvert}}
\newcommand {\A}{{\mathcal{A}}}
\newcommand {\AM}{{\mathrm{AM}}}
\newcommand {\AMp}{{\AM_{p}^{X}\!(\Ri_\w)}}
\newcommand {\B}{{\mathcal{B}}}
\DeclareMathOperator*{\Be}{Bern}
\newcommand {\Br}{{\dot{B}}}
\newcommand {\Ba}{{\mathfrak{B}}}
\newcommand {\C}{{\mathbb C}}
\newcommand {\ce}{\mathrm{c}}
\newcommand {\Ce}{\mathrm{C}}
\newcommand {\Cc}{\mathrm{C_{c}}}
\newcommand {\Ccinf}{\mathrm{C_{c}^{\infty}}}
\DeclareMathOperator{\cov}{Cov}
\DeclareMathOperator{\DEV}{DEV}
\newcommand {\Di}{{\mathbb D}}
\newcommand {\dom}{\mathrm{dom}}
\newcommand{\dist}{\stackrel{\mathrm{dist}}{=}}
\newcommand {\ud}{\mathrm{d}}
\newcommand {\ue}{\mathrm{e}}
\newcommand {\eps}{\varepsilon}
\newcommand {\veps}{\varepsilon}
\newcommand {\vrho}{{\varrho}}
\newcommand {\E}{{\mathbb{E}}}
\newcommand {\Ec}{{\mathcal{E}}}
\newcommand {\Ell}{L}
\newcommand {\Ellp}{{L_{p}[0,1]}}
\newcommand {\Ellpprime}{{L_{p'}([0,1])}}
\newcommand {\Ellq}{{L_{q}([0,1])}}
\newcommand {\Ellqprime}{{L_{q'}([0,1])}}
\newcommand {\Ellr}{L^{r}}
\newcommand {\Ellone}{{L_{1}([0,1])}}
\newcommand{\Elltwo}{{L_{2}([0,1])}}
\newcommand{\Ellinfty}{L^{\infty}}
\newcommand{\Ellinftyc}{L_{\mathrm{c}}^{\infty}}
\newcommand{\exb}[1]{\exp\left\{#1\right\}}
\DeclareMathOperator*{\Ext}{Ext}
\newcommand{\F}{{\mathcal{F}}}
\newcommand{\Fe}{{\mathbb{F}}}
\newcommand{\G}{{\mathcal{G}}}
\newcommand{\HF}{\mathcal{H}_{\text{FIO}}^{1}(\Rd)}
\newcommand{\Hr}{H}
\newcommand{\HT}{\mathcal{H}}
\newcommand{\ui}{\mathrm{i}}
\newcommand{\I}{{I}}
\newcommand{\J}{{\mathcal{J}}}
\newcommand{\id}{{\mathrm{id}}}
\newcommand{\iid}{\stackrel{\mathclap{\normalfont\mbox{iid}}}{\sim}}
\newcommand{\im}{{\text{im }}}
\newcommand{\ind}{{\perp\!\!\!\perp}}
\DeclareMathOperator*{\Int}{int}
\newcommand{\intx}{{\overline{\int_{X}}}}
\newcommand{\inte}{{\overline{\int_{\E}}}}
\newcommand{\la}{\lambda}
\newcommand{\rb}{\rangle}
\newcommand{\lb}{{\langle}}
\newcommand{\La}{\Lambda}
\newcommand{\calL}{{\mathcal{L}}}
\newcommand{\lp}{{\mathcal{L}}^{p}}
\newcommand{\lpo}{{\overline{\mathcal{L}}^{p}\!}}
\newcommand{\Lpo}{{\overline{\Ell}^{p}\!}}
\newcommand{\M}{{\mathbf{M}}}
\newcommand{\Ma}{{\mathcal{M}}}
\newcommand{\N}{{{\mathbb N}}}
\newcommand{\Na}{{{\mathcal{N}}}}
\newcommand{\norm}[1]{\left\|#1\right\|}
\newcommand{\normm}[1]{{\left\vert\kern-0.25ex\left\vert\kern-0.25ex\left\vert #1 
    \right\vert\kern-0.25ex\right\vert\kern-0.25ex\right\vert}}
\newcommand{\Om}{{{\Omega}}}
\newcommand{\one}{{{\bf 1}}}
\newcommand{\pic}{\text{Pic }}
\newcommand{\ph}{{\varphi}}
\newcommand{\Pa}{{\mathbb{P}}}
\newcommand{\Po}{{\mathcal{P}}}
\newcommand{\Q}{{\mathbb{Q}}}
\newcommand{\R}{{\mathbb R}}
\newcommand{\Rd}{{\mathbb{R}^{d}}}
\DeclareMathOperator{\rej}{reject }
\newcommand{\Rn}{{\mathbb{R}^{n}}}
\newcommand{\cR}{{\mathcal{R}}}
\newcommand{\Rad}{{\mathrm{Rad}}}
\newcommand{\ran}{{\mathrm{ran}}}
\newcommand{\Ri}{{\mathrm{R}}}
\newcommand{\supp}{{\mathrm{supp}}}
\newcommand{\Se}{\mathrm{S}}
\newcommand{\Sp}{S^{*}(\Rn)}
\newcommand{\St}{{\mathrm{St}}}
\newcommand{\Sw}{\mathcal{S}}
\newcommand{\T}{{\mathcal{T}}}
\newcommand{\ta}{{\theta}}
\newcommand{\Ta}{{\Theta}}
\newcommand{\topp}{\stackrel{p}{\to}}
\newcommand{\todd}{\stackrel{d}{\to}}
\newcommand{\toL}[1]{\stackrel{L^{#1}}{\to}} 
\newcommand{\toas}{\stackrel{a.s.}{\to}}
\DeclareMathOperator{\V}{Var}
\newcommand {\w}{{\omega}}
\newcommand {\W}{{\mathrm{W}}}
\newcommand {\Wnp}{\text{$\mathrm{W}$\textsuperscript{$n,\!p$}}}
\newcommand {\Wnpeq}{\text{$\mathrm{W}$\textsuperscript{$n\!,\!p$}}}
\newcommand {\Wonep}{\text{$\mathrm{W}$\textsuperscript{$1,\!p$}}}
\newcommand {\Wonepeq}{\text{$\mathrm{W}$\textsuperscript{$1\!,\!p$}}}
\newcommand {\X}{{\mathcal{X}}}
\newcommand {\Z}{{{\mathbb Z}}}
\newcommand {\Za}{{\mathcal{Z}}}
\newcommand {\Zd}{{\Z[\sqrt{d}]}}
\newcommand {\vanish}[1]{\relax}

\newcommand {\wh}{\widehat}
\newcommand {\wt}{\widetilde}
\newcommand {\red}{\color{red}}

% Distributions
\newcommand{\normal}{\mathsf{N}}
\newcommand{\poi}{\mathsf{Poisson}}
\newcommand{\bern}{\mathsf{Bernoulli}}
\newcommand{\bin}{\mathsf{Binomal}}
\newcommand{\multi}{\mathsf{Multinomial}}
\newcommand{\Exp}{\mathsf{Exp}}



% put your command and environment definitions here




% some theorem environments
% remove "[theorem]" if you do not want them to use the same number sequence


  \newtheorem{thrm}{Theorem}
  \newtheorem{lemma}{Lemma}
  \newtheorem{prop}{Proposition}
  \newtheorem{cor}{Corollary}

  \newtheorem{conj}{Conjecture}
  \renewcommand{\theconj}{\Alph{conj}}  % numbered A, B, C etc

  \theoremstyle{definition}
  \newtheorem{defn}{Definition}
  \newtheorem{ex}{Example}
  \newtheorem{exs}{Examples}
  \newtheorem{question}{Question}
  \newtheorem{remark}{Remark}
  \newtheorem{notn}{Notation}
  \newtheorem{exer}{Exercise}




\title{STATS300B -- Lecture 4}
\author{Julia Palacios\\ Scribed by Michael Howes}
\date{01/13/22}

\pagestyle{fancy}
\fancyhf{}
\rhead{STATS300B -- Lecture }
\lhead{01/13/22}
\rfoot{Page \thepage}

\begin{document}
\maketitle
\tableofcontents
\section{Relationships between the modes of convergence}
We ended last lecture with the statement of the following implications,
\[X_n \toas X \Longrightarrow X_n \topp X \Longrightarrow X_n \todd X, \]
and for any $r>0$,
\[X_n \toL{r} X \Longrightarrow X_n \topp X \Longrightarrow X_n \todd X. \]
None of the converse are true in generality, but today we will see some partial converses. Starting with the following,
\begin{proposition}
    Suppose that $X_n \topp X$, then there exists a subsequence $X_{n_k}$ such that $X_{n_k} \toas X$ as $k \to \infty$.
\end{proposition}
\begin{proof}
    Suppose that $X_n \topp X$. Then for every $k \in\N$, there exists $n_k$ such that,
    \[\Pa(\norm{X_{n_k} -X}\ge 1/k) \le 2^{-k}. \]
    We may choose the integers $n_k$ so that they are strictly increasing in $k$. We will now show that $X_{n_k} \toas X$. Let $B$ be the set on which $X_{n_k} \to X$. Note that 
    \[A_m = \bigcap_{k=m}^\infty \{\norm{X_{n_k}-X} < 1/k\} \subseteq B. \]
    Furthermore,
    \[\Pa(A_m^C) \le \sum_{k=m}^\infty \Pa(\norm{X_{n_k}-X} \ge 1/k) \le \sum_{k=m}^\infty \frac{1}{2^k}.\]
    Thus, $\Pa(A_m^C) \to 0$ as $m \to \infty$ and hence $\Pa(A_m) \to 1$ as $m \to \infty$. Since $\Pa(B) \ge \Pa(A_m)$ for every $m$ we have $\Pa(B)=1$. 
\end{proof}
\section{Scales of magnitude}
Recall the following definitions for describing the asymptotic relationship between sequences of numbers.
\begin{definition}
    Let $\{a_n\}_{n \ge 1}$ and $\{b_n\}_{n \ge 1}$ be sequences of constants. Then,
    \begin{enumerate}
        \item $a_n = o(b_n)$ means that $\frac{a_n}{b_n} \to 0$ as $n \to \infty$.
        \item $a_n = O(b_n)$  means that $\limsup\limits_{n \to \infty} \abs{\frac{a_n}{b_n}} < \infty$.
        \item $a_n \sim b_n$ means that $\frac{a_n}{b_n} \to 1$ as $n \to \infty$.
    \end{enumerate}
\end{definition}
These definitions all have probabilistic analogs for sequences of random variables.
\begin{definition}
    Let $\{X_n\}_{n \ge 1}$  be a sequence of random variables and let $\{b_n\}_{n \ge 0}$ be a sequence of constants. Then,
    \begin{enumerate}
        \item $X_n = o_p(b_n)$ means that $\frac{X_n}{a_n} \topp 0$ as $n \to \infty$.
        \item $X_n = O_p(1)$ means that \[\lim_{K \to \infty} \sup_{n}\Pa(\abs{X_n} \ge K) = 0.\]
        \item $X_n = O_p(b_n)$ means that $\frac{X_n}{b_n} = O_p(1)$.
    \end{enumerate}
\end{definition}
The following arithmetic rules are useful and simple to prove.
\begin{lemma}
    We have
    \begin{align*}
        o_p(1)+o_p(1)&=o_p(1)\\
        O_p(1)+O_p(1) &=O_p(1)\\
        O_p(a_n)O_P(b_n)&=O_P(a_nb_n)\\
        O_p(a_n)o_p(b_n)&=o_p(a_nb_n).
    \end{align*}
\end{lemma}
\section{Inequalities for the $L^r$ space}
We will now state and prove some important inequalities and facts about random variables in $L^r$. 
\begin{proposition}
    If $\E\abs{X}^r < \infty$, then $\E\abs{X}^{r'} < \infty$ for all $r' \le r$.
\end{proposition}
\begin{proof}
    If $r'\le r$, then $\abs{X}^{r'} \le 1+\abs{X}^r$. Thus, if $\E\abs{X}^r < \infty$, then
    \[\E\abs{X}^{r'} \le 1+\E\abs{X}^r < \infty \qedhere  \]
\end{proof}
\begin{proposition}
    For any random variable $X$, $\V(X) < \infty$ if and only if $\E[X^2] < \infty$. 
\end{proposition}
\begin{proof}
    If $\E[X^2] < \infty$, then $\E[X] \le \E[\abs{X}] < \infty$. And thus 
    \[\V(X) = \E[X^2]-\E[X]^2 < \infty. \]
    If $\V(X)<\infty$, then $\E[(X-\E[X])^2] < \infty$ in particular $\E[X] \in \R$. Thus, $\E[X^2] = \E[X]^2+\V(X) < \infty$. 
\end{proof}
\begin{proposition}
    For every $r > 0$, $\E\abs{X+Y}^r \le c_r\E\abs{X}^r + c_r \E\abs{X}^r$ where $c_r = 1$ if $0 < r \le 1$ and $c_r=2^{r-1}$ for $r \ge 1$.
\end{proposition}
\begin{proof}
    First suppose that $r > 1$. The function $f(x) = \abs{x}^r$ is convex and thus,
    \[\abs{\frac{X+Y}{2}}^r \le \frac{1}{2}\abs{X}^2+\frac{1}{2}\abs{Y}^r.\]
    Thus, $\abs{X+Y}^r \le 2^{r-1}\abs{X}^r+2^{r-1}\abs{Y}^r$.

    Now suppose $0 < r\le 1$. If $\abs{X+Y}^r \le \abs{X}^r$, then we are done. If $\abs{X+Y}^r > \abs{X}^r$, then 
    \begin{align*}
        \abs{X+Y}^r - \abs{X}^r &= \int_{\abs{X}}^{\abs{X+Y}} rt^{r-1}dt\\
        &\le \int_\abs{X}^{\abs{X}+\abs{Y}} rt^{r-1}dt\\
        &=\int_0^\abs{Y} r(t+\abs{Y})^{r-1}dt\\
        &\le \int_0^\abs{y} rt^{r-1}dt\\
        &= \abs{Y}^r.
    \end{align*}
    Thus, $\abs{X+Y}^r \le \abs{X}^r+\abs{Y}^r$. 
\end{proof}
We next prove H\"{o}lder's inequality.
\begin{proposition}
    Let $r,s \ge 1$ be such that $\frac{1}{r}+\frac{1}{s} = 1$, then 
    \[\E\abs{XY} \le \left(\E\abs{X}^r\right)^{1/r}\left(\E^{1/s}\abs{Y}^s\right)^{1/s}. \]
\end{proposition}
\begin{proof}
    If $\E\abs{X}^r = 0$ or $\E\abs{Y}^s = 0$, then $X=0$ almost surely or $Y=0$ almost surely. Hence, $XY = 0$ almost surely and thus $\E\abs{XY} = 0$. Thus, we may assume $\E\abs{X}^r,\E\abs{Y}^s > 0$. If $\E\abs{X}^r=\infty$ or $\E\abs{Y}^s = \infty$, then we are done. Thus, we will assume that $\E\abs{X}^r, \E\abs{Y}^s \in (0,\infty)$. We will first prove Young's inequality which states for all $a,b \ge 0$,
    \[ab \le \frac{a^r}{r}+\frac{b^s}{s}. \]
    Note that if $a=0$ or $b=0$, then Young's inequality is an equality. Thus assume that $a,b > 0$. We know that the function $x \mapsto e^x$ is convex. Since $\frac{1}{r}+\frac{1}{s} =1$, we thus have
    \begin{align*}
        ab &= e^{\log(ab)}\\
        &= e^{\log(a)+\log(b)}\\
        &=e^{\frac{1}{r}\log(a^r)+\frac{1}{s}\log(b^r)}\\
        &\le \frac{1}{r}e^{\log(a^r)}+\frac{1}{s}e^{\log(b^s)}\\
        &= \frac{a^r}{r}+\frac{b^s}{s},
    \end{align*}
    as claimed. We will now apply Young's inequality point-wise to prove H\"{o}lder's inequality,
    \begin{align*}
        \frac{\E[\abs{XY}]}{\left(\E\abs{X}^r\right)^{1/r}\left(\E^{1/s}\abs{Y}^s\right)^{1/s}} &= \int_\Om \frac{\abs{X}\abs{Y}}{\left(\E\abs{X}^r\right)^{1/r}\left(\E^{1/s}\abs{Y}^s\right)^{1/s}}d\Pa\\
        &\le \int_\Om \frac{1}{r}\frac{\abs{X}^r}{\E\abs{X}^r}+\frac{1}{s}\frac{\abs{Y}^s}{\E\abs{Y}^s}d\Pa\\
        &= \frac{1}{r\E\abs{X}^r}\int_\Om \abs{X}^rd\Pa +\frac{1}{s\E\abs{Y}^s}\int_\Om \abs{Y}^sd\Pa\\
        &=\frac{1}{r}+\frac{1}{s}\\
        &=1. 
    \end{align*}
    Thus ,$\E\abs{XY} \le \left(\E\abs{X}^r\right)^{1/r}\left(\E^{1/s}\abs{Y}^s\right)^{1/s}$.
\end{proof}
\section{Convergence in distribution}
The following theorem shows that we can study convergence in distribution of random vectors by projecting onto one dimensional subspaces.
\begin{theorem}[Cram\'{e}r--Wold device]
    Let $X_n$ and $X$ be random vectors in $\R^d$, then $X_n \todd X$ if and only if $a^T X_n \todd a^T X$ for all constants $a \in \R^d$.
\end{theorem} 
We also have a version of the continuous mapping theorem for convergence in distribution and almost sure convergence. The version for convergence in probability was stated in the first lecture.
\begin{theorem}[Continuous mapping theorem]
    Let $g$ be a continuous function on a set $B$ such that $\Pa(X \in B) =1$. Then 
    \begin{enumerate}
        \item If $X_n \topp X$, then $g(X_n) \topp g(X)$.
        \item If $X_n \todd X$, then $g(X_n) \todd g(X)$.
        \item If $X_n \toas X$, then $g(X_n) \todd g(X)$.
    \end{enumerate}
\end{theorem}
We have already proved 1. For now, we will only prove 3. Once we have Skorokhod's theorem we will see that 3 implies 2. 
\begin{proof}
    Let $A$ be the set on which $X_n \to X$. Since $g$ is continuous on $B$, we have $g(X_n) \to g(X)$ on $A \cap B$. Since $A$ and $B$ both have probability 1, $\Pa(A \cap B) =1 $ and thus $\Pa(g(X_n) \to g(X)) =1$. 
\end{proof}
\begin{definition}
    Given a cumulative distribution function $F$ on $\R$, define $F^{-1}:(0,1)\to \R$ to be the function
    \[F^{-1}(t) = \inf\{x : F(x) \ge t\}. \]
    The function $F^{-1}$ is called the \emph{quantile function} of $F$. 
\end{definition}
\begin{proposition}
    The function $F^{-1}$ is non-decreasing and left-continuous. And for all $t \in (0,1)$ and $x \in \R$,
    \[F^{-1}(t) \le x \Longleftrightarrow t \le F(x).\]
\end{proposition}
\begin{proof}
    The set $\{x: F(x) \ge t\}$ are non-increasing with $t$ and thus if $t\le t'$, then
    \[F^{-1}(t) = \inf\{x : F(x) \ge t\}\le \inf\{x : F(x) \ge t'\} = F^{-1}(t'). \]
    Showing that $F^{-1}$ is non-increasing. Continuity can be proved by considered first the points $t$ such that $F$ is continuous at $F^{-1}(t)$ and then the points $t$ where $F$ is discontinuous at $F^{-1}(t)$. A picture helps.
    
    
    Finally, if $F^{-1}(t) \le x$, then $\inf\{z : F(z) \ge t\} \le x$. Thus, for any $\eps > 0$, there exists $z < x+\eps$ such that $F(z) \le t$. Since $F$ is right-continuous, this implies that 
    \[F(x) = \lim_{z \searrow x} F(z) \le t. \]
    Conversely, if $t \le F(x)$, then $x \in \{z : F(z) \ge t\}$ and so $F^{-1}(t) \le x$. 
\end{proof}
The function $F^{-1}$ also have the following properties.
\begin{proposition}
    Let $X$ be random variable with CDF $F$. Then for all $t \in (0,1)$,
    \[\Pa(F(X) \le t) \le t, \]
    and we have equality if and only if $t$ is in the range of $F$. In particular if $F$ is continuous, then the above holds for all $t \in (0,1)$ and thus $F(X) \sim U(0,1)$. We can also write the above inequality as for all $t \in (0,1)$
    \[F(F^{-1}(t)) \ge t. \]
    We also have $F^{-1}(F(x)) \le x$ for all $x \in \R$ with strict inequality if and only if $F(x-\eps)=F(x)$ for some $\eps > 0$. Thus, $\Pa(F^{-1}(F(X)) \neq X)=0$.
\end{proposition}
We also have
\begin{proposition}
    Let $U \sim U(0,1)$ and let $F$ be some CDF function. Let $X = F^{-1}(U)$. Then $\{X \le x\} = \{U \le F(x)\}$ and so $X \sim F$.
\end{proposition}

We will now state and prove \emph{Skorokhod's representation theorem}.
\begin{theorem}
    Suppose that $X_n \todd X$, then there exist random variables $X_n^*$ and $X^*$ such that $X_n^* \toas X^*$, $X_n^* \dist X_n$ and $X^* \dist X$. 
\end{theorem}
\begin{proof}[Proof]
    Let $F_n$ be the CDF of $X_n$ and let $F$ be the CDF of $X$. Define $X_n^* = F_n^{-1}(U)$ and $X^* = F^{-1}(U)$. We will show that for all but a countable number of $t \in (0,1)$ we have $F_n^{-1}(t) \to F^{-1}(t)$. This will imply that $X_n^* \to X^*$ with probability 1.

    Since $F^{-1}$ is increasing, $F^{-1}$ has at most countably many discontinuities. Let $t \in (0,1)$ be a point such that $F^{-1}$ is continuous at $t$ and let $\eps > 0$. We can find a value $x$ such that $F$ is continuous at $x$ and 
    \[F^{-1}(t)-\eps < x<F^{-1}(t). \]
    It follows that $F(x) < t$. Since $F$ is continuous at $x$ we know that $F_n(x) \to F(x)$. Thus, for large enough $n$, $F_n(x) < t$ which implies $x \le F_n^{-1}(t)$ and so 
    \[\liminf\limits_n F_n^{-1}(t) \ge x \ge F^{-1}(t)-\eps.\] 
    Which implies $\liminf\limits_n F_n^{-1}(t) \ge F^{-1}(t)$. Now consider $s > t$ and choose $y$ such that $F^{-1}(s) < y < F^{-1}(s)+\eps$ and $F$ is continuous at $y$. Thus, $t < s \le F(y)$. Thus, for large enough $n$, $t < F_n(y)$ which implies $F_n^{-1}(t) \le y \le F^{-1}(s)+\eps$. This implies that 
    \[\limsup_{n}F_n^{-1}(t) \le F^{-1}(s)+\eps, \]
    for all $s>t$ and $\eps > 0$ since $F^{-1}$ is continuous at $t$ this implies that 
    \[\limsup_{n}F_n^{-1}(t) \le F^{-1}(t). \]
    Thus, $F_n^{-1}(t) \to F^{-1}(t)$ as required. 
\end{proof}
As a corollary of Skorokhod's theorem we can prove the continuous mapping theorem for convergence in distribution.
\begin{corollary}
    Suppose that $X_n \todd X$ and $g$ is continuous on a set $B$ with $\Pa(X_0 \in B) =1$, then $g(X_n) \todd g(X)$
\end{corollary}
\begin{proof}
    Let $X_n^*$ and $X^*$ be as in Skorokhod's theorem. Then $g(X_n^*) \toas g(X^*)$ be the continuous mapping theorem for almost sure convergence. Thus, $g(X_n^*) \todd g(X^*)$. But $g(X_n^*) \dist g(X_n)$ and $g(X^*) \dist g(X)$ and thus $g(X_n) \todd g(X)$.
\end{proof}
\end{document}
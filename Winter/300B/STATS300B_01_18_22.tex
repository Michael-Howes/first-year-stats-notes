\documentclass{article}
\usepackage{ae,aecompl}
\usepackage{todonotes}
\usepackage{chngcntr}
\usepackage{tikz-cd}
\usepackage{graphicx}
\graphicspath{ {./images/}}
\usepackage[all,cmtip]{xy}
\usepackage{amsmath, amscd}
\usepackage{amsthm}
\usepackage{amssymb}
\usepackage{amsfonts}
\usepackage{bm}
\usepackage{qsymbols}
\usepackage{latexsym}
\usepackage{mathrsfs}
\usepackage{mathtools}
\usepackage{cite}
\usepackage{color}
\usepackage{url}
\usepackage{enumerate}
\usepackage{verbatim}
\usepackage[draft=false, colorlinks=true]{hyperref}
\usepackage{pdfpages}
\usepackage[margin=1.2in]{geometry}
\usepackage{IEEEtrantools}

\usepackage{fancyhdr}


\usepackage[nameinlink]{cleveref}


\DeclareMathOperator*{\ac}{accept}
\DeclareMathOperator*{\amax}{argmax}
\DeclareMathOperator*{\amin}{argmin}
\DeclareMathOperator*{\Aut}{Aut}
\newcommand {\al}{{\alpha}}
\newcommand {\abs}[1]{{\left\lvert#1\right\rvert}}
\newcommand {\A}{{\mathcal{A}}}
\newcommand {\AM}{{\mathrm{AM}}}
\newcommand {\AMp}{{\AM_{p}^{X}\!(\Ri_\w)}}
\newcommand {\B}{{\mathcal{B}}}
\DeclareMathOperator*{\Be}{Bern}
\newcommand {\Br}{{\dot{B}}}
\newcommand {\Ba}{{\mathfrak{B}}}
\newcommand {\C}{{\mathbb C}}
\newcommand {\ce}{\mathrm{c}}
\newcommand {\Ce}{\mathrm{C}}
\newcommand {\Cc}{\mathrm{C_{c}}}
\newcommand {\Ccinf}{\mathrm{C_{c}^{\infty}}}
\DeclareMathOperator{\cov}{Cov}
\DeclareMathOperator{\DEV}{DEV}
\newcommand {\Di}{{\mathbb D}}
\newcommand {\dom}{\mathrm{dom}}
\newcommand{\dist}{\stackrel{\mathrm{dist}}{=}}
\newcommand {\ud}{\mathrm{d}}
\newcommand {\ue}{\mathrm{e}}
\newcommand {\eps}{\varepsilon}
\newcommand {\veps}{\varepsilon}
\newcommand {\vrho}{{\varrho}}
\newcommand {\E}{{\mathbb{E}}}
\newcommand {\Ec}{{\mathcal{E}}}
\newcommand {\Ell}{L}
\newcommand {\Ellp}{{L_{p}[0,1]}}
\newcommand {\Ellpprime}{{L_{p'}([0,1])}}
\newcommand {\Ellq}{{L_{q}([0,1])}}
\newcommand {\Ellqprime}{{L_{q'}([0,1])}}
\newcommand {\Ellr}{L^{r}}
\newcommand {\Ellone}{{L_{1}([0,1])}}
\newcommand{\Elltwo}{{L_{2}([0,1])}}
\newcommand{\Ellinfty}{L^{\infty}}
\newcommand{\Ellinftyc}{L_{\mathrm{c}}^{\infty}}
\newcommand{\exb}[1]{\exp\left\{#1\right\}}
\DeclareMathOperator*{\Ext}{Ext}
\newcommand{\F}{{\mathcal{F}}}
\newcommand{\Fe}{{\mathbb{F}}}
\newcommand{\G}{{\mathcal{G}}}
\newcommand{\HF}{\mathcal{H}_{\text{FIO}}^{1}(\Rd)}
\newcommand{\Hr}{H}
\newcommand{\HT}{\mathcal{H}}
\newcommand{\ui}{\mathrm{i}}
\newcommand{\I}{{I}}
\newcommand{\J}{{\mathcal{J}}}
\newcommand{\id}{{\mathrm{id}}}
\newcommand{\iid}{\stackrel{\mathclap{\normalfont\mbox{iid}}}{\sim}}
\newcommand{\im}{{\text{im }}}
\newcommand{\ind}{{\perp\!\!\!\perp}}
\DeclareMathOperator*{\Int}{int}
\newcommand{\intx}{{\overline{\int_{X}}}}
\newcommand{\inte}{{\overline{\int_{\E}}}}
\newcommand{\la}{\lambda}
\newcommand{\rb}{\rangle}
\newcommand{\lb}{{\langle}}
\newcommand{\La}{\Lambda}
\newcommand{\calL}{{\mathcal{L}}}
\newcommand{\lp}{{\mathcal{L}}^{p}}
\newcommand{\lpo}{{\overline{\mathcal{L}}^{p}\!}}
\newcommand{\Lpo}{{\overline{\Ell}^{p}\!}}
\newcommand{\M}{{\mathbf{M}}}
\newcommand{\Ma}{{\mathcal{M}}}
\newcommand{\N}{{{\mathbb N}}}
\newcommand{\Na}{{{\mathcal{N}}}}
\newcommand{\norm}[1]{\left\|#1\right\|}
\newcommand{\normm}[1]{{\left\vert\kern-0.25ex\left\vert\kern-0.25ex\left\vert #1 
    \right\vert\kern-0.25ex\right\vert\kern-0.25ex\right\vert}}
\newcommand{\Om}{{{\Omega}}}
\newcommand{\one}{{{\bf 1}}}
\newcommand{\pic}{\text{Pic }}
\newcommand{\ph}{{\varphi}}
\newcommand{\Pa}{{\mathbb{P}}}
\newcommand{\Po}{{\mathcal{P}}}
\newcommand{\Q}{{\mathbb{Q}}}
\newcommand{\R}{{\mathbb R}}
\newcommand{\Rd}{{\mathbb{R}^{d}}}
\DeclareMathOperator{\rej}{reject }
\newcommand{\Rn}{{\mathbb{R}^{n}}}
\newcommand{\cR}{{\mathcal{R}}}
\newcommand{\Rad}{{\mathrm{Rad}}}
\newcommand{\ran}{{\mathrm{ran}}}
\newcommand{\Ri}{{\mathrm{R}}}
\newcommand{\supp}{{\mathrm{supp}}}
\newcommand{\Se}{\mathrm{S}}
\newcommand{\Sp}{S^{*}(\Rn)}
\newcommand{\St}{{\mathrm{St}}}
\newcommand{\Sw}{\mathcal{S}}
\newcommand{\T}{{\mathcal{T}}}
\newcommand{\ta}{{\theta}}
\newcommand{\Ta}{{\Theta}}
\newcommand{\topp}{\stackrel{p}{\to}}
\newcommand{\todd}{\stackrel{d}{\to}}
\newcommand{\toL}[1]{\stackrel{L^{#1}}{\to}} 
\newcommand{\toas}{\stackrel{a.s.}{\to}}
\DeclareMathOperator{\V}{Var}
\newcommand {\w}{{\omega}}
\newcommand {\W}{{\mathrm{W}}}
\newcommand {\Wnp}{\text{$\mathrm{W}$\textsuperscript{$n,\!p$}}}
\newcommand {\Wnpeq}{\text{$\mathrm{W}$\textsuperscript{$n\!,\!p$}}}
\newcommand {\Wonep}{\text{$\mathrm{W}$\textsuperscript{$1,\!p$}}}
\newcommand {\Wonepeq}{\text{$\mathrm{W}$\textsuperscript{$1\!,\!p$}}}
\newcommand {\X}{{\mathcal{X}}}
\newcommand {\Z}{{{\mathbb Z}}}
\newcommand {\Za}{{\mathcal{Z}}}
\newcommand {\Zd}{{\Z[\sqrt{d}]}}
\newcommand {\vanish}[1]{\relax}

\newcommand {\wh}{\widehat}
\newcommand {\wt}{\widetilde}
\newcommand {\red}{\color{red}}

% Distributions
\newcommand{\normal}{\mathsf{N}}
\newcommand{\poi}{\mathsf{Poisson}}
\newcommand{\bern}{\mathsf{Bernoulli}}
\newcommand{\bin}{\mathsf{Binomal}}
\newcommand{\multi}{\mathsf{Multinomial}}
\newcommand{\Exp}{\mathsf{Exp}}



% put your command and environment definitions here




% some theorem environments
% remove "[theorem]" if you do not want them to use the same number sequence


  \newtheorem{thrm}{Theorem}
  \newtheorem{lemma}{Lemma}
  \newtheorem{prop}{Proposition}
  \newtheorem{cor}{Corollary}

  \newtheorem{conj}{Conjecture}
  \renewcommand{\theconj}{\Alph{conj}}  % numbered A, B, C etc

  \theoremstyle{definition}
  \newtheorem{defn}{Definition}
  \newtheorem{ex}{Example}
  \newtheorem{exs}{Examples}
  \newtheorem{question}{Question}
  \newtheorem{remark}{Remark}
  \newtheorem{notn}{Notation}
  \newtheorem{exer}{Exercise}




\title{STATS300B -- Lecture 5}
\author{Julia Palacios\\ Scribed by Michael Howes}
\date{01/18/22}

\pagestyle{fancy}
\fancyhf{}
\rhead{STATS300B -- Lecture 5}
\lhead{01/18/22}
\rfoot{Page \thepage}

\begin{document}
\maketitle
\tableofcontents
\section{Convergence of random variables}
\subsection{Skorokhod's theorem}
We ended last lecture with the statement and a proof of Skorokhod's theorem.
\begin{theorem}[Skorokhod]
    Suppose that $X_n \todd X_0$. Then there exist random variables $X_n^*$ such that $X_n^* \dist X_n$ and $X_n^* \toas X_0^*$.
\end{theorem}
The idea behind the proof was to work with the inverse CDFs, 
\[F_n^{-1}(t) = \inf\{x \in \R : F_n(x) \ge t\}, \]
where $F_n(x) = \Pa(X_n \le x)$. We defined $X_n^* = F_n^{-1}(\xi)$ where $\xi \sim U(0,1)$. We showed that if $X_n \todd X_0$, then $F_n^{-1}(\xi) \toas F_0^{-1}(\xi)$.
\subsection{Marginal convergence}
Let $X_n$ and $X$ be random $k$-vectors. Then
\begin{enumerate}
    \item $X_n \topp X$ if and only if $X_{n,j} \topp X_j$ for all $j=1,\ldots, k$.
    \item $X_n \toas X$ if and only if $X_{n,j} \toas X_j$ for all $j=1,\ldots, k$.
    \item $X_n \toL{p} X$ if and only if $X_{n,j} \toL{p} X_j$ for all $j=1,\ldots, k$.
    \item If $X_n \todd X$, then $X_{n,j} \todd X_j$ for all $j=1,\ldots, k$. But (!) it is possible that $X_{n,j} \todd X_j$ for all $j=1,\ldots, k$ and $X_n \not\todd X$. Indeed, $X_n$ need not have any distribution limit (see homework).
\end{enumerate}
So for convergence in distribution, marginal (or element-wise) convergence is not enough to imply joint convergence. The Cramer--Wald device provides one work around. To show that $X_n \todd X$, it suffices to show that $a^TX_n \todd a^TX$ for all $a \in \R^k$ and if $X_n \todd X$, then $a^T X_n \todd a^TX$. There is a special case when marginal convergence in distribution does imply joint convergence in distribution. This is when all but one of the entries of $X$ are constant. More precisely, one can show, that if $X_n,Y_n,X$ are random vectors and $y$ is a constant, then
\[X_n \todd X, Y_n \topp y \Longrightarrow (X_n,Y_n) \todd (X,y). \]
Note that since $y$ is a constant, the condition $Y_n \topp y$ is equivalent to $Y_n \todd y$. This theorem can be combined with the continuous mapping theorem to prove Slutsky's theorem which is a real workhorse of asymptotic statistics.
\begin{theorem}[Slutsky's]
    Suppose $X_n \todd X$ and $Y_n \topp c$, then
    \begin{enumerate}
        \item $X_n+Y_n \todd X+c$
        \item $Y_nX_n \todd cX$,
        \item $X_n/Y_n \todd X/c$ provided $c \neq 0$.
    \end{enumerate}
\end{theorem}
\begin{proof}
    As claimed above, we know that $(X_n,Y_n) \todd (X,c)$. The function $(x,y) \mapsto x+y$, $(x,y)\mapsto xy$ and $(x,y) \mapsto x/y$ are all continuous on their domains. Thus, by the continuous mapping theorem the above results hold.
\end{proof}
\begin{example}[One-sided t-test] Suppose $X_1,\ldots,X_n$ are i.i.d. with $\E[X_i]=\mu$ and $\V(X_i) =\sigma^2 < \infty$. Suppose we wish to test $H_0 : \mu \le \mu_0$ against $H_1 : \mu > \mu_0$. 

    If $X_i$ was normally distributed, we know that the uniformly most powerful unbiased test rejects when $T_n \ge t_{n-1,\al}$ where 
    \[T_n = \frac{\sqrt{n}(\bar{X}_n-\mu_0)}{S_n}, \]
    and 
    \[ S_n^2 = \frac{1}{n-1}\sum_{i=1}^n (X_i-\bar{X}_n)^2.\]
    When $X_i$ are normally distributed and the null holds, we know that $T_n$ has the students $t$ distribution with $n-1$ degrees of freedom. We would like to know the asymptotic distribution of $T_n$ in the non-normal cases. In particular,
    \begin{enumerate}
        \item What is the asymptotic distribution of $T_n$ when $\mu = \mu_0$?
        \item What is the asymptotic distribution of $T_n$ when $\mu >\mu_0$?
    \end{enumerate}
    With Slutsky's theorem we can answer both of these,
    \begin{enumerate}
        \item We have seen before that the weak law of large numbers implies that $S_n \topp \sigma$ and that the central limit theorem implies that $\sqrt{n}(\bar{X}_n-\mu_0)\todd \normal(0,\sigma^2)$. Thus, Slutsky's theorem implies that $T_n \todd \normal(0,1)$.
        \item If $\mu > \mu_0$, then we can write $T_n$ as,
        \[T_n = \frac{\sqrt{n}(\bar{X}_n-\mu)+\sqrt{n}(\mu-\mu_0)}{S_n} =\frac{\sqrt{n}(\bar{X}_n-\mu)}{S_n}+\frac{\sqrt{n}(\mu-\mu_0)}{S_n}. \] 
        We know that $\frac{\sqrt{n}(\bar{X}_n-\mu)}{S_n} \todd \normal(0,1)$ but $\frac{\sqrt{n}(\mu-\mu_0)}{S_n} \topp +\infty$. Thus, when $\mu > \mu_0$, $T_n \todd +\infty$. Thus, for $\mu > \mu_0$,
        \[\Pa_{\mu,\sigma^2}(T_n \ge t_{n-1,\al}) \stackrel{n \to \infty}{\to} 1. \]
        Therefor, under any alternative, the test has asymptotic power 1.
    \end{enumerate}
\end{example}
\begin{example}[Testing variance]
    Suppose $X_1,X_2,\ldots$ are i.i.d. with $\E[X_1]<\infty$, $\V(X_1)=\sigma^2$ and $\E[(X_1-\mu)^4] = \mu_4 < \infty$. Let $Y_i = (X_i-\mu)^2$ then $\E[Y_i] =\sigma^2$ and $\V(Y_i) = \E[(X_i-\mu)^4] - \E[(X_i-\mu)^2]^2 = \mu_4-\sigma^4$. Thus,
    \[T_n = \frac{\sqrt{n}(\bar{Y}_n-\sigma^2)}{\sqrt{\mu_4-\sigma^4}} \to \normal(0,1). \]
\end{example}
\begin{example}[Pearson's chi-squared]
    Suppose $X_1,\ldots,X_n$ are i.i.d. with distribution $\multi_k(1,p)$. That is each $X_i$ is a vector in $\{0,1\}^k$ with exactly one entry equal to one and,
    \[\Pa(X_{i,j} = 1) = p_j,  \]
    for every $j$. Let $N=\sum_{i=1}^n X_i \sim \multi_k(n,p)$ and let $\wh{p} = \frac{N}{n}$. Let $H_0$ be the hypothesis $p= p_0$ and let $H_1$ be $p \neq p_0$. Pearson's chi-squared test statistic of $H_0$ against $H_1$ is,
    \[Q_n = \sum_{i=1}^k \frac{(N_j-np_{0,j})^2}{np_{0,j}}.\]
    We will show that under $H_0$, $Q \todd \chi^2_{k-1}$ as $n \to \infty$. It suffices to write $Q_n = W_n^TW_n$ where $W_n \todd \normal(0,I_{k-1})$. With this in mind, define,
    \[Z_n = \begin{bmatrix}
        \frac{N_1-np_{0,1}}{\sqrt{np_{0,1}}}\\
        \vdots \\
        \frac{N_k-np_{0,k}}{\sqrt{np_{0,k}}}
    \end{bmatrix} \in \R^k.\]
\end{example}
Note that $\E_{H_0}[Z]=0$ and, for $i \neq j$,
\begin{align*}
    \text{Cov}_{H_0}(Z_{n,i}Z_{n,j})&= \frac{1}{n\sqrt{p_{0,i}p_{0,j}}}\text{Cov}_{H_0}(N_{n,i}N_{n,j})\\
    &= \frac{1}{n\sqrt{p_{0,i}p_{0,j}}} np_{0,i}p_{0,j}\\
    &=\sqrt{p_{0,i}p_{0,j}}.
\end{align*}
And \[\V_{H_0}(Z_{n,j}) = \frac{1}{np_{0,j}}\V_{H_0}(N_{n,j}) = \frac{np_{0,j}(1-p_{0,j})}{np_{0,j}} = 1-p_{0,j}.\] 
Thus, $\E[Z_n]=0$ and $\V(Z_n) = \Sigma$ where $\Sigma = I - \sqrt{p_0}\sqrt{p_0}^T$. Furthermore, by the multivariate CLT, $Z_n \todd \normal_K(0,\Sigma)$. Now let $\Gamma$ be an orthogonal matrix with first row equal to $\sqrt{p_0}$. It follows that,
\begin{align*}
    Z_n^TZ_n &= (\Gamma Z_n)^T(\Gamma Z_n)\\
    &=V_n^TV_n.
\end{align*}
By Slutsky's we have $V_n \todd V \sim \Gamma\normal_k(0,\Sigma) = \normal_k(0,\Gamma^T\Sigma \Gamma)$. Furthermore,
\[\Gamma^T\Sigma \Gamma = \Gamma^T\Gamma - \Gamma^T \sqrt{p_0} \sqrt{p_0}^T \Gamma = I - e_1e_1^T = \begin{bmatrix}
    0&0&\ldots & 0\\
    0&1&\ldots & 0\\
    \vdots&\vdots&\ddots&\vdots\\
    0&0&\ldots &1
\end{bmatrix}\]
Thus $V^TV \sim \chi^2_{k-1}$ and $Q_n \todd \chi^2_{k-1}$.
\section{Delta method}
The central limit theorem tells us that $\sqrt{n}(\bar{X_n}-\mu)$ converges in distribution to $\normal(0,\sigma^2)$. Often we aren't just interested in the mean. The \emph{delta method} allows us to study the asymptotic distribution of functions of the mean.
\begin{theorem}[Delta method 1]
    Suppose that $X_1,X_2,\ldots$ are i.i.d. with mean $\mu$ and variance $\sigma^2 < \infty$. Suppose that $f$ is differentiable at $\mu$, then 
    \[\sqrt{n}(f(\bar{X}_n)-f(\mu)) \todd \normal(0,[f'(\mu)]^2\sigma^2). \]
\end{theorem}
\begin{proof}
    We can prove this result by combining Slutsky's theorem with a Taylor's approximation. We have 
    \[f(\bar{X}_n) = f(\mu) + f'(\mu)(\bar{X}_n-\mu) + o(\bar{X}_n-\mu). \]
    By the central limit theorem $\sqrt{n}(\bar{X}_n - \mu) \todd \normal(0,\sigma^2)$ and so $o(\bar{X}_n-\mu) = o_p(\sqrt{n})$. Thus, rearranging the above we get,
    \[ \sqrt{n}(f(\bar{X}_n)-f(\mu)) = f'(\mu)\sqrt{n}(\bar{X}_n-\mu)+o_p(1).\]
    Slutsky's theorem thus implies that $\sqrt{n}(f(\bar{X}_n)-f(\mu)) \todd f'(\mu)\normal(0,\sigma^2) = \normal(0,[f'(\mu)]^2\sigma^2)$.
\end{proof}
The delta method can be generalized to situations other than that of the central limit theorem. There are also higher dimensional versions like the following.
\begin{theorem}[Delta mathod 2 (higher dimensional)]
    Suppose that $X_1,X_2,\ldots$ are random $k$-vectors such that,
    \[a_n(X_n-c) \todd Y. \]
    If $f$ is a real-valued function that is differentiable at $c$, then
    \[a_n(f(X_n) - f(c)) \todd \nabla f(c)^T Y. \]
\end{theorem}
The proof is via a Taylor's approximation like the previous version.
\begin{example}
    Suppose $X_1,X_2,\ldots$ are i.i.d. random vectors with $\E[X_1]=\ta \neq 0$ and $\text{Cov}(X_1)=\Sigma$. Define $\phi(h) = \frac{1}{2}\norm{h}_2^2$. By the multivariate central limit theorem, $\sqrt{n}(\bar{X}_n-\ta) \todd \normal_k(0,\Sigma)$. Thus,
    \[\sqrt{n}(\phi(\bar{X}_n)-\phi(\ta)) \todd \ta^T\normal(0, \Sigma )=\normal(0,\ta^T\Sigma \ta),\]
    since $\nabla h(\ta) = \ta$.
\end{example}
\begin{example}
    Let $S_n^2 = \frac{1}{n}\sum_{i=1}^n X_i^2 - \bar{X}_n^2$. We know that $S_n^2 \topp \sigma^2 = \V(X_1)$, but can we say something about the limiting distribution of $S_n^2$? Note that $S_n^2 = \phi(\bar{X}_n, \bar{X^2}_n)$ where $\phi(x,y) = y-x^2$. Note that $\nabla \phi(x,y) = (-2x,1)^T$. Also, assume $X_1$ has a finite fourth moment,
    \[\sqrt{n}\left(\begin{bmatrix}
        \bar{X}_n\\\bar{X^2}{n}
    \end{bmatrix} - \begin{bmatrix}
        \mu \\ \mu^2+\sigma^2
    \end{bmatrix}\right)\todd \normal_2(0,\Sigma). \]
    Where 
    \[\Sigma = \begin{bmatrix}
        \V(X)& \cov(X,X^2)\\ \cov(X,X^2)&\V(X^2)
    \end{bmatrix} \]
    Note that $\nabla \phi(\mu,\mu^2+\sigma^2) = (-2\mu,1)^T$. Thus,
    \begin{align*}
        \nabla \phi(\mu,\mu^2+\sigma^2)^T\Sigma \nabla \phi(\mu,\mu^2+\sigma^2) &= [-2\mu,1]\begin{bmatrix}
        -2\mu\sigma^2+\cov(X,X^2)\\
        -2\mu\cov(X,X^2)+\V(X^2)
    \end{bmatrix}\\
    & = 4\mu^2\sigma^2-4\mu\cov(X,X^2)+\V(X^2) \\
    &=:\gamma.
\end{align*}
By the delta-method $\sqrt{n}(S_n^2 - \sigma^2) \todd \normal(0,\gamma)$.
\end{example}
If the first derivate of our function is zero, then we can use the higher order delta method to get a better approximation.
\begin{theorem}[Delta method 3 (higher order)]
    Suppose that $X_n$ are random $k$-vectors such that 
    \[r_n(X_n-\ta) \todd X, \]
    where $r_n$ is a deterministic function with $r_n \to +\infty$. Let $\phi$ be a real-valued function that is twice differentiable at $\ta$ with $\phi'(\ta)=0$. Then,
    \[r_n^2(\phi(X_n)-\phi(\ta)) \todd \frac{1}{2}X^T\nabla^2\phi(\ta)T.\]
\end{theorem}
Note that since $r_n \to + \infty$, $r_n^2 > r_n$ for sufficiently large $n$. Thus, the rate of convergence of $\phi(X_n)$ to $\phi(\ta)$ is faster. This is because we have to multiply by large numbers in order to have $\phi(X_n)-\phi(X)$ converge to a non-trivial distribution. We will prove the higher order delta method at the start of the next class. 
\end{document}
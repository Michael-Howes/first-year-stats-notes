\documentclass{article}
\usepackage{ae,aecompl}
\usepackage{todonotes}
\usepackage{chngcntr}
\usepackage{tikz-cd}
\usepackage{graphicx}
\graphicspath{ {./images/}}
\usepackage[all,cmtip]{xy}
\usepackage{amsmath, amscd}
\usepackage{amsthm}
\usepackage{amssymb}
\usepackage{amsfonts}
\usepackage{bm}
\usepackage{qsymbols}
\usepackage{latexsym}
\usepackage{mathrsfs}
\usepackage{mathtools}
\usepackage{cite}
\usepackage{color}
\usepackage{url}
\usepackage{enumerate}
\usepackage{verbatim}
\usepackage[draft=false, colorlinks=true]{hyperref}
\usepackage{pdfpages}
\usepackage[margin=1.2in]{geometry}
\usepackage{IEEEtrantools}

\usepackage{fancyhdr}


\usepackage[nameinlink]{cleveref}


\DeclareMathOperator*{\ac}{accept}
\DeclareMathOperator*{\amax}{argmax}
\DeclareMathOperator*{\amin}{argmin}
\DeclareMathOperator*{\Aut}{Aut}
\newcommand {\al}{{\alpha}}
\newcommand {\abs}[1]{{\left\lvert#1\right\rvert}}
\newcommand {\A}{{\mathcal{A}}}
\newcommand {\AM}{{\mathrm{AM}}}
\newcommand {\AMp}{{\AM_{p}^{X}\!(\Ri_\w)}}
\newcommand {\B}{{\mathcal{B}}}
\DeclareMathOperator*{\Be}{Bern}
\newcommand {\Br}{{\dot{B}}}
\newcommand {\Ba}{{\mathfrak{B}}}
\newcommand {\C}{{\mathbb C}}
\newcommand {\ce}{\mathrm{c}}
\newcommand {\Ce}{\mathrm{C}}
\newcommand {\Cc}{\mathrm{C_{c}}}
\newcommand {\Ccinf}{\mathrm{C_{c}^{\infty}}}
\DeclareMathOperator{\cov}{Cov}
\DeclareMathOperator{\DEV}{DEV}
\newcommand {\Di}{{\mathbb D}}
\newcommand {\dom}{\mathrm{dom}}
\newcommand{\dist}{\stackrel{\mathrm{dist}}{=}}
\newcommand {\ud}{\mathrm{d}}
\newcommand {\ue}{\mathrm{e}}
\newcommand {\eps}{\varepsilon}
\newcommand {\veps}{\varepsilon}
\newcommand {\vrho}{{\varrho}}
\newcommand {\E}{{\mathbb{E}}}
\newcommand {\Ec}{{\mathcal{E}}}
\newcommand {\Ell}{L}
\newcommand {\Ellp}{{L_{p}[0,1]}}
\newcommand {\Ellpprime}{{L_{p'}([0,1])}}
\newcommand {\Ellq}{{L_{q}([0,1])}}
\newcommand {\Ellqprime}{{L_{q'}([0,1])}}
\newcommand {\Ellr}{L^{r}}
\newcommand {\Ellone}{{L_{1}([0,1])}}
\newcommand{\Elltwo}{{L_{2}([0,1])}}
\newcommand{\Ellinfty}{L^{\infty}}
\newcommand{\Ellinftyc}{L_{\mathrm{c}}^{\infty}}
\newcommand{\exb}[1]{\exp\left\{#1\right\}}
\DeclareMathOperator*{\Ext}{Ext}
\newcommand{\F}{{\mathcal{F}}}
\newcommand{\Fe}{{\mathbb{F}}}
\newcommand{\G}{{\mathcal{G}}}
\newcommand{\HF}{\mathcal{H}_{\text{FIO}}^{1}(\Rd)}
\newcommand{\Hr}{H}
\newcommand{\HT}{\mathcal{H}}
\newcommand{\ui}{\mathrm{i}}
\newcommand{\I}{{I}}
\newcommand{\J}{{\mathcal{J}}}
\newcommand{\id}{{\mathrm{id}}}
\newcommand{\iid}{\stackrel{\mathclap{\normalfont\mbox{iid}}}{\sim}}
\newcommand{\im}{{\text{im }}}
\newcommand{\ind}{{\perp\!\!\!\perp}}
\DeclareMathOperator*{\Int}{int}
\newcommand{\intx}{{\overline{\int_{X}}}}
\newcommand{\inte}{{\overline{\int_{\E}}}}
\newcommand{\la}{\lambda}
\newcommand{\rb}{\rangle}
\newcommand{\lb}{{\langle}}
\newcommand{\La}{\Lambda}
\newcommand{\calL}{{\mathcal{L}}}
\newcommand{\lp}{{\mathcal{L}}^{p}}
\newcommand{\lpo}{{\overline{\mathcal{L}}^{p}\!}}
\newcommand{\Lpo}{{\overline{\Ell}^{p}\!}}
\newcommand{\M}{{\mathbf{M}}}
\newcommand{\Ma}{{\mathcal{M}}}
\newcommand{\N}{{{\mathbb N}}}
\newcommand{\Na}{{{\mathcal{N}}}}
\newcommand{\norm}[1]{\left\|#1\right\|}
\newcommand{\normm}[1]{{\left\vert\kern-0.25ex\left\vert\kern-0.25ex\left\vert #1 
    \right\vert\kern-0.25ex\right\vert\kern-0.25ex\right\vert}}
\newcommand{\Om}{{{\Omega}}}
\newcommand{\one}{{{\bf 1}}}
\newcommand{\pic}{\text{Pic }}
\newcommand{\ph}{{\varphi}}
\newcommand{\Pa}{{\mathbb{P}}}
\newcommand{\Po}{{\mathcal{P}}}
\newcommand{\Q}{{\mathbb{Q}}}
\newcommand{\R}{{\mathbb R}}
\newcommand{\Rd}{{\mathbb{R}^{d}}}
\DeclareMathOperator{\rej}{reject }
\newcommand{\Rn}{{\mathbb{R}^{n}}}
\newcommand{\cR}{{\mathcal{R}}}
\newcommand{\Rad}{{\mathrm{Rad}}}
\newcommand{\ran}{{\mathrm{ran}}}
\newcommand{\Ri}{{\mathrm{R}}}
\newcommand{\supp}{{\mathrm{supp}}}
\newcommand{\Se}{\mathrm{S}}
\newcommand{\Sp}{S^{*}(\Rn)}
\newcommand{\St}{{\mathrm{St}}}
\newcommand{\Sw}{\mathcal{S}}
\newcommand{\T}{{\mathcal{T}}}
\newcommand{\ta}{{\theta}}
\newcommand{\Ta}{{\Theta}}
\newcommand{\topp}{\stackrel{p}{\to}}
\newcommand{\todd}{\stackrel{d}{\to}}
\newcommand{\toL}[1]{\stackrel{L^{#1}}{\to}} 
\newcommand{\toas}{\stackrel{a.s.}{\to}}
\DeclareMathOperator{\V}{Var}
\newcommand {\w}{{\omega}}
\newcommand {\W}{{\mathrm{W}}}
\newcommand {\Wnp}{\text{$\mathrm{W}$\textsuperscript{$n,\!p$}}}
\newcommand {\Wnpeq}{\text{$\mathrm{W}$\textsuperscript{$n\!,\!p$}}}
\newcommand {\Wonep}{\text{$\mathrm{W}$\textsuperscript{$1,\!p$}}}
\newcommand {\Wonepeq}{\text{$\mathrm{W}$\textsuperscript{$1\!,\!p$}}}
\newcommand {\X}{{\mathcal{X}}}
\newcommand {\Z}{{{\mathbb Z}}}
\newcommand {\Za}{{\mathcal{Z}}}
\newcommand {\Zd}{{\Z[\sqrt{d}]}}
\newcommand {\vanish}[1]{\relax}

\newcommand {\wh}{\widehat}
\newcommand {\wt}{\widetilde}
\newcommand {\red}{\color{red}}

% Distributions
\newcommand{\normal}{\mathsf{N}}
\newcommand{\poi}{\mathsf{Poisson}}
\newcommand{\bern}{\mathsf{Bernoulli}}
\newcommand{\bin}{\mathsf{Binomal}}
\newcommand{\multi}{\mathsf{Multinomial}}
\newcommand{\Exp}{\mathsf{Exp}}



% put your command and environment definitions here




% some theorem environments
% remove "[theorem]" if you do not want them to use the same number sequence


  \newtheorem{thrm}{Theorem}
  \newtheorem{lemma}{Lemma}
  \newtheorem{prop}{Proposition}
  \newtheorem{cor}{Corollary}

  \newtheorem{conj}{Conjecture}
  \renewcommand{\theconj}{\Alph{conj}}  % numbered A, B, C etc

  \theoremstyle{definition}
  \newtheorem{defn}{Definition}
  \newtheorem{ex}{Example}
  \newtheorem{exs}{Examples}
  \newtheorem{question}{Question}
  \newtheorem{remark}{Remark}
  \newtheorem{notn}{Notation}
  \newtheorem{exer}{Exercise}




\title{STATS300B -- Lecture 1}
\author{Julia Palacios\\ Scribed by Michael Howes}
\date{01/04/22}

\pagestyle{fancy}
\fancyhf{}
\rhead{STATS300B -- Lecture 1}
\lhead{01/04/22}
\rfoot{Page \thepage}

\begin{document}
\maketitle
\tableofcontents
\section{Overview}
This course is about asymptotic statistics and what happens when the number of samples goes to infinity. We'll cover three main topics
\begin{enumerate}
    \item Finite dimensional problems and models.
    \item Optimality and comparisons.
    \item Infinite dimensional and uniform problems.
\end{enumerate}
This course is useful for statistics, machine learning, computer science and data science. The marks for the course will be calculated as follows:
\begin{itemize}
    \item Weekly homework (40\%),
    \item Midterm exam (30\%),
    \item Final exam (25\%),
    \item Participation (5\%).
\end{itemize}
The midterm will take place on February 8 and will be in class. There are three textbooks for the class
\begin{itemize}
    \item \href{https://www.cambridge.org/core/books/asymptotic-statistics/A3C7DAD3F7E66A1FA60E9C8FE132EE1D}{Asymptotic statistics} by van der Vaart (this will be the main text).
    \item \href{https://www.cambridge.org/core/books/highdimensional-probability/797C466DA29743D2C8213493BD2D2102}{High -- dimensional probability} by Vershynin (this will be used towards the end of the course).
    \item \href{https://link.springer.com/book/10.1007/978-0-387-93839-4}{Theoretical statistics} by Keener (this will be used a bit at the start of the course).
\end{itemize}
\section{Convergence of random variables}
See van der Vaart Ch 2,3 and Keener Ch 8.

In STATS 300A we studied properties of estimators that held in an \emph{exact} sense (for example unbiased, MRE, UMRUE, etc.). In this course we will study properties that hold ``approximately'' or ``in the limit.'' To make this precise we have to discuss what we mean by taking limits of estimators and limits of random variables.
\begin{definition}
    Let $X_1,X_2,\ldots$ and $X$ be random vectors. We say that \emph{$X_n$ converges to $X$ in probability} and write $X_n \topp X$ if for every $\eps > 0$,
    \[\Pa(\norm{X_n - X} > \eps) \to 0,\]
    as $n \to \infty$.
\end{definition}
Note that be replacing $\norm{X_n - X}$ with $d(X_n,X)$ we can generalize convergence in probability to arbitrary metric spaces.
\begin{theorem}[Chebyshev's inequality]
    For any random variable $X$ and any constant $a > 0$, 
    \[ \Pa(\abs{X} \ge a ) \le \frac{\E[X^2]}{a^2}.\]
\end{theorem}
\begin{proof}
    Note that $\one_{\abs{X} \ge a} \le \frac{X^2}{a^2}$. Thus, by monotonicity and linearity of expectation we have
    \[\Pa(\abs{X} \ge a) = \E[\one_{\abs{X} \ge a}] \le \E\left[\frac{X^2}{a^2}\right]=\frac{\E[X^2]}{a^2}.\qedhere\] 
\end{proof}
Chebyshev's inequality is a powerful tool for proving convergence in probability.
\begin{proposition}
    If $\E(Y_n-Y)^2 \to 0$ and $n \to \infty$, then $Y_n \topp Y$.
\end{proposition}
\begin{proof}
    Note that for all $\eps > 0$,
    \[0 \le \Pa(\abs{Y_n-Y} \ge \eps) \le \frac{\E(Y_n-Y)^2}{\eps^2}. \]
    Thus, $\lim\limits_{n \to \infty} \Pa(\abs{Y_n-Y} \ge \eps) \to 0$. 
\end{proof}
\begin{example}
    Suppose $X_1,X_2,\ldots$ are i.i.d. with mean $\mu$ and variance $\sigma^2$, then $\bar{X}_n \topp \mu$.
    \begin{proof}
        Note that,
        \[\E(\bar{X}_n-\mu)^2 = \V(\bar{X}_n) = \frac{\V(X_1)}{n} = \frac{\sigma^2}{n}.   \]
        So $\E(\bar{X}_n - \mu)^2 \to 0$.
    \end{proof}
\end{example}
The next definition relates convergence in probability to estimation.
\begin{definition}
    A sequence of estimators $\delta_n$ is consistent for $g(\ta)$ if for all $\ta \in \Om$, $\delta_n \topp g(\ta)$ as $n \to \infty$.
\end{definition}
\begin{remark}
    When using squared error loss, we consider the MSE $R(\ta,\delta_n) = \E_\ta(\delta_n-g(\ta))^2$. By Proposition 1, if the MSE goes to 0 for all $\ta$, then $\delta_n$ is consistent. Recall that we have the decomposition
    \[R(\ta,\delta_n) = b_n^2(\ta) + \V_\ta(\delta_n),\]
    where $b_n(\ta)$ is the bias of $\delta_n$. Thus, if the bias and variance of a sequence of estimators go to zero, then the estimators are consistent.
\end{remark}
\begin{exercise}
    If $X_n \topp X$ and $Y_n \topp Y$, then $X_n+Y_n \topp X+Y$.
    \begin{proof}
        Recall that
        \[\norm{X_n+Y_n-(X+Y)} \le \norm{X_n-X}+\norm{Y_n-Y}.\]
        Thus,
        \[\Pa(\norm{X_n+Y_n - (X+Y)}\ge\eps) \le \Pa( \norm{X_n-X}+\norm{Y_n-Y} \ge \eps). \]
        Also, if $\norm{X_n-X}+\norm{Y_n-Y} \ge \eps$, then $\norm{X_n -X} \ge \eps/2$ or $\norm{Y_n-Y} \ge \eps/2$. Thus,
        \[\Pa(\norm{X_n+Y_n - (X+Y)}\ge\eps) \le \Pa(\norm{X_n-X} \ge \eps/2 )+\Pa(\norm{Y_n-Y} \ge \eps/2) \to 0, \]
        as $n \to \infty$.
    \end{proof}
\end{exercise}
\begin{exercise}
    If $X_n \topp a$ where $a$ is a constant and $g$ is a function that is continuous at $a$, then $g(X_n) \topp g(a)$.
    \begin{proof}
        Let $\eps > 0$. There exists $\delta > 0$ such that $\norm{x-a} < \delta$ implies $\norm{g(x)-g(a)} < \eps$. Thus,
        \[\Pa(\norm{g(X_n)-g(a)} \ge \eps) \le \Pa(\norm{X_n-a} \ge \delta) \to 0, \]
        as $n \to \infty$. 
    \end{proof}
\end{exercise}
\end{document}
\documentclass{article}
\usepackage{ae,aecompl}
\usepackage{todonotes}
\usepackage{chngcntr}
\usepackage{tikz-cd}
\usepackage{graphicx}
\graphicspath{ {./images/}}
\usepackage[all,cmtip]{xy}
\usepackage{amsmath, amscd}
\usepackage{amsthm}
\usepackage{amssymb}
\usepackage{amsfonts}
\usepackage{bm}
\usepackage{qsymbols}
\usepackage{latexsym}
\usepackage{mathrsfs}
\usepackage{mathtools}
\usepackage{cite}
\usepackage{color}
\usepackage{url}
\usepackage{enumerate}
\usepackage{verbatim}
\usepackage[draft=false, colorlinks=true]{hyperref}
\usepackage{pdfpages}
\usepackage[margin=1.2in]{geometry}
\usepackage{IEEEtrantools}

\usepackage{fancyhdr}


\usepackage[nameinlink]{cleveref}


\DeclareMathOperator*{\ac}{accept}
\DeclareMathOperator*{\amax}{argmax}
\DeclareMathOperator*{\amin}{argmin}
\DeclareMathOperator*{\Aut}{Aut}
\newcommand {\al}{{\alpha}}
\newcommand {\abs}[1]{{\left\lvert#1\right\rvert}}
\newcommand {\A}{{\mathcal{A}}}
\newcommand {\AM}{{\mathrm{AM}}}
\newcommand {\AMp}{{\AM_{p}^{X}\!(\Ri_\w)}}
\newcommand {\B}{{\mathcal{B}}}
\DeclareMathOperator*{\Be}{Bern}
\newcommand {\Br}{{\dot{B}}}
\newcommand {\Ba}{{\mathfrak{B}}}
\newcommand {\C}{{\mathbb C}}
\newcommand {\ce}{\mathrm{c}}
\newcommand {\Ce}{\mathrm{C}}
\newcommand {\Cc}{\mathrm{C_{c}}}
\newcommand {\Ccinf}{\mathrm{C_{c}^{\infty}}}
\DeclareMathOperator{\cov}{Cov}
\DeclareMathOperator{\DEV}{DEV}
\newcommand {\Di}{{\mathbb D}}
\newcommand {\dom}{\mathrm{dom}}
\newcommand{\dist}{\stackrel{\mathrm{dist}}{=}}
\newcommand {\ud}{\mathrm{d}}
\newcommand {\ue}{\mathrm{e}}
\newcommand {\eps}{\varepsilon}
\newcommand {\veps}{\varepsilon}
\newcommand {\vrho}{{\varrho}}
\newcommand {\E}{{\mathbb{E}}}
\newcommand {\Ec}{{\mathcal{E}}}
\newcommand {\Ell}{L}
\newcommand {\Ellp}{{L_{p}[0,1]}}
\newcommand {\Ellpprime}{{L_{p'}([0,1])}}
\newcommand {\Ellq}{{L_{q}([0,1])}}
\newcommand {\Ellqprime}{{L_{q'}([0,1])}}
\newcommand {\Ellr}{L^{r}}
\newcommand {\Ellone}{{L_{1}([0,1])}}
\newcommand{\Elltwo}{{L_{2}([0,1])}}
\newcommand{\Ellinfty}{L^{\infty}}
\newcommand{\Ellinftyc}{L_{\mathrm{c}}^{\infty}}
\newcommand{\exb}[1]{\exp\left\{#1\right\}}
\DeclareMathOperator*{\Ext}{Ext}
\newcommand{\F}{{\mathcal{F}}}
\newcommand{\Fe}{{\mathbb{F}}}
\newcommand{\G}{{\mathcal{G}}}
\newcommand{\HF}{\mathcal{H}_{\text{FIO}}^{1}(\Rd)}
\newcommand{\Hr}{H}
\newcommand{\HT}{\mathcal{H}}
\newcommand{\ui}{\mathrm{i}}
\newcommand{\I}{{I}}
\newcommand{\J}{{\mathcal{J}}}
\newcommand{\id}{{\mathrm{id}}}
\newcommand{\iid}{\stackrel{\mathclap{\normalfont\mbox{iid}}}{\sim}}
\newcommand{\im}{{\text{im }}}
\newcommand{\ind}{{\perp\!\!\!\perp}}
\DeclareMathOperator*{\Int}{int}
\newcommand{\intx}{{\overline{\int_{X}}}}
\newcommand{\inte}{{\overline{\int_{\E}}}}
\newcommand{\la}{\lambda}
\newcommand{\rb}{\rangle}
\newcommand{\lb}{{\langle}}
\newcommand{\La}{\Lambda}
\newcommand{\calL}{{\mathcal{L}}}
\newcommand{\lp}{{\mathcal{L}}^{p}}
\newcommand{\lpo}{{\overline{\mathcal{L}}^{p}\!}}
\newcommand{\Lpo}{{\overline{\Ell}^{p}\!}}
\newcommand{\M}{{\mathbf{M}}}
\newcommand{\Ma}{{\mathcal{M}}}
\newcommand{\N}{{{\mathbb N}}}
\newcommand{\Na}{{{\mathcal{N}}}}
\newcommand{\norm}[1]{\left\|#1\right\|}
\newcommand{\normm}[1]{{\left\vert\kern-0.25ex\left\vert\kern-0.25ex\left\vert #1 
    \right\vert\kern-0.25ex\right\vert\kern-0.25ex\right\vert}}
\newcommand{\Om}{{{\Omega}}}
\newcommand{\one}{{{\bf 1}}}
\newcommand{\pic}{\text{Pic }}
\newcommand{\ph}{{\varphi}}
\newcommand{\Pa}{{\mathbb{P}}}
\newcommand{\Po}{{\mathcal{P}}}
\newcommand{\Q}{{\mathbb{Q}}}
\newcommand{\R}{{\mathbb R}}
\newcommand{\Rd}{{\mathbb{R}^{d}}}
\DeclareMathOperator{\rej}{reject }
\newcommand{\Rn}{{\mathbb{R}^{n}}}
\newcommand{\cR}{{\mathcal{R}}}
\newcommand{\Rad}{{\mathrm{Rad}}}
\newcommand{\ran}{{\mathrm{ran}}}
\newcommand{\Ri}{{\mathrm{R}}}
\newcommand{\supp}{{\mathrm{supp}}}
\newcommand{\Se}{\mathrm{S}}
\newcommand{\Sp}{S^{*}(\Rn)}
\newcommand{\St}{{\mathrm{St}}}
\newcommand{\Sw}{\mathcal{S}}
\newcommand{\T}{{\mathcal{T}}}
\newcommand{\ta}{{\theta}}
\newcommand{\Ta}{{\Theta}}
\newcommand{\topp}{\stackrel{p}{\to}}
\newcommand{\todd}{\stackrel{d}{\to}}
\newcommand{\toL}[1]{\stackrel{L^{#1}}{\to}} 
\newcommand{\toas}{\stackrel{a.s.}{\to}}
\DeclareMathOperator{\V}{Var}
\newcommand {\w}{{\omega}}
\newcommand {\W}{{\mathrm{W}}}
\newcommand {\Wnp}{\text{$\mathrm{W}$\textsuperscript{$n,\!p$}}}
\newcommand {\Wnpeq}{\text{$\mathrm{W}$\textsuperscript{$n\!,\!p$}}}
\newcommand {\Wonep}{\text{$\mathrm{W}$\textsuperscript{$1,\!p$}}}
\newcommand {\Wonepeq}{\text{$\mathrm{W}$\textsuperscript{$1\!,\!p$}}}
\newcommand {\X}{{\mathcal{X}}}
\newcommand {\Z}{{{\mathbb Z}}}
\newcommand {\Za}{{\mathcal{Z}}}
\newcommand {\Zd}{{\Z[\sqrt{d}]}}
\newcommand {\vanish}[1]{\relax}

\newcommand {\wh}{\widehat}
\newcommand {\wt}{\widetilde}
\newcommand {\red}{\color{red}}

% Distributions
\newcommand{\normal}{\mathsf{N}}
\newcommand{\poi}{\mathsf{Poisson}}
\newcommand{\bern}{\mathsf{Bernoulli}}
\newcommand{\bin}{\mathsf{Binomal}}
\newcommand{\multi}{\mathsf{Multinomial}}
\newcommand{\Exp}{\mathsf{Exp}}



% put your command and environment definitions here




% some theorem environments
% remove "[theorem]" if you do not want them to use the same number sequence


  \newtheorem{thrm}{Theorem}
  \newtheorem{lemma}{Lemma}
  \newtheorem{prop}{Proposition}
  \newtheorem{cor}{Corollary}

  \newtheorem{conj}{Conjecture}
  \renewcommand{\theconj}{\Alph{conj}}  % numbered A, B, C etc

  \theoremstyle{definition}
  \newtheorem{defn}{Definition}
  \newtheorem{ex}{Example}
  \newtheorem{exs}{Examples}
  \newtheorem{question}{Question}
  \newtheorem{remark}{Remark}
  \newtheorem{notn}{Notation}
  \newtheorem{exer}{Exercise}




\title{STATS300B -- Lecture 3}
\author{Julia Palacios\\ Scribed by Michael Howes}
\date{01/11/22}

\pagestyle{fancy}
\fancyhf{}
\rhead{STATS300B -- Lecture 3}
\lhead{01/11/22}
\rfoot{Page \thepage}

\begin{document}
\maketitle
\tableofcontents
\section{Portmanteau theorem}
Last time we stated the Portmanteau theorem.
\begin{theorem}
    Let $X_n$ and $X$ be a random vectors. The following are equivalent.
    \begin{enumerate}
        \item $X_n \todd X$.
        \item $\E[f(X_n)] \to \E[f(X)]$ for all bounded and continuous $f$.
        \item $\E[f(X_n)] \to \E[f(X)]$ for all Lipschitz $f$ with $f(x) \in [0,1]$ for all $x$.
        \item $\lim\inf_n \E[f(X_n)] \ge \E[f(X)]$ for all continuous non-negative $f$.
        \item $\liminf_n \Pa(X_n \in O) \ge \Pa(X\in O)$ for all open sets $O$.
        \item $\limsup_n \Pa(X_n \in C) \le \Pa(X \in C)$ for all closed sets $C$.
        \item $\lim_n \Pa(X_n \in B) = \Pa(X\in B)$ for all measurable sets $B$ such that $\Pa(X \in \delta B) = 0$ where $\delta B$ denotes the boundary of $B$. 
    \end{enumerate}
\end{theorem}
We will not prove the full theorem, but we will prove some parts to give the flavor of the arguments. Today we will prove that if $\E[f(X_n)] \to \E[f(X)]$ for all bounded and continuous $f$, then $\limsup_n \Pa(X_n \in C) \le \Pa(X \in C)$.
\begin{proof}
    Assume $\E[f(X_n)] \to \E[f(X)]$ for all bounded and continuous $f:\R^d\to \R$ and let $C\subseteq \R^d$ be a closed set. Consider the function $h_C: \R^d \to [0,\infty)$ given by 
    \[h_C(x) = \inf\{\norm{x-y}: y \in C\}.\]
    Since $C$ is closed, we have $h_C(x) = 0$ if and only if $x \in C$. The function $h_C$ is continuous. For each $J \in \N$ define $\phi_J : \R \to \R$ by 
    \[\phi_J(t) = \begin{cases}
        1 & \text{if } t \le 0,\\
        1-Jt & \text{if } 0 < t<\frac{1}{J},\\
        0 & \text{if } t \ge \frac{1}{J}.
    \end{cases} \]
    Also define $f_J(x) = \phi_J(h_C(x))$. The functions $f_J$ are continuous and bounded. Furthermore, $f_J(x)\to \one_C(x)$ and $f_J(x) \ge \one_C(x)$ for all $x \in \R^d$. Thus, for all $J$,
    \begin{align*}
        \Pa(X_n \in C) &= \E[\one_C(X_n)]\\
        &\le \E[f_J(X_n)].
    \end{align*}
    By taking $n$ to infinity, we have $\limsup_n \Pa(X_n \in C) \le \E[f_J(X)]$. But $\abs{f_J(x)} \le 1$ and $f_J(X)$ converges to $\one_C(X)$. By the dominated convergence theorem we therefore have 
    \[\lim_{J \to \infty} \E[f_J(X)] = \E[\one_C(X)] = \Pa(X \in C). \]
    Therefore,
    \[\limsup_{n \to \infty} \Pa(X_n \in C) \ge \Pa(X \in C) \qedhere \]
\end{proof}
\begin{definition}
    A collection of functions $\F$ is a \emph{convergence determining class} if for all random vectors $(X_n)_{n \ge 1}$ and $X$,  $\E[f(X_n)] \to \E[f(X)]$ if and only if $X_n \todd X$.
\end{definition}
The Portmanteau Theorem show that all bounded and continuous functions is a convergence determining class. As is the class of Lipschitz functions taking values in $[0,1]$. Another important class of convergence determining functions are the functions 
\[f_t(x) = e^{it\cdot x}, \]
which is a class indexed by $t$. The function 
\[\phi_X(t) = \E[f_t(X)] = \E[e^{it\cdot X}], \]
is called the characteristic function of $X$. Since $\{f_t\}_{t \in \R^d}$ is a convergence determining class, we know that $X_n \todd X$ if and only if $\phi_{X_n}(t) \to \phi_X(t)$ for all $t$.

Note that the assumption that $f$ is bounded is important. A sequence $X_n$ may converge in distribution to $X$, but this does not imply that $\E[f(X_n)] \to \E[f(X)]$ for continuous unbounded $f$. Indeed, we may not have $\E[X_n] \to \E[X]$. 
\section{Tightness}
\begin{definition}
    A collection of random vectors $\{X_a\}_{a \in \A}$ is \emph{uniformly tight} if for all $\eps > 0$, there exists $M < \infty$ such that 
    \[\sup_{a \in \A} \Pa(\norm{X_a}> M) \le \eps.\]
\end{definition}
A uniformly tight collection of random vectors is sometimes said to be \emph{bounded in probability}. This is because if $\{X_a\}_{a \in \A}$ is uniformly tight, then with probability at least $1-\eps$, $\norm{X_a} \le M$ for every $a$.

We can also define uniform tightness for probability measures instead of random variables.
\begin{definition}
    A collection of probability measures $\{\Pa_a\}_{a \in \A}$ on $\R^d$ is \emph{uniformly tight} if for all $\eps > 0$, there exists a compact set $C$ such that 
    \[\sup_{a \in \A} \Pa_a(C) \ge 1-\eps.\]
\end{definition}
\begin{remark} 
    The following are examples of uniformly tight collections.
    \begin{enumerate}
        \item A single random vector $X$ is uniformly tight since \[\lim_{n \to \infty} \Pa(\norm{X} \le n) = \Pa(\norm{X} < \infty) =1.\]
        \item If $\{X_a\}_{a \in \A_1},\{X_a\}_{a \in \A_2},\ldots \{X_a\}_{a \in \A_m}$ are all uniformly tight, then $\{X_a\}_{a \in \bigcup_{i=1}^m \A_i}$ is also uniformly tight.
        \item If $X_n \todd X$, then $\{X_n\}_{n \ge 1}$ is uniformly tight.
    \end{enumerate}
    The converse of the last remark is almost true. A uniformly tight collection of random vectors need not converge in distribution, but there must be a subsequence which does.
\end{remark}
\begin{theorem}
    If $\{X_n\}_{n \ge 1}$ is uniformly tight, then there exists a random vector $X$ and a subsequence $n_j$ such that $X_{n_j} \todd X$.
\end{theorem}
Note that the original sequence $\{X_n\}$ need not converge to anything as the following example shows.
\[X_n = \begin{cases}
    1+\frac{1}{n} & \text{if $n$ is odd},\\
    2+\frac{1}{n} & \text{if $n$ is even}.
\end{cases} \]
\section{Convergence in $L^p$}
Recall that $X_n \toL{p} X$ if $\E[\norm{X_n-X}_p^p] \to 0$. The following links convergence in $L^p$ to convergence in distribution and convergence in probability.
\begin{definition}
    A sequence $\{X_n\}_{n \ge 1}$ is \emph{uniformly integrable} if
    \[\lim_{\la \to \infty} \limsup_{n \to \infty} \E[\abs{X_n}\one_{\abs{X_n}\ge \la}] = 0. \]
\end{definition}
If $X_n \todd X$, then for every $r > 0$, $\E[\norm{X_n}_r^r] \to \E[\norm{X}_r^r]$ if and only if $\{\norm{X_n}_r^r\}_{n \ge 1}$ is uniformly integrable.
\begin{theorem}[Vitalli]
    Suppose $X_n \in L^r$ for some $r \in(0,\infty)$ and that $X_n \topp X$. Then the following are equivalent,
    \begin{enumerate}
        \item $\{\norm{X_n}_r^r\}$ are uniformly integrable.
        \item $X_n \toL{r} X$
        \item $\limsup_n \E\norm{X_n}_r^r \le \E \norm{X}_r^r$.
    \end{enumerate}
\end{theorem}
\section{Almost sure convergence}
\begin{definition}
    A sequence of random variables $\{X_n\}$ \emph{converge almost surely} to $X$ if 
    \[\Pa\left(\lim_{n \to \infty}X_n \neq X\right) =0. \]
    We denote almost sure convergence by $X_n \toas X$.
\end{definition}
The following are equivalent
\begin{enumerate}
    \item $X_n \toas X$.
    \item For all $\eps > 0$, \[\Pa(\norm{X_n-X}>\eps, \text{ infinitely often})=\Pa\left(\bigcap_{n=1}^\infty \bigcup_{m=n}^\infty \norm{X_m-X} > \eps \right)=0.\]
    \item For all $\eps > 0$, \[\lim\limits_{n \to \infty} \Pa\left(\bigcup_{m=n}^\infty \norm{X_m -X} > \eps\right) = 0.\]
    \item For all $\eps > 0$,
    \[\lim_{n \to \infty}\Pa\left(
    \sup_{m \ge n} \norm{X_m-X} \ge \eps\right) = 0.\]
\end{enumerate}
The following theorem is called the strong law of large numbers.
\begin{theorem}
    Suppose $X_1,\ldots$ are i.i.d. with $\E\abs{X_i} < \infty $ and $\E[X_i] = \mu$. Then $\bar{X}_n \toas \mu$.
\end{theorem}
The Borel--Cantelli lemmas are the main tools for proving almost sure convergence.
\begin{proposition}
    \begin{enumerate}
        \item Let $\{A_n\}_{n \ge 1}$ be any sequence of events. If $\sum_{n=1}^\infty \Pa(A_n) < \infty$, then 
        \[\Pa(A_n \text{ infinitely often}) = \Pa\left(\bigcap_{n=1}^\infty \bigcup_{m=1}^\infty A_m\right)=0.\]
        \item If $\{A_n\}_{n \ge 1}$ is a sequence of independent events and $\sum_{n=1}^\infty \Pa(A_n) = \infty$, then
        \[\Pa(A_n \text{ infinitely often}) = 1.\]
    \end{enumerate}
\end{proposition}
\begin{proof}
    We will only prove 1. Note that, for all $n \in \N$,
    \[\Pa(A_n \text{ infinitely often}) \le \Pa\left(\bigcup_{m=n}^\infty A_m\right)\le \sum_{m=n}^\infty \Pa(A_m).\]
    Since $\sum_{n=1}^\infty \Pa(A_n) < \infty$, the sum $\sum_{m=n}^\infty \Pa(A_m)$ goes to zero as $n$ goes to infinity. Thus, \[\Pa(A_n \text{ infinitely often}) = 0. \qedhere\]
\end{proof}
\section{Standard implications}
We have the following implications
\[X_n \toas X \Longrightarrow X_n \topp X \Longrightarrow X_n \todd X, \]
and for any $p > 0$, 
\[X_n \toL{p} X \Longrightarrow X_n \topp X \Longrightarrow X_n \todd X.\]
The converse implications do not hold in general, but partial converses do hold. For instance if $b$ is a constant, then
\[ X_n \todd b \Longrightarrow X_n \topp b.\]
Also, if $X_n \topp X$, then there exists a subsequence $X_{n_j}$ such that $X_{n_j} \toas X$. Likewise, if $X_n \toL{p} X$, then there exists a subsequence $X_{n_j}$ such that $X_{n_j} \toas X$. Also, if $X_n \toas X$ and $\{\norm{X_n}_p^p\}$ are uniformly integrable, then $X_n \toL{p} X$. But in general, almost sure convergence does not imply convergence in $L^p$ and convergence in $L^p$ does not imply convergence almost surely.

We have already proven some of these implications are others are given as homework. We will prove a few more now. Firstly we will show 
\[ X_n \toas X \Longrightarrow X_n \topp X.\]
\begin{proof}
    Suppose $X_n \toas X$ and let $\eps > 0$. Then 
    \[\Pa(\norm{X_n-X} >\eps) \le \Pa\left(\bigcup_{m=n}^\infty \norm{X_m-X} > \eps\right).\]
    By almost sure convergence, the term on the right goes to 0. Thus, $\Pa(\norm{X_n-X} > \eps) \to 0$ and so $X_n \topp X$.
\end{proof}
We will also prove
\[X_n \topp X \Longrightarrow X_n \todd X. \]
\begin{proof}
    Suppose $X_n \topp X$ and let $t$ be a continuity point of $F$, the cumulative distribution function of $X$. Let $F_n$ be the cumulative distribution function of $X_n$. Fixing $\eps >0$, we have
    \begin{align*}
        F_n(t)&=\Pa(X_n \le t)\\
         &= \Pa(X_n \le t, X \le t+\eps) + \Pa(X_n \le t, X > t+\eps)\\
        &\le \Pa(X \le t+\eps) + \Pa(\abs{X_n -X}>\eps)\\
        &=F(t+\eps) +\Pa(\abs{X_n -X}>\eps)
    \end{align*}
    Since $X_n \topp X$, $\Pa(\abs{X_n-X}>\eps) \to 0$. Thus,
    \[ \limsup_{n \to \infty} F_n(t) \le F(t+\eps).\]
    Similarly,
    \begin{align*}
        F(t-\eps) &= \Pa(X \le t-\eps)\\
        &=\Pa(X \le t-\eps, X_n \le t)+\Pa(X \le t-\eps, X_n \ge t)\\
        &\le \Pa(X_n \le t) + \Pa(\abs{X_n-X}\ge \eps)\\
        &=F_n(t)+\Pa(\abs{X_n-X}\ge \eps).
    \end{align*}
    Thus,
    \[F_n(t) \ge F(t-\eps) - \Pa(\abs{X_n-X} \ge \eps). \]
    Which implies,
    \[\liminf_{n \to \infty} F_n(t) \ge F(t-\eps). \]
    Since $F$ is continuous at $t$, both $F(t-\eps)$ and $F(t+\eps)$ can be made arbitrarily close to $F(t)$ and hence 
    \[\limsup_{n \to \infty} F_n(t) \le F(t) \le \liminf_{n \to \infty}F_n(t). \]
    Thus $\lim\limits_{n \to \infty} F_n(t) = F(t)$. 
\end{proof}
\end{document}
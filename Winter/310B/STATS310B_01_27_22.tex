\documentclass{article}
\usepackage{ae,aecompl}
\usepackage{todonotes}
\usepackage{chngcntr}
\usepackage{tikz-cd}
\usepackage{graphicx}
\graphicspath{ {./images/}}
\usepackage[all,cmtip]{xy}
\usepackage{amsmath, amscd}
\usepackage{amsthm}
\usepackage{amssymb}
\usepackage{amsfonts}
\usepackage{bm}
\usepackage{qsymbols}
\usepackage{latexsym}
\usepackage{mathrsfs}
\usepackage{mathtools}
\usepackage{cite}
\usepackage{color}
\usepackage{url}
\usepackage{enumerate}
\usepackage{verbatim}
\usepackage[draft=false, colorlinks=true]{hyperref}
\usepackage{pdfpages}
\usepackage[margin=1.2in]{geometry}
\usepackage{IEEEtrantools}

\usepackage{fancyhdr}


\usepackage[nameinlink]{cleveref}


\DeclareMathOperator*{\ac}{accept}
\DeclareMathOperator*{\amax}{argmax}
\DeclareMathOperator*{\amin}{argmin}
\DeclareMathOperator*{\Aut}{Aut}
\newcommand {\al}{{\alpha}}
\newcommand {\abs}[1]{{\left\lvert#1\right\rvert}}
\newcommand {\A}{{\mathcal{A}}}
\newcommand {\AM}{{\mathrm{AM}}}
\newcommand {\AMp}{{\AM_{p}^{X}\!(\Ri_\w)}}
\newcommand {\B}{{\mathcal{B}}}
\DeclareMathOperator*{\Be}{Bern}
\newcommand {\Br}{{\dot{B}}}
\newcommand {\Ba}{{\mathfrak{B}}}
\newcommand {\C}{{\mathbb C}}
\newcommand {\ce}{\mathrm{c}}
\newcommand {\Ce}{\mathrm{C}}
\newcommand {\Cc}{\mathrm{C_{c}}}
\newcommand {\Ccinf}{\mathrm{C_{c}^{\infty}}}
\DeclareMathOperator{\cov}{Cov}
\DeclareMathOperator{\DEV}{DEV}
\newcommand {\Di}{{\mathbb D}}
\newcommand {\dom}{\mathrm{dom}}
\newcommand{\dist}{\stackrel{\mathrm{dist}}{=}}
\newcommand {\ud}{\mathrm{d}}
\newcommand {\ue}{\mathrm{e}}
\newcommand {\eps}{\varepsilon}
\newcommand {\veps}{\varepsilon}
\newcommand {\vrho}{{\varrho}}
\newcommand {\E}{{\mathbb{E}}}
\newcommand {\Ec}{{\mathcal{E}}}
\newcommand {\Ell}{L}
\newcommand {\Ellp}{{L_{p}[0,1]}}
\newcommand {\Ellpprime}{{L_{p'}([0,1])}}
\newcommand {\Ellq}{{L_{q}([0,1])}}
\newcommand {\Ellqprime}{{L_{q'}([0,1])}}
\newcommand {\Ellr}{L^{r}}
\newcommand {\Ellone}{{L_{1}([0,1])}}
\newcommand{\Elltwo}{{L_{2}([0,1])}}
\newcommand{\Ellinfty}{L^{\infty}}
\newcommand{\Ellinftyc}{L_{\mathrm{c}}^{\infty}}
\newcommand{\exb}[1]{\exp\left\{#1\right\}}
\DeclareMathOperator*{\Ext}{Ext}
\newcommand{\F}{{\mathcal{F}}}
\newcommand{\Fe}{{\mathbb{F}}}
\newcommand{\G}{{\mathcal{G}}}
\newcommand{\HF}{\mathcal{H}_{\text{FIO}}^{1}(\Rd)}
\newcommand{\Hr}{H}
\newcommand{\HT}{\mathcal{H}}
\newcommand{\ui}{\mathrm{i}}
\newcommand{\I}{{I}}
\newcommand{\J}{{\mathcal{J}}}
\newcommand{\id}{{\mathrm{id}}}
\newcommand{\iid}{\stackrel{\mathclap{\normalfont\mbox{iid}}}{\sim}}
\newcommand{\im}{{\text{im }}}
\newcommand{\ind}{{\perp\!\!\!\perp}}
\DeclareMathOperator*{\Int}{int}
\newcommand{\intx}{{\overline{\int_{X}}}}
\newcommand{\inte}{{\overline{\int_{\E}}}}
\newcommand{\la}{\lambda}
\newcommand{\rb}{\rangle}
\newcommand{\lb}{{\langle}}
\newcommand{\La}{\Lambda}
\newcommand{\calL}{{\mathcal{L}}}
\newcommand{\lp}{{\mathcal{L}}^{p}}
\newcommand{\lpo}{{\overline{\mathcal{L}}^{p}\!}}
\newcommand{\Lpo}{{\overline{\Ell}^{p}\!}}
\newcommand{\M}{{\mathbf{M}}}
\newcommand{\Ma}{{\mathcal{M}}}
\newcommand{\N}{{{\mathbb N}}}
\newcommand{\Na}{{{\mathcal{N}}}}
\newcommand{\norm}[1]{\left\|#1\right\|}
\newcommand{\normm}[1]{{\left\vert\kern-0.25ex\left\vert\kern-0.25ex\left\vert #1 
    \right\vert\kern-0.25ex\right\vert\kern-0.25ex\right\vert}}
\newcommand{\Om}{{{\Omega}}}
\newcommand{\one}{{{\bf 1}}}
\newcommand{\pic}{\text{Pic }}
\newcommand{\ph}{{\varphi}}
\newcommand{\Pa}{{\mathbb{P}}}
\newcommand{\Po}{{\mathcal{P}}}
\newcommand{\Q}{{\mathbb{Q}}}
\newcommand{\R}{{\mathbb R}}
\newcommand{\Rd}{{\mathbb{R}^{d}}}
\DeclareMathOperator{\rej}{reject }
\newcommand{\Rn}{{\mathbb{R}^{n}}}
\newcommand{\cR}{{\mathcal{R}}}
\newcommand{\Rad}{{\mathrm{Rad}}}
\newcommand{\ran}{{\mathrm{ran}}}
\newcommand{\Ri}{{\mathrm{R}}}
\newcommand{\supp}{{\mathrm{supp}}}
\newcommand{\Se}{\mathrm{S}}
\newcommand{\Sp}{S^{*}(\Rn)}
\newcommand{\St}{{\mathrm{St}}}
\newcommand{\Sw}{\mathcal{S}}
\newcommand{\T}{{\mathcal{T}}}
\newcommand{\ta}{{\theta}}
\newcommand{\Ta}{{\Theta}}
\newcommand{\topp}{\stackrel{p}{\to}}
\newcommand{\todd}{\stackrel{d}{\to}}
\newcommand{\toL}[1]{\stackrel{L^{#1}}{\to}} 
\newcommand{\toas}{\stackrel{a.s.}{\to}}
\DeclareMathOperator{\V}{Var}
\newcommand {\w}{{\omega}}
\newcommand {\W}{{\mathrm{W}}}
\newcommand {\Wnp}{\text{$\mathrm{W}$\textsuperscript{$n,\!p$}}}
\newcommand {\Wnpeq}{\text{$\mathrm{W}$\textsuperscript{$n\!,\!p$}}}
\newcommand {\Wonep}{\text{$\mathrm{W}$\textsuperscript{$1,\!p$}}}
\newcommand {\Wonepeq}{\text{$\mathrm{W}$\textsuperscript{$1\!,\!p$}}}
\newcommand {\X}{{\mathcal{X}}}
\newcommand {\Z}{{{\mathbb Z}}}
\newcommand {\Za}{{\mathcal{Z}}}
\newcommand {\Zd}{{\Z[\sqrt{d}]}}
\newcommand {\vanish}[1]{\relax}

\newcommand {\wh}{\widehat}
\newcommand {\wt}{\widetilde}
\newcommand {\red}{\color{red}}

% Distributions
\newcommand{\normal}{\mathsf{N}}
\newcommand{\poi}{\mathsf{Poisson}}
\newcommand{\bern}{\mathsf{Bernoulli}}
\newcommand{\bin}{\mathsf{Binomal}}
\newcommand{\multi}{\mathsf{Multinomial}}
\newcommand{\Exp}{\mathsf{Exp}}



% put your command and environment definitions here




% some theorem environments
% remove "[theorem]" if you do not want them to use the same number sequence


  \newtheorem{thrm}{Theorem}
  \newtheorem{lemma}{Lemma}
  \newtheorem{prop}{Proposition}
  \newtheorem{cor}{Corollary}

  \newtheorem{conj}{Conjecture}
  \renewcommand{\theconj}{\Alph{conj}}  % numbered A, B, C etc

  \theoremstyle{definition}
  \newtheorem{defn}{Definition}
  \newtheorem{ex}{Example}
  \newtheorem{exs}{Examples}
  \newtheorem{question}{Question}
  \newtheorem{remark}{Remark}
  \newtheorem{notn}{Notation}
  \newtheorem{exer}{Exercise}




\title{STATS310B -- Lecture 8}
\author{Sourav Chatterjee\\ Scribed by Michael Howes}
\date{01/27/22}

\pagestyle{fancy}
\fancyhf{}
\rhead{STATS310B -- Lecture 8}
\lhead{01/27/22}
\rfoot{Page \thepage}

\begin{document}
\maketitle
\tableofcontents
\section{Polya's Urn}
Consider an urn of infinite capacity. The urn initially has one white ball and one black ball inside it. At each time step, a ball is picked uniformly at random from the urn and replaced back into the urn with another ball of the same color. Equivalently we choose a color with probability proportional to the number of balls of the same color and then but in an additional ball of the chosen color. 

Let $W_n$ be the proportion of white ball at time $n$ with $W_0=\frac{1}{2}$. We would like to understand the limiting behavior of $W_n$ as $n \to \infty$.
\begin{proposition}
    Let $\F_n = \sigma(W_1,\ldots, W_n)$. Then the sequence $\{W_n\}_{n \ge 0}$ is a martingale with respect to $\{\F_n\}_{n\ge 0}$.
\end{proposition}
\begin{proof}
    Note that the total number of balls at time $n$ is $n+2$. Let $N_n$ be the number of white ball in the urn at time $n$. Thus, $W_n = \frac{1}{n+2}N_n$. It follows that,
    \begin{align*}
        \E(W_{n+1}|\F_n) &= \frac{1}{n+3}\E(N_{n+1}|\F_n)\\
        &=\frac{1}{n+3}\left((N_{n}+1)\frac{N_n}{n+2}+N_n\frac{n+2-N_n}{n+2}\right)\\
        &=\frac{1}{n+3}\left(\frac{1}{n+2}N_n^2 + \frac{N_n}{n+2}+N_n-\frac{1}{n+2}N_n^2\right)\\
        &=\frac{1}{n+3}\left(\frac{n+3}{n+2}N_n\right)\\
        &=\frac{1}{n+2}N_n\\
        &=W_n.\qedhere 
    \end{align*}
\end{proof}
Note that $W_n \in [0,1]$ for every $n$ and thus $\sup_n \E[W_n^+]\le 1< \infty$. It follows that there exists an integrable random variable $W$ such that $W_n \to W$ almost surely. We will in fact prove that $W$ is uniformly distributed on $[0,1]$.
\begin{proof}
    We will show by induction that for all $n$, $N_n$ is uniformly distributed on $\{1,2,\ldots,n+1\}$, where $N_n$ is the number of white balls at time $n$. This is true when $n=0$ since $N_0 = 1$. Now suppose that the result is true for some $n$. Then, for $k=1,\ldots,n+2$,
    \begin{align*}
        \Pa(M_{n+1} = k) &=\sum_{j=1}^{n+1} \Pa(M_{n+1}=k|N_n = j)\Pa(N_n=j)\\
        &=\frac{1}{n+1}\sum_{j=1}^{n+1} \Pa(M_{n+1}=k|N_n = j).
    \end{align*}
    Note that $\Pa(M_{n+1}=k|N_n=j) \neq 0$  only if $j = k-1$ or $j=k$. This is even true if $k=1$ or $k=n+2$ although in these cases one  $\Pa(M_{n+1}=k|N_n=k-1) =0$ or $\Pa(M_{n+1}=k|N_n=k)=0$ respectively which agrees with the calculations below. Thus,
    \begin{align*}
        \Pa(M_{n+1} = k) &=\frac{1}{n+1}\left(\Pa(M_{n+1}=k|N_n=k-1)+\Pa(M_{n+1}=k|N_n=k)\right)\\
        &=\frac{1}{n+1}\left(\frac{k-1}{n+2}+\frac{n+2-k}{n+2}\right)\\
        &=\frac{1}{n+1}\left(\frac{n+1}{n+2}\right)\\
        &=\frac{1}{n+2}.
    \end{align*}
    Thus, $N_{n+1}$ is uniformly distributed on $\{1,\ldots,n+2\}$. Hence, $W_n$ is uniformly distributed on $\left\{\frac{1}{n+2},\ldots,\frac{n+1}{n+2}\right\}$ which implies $W_n$ converges in distribution to $U[0,1]$ but $W_n$ also converges almost surely (and this $W_n \to W$ in distribution). Thus, $W$ must be uniformly distributed on $[0,1]$.
\end{proof}
\section{L\'evy's downwards convergence theorem}
Our next convergence theorem is L\'evy's downwards convergence theorem which is also called the backwards martingale theorem. 
\begin{theorem}[L\'evy's downards convergence theorem]
    Let $X$ be an integrable random variable defined on $(\Om,\F,\Pa)$. Let $\F_0 \supseteq \F_1 \supseteq \F_2 \supseteq \ldots$ be a decreasing sequence of sub-$\sigma$-algebras of $\F$. Define $\F^* = \bigcap_{n=0}^\infty \F_n$, then 
    \[\E(X|\F_n) \to \E(X|\F^*), \]
    almost surely and in $L^1$.
\end{theorem}
\begin{proof}
    Let $X_n = \E(X|\F_n)$. We will first show that $\{X_n\}_{n \ge 0}$ has an almost sure limit $X^*$. We will then prove that $X_n$ converges to $X^*$ in $L^1$ and then finally we will show that $X^*=\E(X|\F^*)$. Fix $n \in \N$ and consider the time reversed finite sequence,
    \[X_{n}, X_{n-1},X_{n-2},\ldots,X_0. \]
    The above sequence is a martingale with respect to $\F_n,\F_{n-1},\ldots,\F_0$. This is because $\F_{k} \subseteq \F_{k-1}$ and thus
    \[\E(X_{k-1}|\F_{k})  = \E(\E(X|\F_{k-1})|\F_k)=\E(X|\F_k). \]
    Fix an interval $[a,b]$ and let $U_n$ be the number of complete upcrossings of $[a,b]$ by $X_{n},X_{n-1},\ldots,X_0$. By the upcrossing lemma,
    \begin{align*}
        \E[U_n] &\le \frac{\E[(X_0-a)^+]-\E[(X_n-a)^+]}{b-a}\\
        &\le \frac{\E[(X_0-a)^+]}{b-a}\\
        &= \frac{\E[(\E(X|\F_0)-a)^+]}{b-a}\\
        &\le \frac{\E[\E((X-a)^+|\F_0)]}{b-a}\\
        &\le \frac{\E[(X-a)^+]}{b-a}.
    \end{align*}
    Note that $U_n \le U_{n+1}$ and thus $U_n \nearrow U$ for some random variable $U$. By the monotone convergence theorem, $\E[U] \le \frac{\E[(X-a)^+]}{b-a}<\infty$. As with Doob's sub-martingale convergence theorem, this implies that $X^* = \lim_{n \to \infty} X_n$ exists almost surely. We will now show that $X^*$ is integrable. Note that
    \begin{align*}
        \E[\abs{X^*}] &= \E\left[\lim_{n \to \infty}\abs{X_n}\right]\\
        &\le \liminf_{n \to \infty}\E[\abs{X_n}]\\
        &=\liminf_{n\to \infty} \E[\abs{\E(X|\F_n)}]\\
        &\le \liminf_{n \to \infty} \E[\E(\abs{X}|\F_n)]\\
        &=\liminf_{n \to \infty} \E[\abs{X}]\\
        &=\E[\abs{X}]<\infty.
    \end{align*} 
    To show that $X^* =\E(X|\F^*)$ and that $X_n \to X^*$ in $L^1$ we need to first review the concept of \emph{uniform integrability}.
\end{proof}

\section{Uniform integrability}
\begin{definition}
    A sequence of random variables $\{X_n\}_{n \ge 1}$ is \emph{uniformly integrable} if for $\epsilon > 0$, there exists $K > 0$ such that,      
    \[\sup_n \E[\abs{X_n}\one_{\{\abs{X_n}>k\}}] < \eps. \]
\end{definition}
Uniform integrability allows one to calculate the expectation of a limit.
\begin{lemma}
    Suppose that $\{X_n\}_{n \ge 0}$ is a uniformly integrable sequence and $X_n \to X$ almost surely. Then $X$ is integrable and $X_n \to X$ in $L^1$.
\end{lemma}
\begin{proof}
    Take any $\eps > 0$ and take $k$ such that $\E[\abs{X_n}\one_{\{\abs{X_n}>k\}}]<\eps$. Then, 
    \[\E[\abs{X_n}] \le \E[\abs{X_n}\one_{\{X_n >k\}}] +\E[\abs{X_n}\one_{\{X_n\le k\}}]\le \eps+k. \]
    Thus, $\E[\abs{X}] \le \liminf_n \E[\abs{X_n}] \le \eps+k<\infty$. So $X$ is integrable. Note that for all $L > 0$, $\abs{X}\one_{\{\abs{X} > L\}} \le \abs{X}$ and, almost surely 
    \[ \lim_{L \to \infty} \abs{X}\one_{\{\abs{X}> L\}} = 0.\]
    Thus, by the dominated convergence theorem,
    \[\lim_{L \to \infty} \E[\abs{X}\one_{\{\abs{X}>L\}}] = 0. \]
    This shows that given $\eps > 0$, we can choose $k>0$ so that $\E[\abs{X}\one_{\{\abs{X}>k\}}]<\eps$ and
    \[\sup_n \E[\abs{X_n}\one_{\{\abs{X_n}>k\}}] <\eps.\] 
    Let $\eps > 0$ be arbitrary and fix such a corresponding $k>0$. Let $\phi : \R \to \R$ be given by
    \[\phi(x) = \begin{cases}
        -k & \text{if } x \le -k,\\
        x & \text{if } x \in (-k,k),\\
        k & \text{if } x \ge k.
    \end{cases} \]  
    The function $\phi$ is bounded and continuous and $\abs{\phi(x)-x}\le \abs{x}\one_{\{\abs{x}>k\}}$ for all $x \in \R$. Thus,
    \begin{align*}
        \E[\abs{X_n-X}]&\le \E[\abs{X_n-\phi(X_n)}] + \E[\abs{\phi(X_n)-\phi(X)}] +\E[\abs{\phi(X)-X}]\\
        &\le \E[\abs{X_n}\one_{\{\abs{X_n}> k\}}] + \E[\abs{\phi(X_n)-\phi(X)}]+\E[\abs{X}\one_{\{\abs{X} > k\}}]\\
        &< 2\eps + \E[\abs{\phi(X_n)-\phi(X)}].
    \end{align*}
    The random variable $\abs{\phi(X_n)-\phi(X)}$ is bounded above by $2k$ and goes to 0 almost surely. Thus, by the dominated convergence theorem,
    \begin{align*}
        \limsup_n \E[\abs{X_n-X}] &\le 2\eps + \limsup_n \E[\abs{\phi(X_n)-\phi(X)}] = 2\eps.
    \end{align*}
    Thus, $X_n \to X$ in $L^1$.
\end{proof}
We will next state a characterization of uniform integrability that we will need in proving L\'evy's downwards convergence theorem.
\begin{proposition}
    Let $\{X_n\}_{n \ge 1}$ be a sequence of random variables. The sequence $\{X_n\}_{n \ge 0}$ is uniformly integrable if and only if the following both hold,
    \begin{enumerate}
        \item $\sup_n \E[\abs{X_n}] < \infty$.
        \item For all $\eps >0$ there exists $\delta > 0$ such that for all $A$ and $n$, if $\Pa(A) <\delta$, then $\E[\abs{X_n}\one_A] < \eps$.
    \end{enumerate}
\end{proposition}
We will prove this proposition in the next lecture.
\end{document}
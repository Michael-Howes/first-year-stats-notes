\documentclass{article}
\usepackage{ae,aecompl}
\usepackage{todonotes}
\usepackage{chngcntr}
\usepackage{tikz-cd}
\usepackage{graphicx}
\graphicspath{ {./images/}}
\usepackage[all,cmtip]{xy}
\usepackage{amsmath, amscd}
\usepackage{amsthm}
\usepackage{amssymb}
\usepackage{amsfonts}
\usepackage{bm}
\usepackage{qsymbols}
\usepackage{latexsym}
\usepackage{mathrsfs}
\usepackage{mathtools}
\usepackage{cite}
\usepackage{color}
\usepackage{url}
\usepackage{enumerate}
\usepackage{verbatim}
\usepackage[draft=false, colorlinks=true]{hyperref}
\usepackage{pdfpages}
\usepackage[margin=1.2in]{geometry}
\usepackage{IEEEtrantools}

\usepackage{fancyhdr}


\usepackage[nameinlink]{cleveref}


\DeclareMathOperator*{\ac}{accept}
\DeclareMathOperator*{\amax}{argmax}
\DeclareMathOperator*{\amin}{argmin}
\DeclareMathOperator*{\Aut}{Aut}
\newcommand {\al}{{\alpha}}
\newcommand {\abs}[1]{{\left\lvert#1\right\rvert}}
\newcommand {\A}{{\mathcal{A}}}
\newcommand {\AM}{{\mathrm{AM}}}
\newcommand {\AMp}{{\AM_{p}^{X}\!(\Ri_\w)}}
\newcommand {\B}{{\mathcal{B}}}
\DeclareMathOperator*{\Be}{Bern}
\newcommand {\Br}{{\dot{B}}}
\newcommand {\Ba}{{\mathfrak{B}}}
\newcommand {\C}{{\mathbb C}}
\newcommand {\ce}{\mathrm{c}}
\newcommand {\Ce}{\mathrm{C}}
\newcommand {\Cc}{\mathrm{C_{c}}}
\newcommand {\Ccinf}{\mathrm{C_{c}^{\infty}}}
\DeclareMathOperator{\cov}{Cov}
\DeclareMathOperator{\DEV}{DEV}
\newcommand {\Di}{{\mathbb D}}
\newcommand {\dom}{\mathrm{dom}}
\newcommand{\dist}{\stackrel{\mathrm{dist}}{=}}
\newcommand {\ud}{\mathrm{d}}
\newcommand {\ue}{\mathrm{e}}
\newcommand {\eps}{\varepsilon}
\newcommand {\veps}{\varepsilon}
\newcommand {\vrho}{{\varrho}}
\newcommand {\E}{{\mathbb{E}}}
\newcommand {\Ec}{{\mathcal{E}}}
\newcommand {\Ell}{L}
\newcommand {\Ellp}{{L_{p}[0,1]}}
\newcommand {\Ellpprime}{{L_{p'}([0,1])}}
\newcommand {\Ellq}{{L_{q}([0,1])}}
\newcommand {\Ellqprime}{{L_{q'}([0,1])}}
\newcommand {\Ellr}{L^{r}}
\newcommand {\Ellone}{{L_{1}([0,1])}}
\newcommand{\Elltwo}{{L_{2}([0,1])}}
\newcommand{\Ellinfty}{L^{\infty}}
\newcommand{\Ellinftyc}{L_{\mathrm{c}}^{\infty}}
\newcommand{\exb}[1]{\exp\left\{#1\right\}}
\DeclareMathOperator*{\Ext}{Ext}
\newcommand{\F}{{\mathcal{F}}}
\newcommand{\Fe}{{\mathbb{F}}}
\newcommand{\G}{{\mathcal{G}}}
\newcommand{\HF}{\mathcal{H}_{\text{FIO}}^{1}(\Rd)}
\newcommand{\Hr}{H}
\newcommand{\HT}{\mathcal{H}}
\newcommand{\ui}{\mathrm{i}}
\newcommand{\I}{{I}}
\newcommand{\J}{{\mathcal{J}}}
\newcommand{\id}{{\mathrm{id}}}
\newcommand{\iid}{\stackrel{\mathclap{\normalfont\mbox{iid}}}{\sim}}
\newcommand{\im}{{\text{im }}}
\newcommand{\ind}{{\perp\!\!\!\perp}}
\DeclareMathOperator*{\Int}{int}
\newcommand{\intx}{{\overline{\int_{X}}}}
\newcommand{\inte}{{\overline{\int_{\E}}}}
\newcommand{\la}{\lambda}
\newcommand{\rb}{\rangle}
\newcommand{\lb}{{\langle}}
\newcommand{\La}{\Lambda}
\newcommand{\calL}{{\mathcal{L}}}
\newcommand{\lp}{{\mathcal{L}}^{p}}
\newcommand{\lpo}{{\overline{\mathcal{L}}^{p}\!}}
\newcommand{\Lpo}{{\overline{\Ell}^{p}\!}}
\newcommand{\M}{{\mathbf{M}}}
\newcommand{\Ma}{{\mathcal{M}}}
\newcommand{\N}{{{\mathbb N}}}
\newcommand{\Na}{{{\mathcal{N}}}}
\newcommand{\norm}[1]{\left\|#1\right\|}
\newcommand{\normm}[1]{{\left\vert\kern-0.25ex\left\vert\kern-0.25ex\left\vert #1 
    \right\vert\kern-0.25ex\right\vert\kern-0.25ex\right\vert}}
\newcommand{\Om}{{{\Omega}}}
\newcommand{\one}{{{\bf 1}}}
\newcommand{\pic}{\text{Pic }}
\newcommand{\ph}{{\varphi}}
\newcommand{\Pa}{{\mathbb{P}}}
\newcommand{\Po}{{\mathcal{P}}}
\newcommand{\Q}{{\mathbb{Q}}}
\newcommand{\R}{{\mathbb R}}
\newcommand{\Rd}{{\mathbb{R}^{d}}}
\DeclareMathOperator{\rej}{reject }
\newcommand{\Rn}{{\mathbb{R}^{n}}}
\newcommand{\cR}{{\mathcal{R}}}
\newcommand{\Rad}{{\mathrm{Rad}}}
\newcommand{\ran}{{\mathrm{ran}}}
\newcommand{\Ri}{{\mathrm{R}}}
\newcommand{\supp}{{\mathrm{supp}}}
\newcommand{\Se}{\mathrm{S}}
\newcommand{\Sp}{S^{*}(\Rn)}
\newcommand{\St}{{\mathrm{St}}}
\newcommand{\Sw}{\mathcal{S}}
\newcommand{\T}{{\mathcal{T}}}
\newcommand{\ta}{{\theta}}
\newcommand{\Ta}{{\Theta}}
\newcommand{\topp}{\stackrel{p}{\to}}
\newcommand{\todd}{\stackrel{d}{\to}}
\newcommand{\toL}[1]{\stackrel{L^{#1}}{\to}} 
\newcommand{\toas}{\stackrel{a.s.}{\to}}
\DeclareMathOperator{\V}{Var}
\newcommand {\w}{{\omega}}
\newcommand {\W}{{\mathrm{W}}}
\newcommand {\Wnp}{\text{$\mathrm{W}$\textsuperscript{$n,\!p$}}}
\newcommand {\Wnpeq}{\text{$\mathrm{W}$\textsuperscript{$n\!,\!p$}}}
\newcommand {\Wonep}{\text{$\mathrm{W}$\textsuperscript{$1,\!p$}}}
\newcommand {\Wonepeq}{\text{$\mathrm{W}$\textsuperscript{$1\!,\!p$}}}
\newcommand {\X}{{\mathcal{X}}}
\newcommand {\Z}{{{\mathbb Z}}}
\newcommand {\Za}{{\mathcal{Z}}}
\newcommand {\Zd}{{\Z[\sqrt{d}]}}
\newcommand {\vanish}[1]{\relax}

\newcommand {\wh}{\widehat}
\newcommand {\wt}{\widetilde}
\newcommand {\red}{\color{red}}

% Distributions
\newcommand{\normal}{\mathsf{N}}
\newcommand{\poi}{\mathsf{Poisson}}
\newcommand{\bern}{\mathsf{Bernoulli}}
\newcommand{\bin}{\mathsf{Binomal}}
\newcommand{\multi}{\mathsf{Multinomial}}
\newcommand{\Exp}{\mathsf{Exp}}



% put your command and environment definitions here




% some theorem environments
% remove "[theorem]" if you do not want them to use the same number sequence


  \newtheorem{thrm}{Theorem}
  \newtheorem{lemma}{Lemma}
  \newtheorem{prop}{Proposition}
  \newtheorem{cor}{Corollary}

  \newtheorem{conj}{Conjecture}
  \renewcommand{\theconj}{\Alph{conj}}  % numbered A, B, C etc

  \theoremstyle{definition}
  \newtheorem{defn}{Definition}
  \newtheorem{ex}{Example}
  \newtheorem{exs}{Examples}
  \newtheorem{question}{Question}
  \newtheorem{remark}{Remark}
  \newtheorem{notn}{Notation}
  \newtheorem{exer}{Exercise}




\title{STATS310B -- Lecture 5}
\author{Sourav Chatterjee\\ Scribed by Michael Howes}
\date{01/18/22}

\pagestyle{fancy}
\fancyhf{}
\rhead{STATS310B -- Lecture 5}
\lhead{01/18/22}
\rfoot{Page \thepage}

\begin{document}
\maketitle
\tableofcontents
\section{Sub-martingales and super-martingales}
We  ended last  lecture  with  the definition of two generalizations  of  martingales. They  were,
\begin{definition}
    Let  $\{\F_n\}_{n \ge 0}$ be a filtration  and let $\{X_n\}_{n \ge 0}$ be  an adapted  sequence    of   integrable random  variables.  
    \begin{enumerate}
        \item The sequence $\{X_n\}_{n \ge 0}$  is a  \emph{sub-martingale}  if  for all $n$, $\E(X_{n+1}|\F_n)  \ge X_n$ almost surely. 
        \item Likewise,  the  sequence $\{X_n\}_{n  \ge  0}$ is a  \emph{super-martingale} if for  all  $n$,  $\E(X_{n+1}|\F_n) \le X_n$  almost surely.
    \end{enumerate}
\end{definition}
Clearly  $\{X_n\}_{n \ge 0}$ is   a  martingale if and only   if, $\{X_n\}_{n \ge 0}$ is both a sub-martingale and a super-martingale.
\subsection{New  martingales from old}
Jensen's inequality allows us to create many sub-martingales and super-martingales from a martingale.
\begin{proposition}
    Let $\{X_n\}_{n \ge 0}$ be  a martingale and  let  $\phi : \R \to \R$ be a function such  that $Y_n = \phi(X_n)$ is integrable for all $n$.  Then,
    \begin{enumerate}
        \item If $\phi$ is convex, then $\{Y_n\}_{n \ge  0}$ is sub-martingale.
        \item If $\phi$ is concave,  then $\{Y_n\}_{n \ge 0}$ is a super-martingale.
    \end{enumerate}
\end{proposition}
\begin{proof}
    If $\phi$ is convex, then
    \[\E(Y_{n+1}|\F_n) = \E(\phi(X_{n+1})|\F_n) \ge \phi(\E(x_{n+1}|\F_n)) =\phi(X_n)=Y_n. \]
    If $\phi$ is concave, then $-\phi$ is convex  and so $\{-Y_n\}_{n \ge 0}$ is a sub-martingale. This  implies that $\{Y_n\}_{n \ge 0}$ is a super-martingale. 
\end{proof}
\begin{example}
    If $\{X_n\}_{n \ge 0}$, then $\abs{X_n}$, $X_n^2$ and $e^{\ta X_n}$ are all sub-martingales (provided the last two are integrable). 
\end{example}
We can also get new sub-martingales from a sub-martingale.
\begin{proposition}
    Let $\{X_n\}_{n \ge 0}$ be a sub-martingale and  let $\phi:\R\to \R$ be a  non-decreasing convex function. If $Y_n = \phi(X_n)$ is integrable from every $n$, then $\{Y_n\}_{n \ge 0}$ is a sub-martingale.
\end{proposition}
\begin{proof}
    By convexity and Jensen's,
    \[\E(Y_{n+1}|\F_n) \ge \phi(\E(X_{n+1}|\F_n)). \]
    Also, $\E(X_{n+1}|\F_n) \ge X_n$ almost surely. Since $\phi$ is non-decreasing, this implies
    \[\phi(\E(X_{n+1}|\F_n)) \ge \phi(X_n)  =Y_n.\qedhere \]
\end{proof}
\begin{example}
    If $\{X_n\}_{n \ge  0}$ is a  sub-martingale, then $X_n^+$ is a sub-martingale. If $\ta > 0$ and $e^{\ta X_n}$ is integrable for every $n$, then $e^{\ta X_n}$ is also a sub-martingale. The random  variables $X_n^2$  and $\abs{X_n}$  need not form sub-martingales  even if they  are integrable. 
\end{example}
We can also  get a martingale  from a sub-martingale.
\begin{proposition}[Doob's  decomposition]
    Let $\{X_n\}_{n \ge  0}$  be  a sub-martingale. Then we can write $X_n = X_0+M_n+A_n$, where $\{M_n\}_{n \ge 0}$ is  a martingale and $\{A_n\}_{n \ge 0}$ is a non-decreasing,  predictable  sequence.
\end{proposition}
The definition  of  a predictable  sequence is given below.
\begin{definition}
    Let $\{F_n\}_{n \ge  0}$  be a   filtration. An adapted  sequence of random variables $\{A_n\}_{n \ge 0}$ is \emph{predictable} if for  every $n \ge 1$, $A_n$ is $\F_{n-1}$-measurable.
\end{definition}
\begin{proof}[Proof  of Doob's  decomposition]
    Define,
    \[ M_n  = \sum_{k=0}^{n-1} X_{k+1}-X_k - \E(X_{k+1}-X_k|\F_k). \]
    Also define,
    \[ A_n =  \sum_{k=0}^{n-1}\E(X_{k+1}-X_k|\F_k). \]
    Then,
    \[M_n+A_n = \sum_{k=0}^{n-1}X_{k+1}-X_k = X_n -X_0. \]
    Thus, it remains to show that $\{M_n\}_{n \ge 0}$ is a martingale and that $\{A_n\}_{n \ge 0}$ is predictable  and non-decreasing. Note  that  for every  $k$,  $\E(X_{k+1}-X_k|\F_k)$ is  $\F_k$-measurable.  Thus, $A_n$  is  $\F_{n-1}$-measurable. We also have,
    \[A_{n+1}-A_n = \E(X_{n+1}-X_n|\F_n) = \E(X_{n+1}|\F_n)-\E(X_n|\F_n)=\E(X_{n+1}|\F_n)-X_n \ge  0.\]
    Also note that,
    \[\E(X_{n+1}-X_n - \E(X_{n+1}-X_n|\F_n)|\F_n) = \E(X_{n+1}-X_n|\F_n)-\E(X_{n+1}-X_n|\F_n) = 0. \]
    Thus,
    \[\E(M_{n+1}-M_n|\F_n) = \E(X_{n+1}-X_n-\E(X_{n+1}-X_n|\F_n)|\F_n) = 0. \]
    Thus, $\E(M_{n+1}|\F_n) =\E(M_n|\F_n)=M_n$ and so $\{M_n\}_{n \ge 0}$  is a martingale.
\end{proof}
Doob's  decomposition  result is  important  because  a  lot  of  properties about  martingales  are well known.  Thus,  if one   can  get a handle on the increasing  predictable sequence $\{A_n\}_{n \ge 0}$,  then  the original sub-martingale  $\{X_n\}_{n  \ge 0}$ can  be studied. 
\subsection{Optional stopping for sub and super-martingales}
\begin{proposition}
    Let  $\{X_n\}_{n \ge   0}$ be an  integrable sequence adapted to $\{\F_n\}_{n  \ge 0}$.  Let $T$  and  $S$  be bounded stopping  times for  $\{\F_n\}_{n \ge 0}$. Then,
    \begin{enumerate}
        \item If  $\{X_n\}_{n \ge 0}$  is  a  sub-martingale,  then $\E(X_T|\F_S)  \ge  X_S$.
        \item If  $\{X_n\}_{n  \ge 0}$ is a super-martingale, then $\E(X_T|\F_S) \le  X_S$.
    \end{enumerate}
\end{proposition}
The  following proposition allows us to rigorously work with sequences  $\{X_n\}_{n \ge  0}$   that  are (sub/super)-martingales up  to  a stopping time $T$.
\begin{proposition}\label{up_to}
    Let  $\{X_n\}_{n \ge  0}$ be a  sequence of integrable random variables adapted to a filtration $\{\F_n\}_{n \ge 0}$. Let $T$ be a stopping time with respect to $\{\F_n\}_{n \ge 0}$, such that on the event $\{T>n\}$, we have
    \begin{equation}
        \label{ineq} X_n \le \E(X_{n+1}|\F_n)~~~\text{a.s.}
    \end{equation}
    By which we mean $\Pa(X_n > \E(X_{n+1}|\F_n), T > n) = 0$. Then the sequence $\{X_{n \land T}\}_{n \ge 0}$ is a sub-martingale with respect to $\{\F_n\}_{n \ge 0}$
\end{proposition}

\begin{proof}
    As an exercise, one can show that $X_{n \land T}$ is $\F_n$-measurable and integrable. Now note that,
    \begin{align*}
        \E(X_{(n+1) \land T}|\F_n) &=\E\left(\sum_{i=0}^n X_{(n+1)\land T}\one_{\{T=i\}} + X_{(n+1)\land T}\one_{\{T > n\}}|\F_n\right)\\
        &=\sum_{i=0}^n\E(X_{(n+1) \land T}\one_{\{T=i\}}|\F_n) +\E(X_{(n+1)\land T}\one_{\{T> n\}}|\F_n)\\
        &=\sum_{i=0}^n\E(X_{i}\one_{\{T=i\}}|\F_n) +\E(X_{n+1}\one_{\{T> n\}}|\F_n)\\
        &=\sum_{i=0}^n X_i\one_{\{T=i\}}+\E(X_{n+1}\one_{\{T> n\}}|\F_n),
    \end{align*}
    since $X_i\one_{\{T=i\}}$ is $\F_n$-measurable. The event $\{T > n\} = \{T \le n\}^c $ is in $\F_n$ and thus, by our assumption \eqref{ineq},
    \begin{align*}
        \E(X_{(n+1) \land T}|\F_n)&=\sum_{i=0}^n X_i\one_{\{T=i\}}+\one_{\{T> n\}}\E(X_{n+1}|\F_n)\\
        &\ge \sum_{i=0}^n X_i\one_{\{T=i\}}+\one_{\{T> n\}}X_n\\
        &=\sum_{i=0}^{n-1}X_i\one_{\{T=i\}} + X_n\one_{\{T > n-1\}}\\
        &=X_{n \land T}. \qedhere
    \end{align*}
\end{proof}
\begin{remark}
    If the inequality in \eqref{ineq} is replaced with an equality, then $\{X_{n \land T}\}_{n \ge 0}$ is a martingale. Likewise, if the inequality in \eqref{ineq} is reversed, then $\{X_{n \land T}\}_{n \ge 0}$ is a super-martingale. The proofs are analogous.
\end{remark}
The idea behind proposition \eqref{up_to} is that even we can ignore what happens after the stopping time $T$. This is useful in examples when the distribution of $X_n$ changes after $T$ occurs. 
\begin{example}
    Suppose a gambler starts with $x>a$ dollars. At each turn the gambler can win or loss a dollar with equal probability. However, will their total is greater $a$, they have to pay $b$ in tax each turn. We wish to know how long it will take for the gambler to have less than $a$. Let $\{X_n\}_{n \ge 0}$ be the total the gambler has at each turn and let $\F_n = \sigma(X_1,\ldots, X_n)$. Let $T = \min\{n : X_n \le a \}$. We wish to bound $\E[T]$. Note that the event $\{T > n\}$ equals $\{X_1 > a,\ldots, X_n > a\}$. Thus, on the event $\{T > n\}$, $X_{n+1}$ is either $X_n+1-b$ or $X_n-1-b$. Furthermore, $X_{n+1}-X_n$ is independent of $\F_n$. Thus, on $\{T > n\}$,
    \[\E(X_{n+1}|\F_n) = X_n +\E[X_{n+1}-X_n]=X_n-b \le X_n. \]
    Now define $Y_n = X_n+nb$. On $\{T > n\}$, we have $\E(Y_{n+1}|\F_n) = X_N-b+(n+1)b=Y_n$. Thus, $\{Y_{n \land T}\}_{n \ge 0}$ is a martingale. Since $0 \land T = 0$, we thus have
    \[\E[Y_{T \land n}] = \E[Y_0]=\E[X_0]=\E[x]=x. \]
    On the other hand,
    \[\E[Y_{T \land n}] = \E[X_{T \land n}]+b\E[T\land n]\ge a+b\E[T\land n]. \]
    The inequality holds because if $T > n$, then $X_{T \land n}=X_n > a$ and if $T \le n$, then $X_{T \land n} = X_T = a$. The random variables $n \land T$ are non-negative and increase up to $T$. Thus, 
    \[x \ge a+b\E[T], \]
    which implies $\E[T] \le \frac{x-a}{b}$. This is agrees with our intuition, since the gambler loses $b$ dollars each turn on average and $T$ records the time it takes the gambler's total to go from $x$ dollars to $a$ dollars. 
\end{example}
\section{Optimal stopping problem}
We will now consider a different kind of problem. Let $(\Om,\F,\Pa)$ be a probability space. The finite horizon optimal stopping problem is as follows. Suppose we have integrable random variables $\{X_n\}_{n=1}^N$ and a filtration $\{\F_n\}_{n = 1}^N$, can we find a stopping time $T$ taking values in $\{1,\ldots,N\}$ that maximizes $\E[X_T]$? Note that we have made no assumptions about $\{X_n\}_{n=1}^N$ beyond integrability. We have not assumed that $\{X_n\}_{n=1}^N$ is adapted to $\{\F_n\}_{n=1}^N$. Nonetheless, this problem has a solution in this very general set up!


\end{document}
\documentclass{article}
\usepackage{ae,aecompl}
\usepackage{todonotes}
\usepackage{chngcntr}
\usepackage{tikz-cd}
\usepackage{graphicx}
\graphicspath{ {./images/}}
\usepackage[all,cmtip]{xy}
\usepackage{amsmath, amscd}
\usepackage{amsthm}
\usepackage{amssymb}
\usepackage{amsfonts}
\usepackage{bm}
\usepackage{qsymbols}
\usepackage{latexsym}
\usepackage{mathrsfs}
\usepackage{mathtools}
\usepackage{cite}
\usepackage{color}
\usepackage{url}
\usepackage{enumerate}
\usepackage{verbatim}
\usepackage[draft=false, colorlinks=true]{hyperref}
\usepackage{pdfpages}
\usepackage[margin=1.2in]{geometry}
\usepackage{IEEEtrantools}

\usepackage{fancyhdr}


\usepackage[nameinlink]{cleveref}


\DeclareMathOperator*{\ac}{accept}
\DeclareMathOperator*{\amax}{argmax}
\DeclareMathOperator*{\amin}{argmin}
\DeclareMathOperator*{\Aut}{Aut}
\newcommand {\al}{{\alpha}}
\newcommand {\abs}[1]{{\left\lvert#1\right\rvert}}
\newcommand {\A}{{\mathcal{A}}}
\newcommand {\AM}{{\mathrm{AM}}}
\newcommand {\AMp}{{\AM_{p}^{X}\!(\Ri_\w)}}
\newcommand {\B}{{\mathcal{B}}}
\DeclareMathOperator*{\Be}{Bern}
\newcommand {\Br}{{\dot{B}}}
\newcommand {\Ba}{{\mathfrak{B}}}
\newcommand {\C}{{\mathbb C}}
\newcommand {\ce}{\mathrm{c}}
\newcommand {\Ce}{\mathrm{C}}
\newcommand {\Cc}{\mathrm{C_{c}}}
\newcommand {\Ccinf}{\mathrm{C_{c}^{\infty}}}
\DeclareMathOperator{\cov}{Cov}
\DeclareMathOperator{\DEV}{DEV}
\newcommand {\Di}{{\mathbb D}}
\newcommand {\dom}{\mathrm{dom}}
\newcommand{\dist}{\stackrel{\mathrm{dist}}{=}}
\newcommand {\ud}{\mathrm{d}}
\newcommand {\ue}{\mathrm{e}}
\newcommand {\eps}{\varepsilon}
\newcommand {\veps}{\varepsilon}
\newcommand {\vrho}{{\varrho}}
\newcommand {\E}{{\mathbb{E}}}
\newcommand {\Ec}{{\mathcal{E}}}
\newcommand {\Ell}{L}
\newcommand {\Ellp}{{L_{p}[0,1]}}
\newcommand {\Ellpprime}{{L_{p'}([0,1])}}
\newcommand {\Ellq}{{L_{q}([0,1])}}
\newcommand {\Ellqprime}{{L_{q'}([0,1])}}
\newcommand {\Ellr}{L^{r}}
\newcommand {\Ellone}{{L_{1}([0,1])}}
\newcommand{\Elltwo}{{L_{2}([0,1])}}
\newcommand{\Ellinfty}{L^{\infty}}
\newcommand{\Ellinftyc}{L_{\mathrm{c}}^{\infty}}
\newcommand{\exb}[1]{\exp\left\{#1\right\}}
\DeclareMathOperator*{\Ext}{Ext}
\newcommand{\F}{{\mathcal{F}}}
\newcommand{\Fe}{{\mathbb{F}}}
\newcommand{\G}{{\mathcal{G}}}
\newcommand{\HF}{\mathcal{H}_{\text{FIO}}^{1}(\Rd)}
\newcommand{\Hr}{H}
\newcommand{\HT}{\mathcal{H}}
\newcommand{\ui}{\mathrm{i}}
\newcommand{\I}{{I}}
\newcommand{\J}{{\mathcal{J}}}
\newcommand{\id}{{\mathrm{id}}}
\newcommand{\iid}{\stackrel{\mathclap{\normalfont\mbox{iid}}}{\sim}}
\newcommand{\im}{{\text{im }}}
\newcommand{\ind}{{\perp\!\!\!\perp}}
\DeclareMathOperator*{\Int}{int}
\newcommand{\intx}{{\overline{\int_{X}}}}
\newcommand{\inte}{{\overline{\int_{\E}}}}
\newcommand{\la}{\lambda}
\newcommand{\rb}{\rangle}
\newcommand{\lb}{{\langle}}
\newcommand{\La}{\Lambda}
\newcommand{\calL}{{\mathcal{L}}}
\newcommand{\lp}{{\mathcal{L}}^{p}}
\newcommand{\lpo}{{\overline{\mathcal{L}}^{p}\!}}
\newcommand{\Lpo}{{\overline{\Ell}^{p}\!}}
\newcommand{\M}{{\mathbf{M}}}
\newcommand{\Ma}{{\mathcal{M}}}
\newcommand{\N}{{{\mathbb N}}}
\newcommand{\Na}{{{\mathcal{N}}}}
\newcommand{\norm}[1]{\left\|#1\right\|}
\newcommand{\normm}[1]{{\left\vert\kern-0.25ex\left\vert\kern-0.25ex\left\vert #1 
    \right\vert\kern-0.25ex\right\vert\kern-0.25ex\right\vert}}
\newcommand{\Om}{{{\Omega}}}
\newcommand{\one}{{{\bf 1}}}
\newcommand{\pic}{\text{Pic }}
\newcommand{\ph}{{\varphi}}
\newcommand{\Pa}{{\mathbb{P}}}
\newcommand{\Po}{{\mathcal{P}}}
\newcommand{\Q}{{\mathbb{Q}}}
\newcommand{\R}{{\mathbb R}}
\newcommand{\Rd}{{\mathbb{R}^{d}}}
\DeclareMathOperator{\rej}{reject }
\newcommand{\Rn}{{\mathbb{R}^{n}}}
\newcommand{\cR}{{\mathcal{R}}}
\newcommand{\Rad}{{\mathrm{Rad}}}
\newcommand{\ran}{{\mathrm{ran}}}
\newcommand{\Ri}{{\mathrm{R}}}
\newcommand{\supp}{{\mathrm{supp}}}
\newcommand{\Se}{\mathrm{S}}
\newcommand{\Sp}{S^{*}(\Rn)}
\newcommand{\St}{{\mathrm{St}}}
\newcommand{\Sw}{\mathcal{S}}
\newcommand{\T}{{\mathcal{T}}}
\newcommand{\ta}{{\theta}}
\newcommand{\Ta}{{\Theta}}
\newcommand{\topp}{\stackrel{p}{\to}}
\newcommand{\todd}{\stackrel{d}{\to}}
\newcommand{\toL}[1]{\stackrel{L^{#1}}{\to}} 
\newcommand{\toas}{\stackrel{a.s.}{\to}}
\DeclareMathOperator{\V}{Var}
\newcommand {\w}{{\omega}}
\newcommand {\W}{{\mathrm{W}}}
\newcommand {\Wnp}{\text{$\mathrm{W}$\textsuperscript{$n,\!p$}}}
\newcommand {\Wnpeq}{\text{$\mathrm{W}$\textsuperscript{$n\!,\!p$}}}
\newcommand {\Wonep}{\text{$\mathrm{W}$\textsuperscript{$1,\!p$}}}
\newcommand {\Wonepeq}{\text{$\mathrm{W}$\textsuperscript{$1\!,\!p$}}}
\newcommand {\X}{{\mathcal{X}}}
\newcommand {\Z}{{{\mathbb Z}}}
\newcommand {\Za}{{\mathcal{Z}}}
\newcommand {\Zd}{{\Z[\sqrt{d}]}}
\newcommand {\vanish}[1]{\relax}

\newcommand {\wh}{\widehat}
\newcommand {\wt}{\widetilde}
\newcommand {\red}{\color{red}}

% Distributions
\newcommand{\normal}{\mathsf{N}}
\newcommand{\poi}{\mathsf{Poisson}}
\newcommand{\bern}{\mathsf{Bernoulli}}
\newcommand{\bin}{\mathsf{Binomal}}
\newcommand{\multi}{\mathsf{Multinomial}}
\newcommand{\Exp}{\mathsf{Exp}}



% put your command and environment definitions here




% some theorem environments
% remove "[theorem]" if you do not want them to use the same number sequence


  \newtheorem{thrm}{Theorem}
  \newtheorem{lemma}{Lemma}
  \newtheorem{prop}{Proposition}
  \newtheorem{cor}{Corollary}

  \newtheorem{conj}{Conjecture}
  \renewcommand{\theconj}{\Alph{conj}}  % numbered A, B, C etc

  \theoremstyle{definition}
  \newtheorem{defn}{Definition}
  \newtheorem{ex}{Example}
  \newtheorem{exs}{Examples}
  \newtheorem{question}{Question}
  \newtheorem{remark}{Remark}
  \newtheorem{notn}{Notation}
  \newtheorem{exer}{Exercise}




\title{STATS310B -- Lecture 6}
\author{Sourav Chatterjee\\ Scribed by Michael Howes}
\date{01/22/22}

\pagestyle{fancy}
\fancyhf{}
\rhead{STATS310B -- Lecture 6}
\lhead{01/22/22}
\rfoot{Page \thepage}

\begin{document}
\maketitle
\tableofcontents
\section{Optimal stopping problem}
We ended last lecture with the statement of the optimal stopping problem with a finite horizon. The problem was as follows. Suppose we are given integrable random variables $X_1,\ldots,X_N$ on a probability space $(\Om,\F,\Pa)$ and a filtration $\F_1\subseteq \ldots \subseteq \F_N$ of sub $\sigma$-algebras. We wish to find a stopping time $T$ with respect to $\{\F_n\}_{n \ge 0}$ taking values in $\{1,\ldots,N\}$ that maximizes $\E[X_T]$. We think of $X_n$ as the reward at time $n$, and thus we wish to find a way of stopping that maximizes the expected reward at time $T$.

Our first step is to replace $\{X_n\}_{n=1}^N$ with a sequence adapted to $\{\F_n\}_{n=1}^N$. With this in mind, define $Y_n = \E(X_n|\F_n)$. The next lemma shows that we can work with $Y_n$ instead of $X_n$.
\begin{lemma}\label{Ys}
    For any stopping time $T$, $\E[X_T]=\E[Y_T]$.
\end{lemma}
\begin{proof}
    Since $\{T=n\} \in \F_n$, we have
    \begin{align*}
        \E[X_T] &=\E\left[\sum_{n=1}^N X_n \one_{\{T=n\}}\right] \\
        &=\sum_{n=1}^N \E\left[X_n\one_{\{T=n\}}\right]\\
        &=\sum_{n=1}^N \E\left[\E(X_n|\F_n)\one_{\{T=n\}}\right]\\
        &=\sum_{n=1}^N\E[Y_n\one_{\{T=n\}}]\\
        &=\E\left[\sum_{n=1}^NY_n\one_{\{T=n\}}\right]\\
        &=\E[Y_T].\qedhere 
    \end{align*}
\end{proof}
Thus it suffices to find a stopping time that maximizes $\E[Y_T]$. We will now define random variable $V_1,\ldots,V_N$ by backwards induction. Let,
\begin{align*}
    V_N &= Y_N,\\
    V_n &= \max\{Y_n, \E(V_{n+1}|\F_n)\} ~~~\text{for } n < N.
\end{align*}
The idea is that $V_n$ is either equal to either the reward at time $n$ ($Y_n$) or the expected future reward given the information up to time $n$ ($\E(V_{n+1}|\F_n)$). The random variable $V_n$ selects whichever reward looks to be bigger. Note the following,
\begin{enumerate}
    \item By definition, $V_N = Y_N$.
    \item For all $n$, $V_n \ge Y_n$ by definition.
\end{enumerate}
Thus, define the following random variable taking values in $\{1,\ldots,N\}$.
\[\tau = \min\{n : V_n = Y_n\}.\]
Note that $\tau$ is a stopping time for $\{\F_n\}_{n=1}^N$, since both $V_n$ and $Y_n$ are $\F_n$-measurable. We will now show that $\tau$ is an optimal stopping time.
\begin{theorem}
    For all stopping times $T$, $\E[X_T] \le \E[X_\tau]$.
\end{theorem}
\begin{proof}
    Fix a stopping time $T$.  By lemma \eqref{Ys}, it suffices to show that $\E[Y_T] \le \E[Y_\tau]$. Suppose that for some $\w \in \Om$ and $n \in \{1,\ldots,N\}$, we have $\tau(\w) > n$. This would imply that $V_n(\w)  \neq Y_n(\w)$. Since $V_n = \max\{Y_n, \E(V_{n+1}|\F_n)\}$, this implies that
    \[V_n(\w) = \E(V_{n+1}|\F_n)(\w). \]
    Thus, one $\{\tau > n\}$, $V_n = \E(V_{n+1}|\F_n)$. By a proposition proved in the previous lemma, this implies that $\{V_{n \land \tau}\}_{n=1}^N$ is a martingale with respect to $\{\F_n\}_{n=1}^N$. By the definition of $\tau$, $V_\tau = Y_\tau$. Also, $\tau \le N$ surely which means $\tau = N \land \tau$. Thus,
    \[\E[Y_\tau] = \E[V_\tau] = \E[V_{N\land \tau}] = \E[V_{1\land \tau}] = \E[V_1]. \]
    So, $\E[Y_\tau] = \E[V_1]$. Note that, by definition,
    \[V_n \ge \E(V_{n+1}|\F_n), \]
    sand thus $\{V_n\}_{n=1}^N$ is a super-martingale. Thus, by the optional stopping theorem,
    \[\E[V_T] \le \E[V_1] = \E[Y_\tau]. \]
    But we have $Y_n \le V_n$ for every $n =1,\ldots,N$ and so $Y_T \le Y_T$. Thus,
    \[\E[Y_T]\le\E[V_T]\le\E[Y_\tau].\qedhere \]
\end{proof}
\section{A hiring problem}
We will now consider an application of the optimal stopping theorem. Suppose we have $N$ candidates for a job, and we are to interview them sequentially. The $N$ candidates have ranks $(r_1,\ldots,r_N)$ which is a uniform permutation of $1,\ldots,N$. After each interview, we have to either hire the current candidate immediately or we have to continue the search with no possibility of hiring the current candidate later.

The information we have up to time $n$ is the relative ranks of the first $n$ candidates which we label $(r_1^n,\ldots, r_n^n)$. For example, if $N=5$ and $(r_1,r_2,r_3,r_4,r_5)= (2,4,1,3,5)$. The for $n=3$, we would have $(r_1^3, r_2^3, r_3^3)= (2,3,1)$. Our filtration is given by $\F_n = \sigma(r_1^n,\ldots, r_n^n)$. 

We wish to maximize the probability that we pick the best candidate (i.e. the candidate with rank 1). Thus, we wish to find a stopping time that maximizes $\Pa(r_T=1)$. To put this into the framework of the optimal stopping problem, define
\[X_n = \one_{\{r_n=1\}}. \]
This gives, $\E(X_T)=\Pa(r_T=1)$ for all stopping times $T$. Our first step is to calculate $Y_n = \E(X_n|\F_n)$. We claim that,
\[Y_n = \begin{cases}
    0 & \text{if } r_n^n \neq 1,\\
    1 & \text{if } r_n^n = 1.
\end{cases} \]
\begin{proof}
Since $Y_n$ is a function of $r_1^n,\ldots, r_n^n$, the random variable $Y_n$ is $\F_n$-measurable. To see that $Y_n = \E(X_n|\F_n)$, note that every event in $\F_n$ can be written as a finite disjoint union of events of the form $A_\sigma = \{(r_1^n,\ldots, r_n^n) = (\sigma_1,\ldots,\sigma_n)\}$ where $\sigma$ is a permutation of $\{1,\ldots,n\}$. Thus, it suffices to show that 
\[\E[X_n \one_{A_\sigma}] = \E[Y_n \one_{A_\sigma}], \]
for all $\sigma$. Fix such a $\sigma$. If $\sigma_n \neq 1$ and $r_n^n = \sigma_n$, then $r_n \neq 1$ and so $\E[X_n \one_{A_\sigma}] = 0 = \E[Y_n \one_{A_\sigma}]$. Thus suppose that $\sigma_n=1$. Then,
\begin{align*}
    \E[X_n \one_{A_\sigma}] &= \Pa(r_n = 1, (r_n^1,\ldots, r_n^n)= (\sigma_1,\ldots,1))\\
    &=\Pa(r_n=1|r_1^n=\sigma_1,\ldots, r_n^{n-1}=\sigma_{n-1}, r_n^n = 1)\Pa(r_1^n=\sigma_1,\ldots, r_n^{n-1}=\sigma_{n-1}, r_n^n = 1)\\
    &=\Pa(r_n=1|r_1^n=\sigma_1,\ldots, r_n^{n-1}=\sigma_{n-1}, r_n^n = 1)\frac{1}{n!}.
\end{align*}
By Bayes's rule, $\Pa(A|B) = \Pa(B|A)\frac{\Pa(A)}{\Pa(B)}$. Thus, 
\begin{align*}
    &\Pa(r_n=1|r_1^n=\sigma_1,\ldots, r_n^{n-1}=\sigma_{n-1}, r_n^n = 1)\\
     &= \Pa(r_1^n=\sigma_1,\ldots, r_n^{n-1}=\sigma_{n-1}, r_n^n = 1|r_n=1) \frac{\Pa(r_n=1)}{\Pa(r_1^n=\sigma_1,\ldots, r_n^{n-1}=\sigma_{n-1}, r_n^n = 1)}\\
    &=\frac{1}{(n-1)!} \frac{1/N}{1/n!}\\
    &= \frac{n}{N}.
\end{align*}
Thus,
\begin{align*}
    \E[X_n \one_{A_\sigma}]&=\frac{n}{N}\cdot \frac{1}{n!}\\
    &=\frac{n}{N}\Pa(_1^n=\sigma_1,\ldots, r_n^{n-1}=\sigma_{n-1}, r_n^n = 1)\\
    &=\E[Y_n\one_{A_\sigma}].
\end{align*}
Thus, $Y_n = \E(X_n|\F_n)$.
\end{proof}
The definition of $Y_n$ can also be seen intuitively. If $r_n^n=1$, then $r_n$ is the highest rank out of $r_1,\ldots,r_n$. The probably that $r_n=1$ given $r_n^n=1$ is thus the probability that one of $r_1,\ldots,r_n$ is 1 and this event has probability $\frac{n}{N}$. Also, clearly if $r_n^n \neq 1$, then we cannot have $r_n =1$. 

We next claim that $Y_n$ is independent of $\F_{n-1}$ for $n \ge 2$. 
\begin{proof}
    Since $Y_n$ only takes two values, it suffices to prove that $\Pa(Y_n = n/N, A) = \Pa(Y_n = n/N)\Pa(A)$ for all $A \in \F_{n-1}$. As before, we can restrict our attention to $A$ of the form $A_\sigma = \{r_1^{n-1}=\sigma_1,\ldots, r_{n-1}^{n-1} = \sigma_{n-1}\}$ for $\sigma$ a permutation of $\{1,\ldots,n-1\}$. Note that,
    \[\Pa(A_\sigma) = \frac{1}{(n-1)!} \text{ and } \Pa(Y_n = n/N) = \Pa(r_n^n = 1) = \frac{1}{n}. \]
    And also,
    \[\Pa(Y_n = n/N, A_\sigma) = \Pa(r_1^n = \sigma_1+1,\ldots, r_{n-1}^n = \sigma_{n-1}+1, r_n^n = 1) = \frac{1}{n!}. \]
    Thus, $\Pa(Y_n = n/N,A_\sigma) = \Pa(Y_n = n/N)\Pa(A_\sigma)$. 
\end{proof}
As in the proof of the optimal stopping theorem, let 
\begin{align*}
    V_N &= Y_N,\\
    V_n &= \max\{Y_n, \E(V_{n+1}|\F_n)\} ~~~\text{for } n < N.
\end{align*}
By the independence of $Y_n$ and $\F_{n-1}$ we can show that for all $n \le N-1$, the random variable $\E(V_{n+1}|\F_n)$ is a number that we will denote by $v_n^N$. 
\begin{proof}[Proof of above claim]
    Note that, by independence, 
    \[\E(V_N|\F_{N-1})=\E(Y_N|\F_{N-1})  =\E[Y_N] =\frac{N}{N}\cdot \Pa(r_N^N=1)=\frac{1}{N}.\]
    Thus, the result holds for $n = N-1$, with $v_{N-1}^N = \frac{1}{N}$. Now suppose that the result holds for some $n$ so that $\E(V_{n+1}\F_n) = v_n^N$. Then $V_n = \max\{Y_n, v_n^N\}$ is independent of $\F_{n-1}$ and so,
    \[\E(V_n|\F_{n-1})=\E[\max\{Y_n, v_n^N\}] = v_{n-1}^N.\qedhere \]
\end{proof}
By our solution to the optimal stopping problem, the optimal stopping time is
\[ \tau = \min\{n : V_n = Y_n\} = \min\{n : Y_n \ge \E(V_{n+1}|\F_{n})\} = \min\{n : Y_n \ge v_n^N\},\]
where we interpret $\E(V_{N+1}|\F_N)$ as $Y_N$ and $v_N^N$ as 0. Note that the condition $v_{n-1}^N = \max\{Y_n, v_n^N\}$ implies that the sequence $v_n^N$ is decreasing. The random variable $Y_n$ take values $\frac{n}{N}$ and 0. Thus, while $\frac{n}{N} < v_n^N$, we will never stop at time $n$. But once $\frac{n}{N} \ge v_n^N$, we will stop once we have $Y_n \neq 0$ which means $r_n^n = 1$. Thus,
\[\tau = \min\{n : n \ge t^N \text{ and } r_n^n=1\}, \]
where $t^N = \min\{n : \frac{n}{N} \ge v_n^N\}$. Thus, our rule is as follows:
\begin{enumerate}
    \item Interview the first $t^N-1$ candidates but do not hire any of them.
    \item Hire the next candidate who is better than all the first  $t^N-1$ candidates.
\end{enumerate}  
It remains to calculate $t^N$ which is a non-random quantity. Recall that $Y_n = \frac{n}{N}$ with probability $\frac{1}{n}$ and $Y_n = 0$ otherwise. Thus,
\[v_{n-1}^n = v_n^N \cdot \frac{n-1}{n} + \frac{n}{N}\cdot \frac{1}{n} = \frac{n-1}{n}v_n^N + \frac{1}{N}. \]
We also know that $v_{N-1}^N = \E[Y_N] = \frac{1}{N}$. Thus, by applying the above recursion, we get
\[v_n^N = \frac{1}{N}\left\{1+n \sum_{k=n+1}^{N-1} \frac{1}{k}\right\} \]
\end{document}
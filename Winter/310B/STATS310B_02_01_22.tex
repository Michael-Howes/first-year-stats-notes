\documentclass{article}
\usepackage{ae,aecompl}
\usepackage{todonotes}
\usepackage{chngcntr}
\usepackage{tikz-cd}
\usepackage{graphicx}
\graphicspath{ {./images/}}
\usepackage[all,cmtip]{xy}
\usepackage{amsmath, amscd}
\usepackage{amsthm}
\usepackage{amssymb}
\usepackage{amsfonts}
\usepackage{bm}
\usepackage{qsymbols}
\usepackage{latexsym}
\usepackage{mathrsfs}
\usepackage{mathtools}
\usepackage{cite}
\usepackage{color}
\usepackage{url}
\usepackage{enumerate}
\usepackage{verbatim}
\usepackage[draft=false, colorlinks=true]{hyperref}
\usepackage{pdfpages}
\usepackage[margin=1.2in]{geometry}
\usepackage{IEEEtrantools}

\usepackage{fancyhdr}


\usepackage[nameinlink]{cleveref}


\DeclareMathOperator*{\ac}{accept}
\DeclareMathOperator*{\amax}{argmax}
\DeclareMathOperator*{\amin}{argmin}
\DeclareMathOperator*{\Aut}{Aut}
\newcommand {\al}{{\alpha}}
\newcommand {\abs}[1]{{\left\lvert#1\right\rvert}}
\newcommand {\A}{{\mathcal{A}}}
\newcommand {\AM}{{\mathrm{AM}}}
\newcommand {\AMp}{{\AM_{p}^{X}\!(\Ri_\w)}}
\newcommand {\B}{{\mathcal{B}}}
\DeclareMathOperator*{\Be}{Bern}
\newcommand {\Br}{{\dot{B}}}
\newcommand {\Ba}{{\mathfrak{B}}}
\newcommand {\C}{{\mathbb C}}
\newcommand {\ce}{\mathrm{c}}
\newcommand {\Ce}{\mathrm{C}}
\newcommand {\Cc}{\mathrm{C_{c}}}
\newcommand {\Ccinf}{\mathrm{C_{c}^{\infty}}}
\DeclareMathOperator{\cov}{Cov}
\DeclareMathOperator{\DEV}{DEV}
\newcommand {\Di}{{\mathbb D}}
\newcommand {\dom}{\mathrm{dom}}
\newcommand{\dist}{\stackrel{\mathrm{dist}}{=}}
\newcommand {\ud}{\mathrm{d}}
\newcommand {\ue}{\mathrm{e}}
\newcommand {\eps}{\varepsilon}
\newcommand {\veps}{\varepsilon}
\newcommand {\vrho}{{\varrho}}
\newcommand {\E}{{\mathbb{E}}}
\newcommand {\Ec}{{\mathcal{E}}}
\newcommand {\Ell}{L}
\newcommand {\Ellp}{{L_{p}[0,1]}}
\newcommand {\Ellpprime}{{L_{p'}([0,1])}}
\newcommand {\Ellq}{{L_{q}([0,1])}}
\newcommand {\Ellqprime}{{L_{q'}([0,1])}}
\newcommand {\Ellr}{L^{r}}
\newcommand {\Ellone}{{L_{1}([0,1])}}
\newcommand{\Elltwo}{{L_{2}([0,1])}}
\newcommand{\Ellinfty}{L^{\infty}}
\newcommand{\Ellinftyc}{L_{\mathrm{c}}^{\infty}}
\newcommand{\exb}[1]{\exp\left\{#1\right\}}
\DeclareMathOperator*{\Ext}{Ext}
\newcommand{\F}{{\mathcal{F}}}
\newcommand{\Fe}{{\mathbb{F}}}
\newcommand{\G}{{\mathcal{G}}}
\newcommand{\HF}{\mathcal{H}_{\text{FIO}}^{1}(\Rd)}
\newcommand{\Hr}{H}
\newcommand{\HT}{\mathcal{H}}
\newcommand{\ui}{\mathrm{i}}
\newcommand{\I}{{I}}
\newcommand{\J}{{\mathcal{J}}}
\newcommand{\id}{{\mathrm{id}}}
\newcommand{\iid}{\stackrel{\mathclap{\normalfont\mbox{iid}}}{\sim}}
\newcommand{\im}{{\text{im }}}
\newcommand{\ind}{{\perp\!\!\!\perp}}
\DeclareMathOperator*{\Int}{int}
\newcommand{\intx}{{\overline{\int_{X}}}}
\newcommand{\inte}{{\overline{\int_{\E}}}}
\newcommand{\la}{\lambda}
\newcommand{\rb}{\rangle}
\newcommand{\lb}{{\langle}}
\newcommand{\La}{\Lambda}
\newcommand{\calL}{{\mathcal{L}}}
\newcommand{\lp}{{\mathcal{L}}^{p}}
\newcommand{\lpo}{{\overline{\mathcal{L}}^{p}\!}}
\newcommand{\Lpo}{{\overline{\Ell}^{p}\!}}
\newcommand{\M}{{\mathbf{M}}}
\newcommand{\Ma}{{\mathcal{M}}}
\newcommand{\N}{{{\mathbb N}}}
\newcommand{\Na}{{{\mathcal{N}}}}
\newcommand{\norm}[1]{\left\|#1\right\|}
\newcommand{\normm}[1]{{\left\vert\kern-0.25ex\left\vert\kern-0.25ex\left\vert #1 
    \right\vert\kern-0.25ex\right\vert\kern-0.25ex\right\vert}}
\newcommand{\Om}{{{\Omega}}}
\newcommand{\one}{{{\bf 1}}}
\newcommand{\pic}{\text{Pic }}
\newcommand{\ph}{{\varphi}}
\newcommand{\Pa}{{\mathbb{P}}}
\newcommand{\Po}{{\mathcal{P}}}
\newcommand{\Q}{{\mathbb{Q}}}
\newcommand{\R}{{\mathbb R}}
\newcommand{\Rd}{{\mathbb{R}^{d}}}
\DeclareMathOperator{\rej}{reject }
\newcommand{\Rn}{{\mathbb{R}^{n}}}
\newcommand{\cR}{{\mathcal{R}}}
\newcommand{\Rad}{{\mathrm{Rad}}}
\newcommand{\ran}{{\mathrm{ran}}}
\newcommand{\Ri}{{\mathrm{R}}}
\newcommand{\supp}{{\mathrm{supp}}}
\newcommand{\Se}{\mathrm{S}}
\newcommand{\Sp}{S^{*}(\Rn)}
\newcommand{\St}{{\mathrm{St}}}
\newcommand{\Sw}{\mathcal{S}}
\newcommand{\T}{{\mathcal{T}}}
\newcommand{\ta}{{\theta}}
\newcommand{\Ta}{{\Theta}}
\newcommand{\topp}{\stackrel{p}{\to}}
\newcommand{\todd}{\stackrel{d}{\to}}
\newcommand{\toL}[1]{\stackrel{L^{#1}}{\to}} 
\newcommand{\toas}{\stackrel{a.s.}{\to}}
\DeclareMathOperator{\V}{Var}
\newcommand {\w}{{\omega}}
\newcommand {\W}{{\mathrm{W}}}
\newcommand {\Wnp}{\text{$\mathrm{W}$\textsuperscript{$n,\!p$}}}
\newcommand {\Wnpeq}{\text{$\mathrm{W}$\textsuperscript{$n\!,\!p$}}}
\newcommand {\Wonep}{\text{$\mathrm{W}$\textsuperscript{$1,\!p$}}}
\newcommand {\Wonepeq}{\text{$\mathrm{W}$\textsuperscript{$1\!,\!p$}}}
\newcommand {\X}{{\mathcal{X}}}
\newcommand {\Z}{{{\mathbb Z}}}
\newcommand {\Za}{{\mathcal{Z}}}
\newcommand {\Zd}{{\Z[\sqrt{d}]}}
\newcommand {\vanish}[1]{\relax}

\newcommand {\wh}{\widehat}
\newcommand {\wt}{\widetilde}
\newcommand {\red}{\color{red}}

% Distributions
\newcommand{\normal}{\mathsf{N}}
\newcommand{\poi}{\mathsf{Poisson}}
\newcommand{\bern}{\mathsf{Bernoulli}}
\newcommand{\bin}{\mathsf{Binomal}}
\newcommand{\multi}{\mathsf{Multinomial}}
\newcommand{\Exp}{\mathsf{Exp}}



% put your command and environment definitions here




% some theorem environments
% remove "[theorem]" if you do not want them to use the same number sequence


  \newtheorem{thrm}{Theorem}
  \newtheorem{lemma}{Lemma}
  \newtheorem{prop}{Proposition}
  \newtheorem{cor}{Corollary}

  \newtheorem{conj}{Conjecture}
  \renewcommand{\theconj}{\Alph{conj}}  % numbered A, B, C etc

  \theoremstyle{definition}
  \newtheorem{defn}{Definition}
  \newtheorem{ex}{Example}
  \newtheorem{exs}{Examples}
  \newtheorem{question}{Question}
  \newtheorem{remark}{Remark}
  \newtheorem{notn}{Notation}
  \newtheorem{exer}{Exercise}




\title{STATS310B -- Lecture 9}
\author{Sourav Chatterjee\\ Scribed by Michael Howes}
\date{02/01/22}

\pagestyle{fancy}
\fancyhf{}
\rhead{STATS310B -- Lecture }
\lhead{02/01/22}
\rfoot{Page \thepage}

\begin{document}
\maketitle
\tableofcontents
\section{Uniform integrability}
We ended last lecture with the statement of the following proposition,
\begin{proposition}\label{charactrized}
    Let $\{X_n\}_{n \ge 1}$ be a sequence of random variables. The sequence $\{X_n\}_{n \ge 0}$ is uniformly integrable if and only if the following both hold,
    \begin{enumerate}
        \item $\sup_n \E[\abs{X_n}] < \infty$.
        \item For all $\eps >0$ there exists $\delta > 0$ such that for all $A$ and $n$, if $\Pa(A) <\delta$, then $\E[\abs{X_n}\one_A] < \eps$.
    \end{enumerate}
\end{proposition}
Which we will now prove.
\begin{proof}
    First suppose that $\{X_n\}_{n \ge 1}$ is uniformly integrable. Fix any $\eps > 0$, then there exists a $k$ such that for all $n$,
    \[\E[\abs{X_n}\one_{\{\abs{X_n}>k\}}] < \eps. \]
    It follows that,
    \[\E[\abs{X_n}] =\E[\abs{X_n}\one_{\{\abs{X_n}\le k\}}]+\E[\abs{X_n}\one_{\{\abs{X_n}>k\}}]<k+\eps. \]
    Thus, $\sup_n \E[\abs{X_n}] < \infty$. Now let $\eps >0$ be arbitrary and fix $k$ so that for all $n$,
    \[\E[\abs{X_n}\one_{\{\abs{X_n}>k\}}] < \eps/2.\]
    Set $\delta = \frac{\epsilon}{2k}$ and suppose $\Pa(A)<\delta$. Then
    \begin{align*}
        \E[\abs{X_n}\one_A]&=\E[\abs{X_n}\one_{A\cap\{\abs{X_n}> k\}}]+\E[\abs{X_n}\one_{A\cap\{\abs{X_n}\le k\}}]\\
        &\le \E[\abs{X_n}\one_{\{\abs{X_n}>k\}}]+\E[k\one_A]\\
        &< \epsilon/2 + k\Pa(A)\\
        &<\epsilon/2+k\delta\\
        &=\epsilon.
    \end{align*}
    Now conversely suppose that the two conditions hold and note that, 
    \[\E[\abs{X_n}\one_{\{\abs{X_n}>k\}}] \ge \E[k\one_{\{\abs{X_n}>k\}}]=k\Pa(\abs{X_n}>k). \]
    Thus, for every $n$ and $k$, $\Pa(\abs{X_n} > k)\le \frac{\sup_m \E[\abs{X_m}]}{k}$. Thus, given $\epsilon >0$ choose $\delta$ so that $\Pa(A)<\delta$ implies that $\E[\abs{X_n}\one_A]<\epsilon$ for every $n$. If $k > \frac{\sup_m \E[\abs{X_m}]}{\delta}$, then $\Pa(\abs{X_n} > k) < \epsilon$ for all $n$ and hence
    \[\E[\abs{X_n}\one_{\{\abs{X_n}>k\}}] < \epsilon.\qedhere \]
\end{proof}
The above proposition has the following important corollary.
\begin{corollary}
    Let $X$ be an integrable random variable. Then for all $\epsilon > 0$, there exists $\delta >0$ such that if $\Pa(A)<\delta$, then $\E[X\one_A]<\epsilon$. 
\end{corollary}
\begin{proof}
    By proposition \eqref{charactrized} it suffices to show that the sequence $X_n=X$ is uniformly integrable. By the dominated convergence theorem
    \[\lim_{k \to \infty} \E[\abs{X}\one_{\{\abs{X}>k\}}] = 0, \]
    showing that for all $\epsilon$ there exists a $k$ such that $\E[\abs{X}\one_{\{\abs{X}>k\}}]<\epsilon$. 
\end{proof}
\section{L\'evy downwards convergence theorem}
With these results about uniform integrability we are ready to finish proving L'evy downwards convergence theorem. Recall the theorem's statement,
\begin{theorem}[L\'evy's downards convergence theorem]
    Let $X$ be an integrable random variable defined on $(\Om,\F,\Pa)$. Let $\F_0 \supseteq \F_1 \supseteq \F_2 \supseteq \ldots$ be a decreasing sequence of sub-$\sigma$-algebras of $\F$. Define $\F^* = \bigcap_{n=0}^\infty \F_n$, then 
    \[\E(X|\F_n) \to \E(X|\F^*), \]
    almost surely and in $L^1$.
\end{theorem}
\begin{proof}
    In the previous lecture we use the upcrossing lemma to show that there exists an integrable random variable $X^*$ such that if $X_n = \E(X|\F_n)$, then $X_n \to X^*$ almost surely. It remains to show that $X_n \to X^*$ in $L^1$ and that $X^*=\E(X|\F^*)$. We will first show that $\{X_n\}_{n \ge 0}$ is uniformly integrable. Thus, fix $\epsilon >0$ and let $\delta$ be such that $\Pa(A) < \delta$ implies that $\E[\abs{X}\one_A] < \epsilon$. Now choose $k$ so that $\frac{\E[\abs{X}]}{k} < \epsilon$. Let $A_n$ be the event $\{\E(\abs{X}|\F_n)>k\}$. Note that, by Markov's inequality,
    \begin{align*}
        \Pa(A_n) &\le \frac{\E[\E(\abs{X}|\F_n)]}{k}\\
        &=\frac{\E[\abs{X}]}{k}\\
        &<\delta.
    \end{align*}
    Thus, $\E[\abs{X}\one_{A_n}]<\epsilon$. Furthermore, since $A_n \in \F_n$ we have
    \[\E[\E(\abs{X}|\F_n)\one_{A_n}] = \E[\abs{X}\one_{A_n}] <\epsilon. \]
    By Jensen's inequality $\abs{X_n} = \abs{\E(X|\F_n)} \le \E(\abs{X}|\F_n)$. Thus,$\E[\abs{X_n}\one_{A_n}] <\epsilon.$
    Furthermore, since $\abs{X_n} < \E(\abs{X}|\F_n)$, we have $\{\abs{X_n}>k\} \subseteq A_n$ and so
    \[\E[\abs{X_n}\one_{\{\abs{X_n}>k\}}] \le \E[\abs{X_n}\one_{A_n}] <\epsilon. \]
    Thus, $\{X_n\}_{n \ge 0}$ are uniformly integrable. Since $X_n \to X^*$ almost surely, this implies that $X_n \to X^*$ in $L^1$. It remains to show that $X^* = \E(X|\F^*)$. First note that $X^*$ is $\F^*$ measurable. This is because for every $m$, $X_n = \E(X|\F_n)$ is $\F_m$ measurable for every $n \ge m$. Thus, $X^* = \lim_{n \ge m} X_n$ is $\F_m$ measurable. Since this holds for every $m$ we must have that $X^*$ is $\F^* = \bigcap_{m=1}^\infty \F_m$ measurable. Now suppose that $A \in \F^*$. Then $A \in \F_n$ for every $n$ and hence 
    \[\E[X_n\one_A]=\E[X\one_A], \]
    for every $m$. Furthermore, 
    \[\abs{\E[X_n\one_A]-\E[X^*\one_A]} \le \E[\abs{X_n-X^*}\one_A]\le \E[\abs{X_n-X^*}]. \]
    Since $X_n \to X^*$ in $L^1$, this implies that
    \[\E[X^*\one_A] = \lim_{n \to \infty} \E[X_n\one_A] = \lim_{n \to \infty}\E[X\one_A]=\E[X\one_A]. \]
    Thus, $X^*=\E(X|\F^*)$.
\end{proof}
We will now discuss and later prove an application of this convergence theorem.
\section{De Finetti's theorem}
\begin{definition}
    
Let $X_1,X_2,\ldots$ be an infinite sequence of random variables. The sequence $\{X_n\}_{n \ge 1}$ is \emph{exchangeable} if for any permutation $\pi : \N \to \N$ that fixes all but finitely many values $(X_{\pi(1)},X_{\pi(2)},\ldots)$ has the same distribution as $(X_1,X_2,\ldots)$.
\end{definition}
More precisely, we can think of the sequence $\{X_n\}_{n \ge 1}$ as a single random variable $X$ taking values in $\R^\N$ given by 
\[X(\w) = (X_1(\w), X_2(\w),\ldots).\]
The distribution of $X$ is the probability measure in induces on $(\R^\N,\F_{\R^\N})$ where $\F_{\R^\N}$ is the product $\sigma$-algebra of countably many copies of the Borel $\sigma$-algebra.  Thus, the law of $X$ is the probability measure $\mu_X$ given by
\[\mu_X(B) = \Pa(X^{-1}(B)). \]
For a permutation $\pi : \N \to \N$ we can define another random variable $X_\pi = (X_{\pi(1)},X_{\pi(2)},\ldots)$. The sequence $\{X_n\}_{n \ge 0}$ is exchangeable if and only if for every $\pi$ that fixes all but finitely many $i \in \N$, $\mu_{X_\pi}=\mu_X$. 

For example, if $\{X_n\}_{n \ge 0}$ is an exchangeable sequence, then
\[\Pa(X_2+X_3+X_5^2 > 20 \text{ and } X_2 < 5) = \Pa(X_1+X_3+X_{10000}^2 > 20\text{ and } X_1 < 5). \]
De Finetti's theorem is a theorem about exchangeable sequences. A special case of de Finetti's theorem concerns $\{0,1\}$-valued exchangeable sequences.
\begin{theorem}[de Finitti's theorem for coin tosses.]
    Let $\{X_n\}_{n \ge 0}$ be an exchangeable sequence of $\{0,1\}$ valued random variables. Then there exists a probability measure $\mu$ on $[0,1]$ such that for every $n$ and every choice of $x_1,\ldots,x_n \in \{0,1\}$,
    \[\Pa\left(X_1=x_1,\ldots, X_n = x_n\right) = \int_0^1 \ta^{\sum_{i=1}^n x_i}(1-\ta)^{n-\sum_{i=1}^n x_i} \mu(d\theta). \]
    Furthermore, the limit $\Ta = \lim_{n\to \infty} \frac{1}{n}\sum_{i=1}^n X_i$ exists almost surely and the random variable $\Ta$ is distributed according to $\Ta$.
\end{theorem}
Informally, conditional on the random variable $\Ta$, $X_1,\ldots,X_n$ are i.i.d. random variables with $\Pa(X_i=1|\Ta=\ta)=\ta$. 
\subsection{The exchangeable $\sigma$-algebra}
Let $X_1,X_2,\ldots$ be any sequence of random variables. Recall that the $\sigma$-algebra $\G = \sigma(X_1,X_2,\ldots)$ consists of all set of the form $X^{-1}(B)$ where $X=(X_1,X_2,\ldots)$ and $B$ is a measurable subset of $\R^\N$.
\begin{definition}
    A set $B \subseteq \R^\N$ is \emph{invariant under permutations of the first $n$ coordinates} if for all permutations $\pi$ of $\{1,\ldots,n\}$, for all $x=(x_1,x_2,\ldots) \in \R^\N$, if $x \in B$, then $(x_{\pi(1)},\ldots, x_{\pi(n)}, x_{n+1},\ldots) \in B$.
\end{definition}
For example $\{x_1+x_2+x_3 > 5\}$ is invariant under permutations of the first 3 coordinates. The set $\{x_1+x_2+x_3^2 > 5\}$ is invariant under permutations of the first 2 coordinates but not invariant under permutations of the first 2 coordinates.
\begin{definition}
    Let $\{X_n\}_{n\ge 0}$ be a sequence of random variables and let $\G = \sigma(X_1,X_2,\ldots)$. Define $\Ec_n$ be the $\sigma$-algebra of all $A \in \G$ such that $A = X^{-1}(B)$ for some $B \subseteq \R^\N$ that is invariant under permutations of the first $n$ coordinates. Also define $\Ec = \bigcap_{n=1}^\infty \Ec_n$. The $\sigma$-algebra $\Ec$ is called the \emph{exchangeable $\sigma$-algebra} of $\{X_n\}_{n \ge 0}$.
\end{definition}
We are now ready to state the general version of de Finetti's theorem.
\begin{theorem}[de Finetti's theorem]
    Let $\{X_n\}_{n \ge 0}$ be an exchangeable sequence of random variables and let $\Ec$ be the exchangable $\sigma$-algebra of $\{X_n\}_{n \ge 0}$. Then $\{X_n\}_{n \ge 0}$ are i.i.d. given $\Ec$ meaning that for every $n$ and every choice of Borel sets $A_1,\ldots,A_n$, we have
    \begin{align*}
        \Pa(X_1 \in A_1,\ldots,X_n \in A_n|\Ec) &=\prod_{i=1}^n \Pa(X_i \in A_i |\Ec)\\
        &=\prod_{i=1}^n \Pa(X_1 \in A|\Ec).
    \end{align*}
\end{theorem}
\end{document}
\documentclass{article}
\usepackage{ae,aecompl}
\usepackage{todonotes}
\usepackage{chngcntr}
\usepackage{tikz-cd}
\usepackage{graphicx}
\graphicspath{ {./images/}}
\usepackage[all,cmtip]{xy}
\usepackage{amsmath, amscd}
\usepackage{amsthm}
\usepackage{amssymb}
\usepackage{amsfonts}
\usepackage{bm}
\usepackage{qsymbols}
\usepackage{latexsym}
\usepackage{mathrsfs}
\usepackage{mathtools}
\usepackage{cite}
\usepackage{color}
\usepackage{url}
\usepackage{enumerate}
\usepackage{verbatim}
\usepackage[draft=false, colorlinks=true]{hyperref}
\usepackage{pdfpages}
\usepackage[margin=1.2in]{geometry}
\usepackage{IEEEtrantools}

\usepackage{fancyhdr}


\usepackage[nameinlink]{cleveref}


\DeclareMathOperator*{\ac}{accept}
\DeclareMathOperator*{\amax}{argmax}
\DeclareMathOperator*{\amin}{argmin}
\DeclareMathOperator*{\Aut}{Aut}
\newcommand {\al}{{\alpha}}
\newcommand {\abs}[1]{{\left\lvert#1\right\rvert}}
\newcommand {\A}{{\mathcal{A}}}
\newcommand {\AM}{{\mathrm{AM}}}
\newcommand {\AMp}{{\AM_{p}^{X}\!(\Ri_\w)}}
\newcommand {\B}{{\mathcal{B}}}
\DeclareMathOperator*{\Be}{Bern}
\newcommand {\Br}{{\dot{B}}}
\newcommand {\Ba}{{\mathfrak{B}}}
\newcommand {\C}{{\mathbb C}}
\newcommand {\ce}{\mathrm{c}}
\newcommand {\Ce}{\mathrm{C}}
\newcommand {\Cc}{\mathrm{C_{c}}}
\newcommand {\Ccinf}{\mathrm{C_{c}^{\infty}}}
\DeclareMathOperator{\cov}{Cov}
\DeclareMathOperator{\DEV}{DEV}
\newcommand {\Di}{{\mathbb D}}
\newcommand {\dom}{\mathrm{dom}}
\newcommand{\dist}{\stackrel{\mathrm{dist}}{=}}
\newcommand {\ud}{\mathrm{d}}
\newcommand {\ue}{\mathrm{e}}
\newcommand {\eps}{\varepsilon}
\newcommand {\veps}{\varepsilon}
\newcommand {\vrho}{{\varrho}}
\newcommand {\E}{{\mathbb{E}}}
\newcommand {\Ec}{{\mathcal{E}}}
\newcommand {\Ell}{L}
\newcommand {\Ellp}{{L_{p}[0,1]}}
\newcommand {\Ellpprime}{{L_{p'}([0,1])}}
\newcommand {\Ellq}{{L_{q}([0,1])}}
\newcommand {\Ellqprime}{{L_{q'}([0,1])}}
\newcommand {\Ellr}{L^{r}}
\newcommand {\Ellone}{{L_{1}([0,1])}}
\newcommand{\Elltwo}{{L_{2}([0,1])}}
\newcommand{\Ellinfty}{L^{\infty}}
\newcommand{\Ellinftyc}{L_{\mathrm{c}}^{\infty}}
\newcommand{\exb}[1]{\exp\left\{#1\right\}}
\DeclareMathOperator*{\Ext}{Ext}
\newcommand{\F}{{\mathcal{F}}}
\newcommand{\Fe}{{\mathbb{F}}}
\newcommand{\G}{{\mathcal{G}}}
\newcommand{\HF}{\mathcal{H}_{\text{FIO}}^{1}(\Rd)}
\newcommand{\Hr}{H}
\newcommand{\HT}{\mathcal{H}}
\newcommand{\ui}{\mathrm{i}}
\newcommand{\I}{{I}}
\newcommand{\J}{{\mathcal{J}}}
\newcommand{\id}{{\mathrm{id}}}
\newcommand{\iid}{\stackrel{\mathclap{\normalfont\mbox{iid}}}{\sim}}
\newcommand{\im}{{\text{im }}}
\newcommand{\ind}{{\perp\!\!\!\perp}}
\DeclareMathOperator*{\Int}{int}
\newcommand{\intx}{{\overline{\int_{X}}}}
\newcommand{\inte}{{\overline{\int_{\E}}}}
\newcommand{\la}{\lambda}
\newcommand{\rb}{\rangle}
\newcommand{\lb}{{\langle}}
\newcommand{\La}{\Lambda}
\newcommand{\calL}{{\mathcal{L}}}
\newcommand{\lp}{{\mathcal{L}}^{p}}
\newcommand{\lpo}{{\overline{\mathcal{L}}^{p}\!}}
\newcommand{\Lpo}{{\overline{\Ell}^{p}\!}}
\newcommand{\M}{{\mathbf{M}}}
\newcommand{\Ma}{{\mathcal{M}}}
\newcommand{\N}{{{\mathbb N}}}
\newcommand{\Na}{{{\mathcal{N}}}}
\newcommand{\norm}[1]{\left\|#1\right\|}
\newcommand{\normm}[1]{{\left\vert\kern-0.25ex\left\vert\kern-0.25ex\left\vert #1 
    \right\vert\kern-0.25ex\right\vert\kern-0.25ex\right\vert}}
\newcommand{\Om}{{{\Omega}}}
\newcommand{\one}{{{\bf 1}}}
\newcommand{\pic}{\text{Pic }}
\newcommand{\ph}{{\varphi}}
\newcommand{\Pa}{{\mathbb{P}}}
\newcommand{\Po}{{\mathcal{P}}}
\newcommand{\Q}{{\mathbb{Q}}}
\newcommand{\R}{{\mathbb R}}
\newcommand{\Rd}{{\mathbb{R}^{d}}}
\DeclareMathOperator{\rej}{reject }
\newcommand{\Rn}{{\mathbb{R}^{n}}}
\newcommand{\cR}{{\mathcal{R}}}
\newcommand{\Rad}{{\mathrm{Rad}}}
\newcommand{\ran}{{\mathrm{ran}}}
\newcommand{\Ri}{{\mathrm{R}}}
\newcommand{\supp}{{\mathrm{supp}}}
\newcommand{\Se}{\mathrm{S}}
\newcommand{\Sp}{S^{*}(\Rn)}
\newcommand{\St}{{\mathrm{St}}}
\newcommand{\Sw}{\mathcal{S}}
\newcommand{\T}{{\mathcal{T}}}
\newcommand{\ta}{{\theta}}
\newcommand{\Ta}{{\Theta}}
\newcommand{\topp}{\stackrel{p}{\to}}
\newcommand{\todd}{\stackrel{d}{\to}}
\newcommand{\toL}[1]{\stackrel{L^{#1}}{\to}} 
\newcommand{\toas}{\stackrel{a.s.}{\to}}
\DeclareMathOperator{\V}{Var}
\newcommand {\w}{{\omega}}
\newcommand {\W}{{\mathrm{W}}}
\newcommand {\Wnp}{\text{$\mathrm{W}$\textsuperscript{$n,\!p$}}}
\newcommand {\Wnpeq}{\text{$\mathrm{W}$\textsuperscript{$n\!,\!p$}}}
\newcommand {\Wonep}{\text{$\mathrm{W}$\textsuperscript{$1,\!p$}}}
\newcommand {\Wonepeq}{\text{$\mathrm{W}$\textsuperscript{$1\!,\!p$}}}
\newcommand {\X}{{\mathcal{X}}}
\newcommand {\Z}{{{\mathbb Z}}}
\newcommand {\Za}{{\mathcal{Z}}}
\newcommand {\Zd}{{\Z[\sqrt{d}]}}
\newcommand {\vanish}[1]{\relax}

\newcommand {\wh}{\widehat}
\newcommand {\wt}{\widetilde}
\newcommand {\red}{\color{red}}

% Distributions
\newcommand{\normal}{\mathsf{N}}
\newcommand{\poi}{\mathsf{Poisson}}
\newcommand{\bern}{\mathsf{Bernoulli}}
\newcommand{\bin}{\mathsf{Binomal}}
\newcommand{\multi}{\mathsf{Multinomial}}
\newcommand{\Exp}{\mathsf{Exp}}



% put your command and environment definitions here




% some theorem environments
% remove "[theorem]" if you do not want them to use the same number sequence


  \newtheorem{thrm}{Theorem}
  \newtheorem{lemma}{Lemma}
  \newtheorem{prop}{Proposition}
  \newtheorem{cor}{Corollary}

  \newtheorem{conj}{Conjecture}
  \renewcommand{\theconj}{\Alph{conj}}  % numbered A, B, C etc

  \theoremstyle{definition}
  \newtheorem{defn}{Definition}
  \newtheorem{ex}{Example}
  \newtheorem{exs}{Examples}
  \newtheorem{question}{Question}
  \newtheorem{remark}{Remark}
  \newtheorem{notn}{Notation}
  \newtheorem{exer}{Exercise}




\title{STATS310B -- Lecture 3}
\author{Sourav Chatterjee\\ Scribed by Michael Howes}
\date{01/11/22}

\pagestyle{fancy}
\fancyhf{}
\rhead{STATS310B -- Lecture 3}
\lhead{01/11/22}
\rfoot{Page \thepage}

\begin{document}
\maketitle
\tableofcontents
\section{Jensen's inequality}
Jensen's inequality is an important result about conditional expectations and convex functions. It states:
\begin{theorem}[Jensen's inquality]
    Let $X$ be an integrable random variable taking values in some interval $I$. Let $\phi : I \to \R$ be a convex function such that $\phi(X)$ is integrable. Then for any sub-$\sigma$-algebra $\G$, we have
    \[\phi(\E(X|\G)) \le \E(\phi(X)|\G), \]
    almost surely.
\end{theorem}
To prove Jensen's inequality, we will first prove some lemmas  about convex functions.
\begin{lemma}
    Let $\A$ be a collection of convex functions $f:I\to \R$. The function 
    \[g(x) = \sup_{f \in \A} f(x), \]
    is also convex.
\end{lemma}
\begin{proof}
    Let $x,y \in I$ and let $t \in [0,1]$. For each $f \in \A$ we have
    \[f(tx+(1-t)y) \le tf(x)+(1-t)f(y) \le tg(x)+(1-t)g(y). \]
    Thus,
    \[g(tx+(1-t)y) = \sup_{f \in \A} f(tx+(1-t)y) \le tg(x) + (1-t)g(y) \qedhere \]
\end{proof}
This lemma has the immediate corollary,
\begin{corollary}
    The supremum of a collection of affine functions is a convex function.
\end{corollary}
We will next see that above corollary has a converse.
\begin{lemma}
    Let $\phi :I \to \R$ be a convex function on some interval $I$. The there exists sequences $a_n,b_n \in \R$ such that 
    \[\phi(x) = \sup_{n \in \N}a_nx+b_n. \]
\end{lemma}
\begin{proof}
    We first will state and prove a fact. If $\phi$ is convex and $y \in I$, then there exists a line $x \mapsto a_yx+b_y$ such that $\phi(y) = a_yy+b_y$ and $\phi(x) \ge a_y x+b_y$ for all $x \in I$. That is, there exists a line through $(y,\phi(y))$ that sits below the graph of $\phi$. The justification is as follows. By convexity, the function $\frac{\phi(y+t)-\phi(y)}{t}$ is a decreasing function of $t$. Thus, the limits 
    \[c = \lim_{t \nearrow 0} \frac{\phi(y+t)-\phi(y)}{t}, ~~ \text{ and } ~~ d = \lim_{t \searrow 0} \frac{\phi(y+t)-\phi(y)}{t},\]
    both exist and $c \le d$. Thus, we can take $a_y$ to be any value in $[c,d]$ and let $b_y = \phi(y)-a_yy$. The line $x \mapsto a_yx+b_y$ sits below the graph $\phi$ by the convexity of $\phi$. 

    Now define $g:I \to \R$ by
    \[g(x) = \sup_{y \in \Q \cap I} a_yx+b_y. \]
    The function $g$ is the supremum of a countable collection of affine functions and thus $g$ is convex. In particular $g$ and $\phi$ are both continuous. Furthermore, $\phi(x)=g(x)$ on $\Q \cap I$ and so $\phi = g$ which completes the proof.
\end{proof}
We are now ready to prove Jensen's inequality.
\begin{proof}[Proof of Jensen's inequality]
    By Lemma 2, we have 
    \[\phi(x) = \sup_{n \in \N}a_nx+b_n. \]
    Thus, for every $n$,
    \[\E(\phi(X)|\G) \ge \E(a_n X + b_n|\G) =a_n\E(X|\G)+b_n, \]
    almost surely. One can show that $\E(X|\G) \in I$ almost surely since $I$ is an interval. Thus, by taking a supremum over $n$, we have
    \[\E(\phi(X)|\G) \ge \sup_{n \in \N} a_n\E(X|\G)+b_n = \phi(\E(X|\G)), \]
    almost surely.
\end{proof}
\section{Martingales}
\begin{definition}
    Let $(\Om,\F,\Pa)$ be a probability space. A \emph{filtration} is a monotone sequence of sub-$\sigma$-algebras,
    \[\F_0 \subseteq \F_1 \subseteq F_2 \subseteq \ldots \subseteq \F.\]
\end{definition}
\begin{example}
    If we have a sequence of random variables $(X_n)_{n=1}^\infty$, then the sequence $\F_n = \sigma(X_1,X_2,\ldots, X_n)$ is a filtration.
\end{example}
\begin{definition}
    A sequence of random variables $(X_n)_{n \ge 0}$ is \emph{adapted} to a filtration $(F_n)_{n \ge 0}$ if for every $n \ge 0$, $X_n$ is $\F_n$-measurable.
\end{definition}
\begin{example}
   Given any sequence $(X_n)_{n \ge 0}$. The sequence $(X_n)_{n \ge 0}$ is adapted to the filtration $\F_n = \sigma(X_0,X_1,\ldots,X_n)$. 
\end{example}
\begin{definition}
    Let $(\F_n)_{n \ge 0}$ be a filtration. A \emph{martingale} is a sequence of random variables $(X_n)_{n \ge 0}$ adapted to $(\F_n)_{n \ge 0}$, such that
    \begin{enumerate}
        \item For every $n$, $\E\abs{X_n} < \infty$.
        \item For every $n$, $\E(X_{n+1}|\F_n) = X_n$, almost surely.
    \end{enumerate}
\end{definition}
\begin{example}
    Let $X_1,X_2,\ldots$ be independent with $\E[X_i]=0$. Let $S_0=0$ and $S_n = \sum_{i=1}^nX_i$ for $n \ge 1$. Then $(S_n)_{n \ge 0}$ is a martingale with respect to the filtration $\F_n = \sigma(X_i, 1\le i \le n)$. Note that 
    \[\abs{S_n} \le \sum_{i=1}^n \abs{X_i}. \]
    Thus, $S_n$ is integrable for each $n$. The random variable $S_n$ is also $\F_n$ measurable. Furthermore,
    \begin{align*}
        \E(S_{n+1}|\F_n) &=\E(X_{n+1}+S_n|\F_n)\\
        &=\E(X_{n+1}|\F_n)+\E(S_n|\F_n)\\
        &=\E[X_{n+1}]+S_n\\
        &=S_n.
    \end{align*}
    More generally if $X_i$ is integrable with mean $\mu_i$, then $S_n = \sum_{i=1}^n X_i-\mu_i$ is a martingale.
\end{example}
\begin{example}
    Let $X_1,\ldots, X_n$ be independent with $\E\abs{X_i}^2 < \infty$. Suppose that $\E[X_i]=0$ and $\E[X_i^2]=\sigma^2$ for all $i$. Let,
    \[Z_n = \left(\sum_{i=1}^n X_i\right)^2-n\sigma^2 = S_n^2-n\sigma^2,\]
    and $Z_0=0$. Then $Z_n$ is a martingale with respect to $\F_n = \sigma(X_1,\ldots, X_n)$.

    Note first that $\E[\abs{Z_n}] \le \E[S_n^2]+n =2n$ and so each $Z_n$ is integrable. Also,
    \begin{align*}
        \E(Z_{n+1}|\F_n)&=\E((S_n+X_{n+1})^2|\F_n) - (n+1)\sigma^2\\
        &=\E(S_n^2|\F_n) + 2\E(S_nX_{n+1}|\F_n)+\E(X_{n+1}^2|\F_n)-(n+1)\sigma^2\\
        &=S_n^2+2S_n\E(X_{n+1}|\F_n)+\E(X_{n+1}^2|\F_n)-(n+1)\sigma^2\\
        &=S_n^2+2S_n\E[X_{n+1}]+\E[X_{n+1}^2]-(n+1)\sigma^2\\
        &=S_n^2-n\sigma^2.
    \end{align*}
\end{example}
\begin{example}
    Let $X_1,X_2,\ldots$ be i.i.d. random variables such that for some $\ta \in \R$, $m(\ta) =\E[e^{\ta X_i}] < \infty$. Then 
    \[M_n = \frac{e^{\ta S_n}}{m(\ta)^n},\]
    is a martingale with respect to $\F_n = \sigma(X_1,\ldots,X_n)$. Note that 
    \[\E[\abs{M_n}] = \frac{1}{m(\ta)^n}\E[e^{\ta S_n}]=\frac{1}{m(\ta)^n}\prod_{i=1}^n \E[e^{\ta X_i}]=1.\]
    Furthermore,
    \begin{align*}
        \E(M_{n+1}|\F_{n+1}) &=\E\left(\frac{e^{\ta X_{n+1}}}{m(\ta)}M_n|\F_n\right)\\
        &=M_n\E\left(\frac{e^{\ta X_{n+1}}}{m(\ta)}|\F_n\right)\\
        &=M_n\E\left[\frac{e^{\ta X_{n+1}}}{m(\ta)}\right]\\
        &=M_n.
    \end{align*}
\end{example}
\section{Stopping times}
\begin{definition}
    Let $\{F_n\}$ be a filtration. A \emph{stopping time} is a random variable $T$ such that $T$ takes values in $\{0,1,\ldots\} \cup\{\infty\}$ and for every $n \in \{0,1,\ldots\}$, then event $\{T = n\} \in \F_n$.
\end{definition}
For a filtration $\{F_n\}$, we can think of each $\sigma$-algebra $\F_n$ as the information available at time $n$. The condition $\F_n \subseteq \F_{n+1}$ means that we gain more information over a time. A stopping time is then a procedure where the decision to stop at time $n$ depends only on the information available at time $n$.
\begin{example}
    Let $\{X_n\}_{n \ge 0}$ be a sequence of random variables adapted to a filtration $\{\F_n\}_{n \ge 0}$. Take any Borel set $A\subseteq \R$, let $T= \inf \{n \ge 0 : X_n \in A\}$ where $\int \emptyset = \infty$. Then $T$ is a stopping time since 
    \[ \{T=n\} = \{X_n\in A\} \cap \left(\bigcap_{i=1}^{n-1} \{X_i \notin A\}\right)\in \F.\]
\end{example}
\begin{example}
    Let $\{S_n\}_{n \ge 0}$ be a simply symmetric random walk (SSRW). That is, $S_0 =0$ and $S_n = X_1+\ldots+X_n$ where $X_i$ are i.i.d. and $\Pa(X_i=1)=\Pa(X_i=-1)=\frac{1}{2}$. Take any integers $a< 0<b$ and define $T=\inf\{n : S_n = a \text{ or } S_n = b\}$. By the previous example, $T$ is a stopping time.
\end{example}
Related to stopping times, is the concept of a stopped $\sigma$-algebra.
\begin{definition}
    Given a filtration $\{\F_n\}_{n \ge 0}$ and a stopping time $T$. The \emph{stopped $\sigma$-algebra} is defined to be
    \[\F_T = \{A \in \F : A \cap \{T=n\} \in \F_n \text{ for all } n \}.\]
\end{definition}
\begin{remark}
    The stopped $\sigma$-algebra is indeed a $\sigma$-algebra. Since
    \begin{enumerate}
        \item $\emptyset \in \F_T$ trivially.
        \item If $A \in \F_T$, then $A^c \cap \{T=n\} = \{T=n\} \setminus (A \cap \{T=n\})$ and $\{T = n \} \in \F_n$ since $T$ is a stopping time.
        \item If $A_i \in \F_T$, then $\bigcup_{i=1}^\infty A_i \in \F_T$ since $\left(\bigcup_{i=1}^\infty A_i\right)\cap \{T=n\} = \bigcup_{i=1}^\infty A_i \cap \{T=n\}$.
    \end{enumerate}
\end{remark}
\begin{example}
    Consider the SSRW from the previous example with $T=\min\{n : S_n = a \text{ or } S_n =b\}$. The event $A = \{S_n \ge 0, \text{ for all } n \le T\}$ is in $\F_T$.
\end{example}
\end{document}
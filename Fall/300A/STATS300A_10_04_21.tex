\documentclass{article}
\usepackage{ae,aecompl}
\usepackage{todonotes}
\usepackage{chngcntr}
\usepackage{tikz-cd}
\usepackage{graphicx}
\graphicspath{ {./images/}}
\usepackage[all,cmtip]{xy}
\usepackage{amsmath, amscd}
\usepackage{amsthm}
\usepackage{amssymb}
\usepackage{amsfonts}
\usepackage{bm}
\usepackage{qsymbols}
\usepackage{latexsym}
\usepackage{mathrsfs}
\usepackage{mathtools}
\usepackage{cite}
\usepackage{color}
\usepackage{url}
\usepackage{enumerate}
\usepackage{verbatim}
\usepackage[draft=false, colorlinks=true]{hyperref}
\usepackage{pdfpages}
\usepackage[margin=1.2in]{geometry}
\usepackage{IEEEtrantools}

\usepackage{fancyhdr}


\usepackage[nameinlink]{cleveref}


\DeclareMathOperator*{\ac}{accept}
\DeclareMathOperator*{\amax}{argmax}
\DeclareMathOperator*{\amin}{argmin}
\DeclareMathOperator*{\Aut}{Aut}
\newcommand {\al}{{\alpha}}
\newcommand {\abs}[1]{{\left\lvert#1\right\rvert}}
\newcommand {\A}{{\mathcal{A}}}
\newcommand {\AM}{{\mathrm{AM}}}
\newcommand {\AMp}{{\AM_{p}^{X}\!(\Ri_\w)}}
\newcommand {\B}{{\mathcal{B}}}
\DeclareMathOperator*{\Be}{Bern}
\newcommand {\Br}{{\dot{B}}}
\newcommand {\Ba}{{\mathfrak{B}}}
\newcommand {\C}{{\mathbb C}}
\newcommand {\ce}{\mathrm{c}}
\newcommand {\Ce}{\mathrm{C}}
\newcommand {\Cc}{\mathrm{C_{c}}}
\newcommand {\Ccinf}{\mathrm{C_{c}^{\infty}}}
\DeclareMathOperator{\cov}{Cov}
\DeclareMathOperator{\DEV}{DEV}
\newcommand {\Di}{{\mathbb D}}
\newcommand {\dom}{\mathrm{dom}}
\newcommand{\dist}{\stackrel{\mathrm{dist}}{=}}
\newcommand {\ud}{\mathrm{d}}
\newcommand {\ue}{\mathrm{e}}
\newcommand {\eps}{\varepsilon}
\newcommand {\veps}{\varepsilon}
\newcommand {\vrho}{{\varrho}}
\newcommand {\E}{{\mathbb{E}}}
\newcommand {\Ec}{{\mathcal{E}}}
\newcommand {\Ell}{L}
\newcommand {\Ellp}{{L_{p}[0,1]}}
\newcommand {\Ellpprime}{{L_{p'}([0,1])}}
\newcommand {\Ellq}{{L_{q}([0,1])}}
\newcommand {\Ellqprime}{{L_{q'}([0,1])}}
\newcommand {\Ellr}{L^{r}}
\newcommand {\Ellone}{{L_{1}([0,1])}}
\newcommand{\Elltwo}{{L_{2}([0,1])}}
\newcommand{\Ellinfty}{L^{\infty}}
\newcommand{\Ellinftyc}{L_{\mathrm{c}}^{\infty}}
\newcommand{\exb}[1]{\exp\left\{#1\right\}}
\DeclareMathOperator*{\Ext}{Ext}
\newcommand{\F}{{\mathcal{F}}}
\newcommand{\Fe}{{\mathbb{F}}}
\newcommand{\G}{{\mathcal{G}}}
\newcommand{\HF}{\mathcal{H}_{\text{FIO}}^{1}(\Rd)}
\newcommand{\Hr}{H}
\newcommand{\HT}{\mathcal{H}}
\newcommand{\ui}{\mathrm{i}}
\newcommand{\I}{{I}}
\newcommand{\J}{{\mathcal{J}}}
\newcommand{\id}{{\mathrm{id}}}
\newcommand{\iid}{\stackrel{\mathclap{\normalfont\mbox{iid}}}{\sim}}
\newcommand{\im}{{\text{im }}}
\newcommand{\ind}{{\perp\!\!\!\perp}}
\DeclareMathOperator*{\Int}{int}
\newcommand{\intx}{{\overline{\int_{X}}}}
\newcommand{\inte}{{\overline{\int_{\E}}}}
\newcommand{\la}{\lambda}
\newcommand{\rb}{\rangle}
\newcommand{\lb}{{\langle}}
\newcommand{\La}{\Lambda}
\newcommand{\calL}{{\mathcal{L}}}
\newcommand{\lp}{{\mathcal{L}}^{p}}
\newcommand{\lpo}{{\overline{\mathcal{L}}^{p}\!}}
\newcommand{\Lpo}{{\overline{\Ell}^{p}\!}}
\newcommand{\M}{{\mathbf{M}}}
\newcommand{\Ma}{{\mathcal{M}}}
\newcommand{\N}{{{\mathbb N}}}
\newcommand{\Na}{{{\mathcal{N}}}}
\newcommand{\norm}[1]{\left\|#1\right\|}
\newcommand{\normm}[1]{{\left\vert\kern-0.25ex\left\vert\kern-0.25ex\left\vert #1 
    \right\vert\kern-0.25ex\right\vert\kern-0.25ex\right\vert}}
\newcommand{\Om}{{{\Omega}}}
\newcommand{\one}{{{\bf 1}}}
\newcommand{\pic}{\text{Pic }}
\newcommand{\ph}{{\varphi}}
\newcommand{\Pa}{{\mathbb{P}}}
\newcommand{\Po}{{\mathcal{P}}}
\newcommand{\Q}{{\mathbb{Q}}}
\newcommand{\R}{{\mathbb R}}
\newcommand{\Rd}{{\mathbb{R}^{d}}}
\DeclareMathOperator{\rej}{reject }
\newcommand{\Rn}{{\mathbb{R}^{n}}}
\newcommand{\cR}{{\mathcal{R}}}
\newcommand{\Rad}{{\mathrm{Rad}}}
\newcommand{\ran}{{\mathrm{ran}}}
\newcommand{\Ri}{{\mathrm{R}}}
\newcommand{\supp}{{\mathrm{supp}}}
\newcommand{\Se}{\mathrm{S}}
\newcommand{\Sp}{S^{*}(\Rn)}
\newcommand{\St}{{\mathrm{St}}}
\newcommand{\Sw}{\mathcal{S}}
\newcommand{\T}{{\mathcal{T}}}
\newcommand{\ta}{{\theta}}
\newcommand{\Ta}{{\Theta}}
\newcommand{\topp}{\stackrel{p}{\to}}
\newcommand{\todd}{\stackrel{d}{\to}}
\newcommand{\toL}[1]{\stackrel{L^{#1}}{\to}} 
\newcommand{\toas}{\stackrel{a.s.}{\to}}
\DeclareMathOperator{\V}{Var}
\newcommand {\w}{{\omega}}
\newcommand {\W}{{\mathrm{W}}}
\newcommand {\Wnp}{\text{$\mathrm{W}$\textsuperscript{$n,\!p$}}}
\newcommand {\Wnpeq}{\text{$\mathrm{W}$\textsuperscript{$n\!,\!p$}}}
\newcommand {\Wonep}{\text{$\mathrm{W}$\textsuperscript{$1,\!p$}}}
\newcommand {\Wonepeq}{\text{$\mathrm{W}$\textsuperscript{$1\!,\!p$}}}
\newcommand {\X}{{\mathcal{X}}}
\newcommand {\Z}{{{\mathbb Z}}}
\newcommand {\Za}{{\mathcal{Z}}}
\newcommand {\Zd}{{\Z[\sqrt{d}]}}
\newcommand {\vanish}[1]{\relax}

\newcommand {\wh}{\widehat}
\newcommand {\wt}{\widetilde}
\newcommand {\red}{\color{red}}

% Distributions
\newcommand{\normal}{\mathsf{N}}
\newcommand{\poi}{\mathsf{Poisson}}
\newcommand{\bern}{\mathsf{Bernoulli}}
\newcommand{\bin}{\mathsf{Binomal}}
\newcommand{\multi}{\mathsf{Multinomial}}
\newcommand{\Exp}{\mathsf{Exp}}



% put your command and environment definitions here




% some theorem environments
% remove "[theorem]" if you do not want them to use the same number sequence


  \newtheorem{thrm}{Theorem}
  \newtheorem{lemma}{Lemma}
  \newtheorem{prop}{Proposition}
  \newtheorem{cor}{Corollary}

  \newtheorem{conj}{Conjecture}
  \renewcommand{\theconj}{\Alph{conj}}  % numbered A, B, C etc

  \theoremstyle{definition}
  \newtheorem{defn}{Definition}
  \newtheorem{ex}{Example}
  \newtheorem{exs}{Examples}
  \newtheorem{question}{Question}
  \newtheorem{remark}{Remark}
  \newtheorem{notn}{Notation}
  \newtheorem{exer}{Exercise}




\title{STATS300A - Lecture 5}
\author{Dominik Rothenhaeusler\\ Scribed by Michael Howes}
\date{10/04/21}

\pagestyle{fancy}
\fancyhf{}
\rhead{STATS300A - Lecture 5}
\lhead{10/04/21}

\begin{document}
\maketitle
\tableofcontents
\section{Recap}
We have the following techniques for finding optimal estimators.

\begin{enumerate}
    \item Conditioning and using Rao-Blackwellisation.
    \item Solving $\E_\ta \delta(T) = g(\ta)$ for $\delta$.
    \item Guessing.
    \item Orthogonality constraints (to be discussed today).
\end{enumerate}

\section{An example from semi-parametrisation}
Semi-parametrisation refers to the set up where $\theta$ is a finite dimensional parameter of interest but our set of measures $\Po$ is infinite dimensional. The following example is semi-parametric.

$X_1, \ldots, X_n \iid F \in \F$ where $\F$ is the collection all cdfs which are symmetric around some $\ta \in \R$ and have finite second moment. The parameter $\ta$ is a function of $F \in \F$ and $\ta = \E_F[X_i]$. We wish to estimate $\ta$ from $X_1,\ldots,X_n$. One can ask, does a UMVUE exist for $\ta$? Suppose one does and call it $T$.
\begin{enumerate}
    \item Consider the submodel $\{N(\ta,1):\ta \in \R\}$, then we know that $\bar{X}_n$ is the UMVUE for $\ta$. This estimator is also unbiased on the full model $\F$.
    \item The risk of $T$ and $\bar{X}_n$ must be equal on the submodel since the are both UMVUE on the submodel. 
    \item Since $\bar{X}_n$ is the unique UMVUE on the submodel we must have $T = \bar{X}_n$.
    \item Repeat (a)-(c) for the new submodel $\{\text{Unif}[\ta-1,\ta + 1]:\ta \in \R\}$. The UMVUE for this model is also unique by completeness and it does not equal $\bar{X}_n$ (see homework for a calculation of this estimator). This estimator is again unbiased on the whole model.
    \item This gives us a contradiction since $T$ cannot be equal to the two different UMVUEs.
\end{enumerate}

\section{Orthogonality}
Suppose $\delta_i$ is a UMVUE for $g_i(\ta)$. Can we conclude that $\sum_i \delta_i$ is UMVUE for $\sum_i g_i(\ta)$?

\begin{defn}
    Define the set $\Delta$ as follows $\Delta = \{\delta(X) : \E_\ta(\delta(X)^2) < \infty, \text{ for all } \ta \}$.
\end{defn}

\begin{thrm}
    \emph{[TPE 2.17]}
    $\delta_0 \in \Delta$ is the UMVUE for $g(\ta) = \E_\ta \delta_0(X)$ if and only if $\E_\ta \delta_0(X)U = 0$ for all $\ta$ and all $U \in \Delta$ such that $\E_\ta U = 0$.
\end{thrm}
\begin{proof}
    See scribed notes.
\end{proof}
We can now answer our question with a yes! If each $\delta_i$ is the UMVUE for $g_i(\ta)$, then $\E_\ta[\delta_i(X)U] = 0$ for all first order ancillary $U$. Furthermore $\E_\ta[\sum_i \delta_i(X)] =\sum_i g_i(\ta)$ and \[\E_\ta[\sum_i \delta_i(X)U] = \sum_i \E_\ta [\delta_i(X)U]=0.\] Thus $\sum_i \delta_i(X)$ is the UMVUE for $\sum_i g_i(\ta)$.

\section{Cramer-Rao lower bound (CRLB)}
\begin{defn}
    We define the \emph{log likelihood} of a density $p(x;\ta)$ to be 
    \[l(x;\ta) = \log (x;\ta). \]
    For this definition we require $p(x;\ta) > 0$ for all $x$ and $\ta$. We also define the \emph{score} or \emph{score function} to be 
    \[S(x,\ta) = \partial_\ta l(x;\ta). \]
\end{defn}
Note that 
\[p(x;\ta_0 + \varepsilon) = p(x;\ta_0)\exb{\varepsilon S(x,\ta_0)+ o(\varepsilon)}. \]
Thus $p(x; \ta_0+\varepsilon)$ ``looks like'' an exponential family with parameter $\varepsilon$ and sufficient statistic $S(x,\ta_0)$.
\begin{thrm}
    \emph{[CRLB - Keener Thrm 4.9]} Let $p(x;\ta)$ be densities with $p(x;\ta) >0$ for all $x, \ta$ and such that $p(x;\ta)$ is differentiable in $\ta$. Suppose furthermore that for some function $g$

    \begin{enumerate}
        \item $\E_\ta[S(X,\ta)] = 0$.
        \item $\E_\ta[S(X,\ta)\delta(X)] = g'(\ta)$.
    \end{enumerate}
    Then 
    \[\V_\ta(\delta) \ge \frac{g'(\ta)^2}{I(\ta)},\]
    where $I(\ta)$ is equal to 
    \[I(\ta) = \E_\ta[S(X,\ta)^2], \]
    and called the \emph{Fisher information}.
\end{thrm}
Some remarks on our two conditions. If $\delta(X)$ is unbiased for $g(\ta)$, then under some regularity conditions
\begin{IEEEeqnarray*}{rCl}
    g'(\ta)&=&\frac{d}{d\ta}\E_\ta[\delta(X)]\\
    &=&\frac{d}{d\ta} \int p(x;\ta)\delta(x)d\mu(x)\\
    &=&\int \frac{d}{d\ta} p(x;ta)\delta(x)d\mu(x)\\
    &=&\int \frac{\frac{d}{d\ta}p(x;\ta)}{p(x;\ta)} \delta(x)p(x;\ta)d\mu(x)\\
    &=&\int S(x,\ta)\delta(x)p(x;\ta)d\mu(x)\\
    &=&\E_\ta[S(X,\ta)\delta(X)].
\end{IEEEeqnarray*}
Thus condition (b) is equivalent to regularity plus unbiased. Condition (a) is equivalent to a regularity condition on $p(x;\ta)$ and can be seen by taking $\delta(X) = 1$ and applying what we have done above. 

We will now prove the CRLB.

\begin{proof}
    By Cauchy-Schwarz
    \begin{IEEEeqnarray*}{rCl}
        \abs{g'(\ta)}&=&\abs{\E_\ta[\delta(X)S(X;\ta)]}\\
        &=&\abs{\text{Cov}_\ta(S(X;\ta),\delta(X))} \quad \text{since}\quad \E_\ta[S(X;\ta)]=0.\\
        &\le&\sqrt{\text{Var}_\ta(S(X;\ta))\text{Var}_\ta(\delta(X))}.
    \end{IEEEeqnarray*}
    Squaring and dividing by $I(\ta) = \text{Var}_\ta(S(X;\ta))$ gives $\text{Var}_{\ta}(\delta(X)) \ge \frac{g'(\ta)^2}{I(\ta)}$.
\end{proof}
Another remark, if $\int \partial_\ta^2 p(x;\ta)d\mu(x) = \partial_\ta^2 \int p(x;\ta) d\mu(x) =0$, then
\[I(\ta) = - \E_\ta[\partial_\ta^2 l(x;\ta)]. \]
Thus we can think of $I(\ta)$ as a measure of curvature. Consider two cases
\begin{enumerate}
    \item Small changes in $\ta$ result in large changes in $l(x;\ta)$ (high Fisher information).
    \item Small changes in $\ta$ result in small changes in $l(x;\ta)$ (low Fisher information).
\end{enumerate}
We want (a) when we are making inferences about $\ta$. Small changes in $\ta$ will result in large changes in the distribution of our data. Thus we can make precise statements about $\ta$ based on our data. 

\begin{ex}
    Suppose $X_1,\ldots, X_n \iid p(x;\ta)$. Then $p(x;\ta) = \prod_{i=1}^n p(x_i;\ta)$ and so $S(x_1,\ldots,x_n;\ta) = \sum_{i=1}^n S(x_i;\ta)$ and furthermore, by our iid assumption,
    \[I_n(\ta) = \text{Var}_\ta(S(X_1,\ldots, X_n);\ta) = n\text{Var}_\ta(S(X_1);\ta) = nI(\ta).\]
    This is one indiciation for why lower bounds scale at a rate of $\frac{1}{n}$ (under our regularity assumptions).
\end{ex}
There is another example in the scribed notes that relates to a Gaussian model.
\section{Equivariance}

We are done with unbiasedness and now we will look at restricting our estimators to respect certain symmetries. Consider the location model $X_1,\ldots,X_n \sim f_\ta(x)$ where $f$ is a known pdf, $\ta \in \R$ is unknown and $f_\ta(x) = f(x_1-\ta,\ldots, x_n - \ta)$. A special case of this is when $X_i$ are iid and thus $f_\ta(x) = \prod_{i=1}^n g(x_i-\ta)$ for some $g$.

\begin{defn}
    A model is called \emph{location invariant} if 
    \[f_{\ta+c}(x+c) = f_\ta(x),\]
    for all $\ta,x$ and $c$. 
\end{defn}
\begin{defn}
    A loss function is called \emph{location invariant} if 
    \[L(\ta+c,d+c) = L(\ta,d), \]
    for all $\ta,d$ and $c$. 
\end{defn}
Note that squared error loss $L(\ta,d)= (\ta-d)^2$ is location invariant as is any other loss that is a function of $\ta - d$. In fact these are the only location invariant losses. Since if $L$ is location in variant then $L(\ta,d) = L(\ta -d,0) =: \rho(\ta - d)$.\
\begin{defn}
    A decision problem is \emph{location invariant} if the model and the loss function are both location invariant.
\end{defn}
\begin{defn}
    An estimator $\delta$ is \emph{location equivariant} if 
    \[\delta(X_1+c,\ldots,X_n+c) = \delta(X)+c. \]
\end{defn}
The sample mean, sample median and sample quartiles are all examples of location equivariant estimators.
\begin{thrm}
    \emph{[TPE 3.1.4]} If $\delta$ is a location equivariant estimator for a location invariant decision problem, then the risk, variance and bias of $\delta$ all are constant as functions of $\ta$.
\end{thrm}
\begin{proof}
    We will prove that the risk is constant.
    \begin{IEEEeqnarray*}{rCl}
        R(\ta, \delta) &=& \E_\ta[L(\ta,\delta(X))]\\
        &=&\E_\ta[L(0,\delta(X)-\ta)]\\
        &=&\E_\ta[\rho(\delta(X)-\ta)]\\
        &=&\int \rho(\delta(x)-\ta)p(x;\ta)d\mu(x)\\
        &=&\int \rho(\delta(x_1-\ta,\ldots,x_n-\ta))p(x;\ta)d\mu(x)\\
        &=&\int \rho(\delta(x_1-\ta,\ldots,x_n-\ta))p(x_1-\ta,\ldots,x_n-\ta;0)d\mu(x)\\
        &=&\int \rho(\delta(x))p(x;0)d\mu(x)\\
        &=&\E_0[\rho(\delta(X))]\\
        &=&R(0,\delta).
    \end{IEEEeqnarray*}
    which does not depend on $\ta$. The bias and variance are similiar.
\end{proof}
The upshot of this theorem is that we can always compare equivariant estimators. We have restricted our class of estimators in such a way so that the risk is just a number. It is no longer a function of $\ta$.
\end{document}
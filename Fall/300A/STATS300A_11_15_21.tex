\documentclass{article}
\usepackage{ae,aecompl}
\usepackage{todonotes}
\usepackage{chngcntr}
\usepackage{tikz-cd}
\usepackage{graphicx}
\graphicspath{ {./images/}}
\usepackage[all,cmtip]{xy}
\usepackage{amsmath, amscd}
\usepackage{amsthm}
\usepackage{amssymb}
\usepackage{amsfonts}
\usepackage{bm}
\usepackage{qsymbols}
\usepackage{latexsym}
\usepackage{mathrsfs}
\usepackage{mathtools}
\usepackage{cite}
\usepackage{color}
\usepackage{url}
\usepackage{enumerate}
\usepackage{verbatim}
\usepackage[draft=false, colorlinks=true]{hyperref}
\usepackage{pdfpages}
\usepackage[margin=1.2in]{geometry}
\usepackage{IEEEtrantools}

\usepackage{fancyhdr}


\usepackage[nameinlink]{cleveref}


\DeclareMathOperator*{\ac}{accept}
\DeclareMathOperator*{\amax}{argmax}
\DeclareMathOperator*{\amin}{argmin}
\DeclareMathOperator*{\Aut}{Aut}
\newcommand {\al}{{\alpha}}
\newcommand {\abs}[1]{{\left\lvert#1\right\rvert}}
\newcommand {\A}{{\mathcal{A}}}
\newcommand {\AM}{{\mathrm{AM}}}
\newcommand {\AMp}{{\AM_{p}^{X}\!(\Ri_\w)}}
\newcommand {\B}{{\mathcal{B}}}
\DeclareMathOperator*{\Be}{Bern}
\newcommand {\Br}{{\dot{B}}}
\newcommand {\Ba}{{\mathfrak{B}}}
\newcommand {\C}{{\mathbb C}}
\newcommand {\ce}{\mathrm{c}}
\newcommand {\Ce}{\mathrm{C}}
\newcommand {\Cc}{\mathrm{C_{c}}}
\newcommand {\Ccinf}{\mathrm{C_{c}^{\infty}}}
\DeclareMathOperator{\cov}{Cov}
\DeclareMathOperator{\DEV}{DEV}
\newcommand {\Di}{{\mathbb D}}
\newcommand {\dom}{\mathrm{dom}}
\newcommand{\dist}{\stackrel{\mathrm{dist}}{=}}
\newcommand {\ud}{\mathrm{d}}
\newcommand {\ue}{\mathrm{e}}
\newcommand {\eps}{\varepsilon}
\newcommand {\veps}{\varepsilon}
\newcommand {\vrho}{{\varrho}}
\newcommand {\E}{{\mathbb{E}}}
\newcommand {\Ec}{{\mathcal{E}}}
\newcommand {\Ell}{L}
\newcommand {\Ellp}{{L_{p}[0,1]}}
\newcommand {\Ellpprime}{{L_{p'}([0,1])}}
\newcommand {\Ellq}{{L_{q}([0,1])}}
\newcommand {\Ellqprime}{{L_{q'}([0,1])}}
\newcommand {\Ellr}{L^{r}}
\newcommand {\Ellone}{{L_{1}([0,1])}}
\newcommand{\Elltwo}{{L_{2}([0,1])}}
\newcommand{\Ellinfty}{L^{\infty}}
\newcommand{\Ellinftyc}{L_{\mathrm{c}}^{\infty}}
\newcommand{\exb}[1]{\exp\left\{#1\right\}}
\DeclareMathOperator*{\Ext}{Ext}
\newcommand{\F}{{\mathcal{F}}}
\newcommand{\Fe}{{\mathbb{F}}}
\newcommand{\G}{{\mathcal{G}}}
\newcommand{\HF}{\mathcal{H}_{\text{FIO}}^{1}(\Rd)}
\newcommand{\Hr}{H}
\newcommand{\HT}{\mathcal{H}}
\newcommand{\ui}{\mathrm{i}}
\newcommand{\I}{{I}}
\newcommand{\J}{{\mathcal{J}}}
\newcommand{\id}{{\mathrm{id}}}
\newcommand{\iid}{\stackrel{\mathclap{\normalfont\mbox{iid}}}{\sim}}
\newcommand{\im}{{\text{im }}}
\newcommand{\ind}{{\perp\!\!\!\perp}}
\DeclareMathOperator*{\Int}{int}
\newcommand{\intx}{{\overline{\int_{X}}}}
\newcommand{\inte}{{\overline{\int_{\E}}}}
\newcommand{\la}{\lambda}
\newcommand{\rb}{\rangle}
\newcommand{\lb}{{\langle}}
\newcommand{\La}{\Lambda}
\newcommand{\calL}{{\mathcal{L}}}
\newcommand{\lp}{{\mathcal{L}}^{p}}
\newcommand{\lpo}{{\overline{\mathcal{L}}^{p}\!}}
\newcommand{\Lpo}{{\overline{\Ell}^{p}\!}}
\newcommand{\M}{{\mathbf{M}}}
\newcommand{\Ma}{{\mathcal{M}}}
\newcommand{\N}{{{\mathbb N}}}
\newcommand{\Na}{{{\mathcal{N}}}}
\newcommand{\norm}[1]{\left\|#1\right\|}
\newcommand{\normm}[1]{{\left\vert\kern-0.25ex\left\vert\kern-0.25ex\left\vert #1 
    \right\vert\kern-0.25ex\right\vert\kern-0.25ex\right\vert}}
\newcommand{\Om}{{{\Omega}}}
\newcommand{\one}{{{\bf 1}}}
\newcommand{\pic}{\text{Pic }}
\newcommand{\ph}{{\varphi}}
\newcommand{\Pa}{{\mathbb{P}}}
\newcommand{\Po}{{\mathcal{P}}}
\newcommand{\Q}{{\mathbb{Q}}}
\newcommand{\R}{{\mathbb R}}
\newcommand{\Rd}{{\mathbb{R}^{d}}}
\DeclareMathOperator{\rej}{reject }
\newcommand{\Rn}{{\mathbb{R}^{n}}}
\newcommand{\cR}{{\mathcal{R}}}
\newcommand{\Rad}{{\mathrm{Rad}}}
\newcommand{\ran}{{\mathrm{ran}}}
\newcommand{\Ri}{{\mathrm{R}}}
\newcommand{\supp}{{\mathrm{supp}}}
\newcommand{\Se}{\mathrm{S}}
\newcommand{\Sp}{S^{*}(\Rn)}
\newcommand{\St}{{\mathrm{St}}}
\newcommand{\Sw}{\mathcal{S}}
\newcommand{\T}{{\mathcal{T}}}
\newcommand{\ta}{{\theta}}
\newcommand{\Ta}{{\Theta}}
\newcommand{\topp}{\stackrel{p}{\to}}
\newcommand{\todd}{\stackrel{d}{\to}}
\newcommand{\toL}[1]{\stackrel{L^{#1}}{\to}} 
\newcommand{\toas}{\stackrel{a.s.}{\to}}
\DeclareMathOperator{\V}{Var}
\newcommand {\w}{{\omega}}
\newcommand {\W}{{\mathrm{W}}}
\newcommand {\Wnp}{\text{$\mathrm{W}$\textsuperscript{$n,\!p$}}}
\newcommand {\Wnpeq}{\text{$\mathrm{W}$\textsuperscript{$n\!,\!p$}}}
\newcommand {\Wonep}{\text{$\mathrm{W}$\textsuperscript{$1,\!p$}}}
\newcommand {\Wonepeq}{\text{$\mathrm{W}$\textsuperscript{$1\!,\!p$}}}
\newcommand {\X}{{\mathcal{X}}}
\newcommand {\Z}{{{\mathbb Z}}}
\newcommand {\Za}{{\mathcal{Z}}}
\newcommand {\Zd}{{\Z[\sqrt{d}]}}
\newcommand {\vanish}[1]{\relax}

\newcommand {\wh}{\widehat}
\newcommand {\wt}{\widetilde}
\newcommand {\red}{\color{red}}

% Distributions
\newcommand{\normal}{\mathsf{N}}
\newcommand{\poi}{\mathsf{Poisson}}
\newcommand{\bern}{\mathsf{Bernoulli}}
\newcommand{\bin}{\mathsf{Binomal}}
\newcommand{\multi}{\mathsf{Multinomial}}
\newcommand{\Exp}{\mathsf{Exp}}



% put your command and environment definitions here




% some theorem environments
% remove "[theorem]" if you do not want them to use the same number sequence


  \newtheorem{thrm}{Theorem}
  \newtheorem{lemma}{Lemma}
  \newtheorem{prop}{Proposition}
  \newtheorem{cor}{Corollary}

  \newtheorem{conj}{Conjecture}
  \renewcommand{\theconj}{\Alph{conj}}  % numbered A, B, C etc

  \theoremstyle{definition}
  \newtheorem{defn}{Definition}
  \newtheorem{ex}{Example}
  \newtheorem{exs}{Examples}
  \newtheorem{question}{Question}
  \newtheorem{remark}{Remark}
  \newtheorem{notn}{Notation}
  \newtheorem{exer}{Exercise}


\newcommand{\poi}{\text{Pois}}


\title{STATS300A - Lecture 16}
\author{Dominik Rothenhaeusler\\ Scribed by Michael Howes}
\date{11/15/21}

\pagestyle{fancy}
\fancyhf{}
\rhead{STATS300A - Lecture 16}
\lhead{11/15/21}
\rfoot{Page \thepage}

\begin{document}
\maketitle
\tableofcontents
\section{Recap}
Our current goal is to find uniformally most powerful unbiased (UMPU) tests for testing $H_0 : \ta \in \Om_0$ against $H_1 : \ta \in \Om_1$. Recall that a test function $\phi$ is unbiased at level $\al$ if 
\[\E_{\ta_0}\phi \le \al \text{ for all } \ta_0 \in \Om_0,\]
and 
\[\E_{\ta_1}\phi \ge \al \text{ for all } \ta_1 \in \Om_1. \]
We also say a test $\phi$ was $\al$-similar if for all $\ta \in W$ where $W = \overline{\Om}_0 \cap \overline{\Om}_1$. We previously proved the following theorem which relates unbiased and $\al$-similar tests.
\begin{thrm}[TSH 4.11]\label{al_similar}
    If $\ta \mapsto \E_\ta \phi$ is continuous for all tests $\phi$ and $\phi$ is uniformly most powerful among level $\al$ $\al$-similar tests, then $\phi$ is UMPU at level $\al$. 
\end{thrm}
Today we will find optimal unbiased tests in multiparameter exponential families. Specifically we will derive optimal one sided tests in the presence of nuisance parameters.
\section{Multiparameter exponential families}
Suppose we have a model $\{P_{\gamma}\}$ where $\gamma = (\ta,\la) \in \R^{k+1}$ is unknown and $P_{\gamma}$ has density
\[p_\gamma(x) = p_{(\ta,\la)}(x) = h(x) \exb{\ta U(x)+\sum_{i=1}^k \la_i T_i(x) - A(\ta,\la)}. \]
We wish to test $H_0 : \ta \le \ta_0$ against $H_1 : \ta > \ta_0$. For a fixed $\ta$, the family $\{p_{\ta,\la}\}$ is an exponential family with sufficient statistics $T = (T_1,\ldots, T_k)$ and so 
\[P_{(\ta,\la)}(X|T) = P_\ta(X|T). \]
In particular we have $P_{(\ta,\la)}(U(X)|T(X)) = P_\ta(U(X)|T(X))$ and so $U(X)|T(X)$ has no $\la$ dependence.
\begin{remark}
    This observation is important. We have shown that conditioning eliminates the nuisance parameters. Thus we can fix $\gamma_0 =(\ta_0,\la_0) \in \Om_1$ and $\gamma_1 = (\ta_1,\la_1)$ and construct a test based on $P_{\ta_0}(X|T)$ against $P_{\ta_1}(X|T)$ which has no $\la$ dependence. Thus we can use tools from one-dimensional hypothesis testing. Even better, conditioning on $T$ gives us a one-dimensional exponential family.
\end{remark}
\begin{lemma}
    For each $t$, $U(X)|T=t$ forms a one-dimensional exponetial family in $\ta$.
\end{lemma}
\begin{proof}
    We will only consider the discrete case. For all $u$ and $t$ let
    \[A_{u,t} = \{x \in \X : U(x)=u, T(x)=t\}~~~\text{and}~~~ A_t = \{x \in \X: T(x)=t\}. \]
    \begin{align*}
        P_{\ta,\la}(U(X)=u|T(X)=t) &= \frac{P_{\ta,\la}(U(X)=u, T(X)=t)}{P_{\ta,\la}(T(X)=t)}\\
        &=\frac{\sum\limits_{x\in A_{u,t}}p_{\ta,\la}(x)}{\sum\limits_{x \in A_t}p_{\ta,\la}(x)}\\
        &=\frac{\sum\limits_{x\in A_{u,t}}\exb{\ta u + \sum_{i=1}^k \la_i t_i}h(x)}{\sum\limits_{x \in A_t}\exb{\ta U(x)+\sum_{i=1}^k \la_i t_i}h(x)}\\
        &=\underbrace{\exb{\ta u}}_{\text{exponential tilt}} \times \underbrace{\sum_{x \in A_{u,t}} h(x)}_{g(t,u) = \text{base measure}} \times ~~\underbrace{\frac{1}{\sum\limits_{x \in A_t} \exb{\ta U(x)}h(x)}}_{c(t,\ta)=\text{normalizing constant}}.
    \end{align*}
    So $U(X)|T(X)=t$ is a one-dimensional exponential family with sufficient statistic $U$.
\end{proof}
 Our general recipe for one sided testing $\ta \le \ta_0$ against $\ta > \ta_0$ is 
\renewcommand\labelenumi{(\theenumi)}
\begin{enumerate}
    \item Fix an alternative $\ta = \ta_1 > \ta_0$ and $\la_1 \in \R^k$.
    \item Condition on $T$ so that $X|T$ does not depend on $\la$ and $U|T$ follows a one dimensional exponential family. 
    \item Construct the MP test for the conditional distribution. That is
    \[\phi_t(u) = \begin{cases}
        1 & \text{if } u > k(t),\\
        \rho(t) & \text{if } u = k(t),\\
        0 & \text{if } u < k(t).
    \end{cases}\]
    where $k(t)$ and $\rho(t)$ are determined by the conditional level constraint
    \begin{align}\label{cond}
        \E_{\ta_0}[\phi_t | T=t] = \al.
    \end{align}
\end{enumerate}
We will next argue that under some assumptions that test $\phi^*(u,t) = \phi_t(u)$ is the UMPU test for $H_0$ against $H_1$. Note that for every test $\phi$
\[\E_{\gamma} \phi = \E_{\gamma}\left[\E_{\gamma}[\phi|T]\right] = \E_{\gamma}\left[\E_{\ta}[\phi|T]\right]. \]
In particular if $\ta \le \ta_0$, then 
\[\E_{\gamma} \phi^* = \E_{\gamma}\left[\E_\ta[\phi|T]\right]\le \E_{\gamma}[\al]=\al, \]
and we have equality if $\ta = \ta_0$. Thus $\phi^*$ is level $\al$ and $\al$-similar.

By Neyman-Pearson, there is no test that satisfies the constraint \eqref{cond} and has strictly large power than $\phi^*$ for any fixed $t$ or $\ta_1 > \ta$. Thus $\phi^*$ is the most powerful test in the class of tests satisfying \eqref{cond} for any fixed $\ta_1 > \ta$. Since $\phi^*$ does not depend on $\ta_1$, the test $\phi^*$ is in fact the UMP test among tests satisfying the constrain \eqref{cond}.

Recall that we are trying to show that $\phi^*$ is the UMPU test. By theorem \eqref{al_similar} it suffices to show that $\phi^*$ is uniformly most powerful among $\al$-similar tests. We thus wish to relate the tests that satisfy condition \eqref{cond} to the $\al$-similar tests. With this in mind we make the following definition.
\begin{defn}
    Let $\phi$ be a test for $H_0 : \gamma \in\Om_0$ against $H_1 : \gamma \in \Om_1$ and let $W = \overline{\Om}_0 \cap \overline{\Om}_1$. Suppose that $T$ is a sufficient statistic for $\{P_\gamma : \gamma \in W\}$. For $\al \in [0,1]$, a test $\phi$ is said to have \emph{Neyman structure} if $\E_\gamma[\phi|T]=\al$ almost surely for all $\gamma \in W$.     
\end{defn}
Thus tests with Neyman structure are precisely those test that satisfy condtion \eqref{cond}. Note that all tests with Neyman structure are $\al$-similar. This is because for $\gamma \in W$,
\begin{align*}
    \E_{\gamma} \phi &= \E_\gamma \left[\E[\phi|T]\right]\\
    &=\E_\gamma [\al]\\
    &=\al.
\end{align*}
The converse is not true in general. Note that we can define the function $g(T) = \E_\gamma[\phi|T]-\al$ for some $\gamma \in W$. The function $g$ is well defined because $T$ is sufficient for $\{P_\gamma : \gamma \in W\}$.The $\phi$ has Neyman structure if $g(t)$ is almost surely 0. On the other hand suppose that $\phi$ is $\al$-similar. Then, for all $\gamma \in W$, we have
\[\E_\gamma[g(t)] = \E_\gamma[\E_\gamma[\phi|T]-\al] = \E_\gamma[\phi]-\al = 0. \]
Thus $\phi$ being $\al$-similar implies that $g(T)$ is first order ancillary. Thus for a converse we need completeness.
\begin{lemma}
    If $T$ is sufficient and complete for $\{P_\gamma : \gamma \in W\}$, then every $\al$-similar test has Neyman structure.
\end{lemma}
\begin{proof}
    As before let $g(T) = \E_\gamma[\phi|T]$ which is well defined by sufficiency. As noted above $g(T)$ is first order ancillary. Since $T$ is complete, this implies that $g(T)=0$ almost surely. Thus $\E_\gamma[\phi|T]=\al$ almost surely and so $\phi$ has Neyman structure.
\end{proof}
Combing what we have done this lecture with Theorem \eqref{al_similar} we have
\begin{thrm}
    Suppose $\beta_\phi(\gamma)= \E_\gamma[\phi]$ is continuous for every test $\phi$. If $\phi^*$ is UMP among level $\al$ tests with Neyman structure, then
    \begin{enumerate}
        \item The test $\phi$ is UMP among $\al$-similar tests.
        \item The test $\phi$ is UMPU at level $\al$.
    \end{enumerate}
\end{thrm}
\section{A Poisson example}
Consider data $(X,Y)$ where $X \sim \poi(v)$, $Y\sim \poi(u)$ and $X$ and $Y$ are independent. For example $Y$ could model the number of people who recovered from a disease after recieving a new medicine and $X$ could model the number of people who recovered from the same disease in a control group. With this application in mind we would like to test the hypotheses
\[H_0 : u \le v \text{ against } H_1 : u > v.\]
Thus rejecting the null would correspond to a belief that our drug increases the chance of recovery. So that we can work with an exponential family we will rewrite these hypotheses as 
\[H_0 : \log\left(\frac{u}{v}\right) \le 0\text{ against } H_1 : \log\left(\frac{u}{v}\right) > 0. \]
The joint density of $(X,Y)$ is
\begin{align*}
    p_{u,v}(x,y) &= \frac{\exb{-v}v^x}{x!}\frac{\exb{-u}v^y}{y!}\\
    &= \frac{1}{x!y!}\exb{x\log(v)+y\log(u)-v-u}\\
    &\propto \frac{1}{x!y!}\exb{y\log\left(\frac{u}{v}\right)+(x+y)\log(u)}.
\end{align*}
If we use the notation of the previous example we have 
\[\gamma = (\ta, \la) = \left(\log\left(\frac{u}{v}\right), \log(v)\right),\]
and 
\[(U,T) = (Y,X+Y).\]
Our goal is to test $\ta < 0$ in the presence of the nuisance parameter $\la$. Our first step is to check that $T$ is sufficient for fixed $\ta$ and that $\ta$ is complete on the boundary $W=\{(0,\la)\}$. One can check that when
\[Y|(X+Y=n) \sim \text{Binomial}\left(n,\frac{u}{u+v}\right).\]
Note that if $\log(\frac{u}{v}) = \ta$ and $\log\left(v\right)=\la$, then $v = \exb{\la}$ and $u = \exb{\ta+\la}$. Thus 
\[\frac{u}{u+v} = \frac{\exb{\ta+\la}}{\exb{\ta+\la}+\exb{\la}}=\frac{\exb{\ta}}{\exb{\ta}+1}.\]
Thus for fixed $\ta$, the distribution of $Y|X+Y$ does not depend on $\la$ and so $T=X+Y$ is sufficient for $Y$. Furthermore on $W=\{(\ta,\la) : \ta = 0\}$ we have
\[p(x,y;\la) \propto \frac{1}{x!y!}\exb{(x+y)\la}, \]
and so $ T$ is complete on the boundary by results on exponential famillies. 


We next have the derive the UMP test with Neyman structure. For all $n$, the family model \[\left\{\text{Binomial}\left(n,\frac{e^{\ta}}{1+e^{\ta}}\right): \ta \in \R\right\},\] has monotone likelihood ratio in $Y$. Furthermore when $\ta=0$, the distribution of $X|X+Y=n$ is $\text{Binomial}(n,0.5)$. Thus the optimal test for $H_0$ against $H_1$ is
\begin{align*}
    \phi(k,n) = \begin{cases}
        1 & \text{if } k > c_n,\\
        \rho_n & \text{if } k = c_n,\\
        0 & \text{if } k < c_n.
    \end{cases}
\end{align*}
where the constants $\rho_n$ and $c_n$ are chosen so that 
\[\Pa(Z>c_n)+\rho_n\Pa(Z=c_n) =\al, \]
where $Z \sim \text{Binomial}(n,0.5)$. The test $\phi$ is the UMP level $\al$ test with Neyman structure. Thus the results of the previous section imply that $\phi$ is UMPU at level $\al$.
\end{document}
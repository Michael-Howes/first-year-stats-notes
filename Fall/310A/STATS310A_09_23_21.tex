\documentclass{article}
\usepackage{ae,aecompl}
\usepackage{todonotes}
\usepackage{chngcntr}
\usepackage{tikz-cd}
\usepackage{graphicx}
\graphicspath{ {./images/}}
\usepackage[all,cmtip]{xy}
\usepackage{amsmath, amscd}
\usepackage{amsthm}
\usepackage{amssymb}
\usepackage{amsfonts}
\usepackage{bm}
\usepackage{qsymbols}
\usepackage{latexsym}
\usepackage{mathrsfs}
\usepackage{mathtools}
\usepackage{cite}
\usepackage{color}
\usepackage{url}
\usepackage{enumerate}
\usepackage{verbatim}
\usepackage[draft=false, colorlinks=true]{hyperref}
\usepackage{pdfpages}
\usepackage[margin=1.2in]{geometry}
\usepackage{IEEEtrantools}

\usepackage{fancyhdr}


\usepackage[nameinlink]{cleveref}


\DeclareMathOperator*{\ac}{accept}
\DeclareMathOperator*{\amax}{argmax}
\DeclareMathOperator*{\amin}{argmin}
\DeclareMathOperator*{\Aut}{Aut}
\newcommand {\al}{{\alpha}}
\newcommand {\abs}[1]{{\left\lvert#1\right\rvert}}
\newcommand {\A}{{\mathcal{A}}}
\newcommand {\AM}{{\mathrm{AM}}}
\newcommand {\AMp}{{\AM_{p}^{X}\!(\Ri_\w)}}
\newcommand {\B}{{\mathcal{B}}}
\DeclareMathOperator*{\Be}{Bern}
\newcommand {\Br}{{\dot{B}}}
\newcommand {\Ba}{{\mathfrak{B}}}
\newcommand {\C}{{\mathbb C}}
\newcommand {\ce}{\mathrm{c}}
\newcommand {\Ce}{\mathrm{C}}
\newcommand {\Cc}{\mathrm{C_{c}}}
\newcommand {\Ccinf}{\mathrm{C_{c}^{\infty}}}
\DeclareMathOperator{\cov}{Cov}
\DeclareMathOperator{\DEV}{DEV}
\newcommand {\Di}{{\mathbb D}}
\newcommand {\dom}{\mathrm{dom}}
\newcommand{\dist}{\stackrel{\mathrm{dist}}{=}}
\newcommand {\ud}{\mathrm{d}}
\newcommand {\ue}{\mathrm{e}}
\newcommand {\eps}{\varepsilon}
\newcommand {\veps}{\varepsilon}
\newcommand {\vrho}{{\varrho}}
\newcommand {\E}{{\mathbb{E}}}
\newcommand {\Ec}{{\mathcal{E}}}
\newcommand {\Ell}{L}
\newcommand {\Ellp}{{L_{p}[0,1]}}
\newcommand {\Ellpprime}{{L_{p'}([0,1])}}
\newcommand {\Ellq}{{L_{q}([0,1])}}
\newcommand {\Ellqprime}{{L_{q'}([0,1])}}
\newcommand {\Ellr}{L^{r}}
\newcommand {\Ellone}{{L_{1}([0,1])}}
\newcommand{\Elltwo}{{L_{2}([0,1])}}
\newcommand{\Ellinfty}{L^{\infty}}
\newcommand{\Ellinftyc}{L_{\mathrm{c}}^{\infty}}
\newcommand{\exb}[1]{\exp\left\{#1\right\}}
\DeclareMathOperator*{\Ext}{Ext}
\newcommand{\F}{{\mathcal{F}}}
\newcommand{\Fe}{{\mathbb{F}}}
\newcommand{\G}{{\mathcal{G}}}
\newcommand{\HF}{\mathcal{H}_{\text{FIO}}^{1}(\Rd)}
\newcommand{\Hr}{H}
\newcommand{\HT}{\mathcal{H}}
\newcommand{\ui}{\mathrm{i}}
\newcommand{\I}{{I}}
\newcommand{\J}{{\mathcal{J}}}
\newcommand{\id}{{\mathrm{id}}}
\newcommand{\iid}{\stackrel{\mathclap{\normalfont\mbox{iid}}}{\sim}}
\newcommand{\im}{{\text{im }}}
\newcommand{\ind}{{\perp\!\!\!\perp}}
\DeclareMathOperator*{\Int}{int}
\newcommand{\intx}{{\overline{\int_{X}}}}
\newcommand{\inte}{{\overline{\int_{\E}}}}
\newcommand{\la}{\lambda}
\newcommand{\rb}{\rangle}
\newcommand{\lb}{{\langle}}
\newcommand{\La}{\Lambda}
\newcommand{\calL}{{\mathcal{L}}}
\newcommand{\lp}{{\mathcal{L}}^{p}}
\newcommand{\lpo}{{\overline{\mathcal{L}}^{p}\!}}
\newcommand{\Lpo}{{\overline{\Ell}^{p}\!}}
\newcommand{\M}{{\mathbf{M}}}
\newcommand{\Ma}{{\mathcal{M}}}
\newcommand{\N}{{{\mathbb N}}}
\newcommand{\Na}{{{\mathcal{N}}}}
\newcommand{\norm}[1]{\left\|#1\right\|}
\newcommand{\normm}[1]{{\left\vert\kern-0.25ex\left\vert\kern-0.25ex\left\vert #1 
    \right\vert\kern-0.25ex\right\vert\kern-0.25ex\right\vert}}
\newcommand{\Om}{{{\Omega}}}
\newcommand{\one}{{{\bf 1}}}
\newcommand{\pic}{\text{Pic }}
\newcommand{\ph}{{\varphi}}
\newcommand{\Pa}{{\mathbb{P}}}
\newcommand{\Po}{{\mathcal{P}}}
\newcommand{\Q}{{\mathbb{Q}}}
\newcommand{\R}{{\mathbb R}}
\newcommand{\Rd}{{\mathbb{R}^{d}}}
\DeclareMathOperator{\rej}{reject }
\newcommand{\Rn}{{\mathbb{R}^{n}}}
\newcommand{\cR}{{\mathcal{R}}}
\newcommand{\Rad}{{\mathrm{Rad}}}
\newcommand{\ran}{{\mathrm{ran}}}
\newcommand{\Ri}{{\mathrm{R}}}
\newcommand{\supp}{{\mathrm{supp}}}
\newcommand{\Se}{\mathrm{S}}
\newcommand{\Sp}{S^{*}(\Rn)}
\newcommand{\St}{{\mathrm{St}}}
\newcommand{\Sw}{\mathcal{S}}
\newcommand{\T}{{\mathcal{T}}}
\newcommand{\ta}{{\theta}}
\newcommand{\Ta}{{\Theta}}
\newcommand{\topp}{\stackrel{p}{\to}}
\newcommand{\todd}{\stackrel{d}{\to}}
\newcommand{\toL}[1]{\stackrel{L^{#1}}{\to}} 
\newcommand{\toas}{\stackrel{a.s.}{\to}}
\DeclareMathOperator{\V}{Var}
\newcommand {\w}{{\omega}}
\newcommand {\W}{{\mathrm{W}}}
\newcommand {\Wnp}{\text{$\mathrm{W}$\textsuperscript{$n,\!p$}}}
\newcommand {\Wnpeq}{\text{$\mathrm{W}$\textsuperscript{$n\!,\!p$}}}
\newcommand {\Wonep}{\text{$\mathrm{W}$\textsuperscript{$1,\!p$}}}
\newcommand {\Wonepeq}{\text{$\mathrm{W}$\textsuperscript{$1\!,\!p$}}}
\newcommand {\X}{{\mathcal{X}}}
\newcommand {\Z}{{{\mathbb Z}}}
\newcommand {\Za}{{\mathcal{Z}}}
\newcommand {\Zd}{{\Z[\sqrt{d}]}}
\newcommand {\vanish}[1]{\relax}

\newcommand {\wh}{\widehat}
\newcommand {\wt}{\widetilde}
\newcommand {\red}{\color{red}}

% Distributions
\newcommand{\normal}{\mathsf{N}}
\newcommand{\poi}{\mathsf{Poisson}}
\newcommand{\bern}{\mathsf{Bernoulli}}
\newcommand{\bin}{\mathsf{Binomal}}
\newcommand{\multi}{\mathsf{Multinomial}}
\newcommand{\Exp}{\mathsf{Exp}}



% put your command and environment definitions here




% some theorem environments
% remove "[theorem]" if you do not want them to use the same number sequence


  \newtheorem{thrm}{Theorem}
  \newtheorem{lemma}{Lemma}
  \newtheorem{prop}{Proposition}
  \newtheorem{cor}{Corollary}

  \newtheorem{conj}{Conjecture}
  \renewcommand{\theconj}{\Alph{conj}}  % numbered A, B, C etc

  \theoremstyle{definition}
  \newtheorem{defn}{Definition}
  \newtheorem{ex}{Example}
  \newtheorem{exs}{Examples}
  \newtheorem{question}{Question}
  \newtheorem{remark}{Remark}
  \newtheorem{notn}{Notation}
  \newtheorem{exer}{Exercise}




\title{STATS 310A - Lecture 2}
\author{Persi Diaconis\\ Scribed by Michael Howes}
\date{09/23/21}

\pagestyle{fancy}
\fancyhf{}
\rhead{STATS310A - Lecture 2}
\lhead{09/23/21}
\rfoot{Page \thepage}

\begin{document}
\maketitle
\tableofcontents
\section{Strong law of large numbers}
Recall that we have $\Om = (0,1]$ and for $\w \in \Om$ we write $\w = 0.d_1(\w)d_2(\w)\ldots$ where $d_i(\w)$ is the $i^{th}$ binary digit of $\Om$. If $\w$ has two binary expansions, we pick the expansion that ends in all 1's. We also defined $r_i = 2d_i -1 \in \{-1,1\}$ and $S_n = \sum_{i=1}^n r_i$. Let $B \subseteq \Om$ be the subset
\[B = \left\{\w \in \Om : \lim_{n \to \infty}\abs{S_n(\w)/n} = 0 \right\}.   \]
Recall also that a subset $A \subseteq \Om$ is said to nelligable if for every $\varepsilon >0$, there exists a countable collection of intervals $\{(a_i,b_i]\}_{i=1}^\infty$ such that $A \subseteq \cup_i (a_i,b_i]$ and $\sum_i b_i-a_i = \sum_i p((a_i,b_i]) < \varepsilon$. The strong law of large numbers (slln) states that $B^c$ is nelligable.
\begin{proof}
    Fix $\delta > 0$ and note that $\{\abs{S_n/n} > \delta\} = \{S_n^4 > \delta^4n^4\}$. By Markov's inequality we can thus conclude that 
    \[p(\{\abs{S_n/n} > \delta\}) \le \frac{1}{\delta^4n^4}\int_0^1 S_n(w)^4d\w. \]
    Since $S_n = \sum_{i=1}^n r_i$ we know that $(S_n)^4 = \sum\limits_{i,j,k,l = 1}^n r_ir_jr_kr_l$. Note that there are fives possibilities for the term $r_ir_jr_kr_l$. These are
    \begin{enumerate}
        \item The case $r_i^4$. This case occurs $n$ times and when it occurs $\int_0^1 r_i(\w)r_j(\w)r_k(\w)r_l(\w)d\w = \int_0^1 1d \w=1$.
        \item The case $r_i^2r_jr_k$ when $i,j,k$ are distinct. In this case $\int_0^1 r_ir_jr_kr_ld\w = \int_0^1r_jr_kd\w = 0$.
        \item The $r_i^2r_j^2$ when $i \neq j$. This case occurs $3n(n-1)$ times and in this case $\int_0^1 r_ir_jr_kr_ld\w = \int_0^1 1d\w = 1$.
        \item The case $r_i^3r_j$ where $i\neq j$. In this case $r_ir_jr_kr_l = r_ir_j$ and thus $\int_0^1 r_ir_jr_kr_ld\w = 0$.
        \item The case $r_ir_jr_kr_l$ and $i,j,k,l$ are all distinct. In this case $\int_0^1r_ir_jr_kr_l = 0$.
    \end{enumerate}
    Combining all of the above we have
    \[\int_0^1 S_n(\w)^4d\w = \sum_{i,j,k,l=1}^n \int_0^1 r_i(\w)r_j(\w)r_k(\w)r_l(\w)d\w = n+3n(n-1) \le 3n^2. \]
    Thus we have $p(\{\abs{S_n/n} > \delta\}) \le \frac{3}{\delta^4n^2}$. Now set $\delta_n = \frac{1}{n^{1/8}}$ so that $\delta_n \to 0$ and $\sum_{n=1}^\infty \frac{3}{\delta_n^4n^2}< \infty$. Note that for all $m \in \N$, we have
    \[\bigcap_{n = m}^\infty\{\w : \abs{S_n(\w)/n} \le \delta_n \} \subseteq B. \]
    By taking complements we see
    \[B^C \subseteq \bigcup_{n = m}^\infty \{\w : \abs{S_n(\w)/n} > \delta_n \}. \]
    The set $\{w : \abs{S_n(\w)/n} > \delta_n \}$ can be written as a finite union of disjoint intervals $\cup_{i=1}^{k_n}I_{n,i}$ such that $\sum_{i=1}^{k_n}p(I_{n,i}) < \frac{3}{\delta_n^4 n^2}$. Thus we have 
    \[ B^c \subseteq \bigcup_{n = m}^\infty \bigcup_{i=1}^{k_n} I_{n,i},\] 
    for all $m \in \N$. Thus given $\varepsilon > 0$ we can choose $m$ such that $\sum_{n= m}^\infty \frac{3}{\delta_n^4 n^2} < \varepsilon$ and note that 
    \[\sum_{n = m}^\infty\sum_{i=1}^{k_n}p(I_{n,i}) \le \sum_{n = m}^\infty \frac{3}{\delta_n^4 n^2} < \varepsilon.  \]
    Showing that $B^c$ is nelligiable,
\end{proof}
Thus we can say that $\lim_{n \to \infty} \frac{S_n(\w)}{n} = 0$ for \emph{almost all} $\w \in \Om$ or simply that $\lim_{n \to \infty} \frac{S_n}{n} = 0$ \emph{almost surely}. But! \grumpy 
This statement does not have any qualitative bounds to it. Also how well does our model actually reflect coin flipping? A true model would have a lot of physics and observations of how people flip coins. Persi has written papers about such things. One is ``Dynamic bias in the coin toss'' written with collaborators.

\section{Assigning Probabilities}
See ``Ten Great Ideas About Chane'' by Persi. 
\begin{defn}
    Let $\Om$ be a set. A collection $\F_0$ of subset of $\Om$ is a \emph{field} is
    \begin{enumerate}
        \item The set $\Om$ is in $\F_0$.
        \item If $A \in \F_0$, then $A^c \in \F_0$.
        \item If $A, B \in \F_0$, then $A \cup B \in \F_0$.
    \end{enumerate}
\end{defn}
We say that $\F_0$ is closed under compliments and finite unions.
\begin{defn}
    Let $\Om$ be a set and $\F$ a collection of subsets of $\Om$. The collection $\F$ is a $\sigma$-field if it is a field and closed under countable unions. That is if $A_1,A_2,\ldots \in \F$, then $\cup_i A_i \in \F$.
\end{defn}
\begin{exs}
\begin{enumerate}
    \item $\Om = (0,1]$ and $\F_0 = \{\text{finite unions of disjoint intervals of the form (a,b]}\}$. This is a field since $(a,b]^c = (0,a] \cup (b,1]$. It is not a $\sigma$-field since $(0,1/2) = \cup_{i=1}^\infty (0,1/2-1/i] \notin \F_0$.
    \item The collection of all subsets of $\Om$ is a $\sigma$-field.
    \item $\{\Om, \emptyset\}$ is a $\sigma$-field.
    \item If $\F_i$ are $\sigma$-fields on $\Om$ for all $i \in I$, then 
    \[\F = \bigcap_{i \in I}\F_i, \]
    is a $\sigma$-field.
    \item If $\mathcal{C}$ is any collection of subsets of $\Om$, then define
     \[\sigma(\mathcal{C}) := \bigcap_{i} \F_i,\]
    where the intersection is of all $\sigma$-fields on $\Om$ that contain $\mathcal{C}$. This is a $\sigma$-field and is called the $\sigma$-field generated by $\mathcal{C}$.
    \item The Borel set (in $(0,1]$) is the $\sigma$-field generated by all the intervals in $(0,1]$. For example our set
    \[B = \bigcap_{k=1}^\infty \bigcup_{m=1}^\infty \bigcap_{n = m}^\infty \{\abs{S_n/n} < 1/k \},\]
    is a Borel set.
\end{enumerate}
\end{exs}
\begin{defn}
    Let $\Om$ be a set and $\F$ a field. The pair $(\Om,\F)$ is called a measurable space. A probability on $(\Om, \F)$ is a function 
    \[P:\F \to [0,1], \quad A \mapsto P(A),\]
    such that\
    \begin{enumerate}
        \item $P(\Om)=1$, $P(\emptyset) = 0$.
        \item $P(A^C) = 1- P(A)$.
        \item If $\{A_i\}_{i=1}^\infty$ are pairwise disjoint and $\cup_{i=1}^\infty A_i \in \F$, then $P(\cup_{i=1}^\infty A_i)= \sum_{i=1}^\infty P(A_i)$.
    \end{enumerate}
\end{defn}
\begin{remark}
    It is often ``easy'' to assign probabilities to a field $\F_0$ of sets and there is a standard way to uniquely extend these probabilites to the $\sigma$-field generated by $\F_0$. 
\end{remark}
\begin{defn}
    Let $P$ be a probability on a field $\F_0$. For all $A \subseteq \Om$, define
    \[P^*[A] := \inf\left\{\sum_{i=1}^\infty P(B_i): A \subseteq \bigcup_{i=1}^\infty B_i \text{ and } B_i \in \F_0\right\}. \]
    The function $P^*$ is called the \emph{outer measure associated to $(\Om, \F, P)$}.
\end{defn}
The function $P^*$ has the following properties
\begin{enumerate}
    \item $P^*(\Om)  =1$, $P^*(\emptyset) = 0$.
    \item $P^*$ is countable subadditive. That if if $A = \cup_{i=1}^\infty A_i$, then $P^*[A] \le \sum_{i=1}^\infty P^*[A_i]$.
    \begin{proof}
        Fix $\varepsilon > 0$ and let $(B_{i,j})_{j=1}^\infty$ be a countable cover of $A_i$ by intervals such that
            \[\sum_{j=1}^\infty P[B_{i,j}] \le P^*[A_i]+\varepsilon 2^{-i}\].
        Thus $\{B_{i,j}\}_{i,j = 1}^\infty$ is a countable cover of $A$ by elements of $\F_0$. Since $P[B_{i,j}] \ge 0$, we can rearrange the infinite series $\sum_{i,j=1}^\infty P[B_{i,j}]$ and thus conclude the following
        \begin{IEEEeqnarray*}{rCl}
            P^*[A] &\le & \sum_{i,j=1}^\infty P[B_{i,j}]\\
            &=&\sum_{i}^\infty\sum_{j=1}^\infty P[B_{i,j}]\\
            &\le&\sum_{i=1}^\infty P^*[A_i]+\varepsilon2^{-i}\\
            &=&\sum_{i=1}^\infty P^*[A_i] + \varepsilon\sum_{i=1}^\infty 2^{-i}\\
            &=&\sum_{i=1}^\infty P^*[A_i] + \varepsilon.
        \end{IEEEeqnarray*}
        Since $\varepsilon > 0$ was arbitrary, we can conclude that $P^*[A] \le \sum_{i=1}^\infty P^*[A_i]$.
    \end{proof}
\end{enumerate}
\begin{defn}
    \emph{[Caratheodory]} Let $P$ be a probability measure on a field $\F_0$. A set $A \in \Om$ is \emph{measurable} if for all $E \subseteq \Om$,
    \[P^*[E] = P^*[E \cap A] + P^*[E \cap A^c]. \]
    We will use $\Ma$ to denote the set of all measurable sets.
\end{defn}
We will prove:
\begin{itemize}
    \item $\Ma$ is a $\sigma$-field that contains $\F_0$.
    \item $P^*$ restricted to $\Ma$ is a probability (in particular $P^*$ is countable additive on measurable sets).
    \item $P^*$ restricted to $\F_0$ equals $P$.
    \item $P^*$ is unique.
\end{itemize}
This goal will preoccupy us for the next lecture but we will start working on it today. With the notation as above we will show that $\Ma$ is a field. 

\begin{proof}
    The one main trick we will use is that to show $A \in \Ma$, it is enough to show
    \[P^*(E \cap A)+P^*(E \cap A^c) \le P^*(E),\]
    for all $E \subseteq \Om$. This is becuase we always have $P^*(E) \le P^*(E \cap A) + P^*(E \cap A^c)$ by subadditivity.


    We can see immediatie that $\Om \in \Ma$ and that $\Ma$ is closed under complements. Thus suppose that $A,B \in \Ma$ and $E \subseteq \Om$, then
    \begin{IEEEeqnarray*}{rCl}
        P^*(E) &=&P^*(E \cap A) + P^*(E \cap A^c)\\
        &=&P^*(E \cap A \cap B) + P^*(E \cap A \cap B^c) + P^*(E \cap A^c \cap B)+ P^*(E \cap A^c \cap B^c)\\
        &\ge&P^*(E \cap ([A \cap B]\cup[A \cap B^c]\cup[A^c \cap B]))+P^*(E \cap [A^c \cap B^c])\\
        &=&P^*(E \cap [A\cup B])+P^*(E \cap [A\cup B]^c).
    \end{IEEEeqnarray*}
    Thus, by the main trick, $A \cup B \in \Ma$.
\end{proof}  
\section{Conclusion}
One might ask why are we doing all this just to talk about probabilities? There are several reasons
\begin{itemize}
    \item People want to work with infinite sequence spaces, random curves, Brownian motion and the set of all probabilities measures on $[0,1]$. These are complicated spaces and it can be hard to assign probabilities on them by hand.
    \item We simply cannot assign a consistent notion of length to all subsets of $[0,1]$.
    \item Keep an eye out for a halloween talk on non-measurable set.
\end{itemize} 
A remark on finite vs countable additivity. Let $\N = \{1,2,\ldots\}$. One would like to say that $j$ choosen ``at random'' from $\N$ has a 50\% chance of being even. We can make sense of this by defining
\[P_n(A) = \frac{\abs{A \cap [n]}}{n}, \]
where $\abs{B}$ is the number of elements in $B$ and $[n] = \{1,2,\ldots,n\}$. If $\lim_{n \to \infty}P_n(A)=l$ exists we say that $A$ has density $D(A) = l$. One can show
\begin{itemize}
    \item $D(\text{multiples of } j) = \frac{1}{j}$,
    \item $D(\{\text{primes}\}) = 0$, and
    \item $D(\{\text{square free numbers}\}) = \frac{6}{\pi^2}$.
\end{itemize} 
Not every set has a density. For example if 
\[A = \{1,10,11,100,101,\ldots,199,1000,10001,\ldots,1999,10000\}. \]
That is, $A$ is the set of numbers that start with a 1 when written in decimal. Then $P_n(A)$ moves up and down between 1/9 and 5/9 infinitely often. The answer to this problem is to use the Hahn-Bananch theorem to extend $D$ to all subsets of $\N$. This is gives us a measure that is finitely additive but not countably additive. This example is highly non-constructive and non unique since there are many possible Hahn Banach extensions and none of them are ``natural.''
\end{document}
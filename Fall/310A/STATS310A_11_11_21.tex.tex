\documentclass{article}
\usepackage{ae,aecompl}
\usepackage{todonotes}
\usepackage{chngcntr}
\usepackage{tikz-cd}
\usepackage{graphicx}
\graphicspath{ {./images/}}
\usepackage[all,cmtip]{xy}
\usepackage{amsmath, amscd}
\usepackage{amsthm}
\usepackage{amssymb}
\usepackage{amsfonts}
\usepackage{bm}
\usepackage{qsymbols}
\usepackage{latexsym}
\usepackage{mathrsfs}
\usepackage{mathtools}
\usepackage{cite}
\usepackage{color}
\usepackage{url}
\usepackage{enumerate}
\usepackage{verbatim}
\usepackage[draft=false, colorlinks=true]{hyperref}
\usepackage{pdfpages}
\usepackage[margin=1.2in]{geometry}
\usepackage{IEEEtrantools}

\usepackage{fancyhdr}


\usepackage[nameinlink]{cleveref}


\DeclareMathOperator*{\ac}{accept}
\DeclareMathOperator*{\amax}{argmax}
\DeclareMathOperator*{\amin}{argmin}
\DeclareMathOperator*{\Aut}{Aut}
\newcommand {\al}{{\alpha}}
\newcommand {\abs}[1]{{\left\lvert#1\right\rvert}}
\newcommand {\A}{{\mathcal{A}}}
\newcommand {\AM}{{\mathrm{AM}}}
\newcommand {\AMp}{{\AM_{p}^{X}\!(\Ri_\w)}}
\newcommand {\B}{{\mathcal{B}}}
\DeclareMathOperator*{\Be}{Bern}
\newcommand {\Br}{{\dot{B}}}
\newcommand {\Ba}{{\mathfrak{B}}}
\newcommand {\C}{{\mathbb C}}
\newcommand {\ce}{\mathrm{c}}
\newcommand {\Ce}{\mathrm{C}}
\newcommand {\Cc}{\mathrm{C_{c}}}
\newcommand {\Ccinf}{\mathrm{C_{c}^{\infty}}}
\DeclareMathOperator{\cov}{Cov}
\DeclareMathOperator{\DEV}{DEV}
\newcommand {\Di}{{\mathbb D}}
\newcommand {\dom}{\mathrm{dom}}
\newcommand{\dist}{\stackrel{\mathrm{dist}}{=}}
\newcommand {\ud}{\mathrm{d}}
\newcommand {\ue}{\mathrm{e}}
\newcommand {\eps}{\varepsilon}
\newcommand {\veps}{\varepsilon}
\newcommand {\vrho}{{\varrho}}
\newcommand {\E}{{\mathbb{E}}}
\newcommand {\Ec}{{\mathcal{E}}}
\newcommand {\Ell}{L}
\newcommand {\Ellp}{{L_{p}[0,1]}}
\newcommand {\Ellpprime}{{L_{p'}([0,1])}}
\newcommand {\Ellq}{{L_{q}([0,1])}}
\newcommand {\Ellqprime}{{L_{q'}([0,1])}}
\newcommand {\Ellr}{L^{r}}
\newcommand {\Ellone}{{L_{1}([0,1])}}
\newcommand{\Elltwo}{{L_{2}([0,1])}}
\newcommand{\Ellinfty}{L^{\infty}}
\newcommand{\Ellinftyc}{L_{\mathrm{c}}^{\infty}}
\newcommand{\exb}[1]{\exp\left\{#1\right\}}
\DeclareMathOperator*{\Ext}{Ext}
\newcommand{\F}{{\mathcal{F}}}
\newcommand{\Fe}{{\mathbb{F}}}
\newcommand{\G}{{\mathcal{G}}}
\newcommand{\HF}{\mathcal{H}_{\text{FIO}}^{1}(\Rd)}
\newcommand{\Hr}{H}
\newcommand{\HT}{\mathcal{H}}
\newcommand{\ui}{\mathrm{i}}
\newcommand{\I}{{I}}
\newcommand{\J}{{\mathcal{J}}}
\newcommand{\id}{{\mathrm{id}}}
\newcommand{\iid}{\stackrel{\mathclap{\normalfont\mbox{iid}}}{\sim}}
\newcommand{\im}{{\text{im }}}
\newcommand{\ind}{{\perp\!\!\!\perp}}
\DeclareMathOperator*{\Int}{int}
\newcommand{\intx}{{\overline{\int_{X}}}}
\newcommand{\inte}{{\overline{\int_{\E}}}}
\newcommand{\la}{\lambda}
\newcommand{\rb}{\rangle}
\newcommand{\lb}{{\langle}}
\newcommand{\La}{\Lambda}
\newcommand{\calL}{{\mathcal{L}}}
\newcommand{\lp}{{\mathcal{L}}^{p}}
\newcommand{\lpo}{{\overline{\mathcal{L}}^{p}\!}}
\newcommand{\Lpo}{{\overline{\Ell}^{p}\!}}
\newcommand{\M}{{\mathbf{M}}}
\newcommand{\Ma}{{\mathcal{M}}}
\newcommand{\N}{{{\mathbb N}}}
\newcommand{\Na}{{{\mathcal{N}}}}
\newcommand{\norm}[1]{\left\|#1\right\|}
\newcommand{\normm}[1]{{\left\vert\kern-0.25ex\left\vert\kern-0.25ex\left\vert #1 
    \right\vert\kern-0.25ex\right\vert\kern-0.25ex\right\vert}}
\newcommand{\Om}{{{\Omega}}}
\newcommand{\one}{{{\bf 1}}}
\newcommand{\pic}{\text{Pic }}
\newcommand{\ph}{{\varphi}}
\newcommand{\Pa}{{\mathbb{P}}}
\newcommand{\Po}{{\mathcal{P}}}
\newcommand{\Q}{{\mathbb{Q}}}
\newcommand{\R}{{\mathbb R}}
\newcommand{\Rd}{{\mathbb{R}^{d}}}
\DeclareMathOperator{\rej}{reject }
\newcommand{\Rn}{{\mathbb{R}^{n}}}
\newcommand{\cR}{{\mathcal{R}}}
\newcommand{\Rad}{{\mathrm{Rad}}}
\newcommand{\ran}{{\mathrm{ran}}}
\newcommand{\Ri}{{\mathrm{R}}}
\newcommand{\supp}{{\mathrm{supp}}}
\newcommand{\Se}{\mathrm{S}}
\newcommand{\Sp}{S^{*}(\Rn)}
\newcommand{\St}{{\mathrm{St}}}
\newcommand{\Sw}{\mathcal{S}}
\newcommand{\T}{{\mathcal{T}}}
\newcommand{\ta}{{\theta}}
\newcommand{\Ta}{{\Theta}}
\newcommand{\topp}{\stackrel{p}{\to}}
\newcommand{\todd}{\stackrel{d}{\to}}
\newcommand{\toL}[1]{\stackrel{L^{#1}}{\to}} 
\newcommand{\toas}{\stackrel{a.s.}{\to}}
\DeclareMathOperator{\V}{Var}
\newcommand {\w}{{\omega}}
\newcommand {\W}{{\mathrm{W}}}
\newcommand {\Wnp}{\text{$\mathrm{W}$\textsuperscript{$n,\!p$}}}
\newcommand {\Wnpeq}{\text{$\mathrm{W}$\textsuperscript{$n\!,\!p$}}}
\newcommand {\Wonep}{\text{$\mathrm{W}$\textsuperscript{$1,\!p$}}}
\newcommand {\Wonepeq}{\text{$\mathrm{W}$\textsuperscript{$1\!,\!p$}}}
\newcommand {\X}{{\mathcal{X}}}
\newcommand {\Z}{{{\mathbb Z}}}
\newcommand {\Za}{{\mathcal{Z}}}
\newcommand {\Zd}{{\Z[\sqrt{d}]}}
\newcommand {\vanish}[1]{\relax}

\newcommand {\wh}{\widehat}
\newcommand {\wt}{\widetilde}
\newcommand {\red}{\color{red}}

% Distributions
\newcommand{\normal}{\mathsf{N}}
\newcommand{\poi}{\mathsf{Poisson}}
\newcommand{\bern}{\mathsf{Bernoulli}}
\newcommand{\bin}{\mathsf{Binomal}}
\newcommand{\multi}{\mathsf{Multinomial}}
\newcommand{\Exp}{\mathsf{Exp}}



% put your command and environment definitions here




% some theorem environments
% remove "[theorem]" if you do not want them to use the same number sequence


  \newtheorem{thrm}{Theorem}
  \newtheorem{lemma}{Lemma}
  \newtheorem{prop}{Proposition}
  \newtheorem{cor}{Corollary}

  \newtheorem{conj}{Conjecture}
  \renewcommand{\theconj}{\Alph{conj}}  % numbered A, B, C etc

  \theoremstyle{definition}
  \newtheorem{defn}{Definition}
  \newtheorem{ex}{Example}
  \newtheorem{exs}{Examples}
  \newtheorem{question}{Question}
  \newtheorem{remark}{Remark}
  \newtheorem{notn}{Notation}
  \newtheorem{exer}{Exercise}



\newcommand{\s}{s}


\title{STATS310A - Lecture 15}
\author{Persi Diaconis\\ Scribed by Michael Howes}
\date{11/11/21}

\pagestyle{fancy}
\fancyhf{}
\rhead{STATS310A - Lecture 15}
\lhead{11/11/21}
\rfoot{Page \thepage}

\begin{document}
\maketitle
\tableofcontents
\section{Final comments on Stein's method}
\subsection{Dependency}
In the Poisson case we used dependency graphs. There is a version where we only require that the disconnected $X_i$ are ``not too dependent'' rather than requiring that disconnected $X_i$ are independent. Consider our familiar set up $X_i \in \{0,1\}$, $\Pa(X_i=1)=p_i$, $\Pa(X_i=1,X_j=1)=p_{ij}$, $W = \sum_{i \in I} X_i$, $\la = \sum_{i \in I}p_i = \E[W]$ and $\Pa_W(A) = \Pa(W \in A)$. We assume that for each $i \in I$ we have a subset $N_i \subseteq I$. Next define
\begin{align*}
    b_1 &= \sum_{i \in I}\sum_{j \in N_i\setminus \{i\}} p_{ij},\\
    b_2 &=\sum_{i\in I}\sum_{j \in N_i} p_ip_j,\\
    b_3 &= \sum_{i \in I} \E[\E[\abs{X_i-p_i}|X_j \in N_i^c]].
\end{align*}
The quantity $b_3$ measures how independent $X_i$ is of $N_i^c$. 
\begin{thrm}
    With notation as above
    \[\norm{P_W-\Po_\la}_{TV} \le 2(b_1+b_2+b_3),\]
    where $\Po_\la$ is the Poisson distribution with parameter $\la$. Also,
    \[\abs{\Pa(W=0)-e^{-\la}} \le (b_1+b_2+b_3)\frac{1-e^{-\la}}{\la}.\]
\end{thrm}
This result can be found in ``\href{https://projecteuclid.org/journals/statistical-science/volume-5/issue-4/Poisson-Approximation-and-the-Chen-Stein-Method/10.1214/ss/1177012015.full}{Poisson Approximation and the Chen-Stein Method}'' by Arratia, Goldstein and Gord. The article ``\href{https://arxiv.org/abs/1404.1392}{A short survey of Stein's method}'' by Sourav Chatterjee was presented at ICM 2014 and is recommended reading.
\subsection{Stein's method and normal approximation}
Although we will not show it here, Stein's method can be used to prove the central limit theorem. See ``\href{https://searchworks.stanford.edu/view/9115286}{Normal approximation by Stein's method}'' by Chen, Goldstein and Shao. To do this we again use a characteristic operator. In particular we need the theorem
\begin{thrm}
    A random variable $Z$ is $\Na(0,1)$ if and only if for all bounded $C_1$ functions $f: \R \to \R$, 
    \[\E[Zf(Z)-f'(Z)]=0. \]
\end{thrm}
In the interest of displaying a variety of a probabilistic techniques we will not use Stein's method to prove the CLT in this class.
\section{Normal approximation and the CLT}
\subsection{Normal heuristic}
Let $X_i$, $i \in I$ has mean 0 and variance $\sigma^2$. Suppose that the $X_i$ are not too wild and not too dependent. Define $S_n = \sum_{i \in I}X_i$, then 
\[\Pa\left(\frac{S_n}{\sqrt{\V(S_n)}} \le x \right) \approx \Phi(x), \]
where $\Phi$ is the cummulative distribution function of the standard normal distribution.
\begin{ex}
    Suppose we have a finite graph (such as an $n\times n$ grid). On each vertex $i$ place independent uniform random variables $U_i \in [0,1]$. For each vertex define $X_i = 1$ if $X_i$ is a local maximum and $X_i=0$ else. Then $\Pa(X_i = 1)=1/(d_i+1)$ where $d_i$ is the number of neighbours of $X_i$. Let $W = \sum_{i \in I} X_i$, then under some assumptions on the graph,
    \[\Pa \left(\frac{W-\sum_{i \in I}\frac{1}{d_i+1}}{\sqrt{\sum_{i \in I}\frac{1}{d_i}\left(1-\frac{1}{d_i}\right)}}\le x\right)\approx \Phi(x). \]
\end{ex}
\begin{ex}
    Pick $i \in [n] = \{1,2,\ldots, n\}$ uniformly at random and let $W$ be the number of 1's in the binary expansion of $i$. Then 
    \[W = \sum_{j=1}^{\log_2(n+1)}X_j,\]
    where 
    \[X_j = \begin{cases}
        1 & \text{if the } j^{th} \text{ bit of } i \text{ equals 1},\\
        0 & \text{else.}  
    \end{cases}\]
    Then,
    \[\Pa\left(\frac{W-\frac{\log_2(n)}{2}}{\sqrt{n/4}}\right) \to \Phi(x). \]
\end{ex}
\subsection{Lindeberg's condition}
We will use Lindeberg's form of the central limit theorem and the proof will use the idea of coupling. First some notation
\begin{defn}
    A \emph{triangular array of random variables} is a collection of random variables
    \[\begin{matrix}
        X_{1,1},\\
        X_{2,1}, &X_{2,2}\\
        \vdots & \vdots&\ddots \\
        X_{n,1} &X_{n,2} & \ldots &X_{n,k_n}\\
        \vdots &\vdots & &\vdots&\ddots  
    \end{matrix} \] 
    We will say that a triangular array has \emph{independent rows} if for every $n$, the random variables $X_{n,1},\ldots,X_{n,k_n}$ are independent.
\end{defn}
The idea of triangular arrays is to generalize the notion of i.i.d. sequences where now we allow for the distribution of $X_i$ to depend on $n$.

Suppose we have a triangular array with independent rows such that $\E[X_{i,n}]=0$ and  $\sigma_{n,i}^2 = \V(X_{n,i}) < \infty$. Then define $\s_n^2 = \sum_{i=1}^{k_n}\sigma_{n,i}^2.$ With this notation we have
\begin{thrm}[Linderberg]
    Suppose that for all $\eps > 0$ we have 
    \begin{equation}\label{lind}\frac{1}{\s_n^2}\sum_{i=1}^{k_n}\int_{\{\abs{X_{i,n}} > \eps\s_n\}}\abs{X_{i,n}}^2d\Pa \stackrel{n \to \infty}{\longrightarrow} 0. 
    \end{equation}
    Then 
    \[\Pa\left(\frac{S_n}{\s_n} \le x\right)\to \Phi(x). \]
\end{thrm}
We will call equation \eqref{lind} \emph{Linderberg's condition}.
\begin{ex}
    Suppose we have an i.i.d. sequence $(X_i)_{i =1}^\infty$ with mean 0 and variance $\sigma^2$. Then 
    \[\Pa\left(\frac{S_n}{\sigma\sqrt{n}} \le x\right)\to \Phi(x). \]
    To see why this holds consider the triangular array 
    \[\begin{matrix}
        X_1\\X_1&X_2\\X_1&X_2&X_3\\ \vdots &&&\ddots
    \end{matrix}\]
    which clearly has independent rows. Note that $\s_n^2 = n \sigma^2$ and so we can check Linderberg's condition \eqref{lind}
    \begin{align*}
        \frac{1}{s_n^2}\sum_{i=1}^n \int_{\{\abs{X_i} > \eps\s_n\}}\abs{X_{i}}^2d\Pa&=\frac{1}{n \sigma^2}\sum_{i=1}^n \int_{\{\abs{X_i} > \eps\sqrt{n} \sigma\}}\abs{X_{i}}^2d\Pa\\
        &= \frac{1}{\sigma^2}\int_{\{\abs{X_1} > \eps \sqrt{n}\sigma\}}\abs{X_1}^2d\Pa\\
        &\to 0,
    \end{align*}
    by the DCT. 
\end{ex}
\begin{ex}
    We can apply this to our ESP card guessing example. Consider cards labelled $1,\ldots, n$. The cards are guessed one at a time and each time complete feedback is given. Define 
    \[X_{n,i} = \begin{cases}
        1 & \text{if guess $i$ is correct},\\
        0 & \text{else}.
    \end{cases} \] 
    Then $\Pa(X_{n,i}=1) = \frac{1}{n-i+1}$ since we are given complete feedback and will never guess a card that we have seen already (and we don't have ESP!). Define 
    \[Y_i = X_{n,i} - \frac{1}{n-i+1}. \]
    We then have $\E[Y_i]=0$ and $\V(Y_i) = \V(X_i) = \frac{1}{n-i+1}\left(1-\frac{1}{n-i+1}\right)$. Thus
    \begin{align*}
        \s_n^2 &= \sum_{i=1}^n \frac{1}{n}\left(1-\frac{1}{n}\right)\\
        &= \log(n)-\gamma+\frac{\pi^2}{6}+O(1/n)\\
        &\sim \log(n),
    \end{align*}
    where $\gamma$ is Euler's constant (Persi directs interested students to ``\href{https://www.ams.org/journals/bull/2013-50-04/S0273-0979-2013-01423-X/S0273-0979-2013-01423-X.pdf}{Euler's constant: Euler's work and modern developments}'' by Lagarias for the AMS Bulliten). Thus we can check Linderberg's condition \eqref{lind}. Note that
    \[\frac{1}{\s_n^2}\sum_{i=1}^{n}\int_{\{\abs{Y_{i,n}} > \eps\s_n\}}\abs{X_{i,n}}^2d\Pa\approx \frac{1}{\log(n)}\sum_{i=1}^{n}\int_{\{\abs{X_{i,n}} > \eps\sqrt{\log(n)}\}}\abs{Y_{i,n}}^2d\Pa, \]
    and the right hand side is 0 for sufficiently large $n$ because $\abs{Y_{i,n}}$ is bounded by 1. Thus
    \[\Pa\left(\frac{S_n}{\s_n}\le x\right)\to \Phi(x). \]
\end{ex}
\subsection{Weak$^*$ convergence}
\begin{defn}
    Let $Q_n,Q$ be probability measures on $\R^k$ with the Borel subsets. We will say that $Q_n$ \emph{converges weak$^*$} to $Q$  and write $Q_n \Rightarrow Q$ if for all bounded and continuous $g: \R^k \to \R$,
    \[\int_{\R^k}g dQ_n \longrightarrow \int_{\R^k}g dQ. \]  
\end{defn}
\begin{thrm}[Portmanteau]
    Let $Q_n,Q$ be probability measures on $\R^k$, then the following are equivalent:
    \begin{enumerate}
        \item The sequence of measures $Q_n$ converges weak$^*$ to $Q$.
        \item For all $f:\R^k \to \R$ which are continuous with compact support, $\int f dQ_n \to \int f dQ$.
        \item For all $f : \R^k \to \R$ which are smooth with compact support, $\int fdQ_n \to \int fdQ$.
        \item If $F_n$ is the cdf of $Q_n$ and $F$ if the cdf of $Q$ (so that $F_n(x) = Q_n(\{y \in \R^k: y_i \le x_i\})$ and $F(x) = Q(\{y \in \R^k: y_i \le x_i\})$), then $F_n(x) \to F(x)$ at all points $x \in \R^k$ where $F$ is continuous. 
    \end{enumerate}
\end{thrm}
\begin{proof}
    We will start the proof today and finish it next week. We will also only prove the case when $k=1$. The general case is essentially the same but one has to work with the norm $\norm{x}=\sqrt{\sum_{i=1}^k x_i^2}$ instead of the absolute value.
    
    Note that we trivially have (a) implies (b) and (b) implies (c). We will first prove (b) implies (a). Let $f:\R \to \R$ be a continuous function such that $\abs{f(x)} \le c$ for all $x$. Given $\eps >0$ there exists $N$ such that 
    \[Q(\{\abs{x}> N\}) \le \frac{\eps}{4C}. \]
    Define \[\ta_N(x) = \begin{cases}
        1 & \text{if } \abs{x} \le N,\\
        0 & \text{if } \abs{x} \ge N+1,\\
        N+1-\abs{x} &\text{if } \abs{x} \in (N,N+1).
    \end{cases}\] 
    The function $\ta_N$ is continuous and has compact support. Thus
    \begin{align*}
        \underline{\lim}Q_n(\{\abs{x} \le N+1\}) &\ge \underline{\lim}\int \ta_N dQ_n\\
        &= \int \ta_n dQ\\
        &\ge 1-\frac{\eps}{4C}.
    \end{align*}
    Thus
    \begin{align*}
        \overline{\lim}Q_n(\{\abs{x}>N+1\})&=1-\underline{\lim}Q_n(\{\abs{x}\le N+1\})\\
        &= \frac{\eps}{4C}.
    \end{align*}
    Now let $g=f\ta_{N+1}$. The function $g$ is continuous with compact support. We have 
    \begin{align*}
        \abs{\int f dQ_n - \int f dQ} &\le \abs{\int f dQ_n - \int g dQ_n} + \abs{\int g dQ_n - \int g dQ}+\abs{\int g dQ-\int fdQ}.
    \end{align*}
    The middle term goes to 0 since we have assumed (b). Furthermore, $f=g$ when $\abs{x} \le N+1$. Thus
    \begin{align*}
        \overline{\lim} \abs{\int f dQ_n - \int g dQ_n} &\le\overline{\lim}\int_{\abs{x} > N+1} \abs{f(x)} dQ_n\\
        &\le C\overline{\lim} Q_n(\{\abs{x}>N+1\})\\
        &\le \frac{\eps}{4}.
    \end{align*}
    Likewise 
    \begin{align*}
        \overline{\lim} \abs{\int f dQ - \int g dQ} &= \abs{\int f dQ - \int g dQ}\\
        &\le\int_{\abs{x} > N+1} \abs{f(x)} dQ\\
        &\le C Q(\{\abs{x}>N+1\})\\
        &\le C Q(\{\abs{x}>N\})\\
        &\le \frac{\eps}{4}.
    \end{align*}
    Thus we have 
    \begin{align*}
        \overline{\lim}\abs{\int f dQ_n - \int f dQ}\le \frac{\eps}{2}.
    \end{align*}
    Letting $\eps \to 0$ we see that $\int f dQ_n \to \int f dQ$ proving (b) implies (a).

    We will now prove that (c) implies (b). Let $f$ be continuous with compact support for $\eta > 0$ define
    \[\rho_\eta(x) = Z(\eta)e^{-\frac{1}{1-\frac{x^2}{\eta^2}}}\delta_{[-\eta,\eta]}(x), \]
    where $Z(\eta)$ is such that $\int_{-\eta}^\eta \rho_\eta(x)dx=1$. One can prove by induction that $\rho_\eta$ is smooth. The function $\rho_\eta$ clearly has compact support. 
    
    Now let $\eps > 0$ be given. Since $f$ is continuous with compact support, $f$ is uniformly continuous. Thus there exists $\eta = \eta(\eps)$ such that for all $x. y \in \R$, $\abs{x-y} \le \eta$ implies that $\abs{f(x)-f(y)}\le \eps$. Define
    \[f^\eps(x) = \int_\R f(y)\rho_\eta(x-y)dy = \int_{-\eta}^\eta f(x-y)\rho_\eta(y)dy.\] 
    The function $f^\eps$ is also smooth and $f^\eps$ has compact support since $f$ and $\rho_\eta$ both have compact support. Furthermore 
    \begin{align*}
        \abs{f(x)-f^\eps(x)} & =\abs{\int_{-\eta}^\eta  f(x)\rho_\eta(y)dy-\int_{-\eta}^\eta f(x-y)\rho_\eta(y)dy}\\
        &\le \int_{-\eta}^\eta \abs{f(x)-f(x-y)}\rho_\eta(y)dy\\
        &\le  \int_{-\eta}^\eta\eps \rho_\eta(y)dy\\
        &= \eps.
    \end{align*}
    Since $Q_n$ and $Q$ are both probability measures we thus have 
    \[\abs{\int f dQ_n - \int f^\eps dQ_n } \le \eps~~~\text{and}~~~\abs{\int f dQ - \int f^\eps dQ} \le \eps.\]
    Thus, by a similiar argument to before we have 
    \begin{align*}
        &~\overline{\lim} \abs{\int f dQ_n - \int f dQ}\\
        \le & ~\overline{\lim} \abs{\int f dQ_n - \int f^\eps dQ_n}+\overline{\lim} \abs{\int f^\eps dQ_n - \int f^\eps dQ}+\overline{\lim} \abs{\int f^\eps dQ - \int f dQ}\\
        \le & ~\eps + 0 + \eps \\
        =& ~2\eps.
    \end{align*}
    Letting $\eps \to 0$ we see that $\int fdQ_n \to \int fdQ$. Thus (c) implies (b).
\end{proof}


We will prove that (d) is also equivalent next lecture.

\end{document}
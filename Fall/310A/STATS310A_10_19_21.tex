\documentclass{article}
\usepackage{ae,aecompl}
\usepackage{todonotes}
\usepackage{chngcntr}
\usepackage{tikz-cd}
\usepackage{graphicx}
\graphicspath{ {./images/}}
\usepackage[all,cmtip]{xy}
\usepackage{amsmath, amscd}
\usepackage{amsthm}
\usepackage{amssymb}
\usepackage{amsfonts}
\usepackage{bm}
\usepackage{qsymbols}
\usepackage{latexsym}
\usepackage{mathrsfs}
\usepackage{mathtools}
\usepackage{cite}
\usepackage{color}
\usepackage{url}
\usepackage{enumerate}
\usepackage{verbatim}
\usepackage[draft=false, colorlinks=true]{hyperref}
\usepackage{pdfpages}
\usepackage[margin=1.2in]{geometry}
\usepackage{IEEEtrantools}

\usepackage{fancyhdr}


\usepackage[nameinlink]{cleveref}


\DeclareMathOperator*{\ac}{accept}
\DeclareMathOperator*{\amax}{argmax}
\DeclareMathOperator*{\amin}{argmin}
\DeclareMathOperator*{\Aut}{Aut}
\newcommand {\al}{{\alpha}}
\newcommand {\abs}[1]{{\left\lvert#1\right\rvert}}
\newcommand {\A}{{\mathcal{A}}}
\newcommand {\AM}{{\mathrm{AM}}}
\newcommand {\AMp}{{\AM_{p}^{X}\!(\Ri_\w)}}
\newcommand {\B}{{\mathcal{B}}}
\DeclareMathOperator*{\Be}{Bern}
\newcommand {\Br}{{\dot{B}}}
\newcommand {\Ba}{{\mathfrak{B}}}
\newcommand {\C}{{\mathbb C}}
\newcommand {\ce}{\mathrm{c}}
\newcommand {\Ce}{\mathrm{C}}
\newcommand {\Cc}{\mathrm{C_{c}}}
\newcommand {\Ccinf}{\mathrm{C_{c}^{\infty}}}
\DeclareMathOperator{\cov}{Cov}
\DeclareMathOperator{\DEV}{DEV}
\newcommand {\Di}{{\mathbb D}}
\newcommand {\dom}{\mathrm{dom}}
\newcommand{\dist}{\stackrel{\mathrm{dist}}{=}}
\newcommand {\ud}{\mathrm{d}}
\newcommand {\ue}{\mathrm{e}}
\newcommand {\eps}{\varepsilon}
\newcommand {\veps}{\varepsilon}
\newcommand {\vrho}{{\varrho}}
\newcommand {\E}{{\mathbb{E}}}
\newcommand {\Ec}{{\mathcal{E}}}
\newcommand {\Ell}{L}
\newcommand {\Ellp}{{L_{p}[0,1]}}
\newcommand {\Ellpprime}{{L_{p'}([0,1])}}
\newcommand {\Ellq}{{L_{q}([0,1])}}
\newcommand {\Ellqprime}{{L_{q'}([0,1])}}
\newcommand {\Ellr}{L^{r}}
\newcommand {\Ellone}{{L_{1}([0,1])}}
\newcommand{\Elltwo}{{L_{2}([0,1])}}
\newcommand{\Ellinfty}{L^{\infty}}
\newcommand{\Ellinftyc}{L_{\mathrm{c}}^{\infty}}
\newcommand{\exb}[1]{\exp\left\{#1\right\}}
\DeclareMathOperator*{\Ext}{Ext}
\newcommand{\F}{{\mathcal{F}}}
\newcommand{\Fe}{{\mathbb{F}}}
\newcommand{\G}{{\mathcal{G}}}
\newcommand{\HF}{\mathcal{H}_{\text{FIO}}^{1}(\Rd)}
\newcommand{\Hr}{H}
\newcommand{\HT}{\mathcal{H}}
\newcommand{\ui}{\mathrm{i}}
\newcommand{\I}{{I}}
\newcommand{\J}{{\mathcal{J}}}
\newcommand{\id}{{\mathrm{id}}}
\newcommand{\iid}{\stackrel{\mathclap{\normalfont\mbox{iid}}}{\sim}}
\newcommand{\im}{{\text{im }}}
\newcommand{\ind}{{\perp\!\!\!\perp}}
\DeclareMathOperator*{\Int}{int}
\newcommand{\intx}{{\overline{\int_{X}}}}
\newcommand{\inte}{{\overline{\int_{\E}}}}
\newcommand{\la}{\lambda}
\newcommand{\rb}{\rangle}
\newcommand{\lb}{{\langle}}
\newcommand{\La}{\Lambda}
\newcommand{\calL}{{\mathcal{L}}}
\newcommand{\lp}{{\mathcal{L}}^{p}}
\newcommand{\lpo}{{\overline{\mathcal{L}}^{p}\!}}
\newcommand{\Lpo}{{\overline{\Ell}^{p}\!}}
\newcommand{\M}{{\mathbf{M}}}
\newcommand{\Ma}{{\mathcal{M}}}
\newcommand{\N}{{{\mathbb N}}}
\newcommand{\Na}{{{\mathcal{N}}}}
\newcommand{\norm}[1]{\left\|#1\right\|}
\newcommand{\normm}[1]{{\left\vert\kern-0.25ex\left\vert\kern-0.25ex\left\vert #1 
    \right\vert\kern-0.25ex\right\vert\kern-0.25ex\right\vert}}
\newcommand{\Om}{{{\Omega}}}
\newcommand{\one}{{{\bf 1}}}
\newcommand{\pic}{\text{Pic }}
\newcommand{\ph}{{\varphi}}
\newcommand{\Pa}{{\mathbb{P}}}
\newcommand{\Po}{{\mathcal{P}}}
\newcommand{\Q}{{\mathbb{Q}}}
\newcommand{\R}{{\mathbb R}}
\newcommand{\Rd}{{\mathbb{R}^{d}}}
\DeclareMathOperator{\rej}{reject }
\newcommand{\Rn}{{\mathbb{R}^{n}}}
\newcommand{\cR}{{\mathcal{R}}}
\newcommand{\Rad}{{\mathrm{Rad}}}
\newcommand{\ran}{{\mathrm{ran}}}
\newcommand{\Ri}{{\mathrm{R}}}
\newcommand{\supp}{{\mathrm{supp}}}
\newcommand{\Se}{\mathrm{S}}
\newcommand{\Sp}{S^{*}(\Rn)}
\newcommand{\St}{{\mathrm{St}}}
\newcommand{\Sw}{\mathcal{S}}
\newcommand{\T}{{\mathcal{T}}}
\newcommand{\ta}{{\theta}}
\newcommand{\Ta}{{\Theta}}
\newcommand{\topp}{\stackrel{p}{\to}}
\newcommand{\todd}{\stackrel{d}{\to}}
\newcommand{\toL}[1]{\stackrel{L^{#1}}{\to}} 
\newcommand{\toas}{\stackrel{a.s.}{\to}}
\DeclareMathOperator{\V}{Var}
\newcommand {\w}{{\omega}}
\newcommand {\W}{{\mathrm{W}}}
\newcommand {\Wnp}{\text{$\mathrm{W}$\textsuperscript{$n,\!p$}}}
\newcommand {\Wnpeq}{\text{$\mathrm{W}$\textsuperscript{$n\!,\!p$}}}
\newcommand {\Wonep}{\text{$\mathrm{W}$\textsuperscript{$1,\!p$}}}
\newcommand {\Wonepeq}{\text{$\mathrm{W}$\textsuperscript{$1\!,\!p$}}}
\newcommand {\X}{{\mathcal{X}}}
\newcommand {\Z}{{{\mathbb Z}}}
\newcommand {\Za}{{\mathcal{Z}}}
\newcommand {\Zd}{{\Z[\sqrt{d}]}}
\newcommand {\vanish}[1]{\relax}

\newcommand {\wh}{\widehat}
\newcommand {\wt}{\widetilde}
\newcommand {\red}{\color{red}}

% Distributions
\newcommand{\normal}{\mathsf{N}}
\newcommand{\poi}{\mathsf{Poisson}}
\newcommand{\bern}{\mathsf{Bernoulli}}
\newcommand{\bin}{\mathsf{Binomal}}
\newcommand{\multi}{\mathsf{Multinomial}}
\newcommand{\Exp}{\mathsf{Exp}}



% put your command and environment definitions here




% some theorem environments
% remove "[theorem]" if you do not want them to use the same number sequence


  \newtheorem{thrm}{Theorem}
  \newtheorem{lemma}{Lemma}
  \newtheorem{prop}{Proposition}
  \newtheorem{cor}{Corollary}

  \newtheorem{conj}{Conjecture}
  \renewcommand{\theconj}{\Alph{conj}}  % numbered A, B, C etc

  \theoremstyle{definition}
  \newtheorem{defn}{Definition}
  \newtheorem{ex}{Example}
  \newtheorem{exs}{Examples}
  \newtheorem{question}{Question}
  \newtheorem{remark}{Remark}
  \newtheorem{notn}{Notation}
  \newtheorem{exer}{Exercise}




\title{STATS310A - Lecture 9}
\author{Persi Diaconis\\ Scribed by Michael Howes}
\date{10/19/21}

\pagestyle{fancy}
\fancyhf{}
\rhead{STATS310A - Lecture 6}
\lhead{10/19/21}
\rfoot{Page \thepage}

\begin{document}
\maketitle
\tableofcontents
\section{Announcements}
\begin{itemize}
    \item Homework
    \begin{itemize}
        \item Read sections 16 and 18.
        \item Do problems 16.4, 16.7, 18.1, 18.2, 18.4, 18.10.
    \end{itemize}
    \item Exam on Thursday 5-7 Oct 28th.
\end{itemize}
\section{Convergence theorems}
Last time we saw the following. If $(\Om,\F,\mu)$ is a measure space and $f: \Om \to \R$ is measurable, then define 
\[\int fd\mu = \int f_+ d\mu - \int f_-d\mu, \]
where if $g \ge 0$, we define
\[\int g d\mu = \sup \sum_{i=1}^n \nu_i \mu(A_i), \]
where $\nu_i = \inf\limits_{\w \in A_i} g(\w)$ and the supremum is over all partitions of $\Om$ by elements of $\F$.


Recall that if $\{x_i\}_{i=1}^\infty$ are real, then 
\[\liminf_{n \to \infty} x_n = \lim_n \inf_{k \ge n}x_k, \]
and 
\[\limsup_{n \to \infty} x_n = -\liminf_{n \to \infty} (-x_n). \]

\begin{lemma}
    \emph{[Fatou's Lemma]} Let $\{f_n\}$ be any sequence of measurable non-negative functions, then 
    \[\int \liminf_n f d\mu \le \liminf_n \int f_nd\mu.  \]
\end{lemma}
\begin{proof}
    Define $g_n = \inf_{k \ge n} f_k$. Then $g_n \le f_n$ for all $n$ and $g_n \nearrow \liminf_n f_n$ by definition. By the monotone convergence theorem and monotonicity:
    \begin{align*}
        \int \liminf_n g_n d\mu &= \int \lim_n g_n d\mu\\
        &=\lim_n\int g_n d\mu\\
        &=\liminf_n \int g_n d\mu \\
        &\le \liminf_n \int f_n d\mu. \qedhere
    \end{align*} 
\end{proof}
\begin{ex}
If $(\Om,\F,\mu) = ([0,1], \B,\la)$ (Lebesgue measure on the Borel subsets of $[0,1]$). Define 
\[f_n = \begin{cases}
    \delta_{[0,1/2)} &\text{if $n$ is even},\\
    \delta_{[1/2,1]} &\text{else}.
\end{cases} \]
Then $\liminf_n f_n =0$ and $\int f_n d\mu = 1/2$. Thus $\int \liminf f_n d\mu \le \liminf_n \int f_nd\mu$ as promised. Note that the sequence $f_n$ is not convergent. In Fatou's lemma the sequence is not required to be convergent. The $\liminf$ of a sequence always exists.
\end{ex}
\begin{ex}
    Note that we need $f_n \ge 0$. If $f_n = -\frac{1}{n}\delta_{[n,2n]}$, then $f_n \to 0$ uniformly but $\int f_nd\mu = -1$ for all $n$ and $0 \le -1$ is not true.
\end{ex}
\begin{ex}
    Let $(r_i)_{i=1}^\infty$  be a the usual enumeration of $\Q \cap [0,1]$,
    \[(r_i) = (0,1,1/2,1/3,2/3,1/4,3/4,1/5,2/5,3/5,4/5,1/6,5/6,\ldots). \]
    Define 
    \[f(\w) = \sum_{i=1}^\infty \frac{1}{i^2\abs{r_i-\w}^{1/2}}. \]
    Note that $f(\w) = \infty$ if $\w$ is rational. Is $f(\w)$ finite for any values of $\w$? Define $f_n(\w) = \sum_{i=1}^n \frac{1}{i^2\abs{r_i-\w}^{1/2}}$, we have $f_n \to f$ pointwise. We also have 
    \[\int_0^1 f_n(\w)d\w = \sum_{i=1}^n \frac{1}{i^2} \int_0^1\frac{1}{\abs{r_i-\w}^2}d\w \le \sum_{i=1}^n \frac{c}{i^2} \le C. \]
    Thus, by Fatou's, $\int f d\w \le C$ and thus $f(\w)$ is finite for almost every $\w \in [0,1]$. Exercise: find a single $\w$ such that $f(\w) < \infty$ [hint: take $\w = 1/\sqrt{2}$].
\end{ex}
\begin{thrm}
    \emph{[Dominated Convergence Theorem (DCT)]} Let $(\Om,\F,\mu)$ be a measure space and let $f,f_n:\Om \to \R$ be measurable and suppose that $f_n(\w) \to f(\w)$ a.e.. Suppose also that there exists a measurable function $g$ such that $\abs{f_n(\w)} \le g(\w)$ for a.e $\w$ and $\int g d\mu < \infty$. Then $f$ is integrable and 
    \[\int f d\mu = \lim_n \int f_n d\mu. \] 
\end{thrm}
\begin{proof}
    Note that $g+f_n, g-f_n \ge 0$. Define $f_* = \liminf_n f_n$ and $f^* = \limsup_n f_n$. We know that $\abs{f_*}, \abs{f^*} \le g$ and so $f_*$ and $f^*$ are integrable. Futhermore
    \begin{align*}
        \int gd\mu + \int f_* d\mu &= \int (g+f_*)d\mu\\
        &= \int \liminf_n (g+f_n)d\mu \\
        &\le \liminf_n \int g+f_nd\mu \\
        &= \int gd\mu + \liminf_n \int f_n d\mu.
    \end{align*}
    Thus $\int f_*d\mu \le \liminf_n \int f_n d\mu$. Likewise we have
    \begin{align*}
        \int g d\mu - \int f^* d\mu &= \int g-f^* d\mu \\
        &= \int g - \limsup_n f_n d\mu \\
        &= \int g + \liminf_n (-f_n)d\mu \\
        &= \int \liminf_n(g-f_n) d\mu\\
        & \le \liminf_n \int g-f_n d\mu \\
        & = \int gd\mu +\liminf_n\left(-\int f_n d\mu\right)\\
        &= \int g d\mu - \limsup_n \int f_n d\mu.
    \end{align*}
    And thus $\int f^* d \mu \ge \limsup_n \int f_n d\mu$. But we know that $f_*=f^*=f$. Thus we have 
    \[\limsup_n \int f_n d\mu \le \int f^*d\mu \le \liminf_n \int f_n d\mu. \]
    Thus $\int f_n d\mu \to \int f d\mu$.
\end{proof}
Note that we couldn't use Fatou's lemma directly since $f_n$ need not be non-negative. We had to apply Fatou's lemma seperately to $g+f_n$ and $g-f_n$.
\section{New measures from old}
\begin{defn}
    Let $(\Om,\F,\mu)$ be a measure space. A \emph{probability density} is a function $f: \Om \to \R$ s.t $f$ is measure, $f(\w) \ge 0$ for all $\w$ and $\int fd\mu = 1$. 
\end{defn}
\begin{prop}
    Suppose that $(\Om,\F,\mu)$ is a measure space and $f$ is a probability density, then the function $\nu : \F \to \R$ given by
    \[\nu(A) = \int_A f d\mu = \int_\Om f\delta_A d\mu, \]
    is a probability measure.
\end{prop}
\begin{proof}
    Note that $\nu(\emptyset) = \int_\Om 0 d\mu = 1$ and $\nu(\Om) = \int_\Om f d\mu = 1$. Furthermore if $\{A_i\}_{i=1}^\infty$ are disjoint and $A = \bigcup_{i=1}^\infty A_i$, then 
    \[f\delta_A =\sum_{i=1}^\infty f\delta_{A_i}.  \]
    Thus the functions $g_n = \sum_{i=1}^n f\delta_{A_i}$ are non-negative and $g_n \nearrow f\delta_A$. Thus by the monotone convergence theorem:
    \[\nu(A) = \int f \delta_A d\mu = \lim_{n} \int \sum_{i=1}^n f\delta_{A_i}d\mu =\lim_n \sum_{i=1}^n \int f\delta_{A_i}d\mu = \lim_n \sum_{i=1}^n \nu(A_i) =\sum_{i=1}^\infty \nu(A_i). \]
    Thus $\nu$ is countably additive. Furthermore $\nu$ is non-negative since $f$ is non-negative.
\end{proof}
\section{Change of measure}
Suppose $(\Om,\F,\mu)$ is a measure space and $(\Om',\F')$ is a measurable space and $T:\Om \to \Om'$ is measurable. We have previously seen that $\mu^{T^{-1}}$ given by
\[\mu^{T^{-1}}(B) = \mu(T^{-1}(B)), \]
is a measure on $(\Om',\F')$. Can we relate integrals on $(\Om',\F', \mu^{T^{-1}})$ to integrals on $(\Om,\F,\mu)$? Yes!
\begin{thrm}
    Suppose that $f: \Om'\to \R$ is measurable and integrable with respect to $\mu^{T^{-1}}$, then $f \circ T:\Om \to \R$ is integrable and 
    \[\int_\Om f\circ T d\mu = \int_{\Om'} f d\mu^{T^{-1}}. \]
\end{thrm}
\begin{proof}
    This is the classic (1), (2), (3) argument.
    \begin{enumerate}
        \item[(0)] First we write $f = f_+-f_-$ and then note that $(f\circ T)_+ = f_+\circ T$ and $(f\circ T)_- = f_- \circ T$. Thus we may assume $f \ge 0$.
        \item[(1)] Suppose $f = \delta_B$, $B \in \F'$.
        \item[(2)] Prove it for finite linear combinations.
        \item[(3)] Monotone convergence + approximation by simple functions.   
    \end{enumerate}
    For (1) note that $\delta_A \circ T = \delta_{T^{-1}(A)}$. Thus 
    \[\int_{\Om'} \delta_A d\mu^{T^{-1}} = \mu^{T^{-1}}(A) = \mu(T^{-1}(A)) =\int_{\Om} \delta_{T^{-1}(A)}d\mu = \int_{\Om} \delta_A \circ Td\mu.  \]
    For (2) note that both sides are linear in $f$. Finally for (3) note that if $f_n \nearrow f$, then $f_n\circ T \nearrow f\circ T $.
\end{proof}
I reckon we will see (1), (2), (3) again when we prove Fubinni's theorem.
\begin{ex}
    Suppose $\Om = \{0,1\}^n$, $\F=$ all subsets of $\Om$, $\mu(A) = \frac{\abs{A}}{2^n}$. Suppose also that $\Om' = \{0,1,\ldots, \}$ and $\F'$ is again the discrete $\sigma$-algebra. Define $T(\w) = \sum_{i=1}^n \w_i$ and thus $\mu^{T^{-1}}(\{j\})=\frac{1}{2^n}\binom{n}{j}$. If $f':\Om' \to \R$, then the change of measure formula states that 
    \[\sum_{\w \in \{0,1\}^n} \frac{1}{2^n} f\left(\sum_{i=1}^n \w_i\right) = \sum_{j=1}^n \frac{1}{2^n}\binom{n}{j}f(j).\]

\end{ex}
\end{document}
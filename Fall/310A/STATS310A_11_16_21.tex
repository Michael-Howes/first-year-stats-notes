\documentclass{article}
\usepackage{ae,aecompl}
\usepackage{todonotes}
\usepackage{chngcntr}
\usepackage{tikz-cd}
\usepackage{graphicx}
\graphicspath{ {./images/}}
\usepackage[all,cmtip]{xy}
\usepackage{amsmath, amscd}
\usepackage{amsthm}
\usepackage{amssymb}
\usepackage{amsfonts}
\usepackage{bm}
\usepackage{qsymbols}
\usepackage{latexsym}
\usepackage{mathrsfs}
\usepackage{mathtools}
\usepackage{cite}
\usepackage{color}
\usepackage{url}
\usepackage{enumerate}
\usepackage{verbatim}
\usepackage[draft=false, colorlinks=true]{hyperref}
\usepackage{pdfpages}
\usepackage[margin=1.2in]{geometry}
\usepackage{IEEEtrantools}

\usepackage{fancyhdr}


\usepackage[nameinlink]{cleveref}


\DeclareMathOperator*{\ac}{accept}
\DeclareMathOperator*{\amax}{argmax}
\DeclareMathOperator*{\amin}{argmin}
\DeclareMathOperator*{\Aut}{Aut}
\newcommand {\al}{{\alpha}}
\newcommand {\abs}[1]{{\left\lvert#1\right\rvert}}
\newcommand {\A}{{\mathcal{A}}}
\newcommand {\AM}{{\mathrm{AM}}}
\newcommand {\AMp}{{\AM_{p}^{X}\!(\Ri_\w)}}
\newcommand {\B}{{\mathcal{B}}}
\DeclareMathOperator*{\Be}{Bern}
\newcommand {\Br}{{\dot{B}}}
\newcommand {\Ba}{{\mathfrak{B}}}
\newcommand {\C}{{\mathbb C}}
\newcommand {\ce}{\mathrm{c}}
\newcommand {\Ce}{\mathrm{C}}
\newcommand {\Cc}{\mathrm{C_{c}}}
\newcommand {\Ccinf}{\mathrm{C_{c}^{\infty}}}
\DeclareMathOperator{\cov}{Cov}
\DeclareMathOperator{\DEV}{DEV}
\newcommand {\Di}{{\mathbb D}}
\newcommand {\dom}{\mathrm{dom}}
\newcommand{\dist}{\stackrel{\mathrm{dist}}{=}}
\newcommand {\ud}{\mathrm{d}}
\newcommand {\ue}{\mathrm{e}}
\newcommand {\eps}{\varepsilon}
\newcommand {\veps}{\varepsilon}
\newcommand {\vrho}{{\varrho}}
\newcommand {\E}{{\mathbb{E}}}
\newcommand {\Ec}{{\mathcal{E}}}
\newcommand {\Ell}{L}
\newcommand {\Ellp}{{L_{p}[0,1]}}
\newcommand {\Ellpprime}{{L_{p'}([0,1])}}
\newcommand {\Ellq}{{L_{q}([0,1])}}
\newcommand {\Ellqprime}{{L_{q'}([0,1])}}
\newcommand {\Ellr}{L^{r}}
\newcommand {\Ellone}{{L_{1}([0,1])}}
\newcommand{\Elltwo}{{L_{2}([0,1])}}
\newcommand{\Ellinfty}{L^{\infty}}
\newcommand{\Ellinftyc}{L_{\mathrm{c}}^{\infty}}
\newcommand{\exb}[1]{\exp\left\{#1\right\}}
\DeclareMathOperator*{\Ext}{Ext}
\newcommand{\F}{{\mathcal{F}}}
\newcommand{\Fe}{{\mathbb{F}}}
\newcommand{\G}{{\mathcal{G}}}
\newcommand{\HF}{\mathcal{H}_{\text{FIO}}^{1}(\Rd)}
\newcommand{\Hr}{H}
\newcommand{\HT}{\mathcal{H}}
\newcommand{\ui}{\mathrm{i}}
\newcommand{\I}{{I}}
\newcommand{\J}{{\mathcal{J}}}
\newcommand{\id}{{\mathrm{id}}}
\newcommand{\iid}{\stackrel{\mathclap{\normalfont\mbox{iid}}}{\sim}}
\newcommand{\im}{{\text{im }}}
\newcommand{\ind}{{\perp\!\!\!\perp}}
\DeclareMathOperator*{\Int}{int}
\newcommand{\intx}{{\overline{\int_{X}}}}
\newcommand{\inte}{{\overline{\int_{\E}}}}
\newcommand{\la}{\lambda}
\newcommand{\rb}{\rangle}
\newcommand{\lb}{{\langle}}
\newcommand{\La}{\Lambda}
\newcommand{\calL}{{\mathcal{L}}}
\newcommand{\lp}{{\mathcal{L}}^{p}}
\newcommand{\lpo}{{\overline{\mathcal{L}}^{p}\!}}
\newcommand{\Lpo}{{\overline{\Ell}^{p}\!}}
\newcommand{\M}{{\mathbf{M}}}
\newcommand{\Ma}{{\mathcal{M}}}
\newcommand{\N}{{{\mathbb N}}}
\newcommand{\Na}{{{\mathcal{N}}}}
\newcommand{\norm}[1]{\left\|#1\right\|}
\newcommand{\normm}[1]{{\left\vert\kern-0.25ex\left\vert\kern-0.25ex\left\vert #1 
    \right\vert\kern-0.25ex\right\vert\kern-0.25ex\right\vert}}
\newcommand{\Om}{{{\Omega}}}
\newcommand{\one}{{{\bf 1}}}
\newcommand{\pic}{\text{Pic }}
\newcommand{\ph}{{\varphi}}
\newcommand{\Pa}{{\mathbb{P}}}
\newcommand{\Po}{{\mathcal{P}}}
\newcommand{\Q}{{\mathbb{Q}}}
\newcommand{\R}{{\mathbb R}}
\newcommand{\Rd}{{\mathbb{R}^{d}}}
\DeclareMathOperator{\rej}{reject }
\newcommand{\Rn}{{\mathbb{R}^{n}}}
\newcommand{\cR}{{\mathcal{R}}}
\newcommand{\Rad}{{\mathrm{Rad}}}
\newcommand{\ran}{{\mathrm{ran}}}
\newcommand{\Ri}{{\mathrm{R}}}
\newcommand{\supp}{{\mathrm{supp}}}
\newcommand{\Se}{\mathrm{S}}
\newcommand{\Sp}{S^{*}(\Rn)}
\newcommand{\St}{{\mathrm{St}}}
\newcommand{\Sw}{\mathcal{S}}
\newcommand{\T}{{\mathcal{T}}}
\newcommand{\ta}{{\theta}}
\newcommand{\Ta}{{\Theta}}
\newcommand{\topp}{\stackrel{p}{\to}}
\newcommand{\todd}{\stackrel{d}{\to}}
\newcommand{\toL}[1]{\stackrel{L^{#1}}{\to}} 
\newcommand{\toas}{\stackrel{a.s.}{\to}}
\DeclareMathOperator{\V}{Var}
\newcommand {\w}{{\omega}}
\newcommand {\W}{{\mathrm{W}}}
\newcommand {\Wnp}{\text{$\mathrm{W}$\textsuperscript{$n,\!p$}}}
\newcommand {\Wnpeq}{\text{$\mathrm{W}$\textsuperscript{$n\!,\!p$}}}
\newcommand {\Wonep}{\text{$\mathrm{W}$\textsuperscript{$1,\!p$}}}
\newcommand {\Wonepeq}{\text{$\mathrm{W}$\textsuperscript{$1\!,\!p$}}}
\newcommand {\X}{{\mathcal{X}}}
\newcommand {\Z}{{{\mathbb Z}}}
\newcommand {\Za}{{\mathcal{Z}}}
\newcommand {\Zd}{{\Z[\sqrt{d}]}}
\newcommand {\vanish}[1]{\relax}

\newcommand {\wh}{\widehat}
\newcommand {\wt}{\widetilde}
\newcommand {\red}{\color{red}}

% Distributions
\newcommand{\normal}{\mathsf{N}}
\newcommand{\poi}{\mathsf{Poisson}}
\newcommand{\bern}{\mathsf{Bernoulli}}
\newcommand{\bin}{\mathsf{Binomal}}
\newcommand{\multi}{\mathsf{Multinomial}}
\newcommand{\Exp}{\mathsf{Exp}}



% put your command and environment definitions here




% some theorem environments
% remove "[theorem]" if you do not want them to use the same number sequence


  \newtheorem{thrm}{Theorem}
  \newtheorem{lemma}{Lemma}
  \newtheorem{prop}{Proposition}
  \newtheorem{cor}{Corollary}

  \newtheorem{conj}{Conjecture}
  \renewcommand{\theconj}{\Alph{conj}}  % numbered A, B, C etc

  \theoremstyle{definition}
  \newtheorem{defn}{Definition}
  \newtheorem{ex}{Example}
  \newtheorem{exs}{Examples}
  \newtheorem{question}{Question}
  \newtheorem{remark}{Remark}
  \newtheorem{notn}{Notation}
  \newtheorem{exer}{Exercise}




\title{STATS310A - Lecture 16}
\author{Persi Diaconis\\ Scribed by Michael Howes}
\date{11/16/21}

\pagestyle{fancy}
\fancyhf{}
\rhead{STATS310A - Lecture 16}
\lhead{11/16/21}
\rfoot{Page \thepage}

\begin{document}
\maketitle
\tableofcontents
\section{Announcements}
Final homeowork uploaded to Canvas and due November 30.
\begin{itemize}
    \item Read chpaters 25,26.
    \item Do 25 / 1,3 and 26 / 1,3,12-14.
    \item The hints contain surprises.
\end{itemize}
\section{The central limit theorem}
We are in the process of proving Lindeberg's version of the central limit theorem. Recall that we have a triangular array of random variables $\{X_{n,i}\}$ where $i=1,\ldots, k_n$ and $n=1,2,\ldots$. We assume that the array has independent rows. That is, for each $n$, $\{X_{n,i}\}_{i=1}^{k_n}$ are independent. Assume also that 
\[\E[X_{n,i}]=0 ~~~\text{and}~~~\sigma_{i,n}^2 = \V(X_{n,i})< \infty. \]
Define $S_n = \sum_{i=1}^{k_n}X_{n,i}$ and $s_n^2=\sum_{i=1}^{k_n} \sigma_{i,n}^2 = \V(S_n)$. 
\begin{defn}
    A triangular array  with independent rows $\{X_{n,i}\}$ is said to satisfy \emph{Lindeberg's condtion} if for all $\eps > 0$
    \[\frac{1}{s_n^2} \sum_{i=1}^{k_n} \int_{\{\abs{X_{n,i}}>\eps s_n\}}\abs{X_{n,i}}^2d\Pa \stackrel{n}{\longrightarrow} 0. \]
\end{defn}
Lindeberg's version of the central limit theorem is:
\begin{thrm}[Lindeberg]
    Let $\{X_{n,i}\}$ be a triangular array with independent rows. If $\{X_{n,i}\}$ satisfy Lindeberg's condition, then for all $x \in \R$,
    \[\Pa\left(\frac{S_n}{s_n} \le x\right) \to \Phi(x),\]
    where $\Phi(x) = \Pa(Z \le x)$ for $Z \sim \Na(0,1)$.
\end{thrm}
\begin{proof}
To prove this we will use the portmanteau theorem. Let $C^\infty_c(\R)$ be the class of infinitely differentiable functions on $\R$ with compact support. By the portmanteau theorem it suffices to show that for all $f \in C^\infty_c(\R)$, $\E[f(S_n/s_n)] \stackrel{n}{\to} \E[f(Z)]$ where $Z \sim \Na(0,1)$. Thus fix such an $f$. Define $Z_{n,i}$ is be independent random variables such that $Z_{n,i} \sim \Na(0,\sigma_{n,i}^2)$. Let $Z_n = \sum_{i=1}^{k_n}Z_{n,i}$, Then
\[Z=\frac{1}{s_n}Z_n \sim \Na(0,1). \]
The idea behind the proof is to swap out $X_{n,i}$ for $Z_{n,i}$ one at a time. With this in mind, define
\[T_{n,i}= X_{n,1}+\ldots+X_{n,i-1}+Z_{n,i}+\ldots+Z_{n,k_n}. \]
Note that $X_i,Z_i$ are independent of $T_{n,i}$ for each $i$. Furthermore we have
\[S_n = T_{n,k_n}+X_{n,k_n}~~~\text{and}~~~Z_n = T_{n,1}+Z_{n,1}. \]
And also
\[T_{n,i}+Z_{n,i} = T_{n,i-1}+X_{n,i-1}, \]
for $i=2,\ldots, k_n$. Thus by telescoping we have
\[f\left(\frac{S_n}{s_n}\right)-f\left(\frac{Z_n}{s_n}\right)=\sum_{i=1}^{k_n} f\left(\frac{T_{n,i}+X_{n,i}}{s_n}\right)-f\left(\frac{T_{n,i}+Z_{n,i}}{s_n}\right). \]
And so 
\begin{equation}\label{telescope}
    \abs{\E\left[f\left(\frac{S_n}{s_n}\right)\right]-\E\left[f(Z)\right]}\le \sum_{i=1}^{k_n}\abs{\E\left[f\left(\frac{T_{n,i}+X_{n,i}}{s_n}\right)\right]-\E\left[f\left(\frac{T_{n,i}+Z_{n,i}}{s_n}\right)\right]}
\end{equation}
We will now use Taylor's approximation to bound each of the terms in the above sum. For $x,h \in \R$, define
\[g(h) = \abs{f(x+h)-f(x)-hf'(x)-\frac{h^2}{2}f''(x)}.\]
Since all deriviates of $f$ are bounded, Taylor's approximation with remainder says that there exists $k > 0$ such that for all $h$ and $x$
\[g(h) \le k \min\{\abs{h}^3,\abs{h}^2\}. \]
Thus for all $x,h_1,h_2\in \R$ we have
\begin{align*}
    \abs{f(x+h_1)-f(x+h_2)-f'(x)(h_1-h_2)-\frac{1}{2}f''(x)(h_1^2-h_2^2)}&=\abs{g(h_1)-g(h_2)} \\
    &\le \abs{g(h_1)}+\abs{g(h_2)}.
\end{align*}
We wish to apply this to equation \eqref{telescope} with $x=\frac{T_{n,i}}{s_n}$, $h_1=\frac{X_{n,i}}{s_n}$ and $h_2 = \frac{Z_{n,i}}{s_n}$. Thus we need to add the high order terms $f'(x)(h_1-h_2)$ and $\frac{1}{2}f''(x)(h_1^2-h_2^2)$. Since $X_{n,i}$ and  $Z_{n,i}$ have the same mean and variance and $X_{n,i},Z_{n,i}$ are independent of $T_{n,i}$, we have
\begin{align*}
    &\abs{\E\left[f\left(\frac{T_{n,i}+X_{n,i}}{s_n}\right)\right]-\E\left[f\left(\frac{T_{n,i}+Z_{n,i}}{s_n}\right)\right]}\\
    =&\Bigg|\E\left[f\left(\frac{T_{n,i}+X_{n,i}}{s_n}\right)\right]-\E\left[f\left(\frac{T_{n,i}+Z_{n,i}}{s_n}\right)\right]-\E\left[f'\left(\frac{T_{n,i}}{s_n}\right)\left(\frac{X_{n,i}}{s_n}-\frac{Z_{n,i}}{s_n}\right)\right]\\
    &~-\frac{1}{2}\E\left[f''\left(\frac{T_{n,i}}{s_n}\right)\left(\frac{X_{n,i}^2}{s_n^2}-\frac{Z_{n,i}^2}{s_n^2}\right)\right]\Bigg|\\
    \le&\E\left[g\left(\frac{X_{n,i}}{s_n}\right)+g\left(\frac{Z_{n,i}}{s_n}\right) \right].
\end{align*}
Thus combining this with equation \eqref{telescope}, we have
\begin{align*}
    \abs{\E\left[f\left(\frac{S_n}{s_n}\right)\right]-\E[f(Z)]}&\le \sum_{i=1}^{k_n}
    \E\left[g\left(\frac{X_{n,i}}{s_n}\right)+g\left(\frac{Z_{n,i}}{s_n}\right) \right]\\
    &=\sum_{i=1}^{k_n}\E\left[g\left(\frac{X_{n,i}}{s_n}\right)\right] +\sum_{i=1}^{k_n}\E\left[g\left(\frac{Z_{n,i}}{s_n}\right)\right]\\
    &=(I)+(II).
\end{align*}
We will deal with the sum $(I)$ first. Recall that $g(h) \le k\min\{h^2,h^3\}$. Thus we will split $\E\left[g\left(\frac{X_{n,i}}{s_n}\right)\right]$ into two regions where we will use two different bounds. For each $\eps > 0$, we have
\begin{align*}
    (I) &=\sum_{i=1}^{k_n} \E\left[g\left(\frac{X_{n,i}}{s_n}\right)\right]\\
    &=\sum_{i=1}^{k_n} \int_{\{X_{n,i} \le \eps s_n\}} g\left(\frac{X_{n,i}}{s_n}\right)d\Pa + \sum_{i=1}^{k_n}\int_{\{X_{n,i}  > \eps s_n \}} g\left(\frac{X_{n,i}}{s_n}\right)d\Pa\\
    &\le k\sum_{i=1}^{k_n} \int_{\{X_{n,i} \le \eps s_n\}} \abs{\frac{X_{n,i}}{s_n}}^3d\Pa + k\sum_{i=1}^{k_n}\int_{\{X_{n,i}  > \eps s_n \}}\abs{\frac{X_{n,i}}{s_n}}^2d\Pa\\
    &=\frac{k}{s_n^2}\sum_{i=1}^{k_n} \int_{\{X_{n,i} \le \eps s_n\}} \abs{\frac{X_{n,i}}{s_n}}\abs{X_{n,i}}^2d\Pa + \frac{k}{s_n^2}\sum_{i=1}^{k_n}\int_{\{X_{n,i}  > \eps s_n \}}\abs{X_{n,i}}^2d\Pa   \\
    &\le  \frac{k\eps }{s_n^2}\sum_{i=1}^{k_n} \int_{\{X_{n,i} \le \eps s_n\}} \abs{X_{n,i}}^2d\Pa + \frac{k}{s_n^2}\sum_{i=1}^{k_n}\int_{\{X_{n,i}  > \eps s_n \}}\abs{X_{n,i}}^2d\Pa \\
    &\le\frac{k\eps }{s_n^2}\sum_{i=1}^{k_n}\V(X_{n,i}) + \frac{k}{s_n^2}\sum_{i=1}^{k_n}\int_{\{X_{n,i}  > \eps s_n \}}\abs{X_{n,i}}^2d\Pa \\
    &= k\eps + \frac{k}{s_n^2}\sum_{i=1}^{k_n}\int_{\{X_{n,i}  > \eps s_n \}}\abs{X_{n,i}}^2d\Pa.
\end{align*}
By Lindeberg's condtion we have that the second term goes to zero for all $\eps > 0$. Thus
\[\overline{\lim_n}\sum_{i=1}^{k_n} \E\left[g\left(\frac{X_{n,i}}{s_n}\right)\right] \le k\eps.  \]
And so the sum $(I)$ goes to 0 as $n$ goes to infinity. For sum $(II)$, we can get a similar bound. If we split each expectation into two regions, then we again get
\[(II) = \sum_{i=1}^{k_n}\E\left[g\left(\frac{Z_{n,i}}{s_n}\right)\right] \le k\eps + \frac{k}{s_n^2}\sum_{i=1}^{k_n}\int_{\{Z_{n,i}  > \eps s_n \}}\abs{Z_{n,i}}^2d\Pa.\]
Note that it thus suffices to prove that $\{Z_{n,i}\}$ satisfy Lindeberg's condtion. This is because if $\{Z_{n,i}\}$ satisfies Lindeberg's condtion, then sum $(II)$ will go to zero by the same argument we used for $(I)$. 

We have only assumed that Lindeberg's condition holds of $\{X_{n,i}\}$ but we can prove this implies Lindeberg's condition holds for $\{Z_{n,i}\}$. Note that for all $\eps' > 0$ we have 
\begin{align*}
    \frac{\sigma_{n,i}^2}{s_n^2} &= \frac{1}{s_n^2}\int \abs{X_{n,i}}^2 d\Pa \\
    &= \frac{1}{s_n^2}\int_{\{\abs{X_{n,i}}\le\eps' s_n\}}\abs{X_{n,i}}^2d\Pa+\frac{1}{s_n^2} \int_{\{\abs{X_{n,i}}>\eps' s_n\}} \abs{X_{n,i}}^2 d\Pa\\
    &\le (\eps')^2+\frac{1}{s_n^2} \int_{\{\abs{X_{n,i}}>\eps' s_n\}} \abs{X_{n,i}}^2 d\Pa.
\end{align*}
Thus, by Lindeberg's condtion on $\{X_{n,i}\}$, we have
\[\overline{\lim_n} \max_{1 \le i \le k_n}\left\{\frac{\sigma_{n,i}^2}{s_n^2}\right\} \le \eps'. \]
It follows that $\max\limits_{1\le i \le k_n} \left\{\frac{\sigma_{n,i}}{s_n}\right\}$ goes to zero as $n$ goes to infinity. Let $Z \sim \Na(0,1)$. To show that $\{Z_{n,i}\}$ satisfy Lindeberg's condition let $\eps > 0$ be given. We then have
\begin{align*}
    \frac{1}{s_n^2}\sum_{i=1}^{k_n} \int_{\{Z_{n,i}  > \eps s_n \}}\abs{Z_{n,i}}^2d\Pa&=\frac{1}{s_n^2}\sum_{i=1}^{k_n} \int_{\{\sigma_{n,i}\abs{Z}  > \eps s_n \}}\abs{\sigma_{n,i}Z}^2d\Pa\\
    &\le \frac{1}{s_n^2}\sum_{i=1}^{k_n} \int_{\{\sigma_{n,i}\abs{Z}  > \eps s_n \}}\abs{\sigma_{n,i}Z}^2\frac{\sigma_{n,i}\abs{Z}}{\eps s_n}d\Pa\\
    &\le \frac{\E[\abs{Z}^3]}{\eps s_n^3}\sum_{i=1}^{k_n} \sigma_{n,i}^3\\
    &\le \frac{\E[\abs{Z}^3]}{\eps s_n^3}\max_{1 \le i \le k_n}\left\{\sigma_{n,i}\right\}\sum_{i=1}^{k_n} \sigma_{n,i}^2\\
    &=  \frac{\E[\abs{Z}^3]}{\eps}\max_{1 \le i \le k_n}\left\{\frac{\sigma_{n,i}}{s_n}\right\}\\
    &\to 0.
\end{align*}
Thus $\{X_{n,i}\}$ satisfies Lindeberg's condtion and we are done.
\end{proof}
\section{Comments}
\subsection{The main idea and generalizations}
In some sense our proof was elementary but the idea is \emph{very general}. Any random variable $Z$ can be expressed as 
\[Y=U(X_1,\ldots, X_n), \]
where $X_i$ are independent and $U$ is a function. If $U$ is smooth and we have bounds on the derivatives of $U$, then $Y$ will be close to 
\[U(Z_1,\ldots,Z_n),\]
where $Z_1,\ldots,Z_n$ are independent normal. Sourav Chatterjee writes about this in ``\href{https://www.jstor.org/stable/25449948
}{A generalization of Lindeberg's principle}'' in the Annals of Probability. The only property of the normal distribution that we really used was that normals have finite third moment and that sums of independent normals are normal. 
\subsection{Comments on the theorem and proof}
There is a converse to Lindeberg's theorem (Lindeberg-Feller). Suppose $s_n \to \infty$ and $\frac{\sigma_{n,i}}{n} \to 0$. Under this assumption, if $\frac{S_n}{s_n}$ converges weakly to a normal random variable, then Lindeberg's condtion holds. This is proved in the textbook.

Our version of the central limit theorem is a limit theorem. It does not have an error bound or tell us anything about finite $n$. It is possible to get explicit error bounds and Lindeberg did do this. For notation, define
\[S(x) = \begin{cases}
    x^3 &\text{if } \abs{x} \le 1,\\
    x^2 &\text{if } \abs{x} \ge 1.
\end{cases}\]
There exists a constant $C > 0$ such that for all $X_1,X_2,\ldots$ independent with mean 0 and variance $\sigma_i^2$, then 
\[\sup_{x \in \R} \abs{\Pa\left(\frac{S_n}{s_n} \le x\right) - \Phi(x)} \le C\left(\sum_{i=1}^n l_i\right)^{1/4}, \]
where $l_i = \E\left[S\left(\frac{X_n}{s_n}\right)\right]$. See S.D. Chatterji ``\href{https://www.sciencedirect.com/science/article/pii/S0723086906000429}{Lindeberg's central limit theorem \`a la Hausdorff}.''

The ``right'' convergence rate was proved in the Berry-Essen theorem.
\begin{thrm}
    The exists a constant $C \in (0.4097,0.4748)$ such that if $X_i$ are i.i.d. with mean $\mu$, variance $\sigma^2$ and finite third moment, then 
    \[\sup_{x \in \R} \abs{\Pa\left(\frac{S_n}{s_n} \le x\right)-\Phi(x)} \le \frac{c \E[\abs{X_1}^3]}{\sigma^3\sqrt{n}}, \]
    where $S_n = \sum_{i=1}^n (X_i - \mu)$ and $s_n^2 = \V(S_n) = n\sigma^2$.
\end{thrm}
Thus when $X_i$ has a third moment, the convergence is at a rate of $\frac{1}{\sqrt{n}}$.
\subsection{The noraml heuristic}
In our central limit thoerem the rows were independent. In many cases if $X_{n,i}$ are not too wild and not too dependent, then $\frac{S_n}{s_n} \Longrightarrow \Na(0,1)$. This is the normal heuristic and multiple examples were discussed previously.
\subsection{Different proof techniques}
There are many ways to prove the central limit theorem. For example
\begin{enumerate}
    \item Lindeberg coupling (our proof).
    \item Stein's method (showing $\E[Wf(W)]\approx  \E[f'(W)]$ where $W = \frac{S_n}{s_n}$).
    \item The method of moments (Laplace's proof). We need the result:
    \begin{thrm}
        If $Q_n$ are a sequence of probabilities on $\R$ and 
        \[ \int x^j Q_n(dx) \to  \E[Z^j],\]
        for $j=1,2,3,\ldots,$ then $Q_n((-\infty,x])\to \Phi(x)$ for all $x$.
    \end{thrm}
    These ideas are used in physics.
\item Fourier analysis and characteristic functions (we will discuss these ideas next week).
\item Entropy. On $\R$ the distribution with a fixed variance $\sigma^2$ and the largest entropy is $\Na(0,\sigma^2)$ where the entropy of a distribution $Q$ with density $q$ is defined to be
\[\E_Q[-\log(q(X))].\]
Convoluting two distributions increases entropy. Thus $\frac{S_n}{s_n}$ has increasing entropy and fixed variance and so it should be approaching the normal distribution. Turning this idea into a proof is Linnik's argument.
\end{enumerate}
\end{document}
\documentclass{article}
\usepackage{ae,aecompl}
\usepackage{todonotes}
\usepackage{chngcntr}
\usepackage{tikz-cd}
\usepackage{graphicx}
\graphicspath{ {./images/}}
\usepackage[all,cmtip]{xy}
\usepackage{amsmath, amscd}
\usepackage{amsthm}
\usepackage{amssymb}
\usepackage{amsfonts}
\usepackage{bm}
\usepackage{qsymbols}
\usepackage{latexsym}
\usepackage{mathrsfs}
\usepackage{mathtools}
\usepackage{cite}
\usepackage{color}
\usepackage{url}
\usepackage{enumerate}
\usepackage{verbatim}
\usepackage[draft=false, colorlinks=true]{hyperref}
\usepackage{pdfpages}
\usepackage[margin=1.2in]{geometry}
\usepackage{IEEEtrantools}

\usepackage{fancyhdr}


\usepackage[nameinlink]{cleveref}


\DeclareMathOperator*{\ac}{accept}
\DeclareMathOperator*{\amax}{argmax}
\DeclareMathOperator*{\amin}{argmin}
\DeclareMathOperator*{\Aut}{Aut}
\newcommand {\al}{{\alpha}}
\newcommand {\abs}[1]{{\left\lvert#1\right\rvert}}
\newcommand {\A}{{\mathcal{A}}}
\newcommand {\AM}{{\mathrm{AM}}}
\newcommand {\AMp}{{\AM_{p}^{X}\!(\Ri_\w)}}
\newcommand {\B}{{\mathcal{B}}}
\DeclareMathOperator*{\Be}{Bern}
\newcommand {\Br}{{\dot{B}}}
\newcommand {\Ba}{{\mathfrak{B}}}
\newcommand {\C}{{\mathbb C}}
\newcommand {\ce}{\mathrm{c}}
\newcommand {\Ce}{\mathrm{C}}
\newcommand {\Cc}{\mathrm{C_{c}}}
\newcommand {\Ccinf}{\mathrm{C_{c}^{\infty}}}
\DeclareMathOperator{\cov}{Cov}
\DeclareMathOperator{\DEV}{DEV}
\newcommand {\Di}{{\mathbb D}}
\newcommand {\dom}{\mathrm{dom}}
\newcommand{\dist}{\stackrel{\mathrm{dist}}{=}}
\newcommand {\ud}{\mathrm{d}}
\newcommand {\ue}{\mathrm{e}}
\newcommand {\eps}{\varepsilon}
\newcommand {\veps}{\varepsilon}
\newcommand {\vrho}{{\varrho}}
\newcommand {\E}{{\mathbb{E}}}
\newcommand {\Ec}{{\mathcal{E}}}
\newcommand {\Ell}{L}
\newcommand {\Ellp}{{L_{p}[0,1]}}
\newcommand {\Ellpprime}{{L_{p'}([0,1])}}
\newcommand {\Ellq}{{L_{q}([0,1])}}
\newcommand {\Ellqprime}{{L_{q'}([0,1])}}
\newcommand {\Ellr}{L^{r}}
\newcommand {\Ellone}{{L_{1}([0,1])}}
\newcommand{\Elltwo}{{L_{2}([0,1])}}
\newcommand{\Ellinfty}{L^{\infty}}
\newcommand{\Ellinftyc}{L_{\mathrm{c}}^{\infty}}
\newcommand{\exb}[1]{\exp\left\{#1\right\}}
\DeclareMathOperator*{\Ext}{Ext}
\newcommand{\F}{{\mathcal{F}}}
\newcommand{\Fe}{{\mathbb{F}}}
\newcommand{\G}{{\mathcal{G}}}
\newcommand{\HF}{\mathcal{H}_{\text{FIO}}^{1}(\Rd)}
\newcommand{\Hr}{H}
\newcommand{\HT}{\mathcal{H}}
\newcommand{\ui}{\mathrm{i}}
\newcommand{\I}{{I}}
\newcommand{\J}{{\mathcal{J}}}
\newcommand{\id}{{\mathrm{id}}}
\newcommand{\iid}{\stackrel{\mathclap{\normalfont\mbox{iid}}}{\sim}}
\newcommand{\im}{{\text{im }}}
\newcommand{\ind}{{\perp\!\!\!\perp}}
\DeclareMathOperator*{\Int}{int}
\newcommand{\intx}{{\overline{\int_{X}}}}
\newcommand{\inte}{{\overline{\int_{\E}}}}
\newcommand{\la}{\lambda}
\newcommand{\rb}{\rangle}
\newcommand{\lb}{{\langle}}
\newcommand{\La}{\Lambda}
\newcommand{\calL}{{\mathcal{L}}}
\newcommand{\lp}{{\mathcal{L}}^{p}}
\newcommand{\lpo}{{\overline{\mathcal{L}}^{p}\!}}
\newcommand{\Lpo}{{\overline{\Ell}^{p}\!}}
\newcommand{\M}{{\mathbf{M}}}
\newcommand{\Ma}{{\mathcal{M}}}
\newcommand{\N}{{{\mathbb N}}}
\newcommand{\Na}{{{\mathcal{N}}}}
\newcommand{\norm}[1]{\left\|#1\right\|}
\newcommand{\normm}[1]{{\left\vert\kern-0.25ex\left\vert\kern-0.25ex\left\vert #1 
    \right\vert\kern-0.25ex\right\vert\kern-0.25ex\right\vert}}
\newcommand{\Om}{{{\Omega}}}
\newcommand{\one}{{{\bf 1}}}
\newcommand{\pic}{\text{Pic }}
\newcommand{\ph}{{\varphi}}
\newcommand{\Pa}{{\mathbb{P}}}
\newcommand{\Po}{{\mathcal{P}}}
\newcommand{\Q}{{\mathbb{Q}}}
\newcommand{\R}{{\mathbb R}}
\newcommand{\Rd}{{\mathbb{R}^{d}}}
\DeclareMathOperator{\rej}{reject }
\newcommand{\Rn}{{\mathbb{R}^{n}}}
\newcommand{\cR}{{\mathcal{R}}}
\newcommand{\Rad}{{\mathrm{Rad}}}
\newcommand{\ran}{{\mathrm{ran}}}
\newcommand{\Ri}{{\mathrm{R}}}
\newcommand{\supp}{{\mathrm{supp}}}
\newcommand{\Se}{\mathrm{S}}
\newcommand{\Sp}{S^{*}(\Rn)}
\newcommand{\St}{{\mathrm{St}}}
\newcommand{\Sw}{\mathcal{S}}
\newcommand{\T}{{\mathcal{T}}}
\newcommand{\ta}{{\theta}}
\newcommand{\Ta}{{\Theta}}
\newcommand{\topp}{\stackrel{p}{\to}}
\newcommand{\todd}{\stackrel{d}{\to}}
\newcommand{\toL}[1]{\stackrel{L^{#1}}{\to}} 
\newcommand{\toas}{\stackrel{a.s.}{\to}}
\DeclareMathOperator{\V}{Var}
\newcommand {\w}{{\omega}}
\newcommand {\W}{{\mathrm{W}}}
\newcommand {\Wnp}{\text{$\mathrm{W}$\textsuperscript{$n,\!p$}}}
\newcommand {\Wnpeq}{\text{$\mathrm{W}$\textsuperscript{$n\!,\!p$}}}
\newcommand {\Wonep}{\text{$\mathrm{W}$\textsuperscript{$1,\!p$}}}
\newcommand {\Wonepeq}{\text{$\mathrm{W}$\textsuperscript{$1\!,\!p$}}}
\newcommand {\X}{{\mathcal{X}}}
\newcommand {\Z}{{{\mathbb Z}}}
\newcommand {\Za}{{\mathcal{Z}}}
\newcommand {\Zd}{{\Z[\sqrt{d}]}}
\newcommand {\vanish}[1]{\relax}

\newcommand {\wh}{\widehat}
\newcommand {\wt}{\widetilde}
\newcommand {\red}{\color{red}}

% Distributions
\newcommand{\normal}{\mathsf{N}}
\newcommand{\poi}{\mathsf{Poisson}}
\newcommand{\bern}{\mathsf{Bernoulli}}
\newcommand{\bin}{\mathsf{Binomal}}
\newcommand{\multi}{\mathsf{Multinomial}}
\newcommand{\Exp}{\mathsf{Exp}}



% put your command and environment definitions here




% some theorem environments
% remove "[theorem]" if you do not want them to use the same number sequence


  \newtheorem{thrm}{Theorem}
  \newtheorem{lemma}{Lemma}
  \newtheorem{prop}{Proposition}
  \newtheorem{cor}{Corollary}

  \newtheorem{conj}{Conjecture}
  \renewcommand{\theconj}{\Alph{conj}}  % numbered A, B, C etc

  \theoremstyle{definition}
  \newtheorem{defn}{Definition}
  \newtheorem{ex}{Example}
  \newtheorem{exs}{Examples}
  \newtheorem{question}{Question}
  \newtheorem{remark}{Remark}
  \newtheorem{notn}{Notation}
  \newtheorem{exer}{Exercise}




\title{STATS310A - Lecture 8}
\author{Dominik Rothenhaeusler\\ Scribed by Michael Howes}
\date{10/07/21}

\pagestyle{fancy}
\fancyhf{}
\rhead{STATS310A - Lecture 8}
\lhead{10/13/21}
\rfoot{Page \thepage}

\begin{document}
\maketitle
\tableofcontents
\section{Announcement}
\begin{itemize}
    \item Recommending reading on Bayesian vs Frequentist modeling in the scribed notes. 
    \item Andrew Gelman's blog is highly recommended.
    \item ``Why isn't everyone a Bayesian?'' by Brad Efron is also recommended.
    \item Midterm is in two weeks in class on Oct 27th.
    \item Last year's midterm will be posted to Canvas.
    \item There will be a section for midterm revision.
    \item The exam is open book.
\end{itemize}
\section{Recap}
We have a new optimality goal: minimize the average risk. That is minimize
\[r(\Lambda, \delta)= \E[r(\Theta, \delta(X))], \]
where $\Theta \sim \Lambda$ and $X |\Theta = \ta \sim \Pa_\ta$. We saw that it suffices to minimize the posterior risk. That is the Bayes estimator is given by 
\[\delta_\Lambda(x) = \arg\min_\delta \E[L(\Theta, \delta)|X=x]. \]
We saw an example yesterday when we had 
\begin{itemize}
    \item Prior: Beta distribution.
    \item Likelihood: Binomial.
    \item Posterior: Beta distribution.
\end{itemize}
This is an example of \emph{conjugate priors}. 
\begin{defn}
    Given a family of distribution $\Po = \{\Pa_\ta :\ta \in \Om\}$, a family of priors $\{\Lambda_\xi : \xi \in \Xi\}$ are \emph{conjugate priors to } $\Po$, if for every  $x \in \X$ and $\xi \in \Xi$, there exists $\xi' \in \Xi$, such that $\Lambda_\xi | X=x $ has distribution $\Lambda_{\xi'}$.
\end{defn}
Such priors are computational convenient.
\section{Examples}
\subsection{Absolute error}
Suppose that $L(\ta,d) = \abs{\ta - d}$, then $\delta_\Lambda$ is the median of the posterior distribution.
\subsection{Weighted squared error loss}
Suppose that $L(\ta,d) = w(\ta)(\ta - d)^2$ for some weight function $w(\ta) \ge 0$. The Bayes estimator for this loss function minimises
\[\E[w(\ta)(\ta-\delta)^2|X=x]. \]
Note that
\begin{align*}
    \E[w(\ta)(\ta-\delta)^2|X=x]=& \E[w(\ta)\ta^2|X=x]-2\E[\delta w(\ta)\ta|X=x] + \E[w(\ta)\delta^2|X=x]\\
    =& \E[w(\ta)\ta^2|X=x]-2\delta\E[w(\ta)\ta|X=x]+\delta^2\E[w(\ta)\delta^2|X=x].
\end{align*}
Thus the Bayes estimator is the value of $\delta$ that minimizes $-2\delta \E[w(\ta)\ta|X=x]+\delta^2\E[w(\ta)|X=x]$. Differentiating this with respect to $\delta$ we see that $-2[w(\ta)\ta|X=x]+2\delta \E[w(\ta)|X=x]$ and so
\[\delta_{\Lambda}(x) = \frac{\E[\ta w(\ta)|X=x]}{\E[w(\ta)|X=x]}. \]
An alternative solution to this problem is the define a new measure
\[\Theta \sim w(\ta)d\Lambda(\ta). \]
Then $\delta_\Lambda$ is the conditional expectation under the new measure. Thinking of $w(\ta)d\Lambda(\ta)$ as a new measure is a useful computational trick.
\subsection{Binary classification}
Suppose $\Om = \{0,1\}$. For example we may be classifying emails as either spam or not span. $X \sim f_0$ or $X\sim f_1$, $D = \{0,1\}$ and $L(\ta,d) = \one(\ta = d)$, where $\one(A)$ is the indicator function of $A$. Our prior $\Lambda$ is specified by $\pi(1) = p$, $\pi(0) = 1-p$. The average risk of an estimator is $P(\delta(X) \neq \Theta)$. We want to minimise
\[P(\delta \neq \Theta | X=x) = P(\Theta = 1|X=x)\one(\delta = 0)+P(\Theta = 0|X=x)\one(\delta=1). \]
Thus we can see that if $P(\Theta = 1|X=x) > P(\Theta = 0|X=x)$, then we will pick $\delta=1$ and otherwise we will pick $\delta = 0$. Note that $P(\Theta = 1|X=x) \propto pf_1(x)$ and $P(\Theta = 0|X=x) \propto (1-p)f_2(x)$. Thus we will set $\delta = 1$ if and only if 
\[\frac{f_1(x)}{f_0(x)} > \frac{1-p}{p}. \]
Continuing this example suppose we have $\la_0, \la_1 > 0$ and we know $X_1,\ldots,X_n \iid \text{Exp}(\la_\ta)$ where $\ta \in \{0,1\}$. With the same loss and prior as before we know that we will set $\delta = 1$ if and only if
\[\frac{1-p}{p} < \frac{f_1(x)}{f_0(x)} = \frac{\la_1}{\la_0}\exb{-(\la_1-\la_0)\sum_{i=1}^n x_i}. \]
Thus we set $\delta = 1$ if and only if $\sum_{i=1}^n x_i$ exceeds some value which is a function of $p,\la_0$ and $\la_1$. Thus the Bayes estimator depends only on the sufficient statistic $\sum_{i=1}^n X_i$. This is a general result. It holds because
\[\pi(\ta | x) \propto f(x|\ta)\pi(\ta) \al g_\ta(T(x))h(x)\pi(\ta) \propto g_\ta(T(x))\pi(\ta), \]
by the NFFC. 
\subsection{Normal likelihood and normal prior}
Suppose $X_1,\ldots, X_n \iid N(\ta, \sigma^2)$ where $\sigma^2$ is known and $\Theta \sim N(\mu, b^2)$ where both $\mu$ and $b^2$ are known. We then have
\begin{align*}
    \text{posterior }\propto&\text{prior}\times \text{likelihood}\\
    \propto&e^{-\frac{1}{2b^2}(\ta-\mu)^2}\prod_{i=1}^n e^{-\frac{1}{2\sigma^2}(x_i-\ta)^2}\\
    \propto&\exb{-\frac{1}{2}\ta^2\left(\frac{1}{b^2}+\frac{n}{\sigma^2}\right)+\ta\left(\frac{\mu}{b^2}+\sum_{i=1}^n \frac{x_i}{\sigma^2}\right)}.
\end{align*}
Thus the posterior has a normal distribution with parameters.
\begin{align*}
    \text{Variance } =&\frac{1}{\frac{1}{b^2}+\frac{n}{\sigma^2}},\\
    \text{Mean } = & \left(\frac{\mu}{b^2}+\frac{n\bar{x}}{\sigma^2}\right)\cdot \frac{1}{\frac{1}{b^2}+\frac{n}{\sigma^2}}\\
    =&\bar{x} \frac{\frac{n}{\sigma^2}}{\frac{1}{b^2}+\frac{n}{\sigma^2}}+\mu \frac{\frac{1}{b^2}}{\frac{1}{b^2}+\frac{n}{\sigma^2}}\\
    =&\bar{x} w_1 + \mu w_2.
\end{align*}
Since $w_1+w_2=1$, we see that the Bayes estimator under squared error loss is again a convex combination of the UMVUE and the prior mean. We see again that the Bayes estimator depends only on the sufficient statistic $\bar{x}$. 
\section{Evaluating Bayes estimators}
\subsection{Bias}
Under squared error loss Bayes estimators are always biased. More precisely:
\begin{thrm}
    If $\delta$ is unbiased for $g(\ta)$ with $r(\Lambda, \delta) < \infty$ and $\E[g(\Theta)^2]<\infty$, then $\delta$ is not the Bayes estimator under squared error loss unles $r(\Lambda, \delta) = 0$ (ie $g(\Theta) = \delta(X)$ with probability 1).
\end{thrm}
\begin{proof}
    We will proceed by contradiction. Suppose that $\delta$ is both unbiased and the Bayes estimator. We know that the Bayes estimator under squared error loss is given by 
    \[\delta(x) = \E[g(\Theta)|X=x].\]
    It follows that
    \[\E[g(\Theta)\delta(X)] = \E[\delta(X)\E[g(\Theta)|X=x]] =\E[\delta(X)^2]. \]
    Also since $\delta(X)$ is unbiased, we have $\E[\delta(X)|\Theta = \ta] = g(\ta)$. This implies
    \[\E[g(\Theta)\delta(X)] = \E[g(\Theta)\E[\delta(X)\Theta]] =\E[g(\Theta)^2].\]
    Hence
    \begin{align*}
        r(\Lambda, \delta) &= \E[\left(g(\Theta)-\delta(X)\right)^2]\\
        &=\E[g(\Theta)^2]-2\E[g(\Theta)\delta(X)]-\E[\delta(X)^2]\\
        &=\E[g(\Theta)\delta(X)]-2\E[g(\Theta)\delta(X)]-\E[g(\Theta)\delta(X)]\\
        &=0,
    \end{align*}
    as required.
\end{proof}
Suppose that $X_i \iid N(\ta,\sigma^2)$. The estimator $\delta(X) = \frac{1}{n}\sum_{i=1}^n X_i$ is not the Bayes estimator since it has risk $\frac{\sigma^2}{n} >0$ and $\delta$ is unbiased.
\begin{thrm}
A Bayes estimator with finite risk is admissible on the support of $\Lambda$. That is there are no estimators $\delta'$ satifying
\begin{enumerate}
    \item $R(\ta, \delta') \le R(\ta, \delta_\Lambda)$ with probability 1 under $\Lambda$.
    \item $R(\ta,\delta') < R(\ta,\delta_\Lambda)$ for all $\ta \in \Om'$ for some $\Om' \subseteq \Om$ with $\Lambda(\Om') > 0$. 
\end{enumerate}
\begin{proof}
    Suppose that there did exist such a $\delta'$. Then
    \[r(\Lambda, \delta') = \int R(\ta,\delta')d\Lambda(\ta) < \int R(\ta, \delta_\Lambda)d\Lambda(\ta) =r(\Lambda, \delta_\Lambda),\]
    a contradiction.
\end{proof}
\end{thrm}
\end{document}
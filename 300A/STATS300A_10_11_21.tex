\documentclass{article}
\usepackage{ae,aecompl}
\usepackage{todonotes}
\usepackage{chngcntr}
\usepackage{tikz-cd}
\usepackage{graphicx}
\graphicspath{ {./images/}}
\usepackage[all,cmtip]{xy}
\usepackage{amsmath, amscd}
\usepackage{amsthm}
\usepackage{amssymb}
\usepackage{amsfonts}
\usepackage{bm}
\usepackage{qsymbols}
\usepackage{latexsym}
\usepackage{mathrsfs}
\usepackage{mathtools}
\usepackage{cite}
\usepackage{color}
\usepackage{url}
\usepackage{enumerate}
\usepackage{verbatim}
\usepackage[draft=false, colorlinks=true]{hyperref}
\usepackage{pdfpages}
\usepackage[margin=1.2in]{geometry}
\usepackage{IEEEtrantools}

\usepackage{fancyhdr}


\usepackage[nameinlink]{cleveref}


\DeclareMathOperator*{\ac}{accept}
\DeclareMathOperator*{\amax}{argmax}
\DeclareMathOperator*{\amin}{argmin}
\DeclareMathOperator*{\Aut}{Aut}
\newcommand {\al}{{\alpha}}
\newcommand {\abs}[1]{{\left\lvert#1\right\rvert}}
\newcommand {\A}{{\mathcal{A}}}
\newcommand {\AM}{{\mathrm{AM}}}
\newcommand {\AMp}{{\AM_{p}^{X}\!(\Ri_\w)}}
\newcommand {\B}{{\mathcal{B}}}
\DeclareMathOperator*{\Be}{Bern}
\newcommand {\Br}{{\dot{B}}}
\newcommand {\Ba}{{\mathfrak{B}}}
\newcommand {\C}{{\mathbb C}}
\newcommand {\ce}{\mathrm{c}}
\newcommand {\Ce}{\mathrm{C}}
\newcommand {\Cc}{\mathrm{C_{c}}}
\newcommand {\Ccinf}{\mathrm{C_{c}^{\infty}}}
\DeclareMathOperator{\cov}{Cov}
\DeclareMathOperator{\DEV}{DEV}
\newcommand {\Di}{{\mathbb D}}
\newcommand {\dom}{\mathrm{dom}}
\newcommand{\dist}{\stackrel{\mathrm{dist}}{=}}
\newcommand {\ud}{\mathrm{d}}
\newcommand {\ue}{\mathrm{e}}
\newcommand {\eps}{\varepsilon}
\newcommand {\veps}{\varepsilon}
\newcommand {\vrho}{{\varrho}}
\newcommand {\E}{{\mathbb{E}}}
\newcommand {\Ec}{{\mathcal{E}}}
\newcommand {\Ell}{L}
\newcommand {\Ellp}{{L_{p}[0,1]}}
\newcommand {\Ellpprime}{{L_{p'}([0,1])}}
\newcommand {\Ellq}{{L_{q}([0,1])}}
\newcommand {\Ellqprime}{{L_{q'}([0,1])}}
\newcommand {\Ellr}{L^{r}}
\newcommand {\Ellone}{{L_{1}([0,1])}}
\newcommand{\Elltwo}{{L_{2}([0,1])}}
\newcommand{\Ellinfty}{L^{\infty}}
\newcommand{\Ellinftyc}{L_{\mathrm{c}}^{\infty}}
\newcommand{\exb}[1]{\exp\left\{#1\right\}}
\DeclareMathOperator*{\Ext}{Ext}
\newcommand{\F}{{\mathcal{F}}}
\newcommand{\Fe}{{\mathbb{F}}}
\newcommand{\G}{{\mathcal{G}}}
\newcommand{\HF}{\mathcal{H}_{\text{FIO}}^{1}(\Rd)}
\newcommand{\Hr}{H}
\newcommand{\HT}{\mathcal{H}}
\newcommand{\ui}{\mathrm{i}}
\newcommand{\I}{{I}}
\newcommand{\J}{{\mathcal{J}}}
\newcommand{\id}{{\mathrm{id}}}
\newcommand{\iid}{\stackrel{\mathclap{\normalfont\mbox{iid}}}{\sim}}
\newcommand{\im}{{\text{im }}}
\newcommand{\ind}{{\perp\!\!\!\perp}}
\DeclareMathOperator*{\Int}{int}
\newcommand{\intx}{{\overline{\int_{X}}}}
\newcommand{\inte}{{\overline{\int_{\E}}}}
\newcommand{\la}{\lambda}
\newcommand{\rb}{\rangle}
\newcommand{\lb}{{\langle}}
\newcommand{\La}{\Lambda}
\newcommand{\calL}{{\mathcal{L}}}
\newcommand{\lp}{{\mathcal{L}}^{p}}
\newcommand{\lpo}{{\overline{\mathcal{L}}^{p}\!}}
\newcommand{\Lpo}{{\overline{\Ell}^{p}\!}}
\newcommand{\M}{{\mathbf{M}}}
\newcommand{\Ma}{{\mathcal{M}}}
\newcommand{\N}{{{\mathbb N}}}
\newcommand{\Na}{{{\mathcal{N}}}}
\newcommand{\norm}[1]{\left\|#1\right\|}
\newcommand{\normm}[1]{{\left\vert\kern-0.25ex\left\vert\kern-0.25ex\left\vert #1 
    \right\vert\kern-0.25ex\right\vert\kern-0.25ex\right\vert}}
\newcommand{\Om}{{{\Omega}}}
\newcommand{\one}{{{\bf 1}}}
\newcommand{\pic}{\text{Pic }}
\newcommand{\ph}{{\varphi}}
\newcommand{\Pa}{{\mathbb{P}}}
\newcommand{\Po}{{\mathcal{P}}}
\newcommand{\Q}{{\mathbb{Q}}}
\newcommand{\R}{{\mathbb R}}
\newcommand{\Rd}{{\mathbb{R}^{d}}}
\DeclareMathOperator{\rej}{reject }
\newcommand{\Rn}{{\mathbb{R}^{n}}}
\newcommand{\cR}{{\mathcal{R}}}
\newcommand{\Rad}{{\mathrm{Rad}}}
\newcommand{\ran}{{\mathrm{ran}}}
\newcommand{\Ri}{{\mathrm{R}}}
\newcommand{\supp}{{\mathrm{supp}}}
\newcommand{\Se}{\mathrm{S}}
\newcommand{\Sp}{S^{*}(\Rn)}
\newcommand{\St}{{\mathrm{St}}}
\newcommand{\Sw}{\mathcal{S}}
\newcommand{\T}{{\mathcal{T}}}
\newcommand{\ta}{{\theta}}
\newcommand{\Ta}{{\Theta}}
\newcommand{\topp}{\stackrel{p}{\to}}
\newcommand{\todd}{\stackrel{d}{\to}}
\newcommand{\toL}[1]{\stackrel{L^{#1}}{\to}} 
\newcommand{\toas}{\stackrel{a.s.}{\to}}
\DeclareMathOperator{\V}{Var}
\newcommand {\w}{{\omega}}
\newcommand {\W}{{\mathrm{W}}}
\newcommand {\Wnp}{\text{$\mathrm{W}$\textsuperscript{$n,\!p$}}}
\newcommand {\Wnpeq}{\text{$\mathrm{W}$\textsuperscript{$n\!,\!p$}}}
\newcommand {\Wonep}{\text{$\mathrm{W}$\textsuperscript{$1,\!p$}}}
\newcommand {\Wonepeq}{\text{$\mathrm{W}$\textsuperscript{$1\!,\!p$}}}
\newcommand {\X}{{\mathcal{X}}}
\newcommand {\Z}{{{\mathbb Z}}}
\newcommand {\Za}{{\mathcal{Z}}}
\newcommand {\Zd}{{\Z[\sqrt{d}]}}
\newcommand {\vanish}[1]{\relax}

\newcommand {\wh}{\widehat}
\newcommand {\wt}{\widetilde}
\newcommand {\red}{\color{red}}

% Distributions
\newcommand{\normal}{\mathsf{N}}
\newcommand{\poi}{\mathsf{Poisson}}
\newcommand{\bern}{\mathsf{Bernoulli}}
\newcommand{\bin}{\mathsf{Binomal}}
\newcommand{\multi}{\mathsf{Multinomial}}
\newcommand{\Exp}{\mathsf{Exp}}



% put your command and environment definitions here




% some theorem environments
% remove "[theorem]" if you do not want them to use the same number sequence


  \newtheorem{thrm}{Theorem}
  \newtheorem{lemma}{Lemma}
  \newtheorem{prop}{Proposition}
  \newtheorem{cor}{Corollary}

  \newtheorem{conj}{Conjecture}
  \renewcommand{\theconj}{\Alph{conj}}  % numbered A, B, C etc

  \theoremstyle{definition}
  \newtheorem{defn}{Definition}
  \newtheorem{ex}{Example}
  \newtheorem{exs}{Examples}
  \newtheorem{question}{Question}
  \newtheorem{remark}{Remark}
  \newtheorem{notn}{Notation}
  \newtheorem{exer}{Exercise}




\title{STATS300A - Lecture 7}
\author{Dominik Rothenhaeusler\\ Scribed by Michael Howes}
\date{10/11/21}

\pagestyle{fancy}
\fancyhf{}
\rhead{STATS310A - Lecture 7}
\lhead{10/11/21}
\rfoot{Page \thepage}

\begin{document}
\maketitle
\tableofcontents
\section{Overview of optimal estimation}
\begin{itemize}
    \item We have seen that uniformly best estimators do not exist.
    \item We can constrain our class of estimators eg unbiased, equivariant.
    \item We can also collapse the risk via Bayesian estimation or minimax estimation.
\end{itemize}
Today we will finish up on equivariance and discuss Bayesian estimation.
\section{Equivariance}
Recall that if $\delta_0$ is any equivariant estimator, and $v^*(y)$ is defined to be the value $v$ that minimises
\[\E_0[\rho(\delta_0(X)-v)|Y=y], \]
where $Y=(X_1-X_n,\ldots,X_{n-1}-X_n)$, then $\delta^*(X) = \delta_0(X) - v^*(Y)$ is MRE.
\subsection{MREs vs UMRUES}
\begin{itemize}
    \item MREs depend on the loss function.
    \item UMRUES do not depend on the loss function provided the loss function is strictly convex. This is because the UMRUE is often the unique unbiased estimator that is a function of a complete sufficient statistic.
    \item UMRUES do not always exist.
    \item MREs usually do exist and we can find them via an optimisation procedure.
    \item UMRUES are often \emph{inadmissible}.
    \item Pitman's estimator is admissible under weak regularity conditions (Stein 1959).
    \item MREs are often biased if something other squared error loss is being used.
\end{itemize}
\subsection{Risk unbiasedness}
\begin{defn}
    An estimator $\delta$  is risk unbiased for the loss $L(\ta,d)$ if all $\ta,\ta'$
    \[\E_\ta[L(\ta,\delta(X))] \le \E_{\ta}[L(\ta',\delta(X))]. \]
\end{defn}

Intuitively this says the true parameter $\ta$ penalises less than the false parameter $\ta'$. If $L$ is squared error loss, then this is the same as regular unbiasedness.
\begin{thrm}
    \emph{[TPE 3.127]} If $\delta$ if MRE for a location invariant decision problem, then $\delta$ is risk unbiased.
\end{thrm}
\begin{proof}
    [Sketch] Prove the contrapositive. Show that if $\delta$ is not risk unbiased, then a shifted version of $\delta$ has strictly lower risk.
\end{proof}
\section{Location-Scale Models}
Suppose that $X=(X_1,\ldots, X_n)$ has the joint density
\[f_{\ta,\tau}(x_1,\ldots, x_n) = \frac{1}{\tau^n}f\left(\frac{x_1-\ta}{\tau},\ldots, \frac{x_n-\ta}{\tau}\right). \]
Our parameters are the \emph{location} $\ta \in \R$ and \emph{scale} $\tau > 0$. 
\begin{ex}
    Suppose $X_1,\ldots, X_n \iid \Na(\ta, \tau^2)$, then 
    \begin{IEEEeqnarray*}{rCl}
        f(x_1,\ldots, x_n) &=& \frac{1}{(\sqrt{2 \pi} \tau)^n} \exb{-\frac{1}{2\tau^2}\sum_{i=1}^n (x_i-\ta)^2}\\
        &=&\frac{1}{\tau^n}\frac{1}{\sqrt{2 \pi}^n} \exb{-\frac{1}{2}\sum_{i=1}^n \left(\frac{x_i-\ta}{\tau}\right)^2}\\
        &=&\frac{1}{\tau^n}g\left(\frac{x_1-\ta}{\tau},\ldots,\frac{x_n-\ta}{\tau}\right),
    \end{IEEEeqnarray*}
    where $g \sim \Na(0,I_n)$.
\end{ex}
Note that if $X_i$ has a location-scale distribution $(\ta,\tau)$, then $X'_i = aX_i+b$ has a loc-scale distribution $(a\ta+b, a\tau)$ if $a>0$ and $b \in \R$. Our goal is to estimate $\ta$, we are not interested in estimating $\tau$ and we call $\tau$ a \emph{nuissance parameter}.

\begin{defn}
    A loss function is \emph{loc-scale invariant} if 
    \[L((a\ta+b,a\tau), ad+b) = L((\ta,\tau),d), \]
    for all $a >0$ and $b \in \R$. This is equivalent to requiring that $L((\ta,\tau),d)$ is a function of $\frac{\ta-d}{\tau}$.
\end{defn}

\begin{defn}
    A model $\Po = \{P_{(\ta,\tau)} : (\ta,\tau) \in \Om \}$ is \emph{loc-scale invariant} if 
    \[f_{(a\ta+b,a\tau)}(ax+b) = f_{(\ta,\tau)}(x), \]
    for all $a >0$ and $b \in \R$.
\end{defn}
\begin{defn}
    An estimator $\delta$, is \emph{loc-scale equivariant} if 
    \[\delta(aX+b) = a\delta(X)+b, \]
    for all $a > 0$ and $b \in \R$.
\end{defn}

\begin{thrm}
    Suppose we have a loc-scale invariant loss and model. Let $\delta_\tau^*$ be the MRE in the submodel where $\tau$ is fixed.

    If $\delta_\tau^*$ does not depend on the scale $\tau$, then $\delta^* = \delta_\tau^*$ is the MRE for the full mode. That is for any loc-scale equivariant estimator $\delta'$,
    \[R((\ta,\tau), \delta^*) \le R((\ta,\tau), \delta'). \]
\end{thrm}
This is another example of the technique of restricting attention to a submodel and then concluding something about the full model (recall our semi-parametric example from earlier).
\begin{proof}
    Assume $\delta'$ is strictly better at $(\ta_0,\tau_0)$ and that $\delta'$ is loc-scale equivariant. Then $\delta'$ has strictly better risk on the submodel $\{(\ta,\tau): \tau= \tau_0\}$ and $\delta'$ is loc equivaraint on the submodel. This is contradiction to $\delta^*$ being the MRE on the submodel.
\end{proof}

\begin{ex}
    In the model $\Na(\ta,\tau^2)$, $\bar{X}$ is the MRE under squared error loss for fixed $\tau$. It does not depend on $\tau$ and thus $\bar{X}$ is the MRE in the full model.
\end{ex}

\begin{ex}
    If $f_{\ta,\tau} \sim \frac{1}{\tau} \exb{-\frac{1}{\tau}(x-\ta)}\mathbb{I}(x\ge \ta)$, then the loc MRE under squared error loss if $X_{(1)}-\frac{\tau}{n}$ depends on $\tau$. We will study this example more on the next assignment.
\end{ex}

\section{Bayes Estimators}
As before we have our data $X \in \X$ and model $\Po = \{P_\ta : \ta \in \Om\}$. Let $\Lambda$ be a measure on $\Om$. We wish to find an estimator $\delta$ that minimizes the \emph{average risk}
\[r(\Lambda, \delta) = \int_{\Om} R(\ta, \delta)d\Lambda(\ta). \]
If $\Lambda$ is a probability distribution, then $\Lambda$ is called a prior distribution. Is $\delta_\Lambda$ minimises $r(\Lambda, \delta)$, then $\delta_\Lambda$ is called a \emph{Bayes estimator} and we call $r(\Lambda, \delta_\Lambda)$ the \emph{Bayes risk}. 

\begin{prop}
    If $\Lambda$ is a probability distribution, then we have
    \[r(\Lambda, \delta) = \E L(\Theta, \delta(X)), \]
    where $\Theta \sim \Lambda$ and $X | \Theta = \ta \sim P_\ta$. 
\end{prop}
Note that the above expectation is with respect to both $\Theta$ and $X$. This is different to the regular risk which is only an expectation with respect to $X$.
\begin{proof}
    Note that
    \begin{IEEEeqnarray*}{rCl}
    \E(L(\Theta, \delta(X))) &=& \E[\E[L(\Theta, \delta(X))|\Theta]]\\
    &=& \E[R(\Theta, \delta)]\\
    &=&\int_\Om R(\ta, \delta) d\Lambda(\ta)\\
    &=&r(\Lambda, \delta).       
    \end{IEEEeqnarray*}
\end{proof}
The usual interpretation is that $\Lambda$ encodes prior beliefs about $\ta$ that we have before seeing the data $X$. Our main result is the following:

\begin{thrm}
    Suppose $X \sim \Lambda$ and $X|\Theta = \ta \sim P_\ta$. If there exists an estimator $\delta_0$ with finite risk and if for almost every $x$, there exists a value $\delta(x)$ that minimises 
    \[\E[L(\Theta,d)|X=x], \]
    over $d$, then $\delta(X)$ is the Bayes estimator.
\end{thrm}
\begin{proof}
    For a.e. $x$ and every estimator $\delta'$ we have 
    \[\E[L(\Theta, \delta'(X))|X=x] \ge \E[L(\Theta, \delta(X))|X=x]. \]
    Thus,
    \begin{IEEEeqnarray*}{rCl}
        r(\Lambda, \delta')&=& \E\left[\E\left[L(\ta, \delta'(X))|X=x\right]\right]\\
        &\ge& \E\left[\E\left[L(\ta, \delta(X))|X=x\right]\right]\\
        &=&r(\Lambda, \delta).
    \end{IEEEeqnarray*}
    Thus $\delta$ is the Bayes estimator.
\end{proof}
Note that under squared error loss the minimizer of $\E[(g(\Theta)-\delta(x))^2|X=x]$ is the conditional expectation $\E[g(\Theta)|X=x]$ which we call the posterior mean.
\subsection{A binomial example}
Suppose that $X \sim \text{Bin}(n, \ta)$ where $\ta \sim \Lambda =  \text{Beta}(a,b)$. That is the prior for $\ta$ has density
\[\pi(\ta) = \frac{\Gamma(a+b)}{\Gamma(a)\Gamma(b)}\ta^{a-1}(1-\ta)^{b-1}. \]
The likelihood of $\ta$ is
\[f(x|\ta) = \binom{n}{x}\ta^x(1-\ta)^{n-x}. \]
Recall Bayes rule
\begin{IEEEeqnarray*}{rCl}
    \text{posterior}&=&\frac{\text{joint density}}{\text{marginal of } x}\\
    &=&\frac{\text{prior} \times \text{likelihood}}{\text{marginal of } x}\\
    \therefore p(\ta|x) &=& \frac{p(x,\ta)}{p(x)}\\
    &=&\frac{\pi(\ta)f(x|\ta)}{p(x)}\\
    &=& \frac{\pi(\ta)f(x|\ta)}{\int \pi(\ta')f(x|\ta')d\ta'}
\end{IEEEeqnarray*}
Thus 
\begin{IEEEeqnarray*}{rCl}
    \pi(\ta | x)&\propto& \text{likelihood} \times \text{prior}\\
    &\propto& \ta^{x+a-1}(1-\ta)^{n-x+b}\\
    &\propto& \text{Beta}(x+a, n-x+b).
\end{IEEEeqnarray*}
Thus the Bayes estimator under squared error loss is the mean of $\text{Beta}(x+a, n-x+b)$ which is $\frac{x+a}{n+a+b}$ (exercise). Note that
\[\frac{x+a}{n+a+b} = \frac{n}{n+a+b}\frac{x}{n}+ \frac{a+b}{n+a+b}\frac{a}{a+b} =  \la_n \cdot \text{UMVUE}+ (1-\la_n)\cdot \text{prior mean}. \]
Thus the Bayes estimator is a convex combination of the UMVUE and the prior. Also $\la_n \to 1$ and $n \to \infty$ and so with enough data we approach the UMVUE. If $n$ is small compared to $a+b$, then the Bayes estimator is closer to the mean of the Bayes's prior.
  
\end{document}
\documentclass{article}
\usepackage{ae,aecompl}
\usepackage{todonotes}
\usepackage{chngcntr}
\usepackage{tikz-cd}
\usepackage{graphicx}
\graphicspath{ {./images/}}
\usepackage[all,cmtip]{xy}
\usepackage{amsmath, amscd}
\usepackage{amsthm}
\usepackage{amssymb}
\usepackage{amsfonts}
\usepackage{bm}
\usepackage{qsymbols}
\usepackage{latexsym}
\usepackage{mathrsfs}
\usepackage{mathtools}
\usepackage{cite}
\usepackage{color}
\usepackage{url}
\usepackage{enumerate}
\usepackage{verbatim}
\usepackage[draft=false, colorlinks=true]{hyperref}
\usepackage{pdfpages}
\usepackage[margin=1.2in]{geometry}
\usepackage{IEEEtrantools}

\usepackage{fancyhdr}


\usepackage[nameinlink]{cleveref}


\DeclareMathOperator*{\ac}{accept}
\DeclareMathOperator*{\amax}{argmax}
\DeclareMathOperator*{\amin}{argmin}
\DeclareMathOperator*{\Aut}{Aut}
\newcommand {\al}{{\alpha}}
\newcommand {\abs}[1]{{\left\lvert#1\right\rvert}}
\newcommand {\A}{{\mathcal{A}}}
\newcommand {\AM}{{\mathrm{AM}}}
\newcommand {\AMp}{{\AM_{p}^{X}\!(\Ri_\w)}}
\newcommand {\B}{{\mathcal{B}}}
\DeclareMathOperator*{\Be}{Bern}
\newcommand {\Br}{{\dot{B}}}
\newcommand {\Ba}{{\mathfrak{B}}}
\newcommand {\C}{{\mathbb C}}
\newcommand {\ce}{\mathrm{c}}
\newcommand {\Ce}{\mathrm{C}}
\newcommand {\Cc}{\mathrm{C_{c}}}
\newcommand {\Ccinf}{\mathrm{C_{c}^{\infty}}}
\DeclareMathOperator{\cov}{Cov}
\DeclareMathOperator{\DEV}{DEV}
\newcommand {\Di}{{\mathbb D}}
\newcommand {\dom}{\mathrm{dom}}
\newcommand{\dist}{\stackrel{\mathrm{dist}}{=}}
\newcommand {\ud}{\mathrm{d}}
\newcommand {\ue}{\mathrm{e}}
\newcommand {\eps}{\varepsilon}
\newcommand {\veps}{\varepsilon}
\newcommand {\vrho}{{\varrho}}
\newcommand {\E}{{\mathbb{E}}}
\newcommand {\Ec}{{\mathcal{E}}}
\newcommand {\Ell}{L}
\newcommand {\Ellp}{{L_{p}[0,1]}}
\newcommand {\Ellpprime}{{L_{p'}([0,1])}}
\newcommand {\Ellq}{{L_{q}([0,1])}}
\newcommand {\Ellqprime}{{L_{q'}([0,1])}}
\newcommand {\Ellr}{L^{r}}
\newcommand {\Ellone}{{L_{1}([0,1])}}
\newcommand{\Elltwo}{{L_{2}([0,1])}}
\newcommand{\Ellinfty}{L^{\infty}}
\newcommand{\Ellinftyc}{L_{\mathrm{c}}^{\infty}}
\newcommand{\exb}[1]{\exp\left\{#1\right\}}
\DeclareMathOperator*{\Ext}{Ext}
\newcommand{\F}{{\mathcal{F}}}
\newcommand{\Fe}{{\mathbb{F}}}
\newcommand{\G}{{\mathcal{G}}}
\newcommand{\HF}{\mathcal{H}_{\text{FIO}}^{1}(\Rd)}
\newcommand{\Hr}{H}
\newcommand{\HT}{\mathcal{H}}
\newcommand{\ui}{\mathrm{i}}
\newcommand{\I}{{I}}
\newcommand{\J}{{\mathcal{J}}}
\newcommand{\id}{{\mathrm{id}}}
\newcommand{\iid}{\stackrel{\mathclap{\normalfont\mbox{iid}}}{\sim}}
\newcommand{\im}{{\text{im }}}
\newcommand{\ind}{{\perp\!\!\!\perp}}
\DeclareMathOperator*{\Int}{int}
\newcommand{\intx}{{\overline{\int_{X}}}}
\newcommand{\inte}{{\overline{\int_{\E}}}}
\newcommand{\la}{\lambda}
\newcommand{\rb}{\rangle}
\newcommand{\lb}{{\langle}}
\newcommand{\La}{\Lambda}
\newcommand{\calL}{{\mathcal{L}}}
\newcommand{\lp}{{\mathcal{L}}^{p}}
\newcommand{\lpo}{{\overline{\mathcal{L}}^{p}\!}}
\newcommand{\Lpo}{{\overline{\Ell}^{p}\!}}
\newcommand{\M}{{\mathbf{M}}}
\newcommand{\Ma}{{\mathcal{M}}}
\newcommand{\N}{{{\mathbb N}}}
\newcommand{\Na}{{{\mathcal{N}}}}
\newcommand{\norm}[1]{\left\|#1\right\|}
\newcommand{\normm}[1]{{\left\vert\kern-0.25ex\left\vert\kern-0.25ex\left\vert #1 
    \right\vert\kern-0.25ex\right\vert\kern-0.25ex\right\vert}}
\newcommand{\Om}{{{\Omega}}}
\newcommand{\one}{{{\bf 1}}}
\newcommand{\pic}{\text{Pic }}
\newcommand{\ph}{{\varphi}}
\newcommand{\Pa}{{\mathbb{P}}}
\newcommand{\Po}{{\mathcal{P}}}
\newcommand{\Q}{{\mathbb{Q}}}
\newcommand{\R}{{\mathbb R}}
\newcommand{\Rd}{{\mathbb{R}^{d}}}
\DeclareMathOperator{\rej}{reject }
\newcommand{\Rn}{{\mathbb{R}^{n}}}
\newcommand{\cR}{{\mathcal{R}}}
\newcommand{\Rad}{{\mathrm{Rad}}}
\newcommand{\ran}{{\mathrm{ran}}}
\newcommand{\Ri}{{\mathrm{R}}}
\newcommand{\supp}{{\mathrm{supp}}}
\newcommand{\Se}{\mathrm{S}}
\newcommand{\Sp}{S^{*}(\Rn)}
\newcommand{\St}{{\mathrm{St}}}
\newcommand{\Sw}{\mathcal{S}}
\newcommand{\T}{{\mathcal{T}}}
\newcommand{\ta}{{\theta}}
\newcommand{\Ta}{{\Theta}}
\newcommand{\topp}{\stackrel{p}{\to}}
\newcommand{\todd}{\stackrel{d}{\to}}
\newcommand{\toL}[1]{\stackrel{L^{#1}}{\to}} 
\newcommand{\toas}{\stackrel{a.s.}{\to}}
\DeclareMathOperator{\V}{Var}
\newcommand {\w}{{\omega}}
\newcommand {\W}{{\mathrm{W}}}
\newcommand {\Wnp}{\text{$\mathrm{W}$\textsuperscript{$n,\!p$}}}
\newcommand {\Wnpeq}{\text{$\mathrm{W}$\textsuperscript{$n\!,\!p$}}}
\newcommand {\Wonep}{\text{$\mathrm{W}$\textsuperscript{$1,\!p$}}}
\newcommand {\Wonepeq}{\text{$\mathrm{W}$\textsuperscript{$1\!,\!p$}}}
\newcommand {\X}{{\mathcal{X}}}
\newcommand {\Z}{{{\mathbb Z}}}
\newcommand {\Za}{{\mathcal{Z}}}
\newcommand {\Zd}{{\Z[\sqrt{d}]}}
\newcommand {\vanish}[1]{\relax}

\newcommand {\wh}{\widehat}
\newcommand {\wt}{\widetilde}
\newcommand {\red}{\color{red}}

% Distributions
\newcommand{\normal}{\mathsf{N}}
\newcommand{\poi}{\mathsf{Poisson}}
\newcommand{\bern}{\mathsf{Bernoulli}}
\newcommand{\bin}{\mathsf{Binomal}}
\newcommand{\multi}{\mathsf{Multinomial}}
\newcommand{\Exp}{\mathsf{Exp}}



% put your command and environment definitions here




% some theorem environments
% remove "[theorem]" if you do not want them to use the same number sequence


  \newtheorem{thrm}{Theorem}
  \newtheorem{lemma}{Lemma}
  \newtheorem{prop}{Proposition}
  \newtheorem{cor}{Corollary}

  \newtheorem{conj}{Conjecture}
  \renewcommand{\theconj}{\Alph{conj}}  % numbered A, B, C etc

  \theoremstyle{definition}
  \newtheorem{defn}{Definition}
  \newtheorem{ex}{Example}
  \newtheorem{exs}{Examples}
  \newtheorem{question}{Question}
  \newtheorem{remark}{Remark}
  \newtheorem{notn}{Notation}
  \newtheorem{exer}{Exercise}




\title{STATS3100A - Lecture 10}
\author{Dominik Rothenhaeusler\\ Scribed by Michael Howes}
\date{10/20/21}

\pagestyle{fancy}
\fancyhf{}
\rhead{STATS300A - Lecture 10}
\lhead{10/20/21}
\rfoot{Page \thepage}

\begin{document}
\maketitle
\tableofcontents
\section{Announcements}
\begin{itemize}
    \item No new homework this week.
    \item Last years midterm+solutions are on Canvas.
    \item Midterm review session Monday 6pm on Zoom.
    \item Midterm in one week time. Midterm is open book and 90 minutes long.
    \item The relevant content is everything up to and including today.
\end{itemize}
\section{Minimaxity and limits of Bayes estimators}
Recall that if a Bayes estimator $\delta_\Lambda$ has constant risk, then $\delta_\Lambda$ is minimax. 
\begin{ex}
    Suppose $X_1,\ldots, X_n \iid N(\ta,\sigma^2)$ where $\sigma^2$ is known and $L(\ta,d)=(\ta-d)^2$. We have seen that $\bar{X}$ has constant risk but it cannot be the Bayes estimator for any proper prior since it is unbiased. One can show that $\bar{X}$ is a generalized Bayes estimator w.r.t. the improper prior $\pi(\ta)=1$. We wish to show that $\bar{X}$ is minimax. To do this we will look at limits of Bayes estimators
\end{ex}
\begin{defn}
    Let $(\Lambda_m)_{m=1}^\infty$ be a sequence of priors with $r_{\Lambda_m} :=\inf_\delta r(\Lambda_m, \delta)$. The sequence $(\Lambda_m)$ is called \emph{least favourable} if $r_{\Lambda_m} \to r$ and $r \ge r_{\Lambda'}$ for all priors $\Lambda'$.
\end{defn}
\begin{thrm}
    \emph{[TPE 5.1.12]} Suppose $(\Lambda_m)$ is a sequence of priors with $r_{\Lambda_m} \to r < \infty$. Let $\delta$ be an estimator such that $\sup_\ta R(\ta,\delta) = r$, Then $\delta$ is minimax and $(\Lambda_m)$ is least favourable.
\end{thrm}
\begin{proof}
    Let $\delta'$ be an estimator. We know that 
    \[\sup_\ta R(\ta,\delta') \ge \int_\Om R(\ta,\delta')d\Lambda_m \ge r_{\Lambda_m}. \]
    Letting $m \to \infty$ we can conclude that 
    \[\sup_\ta R(\ta,\delta') \ge r = \sup_{\ta}R(\ta,\delta). \]
    Thus $\delta$ is minimax. Now let $\Lambda'$ be a prior. We know that 
    \begin{align*}
        r_{\Lambda'}&\le \int_\Om R(\ta,\delta)d\Lambda'\\
        &\le\sup_\ta R(\ta,\delta)\\
        &=r.
    \end{align*}
    Thus $(\Lambda_m)$ is least favourable.
\end{proof}
\begin{ex}
    Returning to our example with $X_1,\ldots,X_n \iid N(\ta,\sigma^2)$ and $L(\ta,d) = (\ta-d)^2$. Consider the prior $\Lambda_m \sim N(0,m^2)$. We have seen that 
    \[\Ta|X \sim N\left(\frac{\frac{n}{\sigma^2}}{\frac{n}{\sigma^2}+\frac{1}{m^2}}\bar{X}, \frac{1}{\frac{n}{\sigma^2}+\frac{1}{m^2}}\right).\]
    Thus $\delta_{\Lambda_m} = \frac{n}{\sigma^2}{\frac{n}{\sigma^2}+\frac{1}{m^2}}\bar{X}$ is the posterior mean and hence the Bayes estimator for $\Lambda_m$. Note that $\delta_{\Lambda_n}(X)$ is the mean of $\Ta|X$ and thus
    \begin{align*}
        r_{\Lambda_m}&=\E\left[L(\Ta,\delta_{\Lambda_m}(X))\right]\\
        &=\E\left[(\Ta-\delta_{\Lambda_n}(X))^2\right]\\
        &=\E\left[\E[(\Ta-\delta_{\Lambda_n}(x))^2|X=x]\right]\\
        &=\E\left[\V(\Ta|X=x)\right]\\
        &=\frac{1}{\frac{n}{\sigma^2}+\frac{1}{m^2}}.
    \end{align*} 
    Thus as $m\to \infty$ we have $r_{\Lambda_m} \to \frac{\sigma^2}{n} = R(\ta,\bar{X})$. Thus $\bar{X}$ is minimax.
\end{ex}
\section{Randomized estimators}
Recall that a randomized estimator is one of the form $\delta(X,U)$ where $U \sim \text{Unif}([0,1])$ and $U\ind X$. This is in constrast to estimators of the form $\delta(X)$ which are deterministic functions of the data. We saw that for convex loss functions we can ignore randomized estimators since by Jensen's inequality
\[\E_\ta L(\ta,\delta(X,U)) \ge \E_\ta L(\ta, \E[\delta(X,U)|X]) = \E_\ta L(\ta, \eta(X)). \]
For non-convex loss functions the minimax estimator may be randomized.
\begin{ex}
    Consider $X$ a binomial random variable with parameters $(n,\ta)$ where $n$ is fixed and $\ta \in [0,1]$. Consider the 0-1 loss
    \begin{align*}
        L(\ta,d) = \begin{cases}
            0&\text{if } \abs{d-\ta} < \al,\\
            1&\text{else}.
        \end{cases}
    \end{align*}
    If $\delta$ is a non-randomized estimator, $\delta$ can take at most $n+1$ values. For $\al < \frac{1}{2(n+1)}$, there exists $\ta_0$ such that $\abs{\delta(x)-\ta_0} \ge \al$ for all $x$ and thus the worst case risk of $\delta$ is 1. Consider $\delta'(X,U) = U$. For every $\ta$,
    \[R(\ta,\delta') = \Pa(\abs{U-\ta} \ge \al) \le 1-\al. \]
    Thus the worst case risk of $\delta'$ is $1-\al < 1$.
\end{ex}
\section{``Boosting'' via submodel restrictions}
Consider $X_1,\ldots, X_n \iid N(\ta,\sigma^2)$ where $\sigma^2$ and $\ta$ are both unknown but $\sigma^2 \le B_0$. Consider squared error loss $L(\ta,d)= (\ta-d)^2$. Is $\bar{X}$ minimax? To prove this we need to show that
\[\sup_{\ta,\sigma^2 \le B_0} R((\ta,\sigma^2), \delta) \ge \sup_{\ta,\sigma^2\le B_0}R((\ta,\sigma^2),\bar{X}), \]
where $\delta$ is any estimator. We know that 
\[\sup_{\ta,\sigma^2\le B_0}R((\ta,\sigma^2),\bar{X}) = \sup_{\ta,\sigma^2\le B_0} \frac{\sigma^2}{n} = \frac{B_0}{n}. \]
Thus (and this is crucial)
\[\sup_{\ta,\sigma^2\le B_0}R((\ta,\sigma^2),\bar{X}) =\sup_{\ta,\sigma^2= B_0}R((\ta,\sigma^2),\bar{X}).  \]
We know that for fixed variance $\bar{X}$ is minimax. Thus for any estimator $\delta$ we have
\begin{align*}
    \sup_{\ta,\sigma^2 \le B_0} R((\ta,\sigma^2), \delta)&\ge \sup_{\ta,\sigma^2 = B_0} R((\ta,\sigma^2), \delta)\\
    &\ge \sup_{\ta,\sigma^2 = B_0} R((\ta,\sigma^2), \bar{X})\\
    &=\sup_{\ta,\sigma^2 \le B_0} R((\ta,\sigma^2), \bar{X}).
\end{align*}
Thus $\bar{X}$ is minimax. Formalizing this example, we have:
\begin{lemma}
    \emph{[TPE 5.1.15]} If $\delta$ is minimax for a submodel $\Om_0 \subseteq \Om$ and $\sup_{\ta \in \Om}R(\ta,\delta)= \sup_{\ta \in \Om_0}R(\ta,\delta)$, then $\delta$ is minimax for the full model $\Om$.
\end{lemma}
\begin{proof}
    This is the same argument we saw in the example. For any estimator $\delta'$,
    \begin{align*}
        \sup_{\ta \in \Om} R(\ta,\delta')&\ge \sup_{\ta \in \Om_0} R(\ta,\delta')\\
        &\ge \sup_{\ta \in \Om_0}R(\ta,\delta)\\
        &=\sup_{\ta \in \Om} R(\ta,\delta) \qedhere
    \end{align*}
\end{proof}
\begin{ex}
    [Non-parametric] Suppse $X_1,\ldots, X_n \iid F$, where $F \in \F$ has mean $\mu(F)$ and variance $\sigma^2(F) < B$. Our goal is to estimate $\mu(F)$ under $L(F,d) = (\mu(F)-d)^2$. We wish to show that the minimax estimator is $\delta(X)=\bar{X}$.  We know that $\bar{X}$ is minimax on the submodel $\F_0 = \{N(\ta,\sigma^2):\sigma^2 \le B\}$. On the full model $\F$, the estimator $\bar{X}$ has risk $R(F,\bar{X}) = \frac{\sigma^2(F)}{n} \le \frac{B}{n}$. Thus we hav 
    \[\sup_{F\in \F_0}R(F,\bar{X}) = \sup_{F \in \F} R(F,\bar{X}). \]
    Thus by boosting, $\bar{X}$ is sufficient on the full model.
\end{ex}
\begin{ex}
    [Another non-parametric example] Now suppose that $\F$ is the set of all distributions with support on $[0,1]$. Again we will do estimation with squared error loss $L(F,d) = (\mu(F)-d)^2$. Suppose $X_i \iid \F$. Consider the submodel $\F_0$ of all Bernoulli distributions with parameter $\ta \in (0,1)$. We have seen that in this setting the minimax estimator is
    \[\delta_n(X) = \frac{\sum_{i=1}^n X_i+\frac{\sqrt{n}}{2}}{n+\sqrt{n}}. \]
    By studying the risk of $\delta_n$ on the full model we can show that $\delta_n$ is minimax on $\F$.
\end{ex}
\end{document}
\documentclass{article}
\usepackage{ae,aecompl}
\usepackage{todonotes}
\usepackage{chngcntr}
\usepackage{tikz-cd}
\usepackage{graphicx}
\graphicspath{ {./images/}}
\usepackage[all,cmtip]{xy}
\usepackage{amsmath, amscd}
\usepackage{amsthm}
\usepackage{amssymb}
\usepackage{amsfonts}
\usepackage{bm}
\usepackage{qsymbols}
\usepackage{latexsym}
\usepackage{mathrsfs}
\usepackage{mathtools}
\usepackage{cite}
\usepackage{color}
\usepackage{url}
\usepackage{enumerate}
\usepackage{verbatim}
\usepackage[draft=false, colorlinks=true]{hyperref}
\usepackage{pdfpages}
\usepackage[margin=1.2in]{geometry}
\usepackage{IEEEtrantools}

\usepackage{fancyhdr}


\usepackage[nameinlink]{cleveref}


\DeclareMathOperator*{\ac}{accept}
\DeclareMathOperator*{\amax}{argmax}
\DeclareMathOperator*{\amin}{argmin}
\DeclareMathOperator*{\Aut}{Aut}
\newcommand {\al}{{\alpha}}
\newcommand {\abs}[1]{{\left\lvert#1\right\rvert}}
\newcommand {\A}{{\mathcal{A}}}
\newcommand {\AM}{{\mathrm{AM}}}
\newcommand {\AMp}{{\AM_{p}^{X}\!(\Ri_\w)}}
\newcommand {\B}{{\mathcal{B}}}
\DeclareMathOperator*{\Be}{Bern}
\newcommand {\Br}{{\dot{B}}}
\newcommand {\Ba}{{\mathfrak{B}}}
\newcommand {\C}{{\mathbb C}}
\newcommand {\ce}{\mathrm{c}}
\newcommand {\Ce}{\mathrm{C}}
\newcommand {\Cc}{\mathrm{C_{c}}}
\newcommand {\Ccinf}{\mathrm{C_{c}^{\infty}}}
\DeclareMathOperator{\cov}{Cov}
\DeclareMathOperator{\DEV}{DEV}
\newcommand {\Di}{{\mathbb D}}
\newcommand {\dom}{\mathrm{dom}}
\newcommand{\dist}{\stackrel{\mathrm{dist}}{=}}
\newcommand {\ud}{\mathrm{d}}
\newcommand {\ue}{\mathrm{e}}
\newcommand {\eps}{\varepsilon}
\newcommand {\veps}{\varepsilon}
\newcommand {\vrho}{{\varrho}}
\newcommand {\E}{{\mathbb{E}}}
\newcommand {\Ec}{{\mathcal{E}}}
\newcommand {\Ell}{L}
\newcommand {\Ellp}{{L_{p}[0,1]}}
\newcommand {\Ellpprime}{{L_{p'}([0,1])}}
\newcommand {\Ellq}{{L_{q}([0,1])}}
\newcommand {\Ellqprime}{{L_{q'}([0,1])}}
\newcommand {\Ellr}{L^{r}}
\newcommand {\Ellone}{{L_{1}([0,1])}}
\newcommand{\Elltwo}{{L_{2}([0,1])}}
\newcommand{\Ellinfty}{L^{\infty}}
\newcommand{\Ellinftyc}{L_{\mathrm{c}}^{\infty}}
\newcommand{\exb}[1]{\exp\left\{#1\right\}}
\DeclareMathOperator*{\Ext}{Ext}
\newcommand{\F}{{\mathcal{F}}}
\newcommand{\Fe}{{\mathbb{F}}}
\newcommand{\G}{{\mathcal{G}}}
\newcommand{\HF}{\mathcal{H}_{\text{FIO}}^{1}(\Rd)}
\newcommand{\Hr}{H}
\newcommand{\HT}{\mathcal{H}}
\newcommand{\ui}{\mathrm{i}}
\newcommand{\I}{{I}}
\newcommand{\J}{{\mathcal{J}}}
\newcommand{\id}{{\mathrm{id}}}
\newcommand{\iid}{\stackrel{\mathclap{\normalfont\mbox{iid}}}{\sim}}
\newcommand{\im}{{\text{im }}}
\newcommand{\ind}{{\perp\!\!\!\perp}}
\DeclareMathOperator*{\Int}{int}
\newcommand{\intx}{{\overline{\int_{X}}}}
\newcommand{\inte}{{\overline{\int_{\E}}}}
\newcommand{\la}{\lambda}
\newcommand{\rb}{\rangle}
\newcommand{\lb}{{\langle}}
\newcommand{\La}{\Lambda}
\newcommand{\calL}{{\mathcal{L}}}
\newcommand{\lp}{{\mathcal{L}}^{p}}
\newcommand{\lpo}{{\overline{\mathcal{L}}^{p}\!}}
\newcommand{\Lpo}{{\overline{\Ell}^{p}\!}}
\newcommand{\M}{{\mathbf{M}}}
\newcommand{\Ma}{{\mathcal{M}}}
\newcommand{\N}{{{\mathbb N}}}
\newcommand{\Na}{{{\mathcal{N}}}}
\newcommand{\norm}[1]{\left\|#1\right\|}
\newcommand{\normm}[1]{{\left\vert\kern-0.25ex\left\vert\kern-0.25ex\left\vert #1 
    \right\vert\kern-0.25ex\right\vert\kern-0.25ex\right\vert}}
\newcommand{\Om}{{{\Omega}}}
\newcommand{\one}{{{\bf 1}}}
\newcommand{\pic}{\text{Pic }}
\newcommand{\ph}{{\varphi}}
\newcommand{\Pa}{{\mathbb{P}}}
\newcommand{\Po}{{\mathcal{P}}}
\newcommand{\Q}{{\mathbb{Q}}}
\newcommand{\R}{{\mathbb R}}
\newcommand{\Rd}{{\mathbb{R}^{d}}}
\DeclareMathOperator{\rej}{reject }
\newcommand{\Rn}{{\mathbb{R}^{n}}}
\newcommand{\cR}{{\mathcal{R}}}
\newcommand{\Rad}{{\mathrm{Rad}}}
\newcommand{\ran}{{\mathrm{ran}}}
\newcommand{\Ri}{{\mathrm{R}}}
\newcommand{\supp}{{\mathrm{supp}}}
\newcommand{\Se}{\mathrm{S}}
\newcommand{\Sp}{S^{*}(\Rn)}
\newcommand{\St}{{\mathrm{St}}}
\newcommand{\Sw}{\mathcal{S}}
\newcommand{\T}{{\mathcal{T}}}
\newcommand{\ta}{{\theta}}
\newcommand{\Ta}{{\Theta}}
\newcommand{\topp}{\stackrel{p}{\to}}
\newcommand{\todd}{\stackrel{d}{\to}}
\newcommand{\toL}[1]{\stackrel{L^{#1}}{\to}} 
\newcommand{\toas}{\stackrel{a.s.}{\to}}
\DeclareMathOperator{\V}{Var}
\newcommand {\w}{{\omega}}
\newcommand {\W}{{\mathrm{W}}}
\newcommand {\Wnp}{\text{$\mathrm{W}$\textsuperscript{$n,\!p$}}}
\newcommand {\Wnpeq}{\text{$\mathrm{W}$\textsuperscript{$n\!,\!p$}}}
\newcommand {\Wonep}{\text{$\mathrm{W}$\textsuperscript{$1,\!p$}}}
\newcommand {\Wonepeq}{\text{$\mathrm{W}$\textsuperscript{$1\!,\!p$}}}
\newcommand {\X}{{\mathcal{X}}}
\newcommand {\Z}{{{\mathbb Z}}}
\newcommand {\Za}{{\mathcal{Z}}}
\newcommand {\Zd}{{\Z[\sqrt{d}]}}
\newcommand {\vanish}[1]{\relax}

\newcommand {\wh}{\widehat}
\newcommand {\wt}{\widetilde}
\newcommand {\red}{\color{red}}

% Distributions
\newcommand{\normal}{\mathsf{N}}
\newcommand{\poi}{\mathsf{Poisson}}
\newcommand{\bern}{\mathsf{Bernoulli}}
\newcommand{\bin}{\mathsf{Binomal}}
\newcommand{\multi}{\mathsf{Multinomial}}
\newcommand{\Exp}{\mathsf{Exp}}



% put your command and environment definitions here




% some theorem environments
% remove "[theorem]" if you do not want them to use the same number sequence


  \newtheorem{thrm}{Theorem}
  \newtheorem{lemma}{Lemma}
  \newtheorem{prop}{Proposition}
  \newtheorem{cor}{Corollary}

  \newtheorem{conj}{Conjecture}
  \renewcommand{\theconj}{\Alph{conj}}  % numbered A, B, C etc

  \theoremstyle{definition}
  \newtheorem{defn}{Definition}
  \newtheorem{ex}{Example}
  \newtheorem{exs}{Examples}
  \newtheorem{question}{Question}
  \newtheorem{remark}{Remark}
  \newtheorem{notn}{Notation}
  \newtheorem{exer}{Exercise}




\title{STATS300A - Lecture 18}
\author{Dominik Rothenhaeusler\\ Scribed by Michael Howes}
\date{11/29/21}

\pagestyle{fancy}
\fancyhf{}
\rhead{STATS300A - Lecture 18}
\lhead{11/29/21}
\rfoot{Page \thepage}

\begin{document}
\maketitle
\tableofcontents
\section{Announcements}
\begin{itemize}
    \item The final exam is at 3:30pm December $8^{th}$ Wednesday.
    \item The exam is an online timed assignment.
    \item The exam is three hours long and has the same rules as the midterm.
    \item Approximately 1/3 of the exam will be on the first half of the course and approximately 2/3 of the exam will be on the second half.
\end{itemize}
\section{Multiple testing}
Historically people haved worked in a setting where first they will fix a question, then collect data and then perform inference. Today people are more likely to collect a lot of data, then ask data dependent questions and then do inference. This can be viewed as asking many many questions about the data and requires different techniques. 
\subsection{Setting}
As before we have data $X \sim \Pa \in \Po$ where we observe $X$ but we do not know $\Pa$. We are given $n$ null hypotheses $H_{0,i}$ for $i=1,\ldots, n$. Rather than thinking of each null as a partition of our parameter space we will work directly with $\Po$. That is $\Po = H_{0,i} \cup H_{1,i}$ where $H_{1,i} = H_{0,i}^c$. 

For each null hypothesis we have a p-value $p_i$ such that under $H_{0,i}$,
\[\Pa(p_i \le t) \le t. \]
That is, under the null $H_{0,i}$, the p-value $p_i$ stochastically dominante the uniform distribution. For simplicity we will in fact assume that $p_i$ is uniformly distributed under $H_{0,i}$ so that $\Pa(p_i \le t)=t$ under $H_{0,i}$. 
\subsection{Motivation}
What is the problem that multiple testing is meant to solve? Suppose that we have $X_i \sim \Na(\ta_i,1)$ $i=1,\ldots,n$ where $n=10,000$ and $X_i$ are independent. Suppose we want to test $H_{0,i} :\ta_i=0$ against $H_{0,i}: \ta_i <0$. We can define our p-values as 
\[p_i = \psi(X_i), \]
where $\psi$ is the CDF of a standard Gaussian distribution. The test $\phi_i = \one_{p_i \le \al}$ is thus the UMP level-$\al$ test for $H_{0,i}$. If $\al = 0.05$ and $\ta_i=0$ for all $i$, then we would expect approximately $n\times \al = 500$ false discoveries.

\subsection{Different goals}
There are different quantities we can work with in multiple testing. For example:
\begin{enumerate}
    \item We can test the \emph{global null}. That is we wish to test the null $H_0 = \bigcap_{i=1}^n H_{0,i}$ where every null is true. In this setting we wish to find a function $\Phi_G : [0,1]^n \to \{0,1\}$ where $\Phi_G$ is a function of our p-values $p=(p_1,\ldots,p_n)$ and $\Phi_G(p)=1$ means that for the given p-values $p$ we reject the global null and $\Phi_G(p)=0$ means we do not reject the global null. In this setting we wish to control the \emph{global type I error} of $\Phi_G$ which is 
    \[\text{Global type I error}(\Phi_G) = \begin{cases}
        \Pa(\Phi_G=1)&\text{if } H_0 \text{ is true,}\\
        0 & \text{if $H_0$ is false.}
    \end{cases} \]
    \item We can also work with the \emph{family wise error rate (FWER)}. This is the probability of making one or more false discoveries. In this case we want a function $\Phi:[0,1]^n \to \{0,1\}^n$ where each component $\Phi_i$ is a function of our p-values and $\Phi_i(p)=1$ means we reject the null $H_{0,i}$ and $\Phi_i(p)=0$ means we do not reject the null $H_{0,i}$. In this setting we can define a random variable $V$ which counts the $i$'s such that $\Phi_i(p)=1$ and $H_{0,i}$ is true. Thus $V$ is the number of false discoveries. We then define
    \[FWER = FWER(\Phi) := \Pa(V > 1).\]
    We wish to find powerful procedures $\Phi$ such that $FWER \le \al$. 
    \item We can also work with the \emph{false discover rate (FDR)}. Let $R$ be the total number of rejections and define
    \[FDR =FDR(\Phi) := \E\left[\frac{V}{\max\{R,1\}}\right].\]
    Again we are interested in powerful procedures $\Phi$ such that $FDR \le \al$.
\end{enumerate}
\subsection{Comparing error rates}
Given a procedure $\Phi:[0,1]^n \to \{0,1\}$ for $H_{0,i}$, $i=1,\ldots,n$, we can define a global procedure for $H_0 = \bigcap_{i=1}^n H_{0,i}$ by
\[\Phi_G(p)=\max\{\Phi_1(p),\ldots, \Phi_n(p)\}. \]
Thus $\Phi_G$ rejects the global null $H_0$ if and only if for some $i$, $\Phi_i$ rejects the null $H_{0,i}$. This procedure is natural in settings were we expect the false nulls to be sparse. For this choice of $\Phi_G$, we have the following comparison between the different quantities we want to control:
\[\text{Global null type I error}(\Phi_G) \le FDR(\Phi) \le FWER(\Phi). \]
\begin{proof}
    If the global null $H_0$ is false, then $\text{Global null type I error}(\Phi_G)=0$ so we automatically have $\text{Global null type I error}(\Phi_G) \le FDR$. If $H_0$ is true, then all of nulls $H_{0,i}$ are true and so every rejection is a false rejection. This implies that $V=R$ and so 
    \[FDR = \Pa(V>0) = \Pa(\Phi_G=1) = \text{Global null type I error}(\Phi_G). \]
    For the second inequality we have $V \le R$ and so 
    \[\frac{V}{\max\{R,1\}}\le \frac{V}{\max\{V,1\}} = \one_{V > 0}. \]
    Thus 
    \[FDR = \E\left[\frac{V}{\max\{R,1\}}\right] \le \E[\one_{V > 0}]=\Pa(V>0)=FWER.\qedhere\]
\end{proof}
The different error criteria have different uses.
\begin{itemize}
    \item Testing the global null is for ``detecting.''
    \item Controlling the FWER or FDR is for ``locating.''
\end{itemize}
Multiple testing is an active area of research and if you are interested, you should consider attending the \href{https://www.selectiveinferenceseminar.com/home}{International seminar on selective inference}.
\section{Multiple testing proceedures}
We will now consider a number of methods that can be used when doing multiple testing.
\subsection{Bonferroni correction}
Define $\Phi_i = \one_{p_i \le \frac{\al}{n}}$. We will show that this proceedure control FWER at $\al$. That is, $FWER \le \al$. Note that
\begin{align*}
    \Pa(V > 0)&=\Pa(\Phi_i = 1 \text{ for some } i \text{ such that $H_{0,i}$ is true})\\
    &\le \sum_{i, H_{0,i} \text{ is true}}\Pa(\Phi_i = 1)\\
    &= \sum_{i, H_{0,i} \text{ is true}}\Pa(p_i \le \al/n)\\
    &= \frac{n_0}{n}\al\\
    &\le \al,
\end{align*}
where $n_0 \le n$ is the number of true nulls. Note that we did not put any independence assumptions on our p-values. The optimality of Bonferroni depends on the correlation between our p-values. Suppose that $p_i$ are all independent and uniform and consider a test of the form $\Phi_i(p) = \one_{p_i \le t}$ for some value of $t$ that does not depend on $i$. Suppose that the global null is true. Under this assumption, we have
\begin{align*}
    FWER &= 1-\Pa(V=0)\\
    &=1-\Pa(p_i > t, \text{ for all } i)\\
    &=1-(1-t)^n.
\end{align*}
If we wish to have $FWER=\al$ we get $t = 1-(1-\al)^{1/n}\approx \al/n$. So that Bonferroni is approximately optimal for small $\al$, large $n$ and independent p-values. If instead the p-values have positive dependence, then Bonferroni is sub-optimal. Suppose in an extreme case that $p_1=\ldots =p_n$. Then
\[FWER = 1-\Pa(V=0)=1-\Pa(p_1 > t) =t.\]
So the optimal choice of $t$ is $\al$ which is much larger than $\al/n$. In the case of negative dependence, it can be shown that Bonferroni is optimal.
\subsection{Holm's procedure}
How can we improve Bonferroni? Note that after we reject $H_{0,i}$ we have two possibilities. Either we have made a false discovery or we have made a true discovery and the remaining hypotheses become a multiple testing problem with $n-1$ null hypotheses. Thus after making on rejection we can ``relax'' the rejection criteria. More formally, we first order the p values
\[p_{(1)}\le p_{(2)}\le \ldots \le p_{(n)}, \]
and relabel the corresponding null hypotheses $H_{0,(1)}, H_{0,(2)},\ldots, H_{0,(n)}$. Let 
\[j = \min\left\{i : p_{(i+1)} > \frac{\al}{n-i}, i=0,1,\ldots,n-1, \right\}. \]
We then reject the nulls $H_{0,(1)},\ldots,H_{0,(j)}$. Thus $p_{(i)} \le \frac{\al}{n-i+1}$ for rejected $H_{0,(i)}$.
\begin{prop}
Holm's procedure controls FWER at level $\al$.
\end{prop}
\begin{proof}
    Let $i_0$ be the first index $i$ for which $H_{0,(i)}$ is true. The quantity $i_0$ is a random variable since it depends on the ordering of the random variables $p_i$. Let $n_0$ be the number of true nulls, we thus have $i_0 \le n-n_0+1$ and so $n_0 \le n-i_0+1$ and $\frac{\al}{n_0} \ge \frac{\al}{n-i_0+1}$. Now note that
    \begin{align*}
        FWER &=\Pa(V > 0)\\
        &=\Pa\left(p_{(1)} \le \frac{\al}{n}, p_{(2)} \le \frac{\al}{n-1},\ldots, p_{(i_0)} \le \frac{\al}{n-i_0+1}\right)\\
        &\le \Pa\left(p_{(i_0)} \le \frac{\al}{n-i_0+1}\right)\\
        &\le \Pa\left(p_{(i_0)} \le \frac{\al}{n_0}\right)\\
        &= \Pa\left(p_{i} \le \frac{\al}{n_0}, \text{ for some $i$ such that $H_{0,i}$ is true}\right)\\
        &\le \sum_{i, H_{0,i} \text{ is true}} \Pa\left(p_{i}\le \frac{\al}{n_0}\right)\\
        &=n_0 \cdot \frac{\al}{n_0}\\
        &=\al. \qedhere
    \end{align*}
\end{proof}
\subsection{Hochberg's procedure}
As before order the p-value and null hypotheses so that $p_{(1)}\le p_{(2)}\le \ldots \le p_{(n)}$ and $p_{(i)}$ corresponds to $H_{0,(i)}$. Define
\[j = \max\left\{i: p_{(i)} \le \frac{\al}{n-i+1}, i =1,\ldots, n\right\}, \]
where we define $\max \emptyset=0$. We the reject $H_{0,(1)},\ldots H_{0,(j)}$. If the p-values are independent, then this procedure also has level $\al$ FWER control. This procedure is more powerful that Holm's procedure in the sense that it rejects more often.
\end{document}
\documentclass{article}
\usepackage{ae,aecompl}
\usepackage{todonotes}
\usepackage{chngcntr}
\usepackage{tikz-cd}
\usepackage{graphicx}
\graphicspath{ {./images/}}
\usepackage[all,cmtip]{xy}
\usepackage{amsmath, amscd}
\usepackage{amsthm}
\usepackage{amssymb}
\usepackage{amsfonts}
\usepackage{bm}
\usepackage{qsymbols}
\usepackage{latexsym}
\usepackage{mathrsfs}
\usepackage{mathtools}
\usepackage{cite}
\usepackage{color}
\usepackage{url}
\usepackage{enumerate}
\usepackage{verbatim}
\usepackage[draft=false, colorlinks=true]{hyperref}
\usepackage{pdfpages}
\usepackage[margin=1.2in]{geometry}
\usepackage{IEEEtrantools}

\usepackage{fancyhdr}


\usepackage[nameinlink]{cleveref}


\DeclareMathOperator*{\ac}{accept}
\DeclareMathOperator*{\amax}{argmax}
\DeclareMathOperator*{\amin}{argmin}
\DeclareMathOperator*{\Aut}{Aut}
\newcommand {\al}{{\alpha}}
\newcommand {\abs}[1]{{\left\lvert#1\right\rvert}}
\newcommand {\A}{{\mathcal{A}}}
\newcommand {\AM}{{\mathrm{AM}}}
\newcommand {\AMp}{{\AM_{p}^{X}\!(\Ri_\w)}}
\newcommand {\B}{{\mathcal{B}}}
\DeclareMathOperator*{\Be}{Bern}
\newcommand {\Br}{{\dot{B}}}
\newcommand {\Ba}{{\mathfrak{B}}}
\newcommand {\C}{{\mathbb C}}
\newcommand {\ce}{\mathrm{c}}
\newcommand {\Ce}{\mathrm{C}}
\newcommand {\Cc}{\mathrm{C_{c}}}
\newcommand {\Ccinf}{\mathrm{C_{c}^{\infty}}}
\DeclareMathOperator{\cov}{Cov}
\DeclareMathOperator{\DEV}{DEV}
\newcommand {\Di}{{\mathbb D}}
\newcommand {\dom}{\mathrm{dom}}
\newcommand{\dist}{\stackrel{\mathrm{dist}}{=}}
\newcommand {\ud}{\mathrm{d}}
\newcommand {\ue}{\mathrm{e}}
\newcommand {\eps}{\varepsilon}
\newcommand {\veps}{\varepsilon}
\newcommand {\vrho}{{\varrho}}
\newcommand {\E}{{\mathbb{E}}}
\newcommand {\Ec}{{\mathcal{E}}}
\newcommand {\Ell}{L}
\newcommand {\Ellp}{{L_{p}[0,1]}}
\newcommand {\Ellpprime}{{L_{p'}([0,1])}}
\newcommand {\Ellq}{{L_{q}([0,1])}}
\newcommand {\Ellqprime}{{L_{q'}([0,1])}}
\newcommand {\Ellr}{L^{r}}
\newcommand {\Ellone}{{L_{1}([0,1])}}
\newcommand{\Elltwo}{{L_{2}([0,1])}}
\newcommand{\Ellinfty}{L^{\infty}}
\newcommand{\Ellinftyc}{L_{\mathrm{c}}^{\infty}}
\newcommand{\exb}[1]{\exp\left\{#1\right\}}
\DeclareMathOperator*{\Ext}{Ext}
\newcommand{\F}{{\mathcal{F}}}
\newcommand{\Fe}{{\mathbb{F}}}
\newcommand{\G}{{\mathcal{G}}}
\newcommand{\HF}{\mathcal{H}_{\text{FIO}}^{1}(\Rd)}
\newcommand{\Hr}{H}
\newcommand{\HT}{\mathcal{H}}
\newcommand{\ui}{\mathrm{i}}
\newcommand{\I}{{I}}
\newcommand{\J}{{\mathcal{J}}}
\newcommand{\id}{{\mathrm{id}}}
\newcommand{\iid}{\stackrel{\mathclap{\normalfont\mbox{iid}}}{\sim}}
\newcommand{\im}{{\text{im }}}
\newcommand{\ind}{{\perp\!\!\!\perp}}
\DeclareMathOperator*{\Int}{int}
\newcommand{\intx}{{\overline{\int_{X}}}}
\newcommand{\inte}{{\overline{\int_{\E}}}}
\newcommand{\la}{\lambda}
\newcommand{\rb}{\rangle}
\newcommand{\lb}{{\langle}}
\newcommand{\La}{\Lambda}
\newcommand{\calL}{{\mathcal{L}}}
\newcommand{\lp}{{\mathcal{L}}^{p}}
\newcommand{\lpo}{{\overline{\mathcal{L}}^{p}\!}}
\newcommand{\Lpo}{{\overline{\Ell}^{p}\!}}
\newcommand{\M}{{\mathbf{M}}}
\newcommand{\Ma}{{\mathcal{M}}}
\newcommand{\N}{{{\mathbb N}}}
\newcommand{\Na}{{{\mathcal{N}}}}
\newcommand{\norm}[1]{\left\|#1\right\|}
\newcommand{\normm}[1]{{\left\vert\kern-0.25ex\left\vert\kern-0.25ex\left\vert #1 
    \right\vert\kern-0.25ex\right\vert\kern-0.25ex\right\vert}}
\newcommand{\Om}{{{\Omega}}}
\newcommand{\one}{{{\bf 1}}}
\newcommand{\pic}{\text{Pic }}
\newcommand{\ph}{{\varphi}}
\newcommand{\Pa}{{\mathbb{P}}}
\newcommand{\Po}{{\mathcal{P}}}
\newcommand{\Q}{{\mathbb{Q}}}
\newcommand{\R}{{\mathbb R}}
\newcommand{\Rd}{{\mathbb{R}^{d}}}
\DeclareMathOperator{\rej}{reject }
\newcommand{\Rn}{{\mathbb{R}^{n}}}
\newcommand{\cR}{{\mathcal{R}}}
\newcommand{\Rad}{{\mathrm{Rad}}}
\newcommand{\ran}{{\mathrm{ran}}}
\newcommand{\Ri}{{\mathrm{R}}}
\newcommand{\supp}{{\mathrm{supp}}}
\newcommand{\Se}{\mathrm{S}}
\newcommand{\Sp}{S^{*}(\Rn)}
\newcommand{\St}{{\mathrm{St}}}
\newcommand{\Sw}{\mathcal{S}}
\newcommand{\T}{{\mathcal{T}}}
\newcommand{\ta}{{\theta}}
\newcommand{\Ta}{{\Theta}}
\newcommand{\topp}{\stackrel{p}{\to}}
\newcommand{\todd}{\stackrel{d}{\to}}
\newcommand{\toL}[1]{\stackrel{L^{#1}}{\to}} 
\newcommand{\toas}{\stackrel{a.s.}{\to}}
\DeclareMathOperator{\V}{Var}
\newcommand {\w}{{\omega}}
\newcommand {\W}{{\mathrm{W}}}
\newcommand {\Wnp}{\text{$\mathrm{W}$\textsuperscript{$n,\!p$}}}
\newcommand {\Wnpeq}{\text{$\mathrm{W}$\textsuperscript{$n\!,\!p$}}}
\newcommand {\Wonep}{\text{$\mathrm{W}$\textsuperscript{$1,\!p$}}}
\newcommand {\Wonepeq}{\text{$\mathrm{W}$\textsuperscript{$1\!,\!p$}}}
\newcommand {\X}{{\mathcal{X}}}
\newcommand {\Z}{{{\mathbb Z}}}
\newcommand {\Za}{{\mathcal{Z}}}
\newcommand {\Zd}{{\Z[\sqrt{d}]}}
\newcommand {\vanish}[1]{\relax}

\newcommand {\wh}{\widehat}
\newcommand {\wt}{\widetilde}
\newcommand {\red}{\color{red}}

% Distributions
\newcommand{\normal}{\mathsf{N}}
\newcommand{\poi}{\mathsf{Poisson}}
\newcommand{\bern}{\mathsf{Bernoulli}}
\newcommand{\bin}{\mathsf{Binomal}}
\newcommand{\multi}{\mathsf{Multinomial}}
\newcommand{\Exp}{\mathsf{Exp}}



% put your command and environment definitions here




% some theorem environments
% remove "[theorem]" if you do not want them to use the same number sequence


  \newtheorem{thrm}{Theorem}
  \newtheorem{lemma}{Lemma}
  \newtheorem{prop}{Proposition}
  \newtheorem{cor}{Corollary}

  \newtheorem{conj}{Conjecture}
  \renewcommand{\theconj}{\Alph{conj}}  % numbered A, B, C etc

  \theoremstyle{definition}
  \newtheorem{defn}{Definition}
  \newtheorem{ex}{Example}
  \newtheorem{exs}{Examples}
  \newtheorem{question}{Question}
  \newtheorem{remark}{Remark}
  \newtheorem{notn}{Notation}
  \newtheorem{exer}{Exercise}




\title{STATS 300A - Lecture 3}
\author{Dominik Rothenhaeusler\\
Scribed by Michael Howes}
\date{9/27/2021}

\pagestyle{fancy}
\fancyhf{}
\rhead{STATS 300A - Lecture 3}
\lhead{9/27/2021}

\begin{document}
\maketitle
\section{Announcements}
\begin{itemize}
    \item HW1 due Wednesday.
    \item Sign up on gradescope
    \item Form study groups
\end{itemize}
\section{Recap}
\begin{itemize}
    \item Decision theoretic framework
    \item Sufficiency and NFFC
    \item Exponential families
    \item Minimal sufficiency and optimal data reduction.
\end{itemize}
We will soon see a relationship between optimal data reduction and optimal estimation.

\section{Minimal Sufficiency}
\begin{defn}
    A sufficient statistic $T$ is minimal if for every sufficient statistic $T'$, if $T'(x) = T'(y)$, then $T(x) = T(y)$.
\end{defn}
That is, $T$ is the coarsest sufficient statistic. Last time we stated the following theorem
\begin{thrm}
    Let $p(x;\ta)$ be the density of $X$ (w.r.t. to $\mu$) and let $T$ be a statistic such that for all $x, y \in \X$, 
    \[\text{ there exists } C_{x,y} \text{ such that } p(x;\theta) = C_{x,y}p(y;\theta) \text{ for all }\theta \Longleftrightarrow T'(x) = T'(y). \]
    Then $T$ is a minimal sufficient statistic.
\end{thrm}
In the case when $p(x;\ta) > 0$ for all $x$ and $\ta$, the condition on the left is equivalent to the ratio $\frac{p(x;\ta)}{p(y;\ta)}$ being constant as a function of $\ta$.
\begin{proof}
    \emph{[Discrete case where the densities are strictly positive]} We will first show that $T$ is sufficient. To do this we will use NFFC and show that for some $h$ and $g_\ta$ we have
    \[p(y;\ta) = h(y)g_\ta(T(y)), \]
    for all $y \in \X$. For each $t$ in the range of $T$, define $A_t = \{x \in \X : T(x) = t\}$ and choose $x_t \in A_t$. Then for every $y$ we have $T(y) = T(x_{T(y)})$. Thus there exists $C_{y, x_{T(y)}}$ such that $p(y;\ta) = C_{y, x_{T(y)}}p(x_{T(y)}; \ta)$. Hence we can define $h(y) = C_{y, x_{T(y)}}$ and $g_\ta(t) = p(x_t;\ta)$ giving us our factorisation 
    \[p(y;\ta) = C_{y,x_{T(y)}}p(x_{T(y)};\ta) = h(y)g_\ta(T(y)).\] 
    Thus $T$ is sufficient.

    Now suppose that $T'$ is another sufficient statistic. Then there exists a factorisation $p(x;\ta) = \wt{h}(x)\wt{g}_\ta(T'(x))$. Now suppose that $T'(x) = T'(y)$ for some $x,y \in \X$. Then 
    \[\frac{p(x;\ta)}{p(y;\ta)} = \frac{\wt{h}(x)\wt{g}_\ta(T'(x))}{\wt{h}(y)\wt{g}_\ta(T'(y))} =\frac{\wt{h}(x)}{\wt{h}(y)}, \]
    and thus $T(x) = T(y)$ since the above ratio does not depend on $\ta$.
\end{proof}
\begin{ex}
    A natural question is: what are the minimal sufficient statistics for exponential families? The answer (Keener 3.17) is that for $n$ iid samples from a minimal $s$-dimensional exponential family the statistics
    \[\left(\sum_{i=1}^n T_1(X_i), \sum_{i=1}^n T_2(X_i),\ldots, \sum_{i=1}^n T_s(X_i)\right), \]
    are minimal sufficient for $(X_i)_{i=1}^n$.
\end{ex}   
\begin{ex}
        Consider now the case $X_1,\ldots, X_n \iid N(\sigma, \sigma^2)$ where $\ta = \sigma > 0$. This is a curved exponential family and so the above result does not apply. We will show that $T(x) = \left(\sum_{i=1}^n x_i, \sum_{i=1}^n x_i^2\right)$ is minimal sufficient. Note that 
        \[\frac{p(x;\ta)}{p(x;\ta)} = \exb{\frac{1}{2\sigma^2}\left(\sum_{i=1}^n y_i^2 - \sum_{i=1}^n x_i^2\right)+\frac{1}{\sigma^2}\left(\sum_{i=1}^n x_i - \sum_{i=1}^n y_i\right)}. \]
        Thus if $T(x) = T(y)$, then the above does not depend on $\ta$. Suppose now that the above ratio is constant. Then 
        \[ 0=\lim_{\sigma \to 0}\sigma^2 \log\frac{p(x;\ta)}{p(y;\ta)} = \frac{1}{2}\left( \sum_{i=1}^n y_i^2-\sum_{i=1}^n x_i^2\right)+\lim_{\sigma \to 0} \sigma\left(\sum_{i=1}^n x_i -\sum_{i=1}^n y_i\right) = \frac{1}{2}\left( \sum_{i=1}^n y_i^2-\sum_{i=1}^n x_i^2\right). \]
        Thus $\sum_{i=1}^n x_i^2 = \sum_{i=1}^n y_I^2$. Using this we can see that 
        \[0 =\lim_{\sigma \to 0} \sigma \log{\frac{p(x;\ta)}p(y;\ta)} =  \sum_{i=1}^n y_i^2-\sum_{i=1}^n x_i^2,\]
        and thus $T(x) = T(y)$. 
\end{ex}
\begin{ex}
    Suppose $(X_i)_{i=1}^n \iid \text{Unif}([0,\ta])$ this is not an exponential family and the support of $p(x;\ta)$ depend on $\ta$. In tis case $T(x) = \max(X_1,\ldots, X_n)$ is minimal sufficient.
\end{ex}
\begin{ex}
    Sometimes data reduction is not possible. If $(X_i)_{i=1}^n \iid \text{Cauchy}$ with mean $\ta$, then the order statistics $X_{(1)} \le X_{(2)} \le \ldots \le X_{(n)}$ are minimal sufficient.
\end{ex}

\section{``Useless'' data}
The next definition captures the idea of useless information

\begin{defn}
    A statistic $A$ is \emph{ancillary} for $X \sim P_\ta$ if the distribution of $A(X)$ does not depend on $\ta$.
\end{defn}
\begin{ex}
    Continuing the Cauchy example. The order statistics are minimal sufficient but the differences $X_{(i)} - X_{(j)}$ are ancillary for $\ta$. This is because we can write $X_i = Z_i +\ta$ where $Z_i$ is Cauchy with mean 0. This implies that $X_{(i)} = Z_{(i)}+\ta$ and thus $X_{(i)}-X_{(j)} = Z_{(i)}-Z_{(j)}$ which has no $\ta$ dependence. Thus we can see that sometimes minimal sufficient statistics contain ``useless'' information.
\end{ex}

Later we will see the following general stratergy for finding an optimal estimator and proving optimality. We want to use the following two part stratergy
\begin{enumerate}
    \item Show that the optimal unbiased estimator is a function of a minimal statistic.
    \item Show there exists only one function of a minimal statistic that is unbiased.
\end{enumerate}

We won't be able to show (b) if our minimal statistics contain an ancillary part and thus the next definitions will give us a new notion of minimal sufficiency. We first need a weaker notion of ancillary.

\begin{defn}
    A statistic $A$ is \emph{first order ancillary} for $X \sim P_\ta$ if $\E_\ta[A(X)]$ does not depend on $\ta$.
\end{defn}

\begin{defn}
    A statistic $T$ is complete if any non-constant function of $T$ is \emph{not} first order ancillary.
\end{defn}

Equivalenty a statistic $T$ is complete if for all functions $f$ if $\E_\ta[f(T(X))] = c$ for all $\ta$, then $f \equiv c$. Note that since $\E_\ta[c] = c$ and $\E_\ta$ is linear this is equivalent to stating that for all functions $f$, if $\E_\ta[f(T(X))]=0$ for all $\ta$, then $f \equiv 0$. Complete statistics contain no useless info.

Bahadur's theorem states that every complete sufficient statistic is minimal sufficient.

\begin{ex}
    Suppose $X_1,\ldots, X_n \iid \text{Bern}(\ta)$. We have seen that
    \[T(X) = \sum_{i=1}^n X_i\]
    is sufficient for $(X_i)_{i=1}^n$. We know that $T \sim \text{Binom}(n,\ta)$. Suppose $f$ is a function of $T$ satisfying $\E_\ta[f(T)] = 0$. Then, since $T$ is discrete
    \begin{IEEEeqnarray*}{rCl}
        0&=&\E_\ta[f(T)]\\
        &=&\sum_{k=0}^n \binom{n}{k}\ta^k(1-\ta)^{n-k} f(k)\\
        &=&(1-\ta)^n \sum_{k=0}^n \binom{n}{k} \left(\frac{\ta}{1-\ta}\right)^k f(k)\\
        \therefore 0 &=& \sum_{k=0}^n \binom{n}{k}f(k)\beta^k.
    \end{IEEEeqnarray*}
    Where $\beta = \frac{\ta}{1-\ta} \in [0,\infty)$. Thus the above polynoimal is $\beta$ is zero on an uncountable set and hence $\binom{n}{k} f(k) = 0$ for all $k$ and so $f \equiv 0$. Thus $T$ is complete.
\end{ex}
Another example when $X_1,X_2,\ldots, X_n \iid N(\ta,\sigma^2)$ with $\sigma^2$ known. Then $\bar{x} = \frac{1}{n}\sum_{i=1}^nx_i$ is a complete statistic (proof in scribed notes).

\end{document}
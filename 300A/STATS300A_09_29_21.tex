\documentclass{article}
\usepackage{ae,aecompl}
\usepackage{todonotes}
\usepackage{chngcntr}
\usepackage{tikz-cd}
\usepackage{graphicx}
\graphicspath{ {./images/}}
\usepackage[all,cmtip]{xy}
\usepackage{amsmath, amscd}
\usepackage{amsthm}
\usepackage{amssymb}
\usepackage{amsfonts}
\usepackage{bm}
\usepackage{qsymbols}
\usepackage{latexsym}
\usepackage{mathrsfs}
\usepackage{mathtools}
\usepackage{cite}
\usepackage{color}
\usepackage{url}
\usepackage{enumerate}
\usepackage{verbatim}
\usepackage[draft=false, colorlinks=true]{hyperref}
\usepackage{pdfpages}
\usepackage[margin=1.2in]{geometry}
\usepackage{IEEEtrantools}

\usepackage{fancyhdr}


\usepackage[nameinlink]{cleveref}


\DeclareMathOperator*{\ac}{accept}
\DeclareMathOperator*{\amax}{argmax}
\DeclareMathOperator*{\amin}{argmin}
\DeclareMathOperator*{\Aut}{Aut}
\newcommand {\al}{{\alpha}}
\newcommand {\abs}[1]{{\left\lvert#1\right\rvert}}
\newcommand {\A}{{\mathcal{A}}}
\newcommand {\AM}{{\mathrm{AM}}}
\newcommand {\AMp}{{\AM_{p}^{X}\!(\Ri_\w)}}
\newcommand {\B}{{\mathcal{B}}}
\DeclareMathOperator*{\Be}{Bern}
\newcommand {\Br}{{\dot{B}}}
\newcommand {\Ba}{{\mathfrak{B}}}
\newcommand {\C}{{\mathbb C}}
\newcommand {\ce}{\mathrm{c}}
\newcommand {\Ce}{\mathrm{C}}
\newcommand {\Cc}{\mathrm{C_{c}}}
\newcommand {\Ccinf}{\mathrm{C_{c}^{\infty}}}
\DeclareMathOperator{\cov}{Cov}
\DeclareMathOperator{\DEV}{DEV}
\newcommand {\Di}{{\mathbb D}}
\newcommand {\dom}{\mathrm{dom}}
\newcommand{\dist}{\stackrel{\mathrm{dist}}{=}}
\newcommand {\ud}{\mathrm{d}}
\newcommand {\ue}{\mathrm{e}}
\newcommand {\eps}{\varepsilon}
\newcommand {\veps}{\varepsilon}
\newcommand {\vrho}{{\varrho}}
\newcommand {\E}{{\mathbb{E}}}
\newcommand {\Ec}{{\mathcal{E}}}
\newcommand {\Ell}{L}
\newcommand {\Ellp}{{L_{p}[0,1]}}
\newcommand {\Ellpprime}{{L_{p'}([0,1])}}
\newcommand {\Ellq}{{L_{q}([0,1])}}
\newcommand {\Ellqprime}{{L_{q'}([0,1])}}
\newcommand {\Ellr}{L^{r}}
\newcommand {\Ellone}{{L_{1}([0,1])}}
\newcommand{\Elltwo}{{L_{2}([0,1])}}
\newcommand{\Ellinfty}{L^{\infty}}
\newcommand{\Ellinftyc}{L_{\mathrm{c}}^{\infty}}
\newcommand{\exb}[1]{\exp\left\{#1\right\}}
\DeclareMathOperator*{\Ext}{Ext}
\newcommand{\F}{{\mathcal{F}}}
\newcommand{\Fe}{{\mathbb{F}}}
\newcommand{\G}{{\mathcal{G}}}
\newcommand{\HF}{\mathcal{H}_{\text{FIO}}^{1}(\Rd)}
\newcommand{\Hr}{H}
\newcommand{\HT}{\mathcal{H}}
\newcommand{\ui}{\mathrm{i}}
\newcommand{\I}{{I}}
\newcommand{\J}{{\mathcal{J}}}
\newcommand{\id}{{\mathrm{id}}}
\newcommand{\iid}{\stackrel{\mathclap{\normalfont\mbox{iid}}}{\sim}}
\newcommand{\im}{{\text{im }}}
\newcommand{\ind}{{\perp\!\!\!\perp}}
\DeclareMathOperator*{\Int}{int}
\newcommand{\intx}{{\overline{\int_{X}}}}
\newcommand{\inte}{{\overline{\int_{\E}}}}
\newcommand{\la}{\lambda}
\newcommand{\rb}{\rangle}
\newcommand{\lb}{{\langle}}
\newcommand{\La}{\Lambda}
\newcommand{\calL}{{\mathcal{L}}}
\newcommand{\lp}{{\mathcal{L}}^{p}}
\newcommand{\lpo}{{\overline{\mathcal{L}}^{p}\!}}
\newcommand{\Lpo}{{\overline{\Ell}^{p}\!}}
\newcommand{\M}{{\mathbf{M}}}
\newcommand{\Ma}{{\mathcal{M}}}
\newcommand{\N}{{{\mathbb N}}}
\newcommand{\Na}{{{\mathcal{N}}}}
\newcommand{\norm}[1]{\left\|#1\right\|}
\newcommand{\normm}[1]{{\left\vert\kern-0.25ex\left\vert\kern-0.25ex\left\vert #1 
    \right\vert\kern-0.25ex\right\vert\kern-0.25ex\right\vert}}
\newcommand{\Om}{{{\Omega}}}
\newcommand{\one}{{{\bf 1}}}
\newcommand{\pic}{\text{Pic }}
\newcommand{\ph}{{\varphi}}
\newcommand{\Pa}{{\mathbb{P}}}
\newcommand{\Po}{{\mathcal{P}}}
\newcommand{\Q}{{\mathbb{Q}}}
\newcommand{\R}{{\mathbb R}}
\newcommand{\Rd}{{\mathbb{R}^{d}}}
\DeclareMathOperator{\rej}{reject }
\newcommand{\Rn}{{\mathbb{R}^{n}}}
\newcommand{\cR}{{\mathcal{R}}}
\newcommand{\Rad}{{\mathrm{Rad}}}
\newcommand{\ran}{{\mathrm{ran}}}
\newcommand{\Ri}{{\mathrm{R}}}
\newcommand{\supp}{{\mathrm{supp}}}
\newcommand{\Se}{\mathrm{S}}
\newcommand{\Sp}{S^{*}(\Rn)}
\newcommand{\St}{{\mathrm{St}}}
\newcommand{\Sw}{\mathcal{S}}
\newcommand{\T}{{\mathcal{T}}}
\newcommand{\ta}{{\theta}}
\newcommand{\Ta}{{\Theta}}
\newcommand{\topp}{\stackrel{p}{\to}}
\newcommand{\todd}{\stackrel{d}{\to}}
\newcommand{\toL}[1]{\stackrel{L^{#1}}{\to}} 
\newcommand{\toas}{\stackrel{a.s.}{\to}}
\DeclareMathOperator{\V}{Var}
\newcommand {\w}{{\omega}}
\newcommand {\W}{{\mathrm{W}}}
\newcommand {\Wnp}{\text{$\mathrm{W}$\textsuperscript{$n,\!p$}}}
\newcommand {\Wnpeq}{\text{$\mathrm{W}$\textsuperscript{$n\!,\!p$}}}
\newcommand {\Wonep}{\text{$\mathrm{W}$\textsuperscript{$1,\!p$}}}
\newcommand {\Wonepeq}{\text{$\mathrm{W}$\textsuperscript{$1\!,\!p$}}}
\newcommand {\X}{{\mathcal{X}}}
\newcommand {\Z}{{{\mathbb Z}}}
\newcommand {\Za}{{\mathcal{Z}}}
\newcommand {\Zd}{{\Z[\sqrt{d}]}}
\newcommand {\vanish}[1]{\relax}

\newcommand {\wh}{\widehat}
\newcommand {\wt}{\widetilde}
\newcommand {\red}{\color{red}}

% Distributions
\newcommand{\normal}{\mathsf{N}}
\newcommand{\poi}{\mathsf{Poisson}}
\newcommand{\bern}{\mathsf{Bernoulli}}
\newcommand{\bin}{\mathsf{Binomal}}
\newcommand{\multi}{\mathsf{Multinomial}}
\newcommand{\Exp}{\mathsf{Exp}}



% put your command and environment definitions here




% some theorem environments
% remove "[theorem]" if you do not want them to use the same number sequence


  \newtheorem{thrm}{Theorem}
  \newtheorem{lemma}{Lemma}
  \newtheorem{prop}{Proposition}
  \newtheorem{cor}{Corollary}

  \newtheorem{conj}{Conjecture}
  \renewcommand{\theconj}{\Alph{conj}}  % numbered A, B, C etc

  \theoremstyle{definition}
  \newtheorem{defn}{Definition}
  \newtheorem{ex}{Example}
  \newtheorem{exs}{Examples}
  \newtheorem{question}{Question}
  \newtheorem{remark}{Remark}
  \newtheorem{notn}{Notation}
  \newtheorem{exer}{Exercise}




\title{STATS300A - Lecture 4}
\author{Dominik Rothenhaeusler\\ Scribed by Michael Howes}
\date{09/29/21}

\pagestyle{fancy}
\fancyhf{}
\rhead{STATS300A - Lecture 4}
\lhead{09/29/21}

\begin{document}
\maketitle
\tableofcontents
\section{Announcements}
\begin{itemize}
    \item HW1 due today.
    \item HW2 will be posted today.
\end{itemize}
\section{Recap}
We have seen
\begin{itemize}
    \item Exponential families.
    \item Data reduction:
    \begin{itemize}
        \item Sufficiency (all necessary information),
        \item Minimal sufficiency (coarest sufficient statistic),
        \item Ancillary (useless data),
        \item Completeness (contains no useless data).
    \end{itemize}
\end{itemize}
Today we will look at optimal unbiased estimation but first a refresher on conditional expectations.
\section{Conditional expectation}
(See Keener Chp 1.1) Let $X$ and $Z$ be random variables with density $p(x,z) = p(z|x)p(x)$. Let $h$ be a function with finite expectation, that is
\[\int\int \abs{h(x,z)}p(z|x)p(x)dzdx < \infty. \]
Then we define the conditional expectation of $h$ given $Z$ as
\[\E[h(X,Z)|X=x] = \int h(x,z)p(z|x)dx. \]
Note that $\E[h(X,Z)|X]$ is a function of $x$. Some properties of conditional expectation are 
\begin{itemize}
    \item (Pull out property) $\E[h_1(X)h_2(X,Y)|X=x] = h_1(x)\E[h_2(X,Y)|X=x]$.
    \item (Tower property) $\E[\E[h(X,Z)|X]] = \E[h(X,Z)]$.
    \item (Independence) If $X$ and $Z$ are independent ($p(z|x) = p(z)$ for all $x,z$), then \[\E[h(Z)|X=x] = \E[h(Z)].\] 
\end{itemize}
We will use these ideas when studying estimation.
\section{Estimation}
\subsection{Complete statistics}
See TSH Theorem 4.3.1. In a full rank exponential family, the statistic $(T_1,\ldots, T_s)$ is complete. 
\begin{thrm}
    \emph{[Basu's Theorem]} If $T$ is complete and sufficient for $\Po = \{P_\ta : \ta \in \Om \}$ and $A$ is ancillary for $\Po$, then $T(X)$ is independent of $A(X)$ which we write as $A(X) \ind T(X)$.
\end{thrm}
By independent we mean for all events $C$, $P_\ta(A(X) \in C|T(X)=t) = P_\ta(A(X) \in C)$. We will not prove this here but it can be done using the tower property. Here is an application of this theorem.
\begin{ex}
    Let $X_1,\ldots,X_n \iid N(\mu,\sigma^2)$ where both $\mu$ and $\sigma^2$ are unknown. We wish to show that $\bar{X} = \frac{1}{n}\sum_{i=1}^n X_i$ and $S^2 = \frac{1}{n-1}\sum_{i=1}^n (X_i-\bar{X})^2$ are independent.
    \begin{proof}
        Fix $\sigma^2$ and consider the model where only $\mu$ is unknown. This is an exponential family and thus the statistic $\bar{X}$ is complete and sufficient. Also since we are working with a location model the statistic 
        \[ S^2 = \frac{1}{n-1}\sum_{i=1}^n (X_i-\bar{X}),\]
        has the same distribution as 
        \[ \frac{1}{n-1}\sum_{i=1}^n (Z_i-\bar{Z}),\]
        where $Z_i \iid N(0,\sigma^2)$. Thus, by Basu's theorem $S^2 \ind \bar{X}$. Although we fixed $\sigma^2$ in the submodel, $\sigma^2$ was arbitrary and thus $S^2 \ind \bar{X}$ regardless of $\mu$ and $\sigma^2$.
    \end{proof}
\end{ex}
\subsection{Risk reduction via conditioning}
\begin{defn}
    Let $C$ be a convex subspace of a vector space. A function $f : C \to \R$ is \emph{convex} if 
    \[f(\gamma x + (1-\gamma)y) \le \gamma f(x) + (1-\gamma)f(y), \]
    for all $x,y \in C$ and all $\gamma \in [0,1]$. If we have strict inequality for all $x,y \in C$ and $\gamma \in (0,1)$, then we say that $f$ is \emph{strictly convex}.
\end{defn}
\begin{thrm}
    \emph{[Jensen's Inequality]} Let $f :C \to \R$ be convex on an open subset $C$ with $P(X \in C) = 1$. If $\E[X]$ exists, then \[f(\E[X])\le \E[f(X)].\] 
    If $f$ is strictly convex then 
    \[f(\E[X]) < \E[f(X)], \]
    unless $P(X = \E[X]) = 1$.
\end{thrm}
Recall that $L(\ta,d)$ is the penalty when $\ta$ is the true parameter and decision $d$ is made. Also recall that $R(\ta, \delta) = \E_\ta[L(\ta,\delta(X))]$ is the risk of the decision procedure $\delta$.
\begin{thrm}
    \emph{[Rao-Blackwell]} Suppose that $T$ is sufficient for $\{P_\ta : \ta \in \Om\}$ and that $\delta(X)$ is an estimator with $\E_\ta|\delta(X)| < \infty$ and $R(\theta,\delta)< \infty$. Let $\eta(T) = \E[\delta(X)|T]$ (which is well-defined because $T$ is sufficient). Then
    \begin{enumerate}
        \item If $L(\ta,\cdot)$ is convex for a fixed $\ta$, then 
        \[R(\ta, \eta) \le R(\ta, \delta).\]
        \item If $L(\ta, \cdot)$ is strictly convex for a fixed $\ta$, then 
        \[R(\ta,\eta) < R(\ta,\delta), \]
        unless $\eta(T(X))=\delta(X)$ with probability 1.
    \end{enumerate}
\end{thrm}
\begin{proof}
    For (a),
    \begin{IEEEeqnarray*}{rCl}
        R(\ta,\eta)&=&\E_\ta[L(\ta,\eta(X))]\\
        &=& \E_\ta[L(\ta, \E[\delta(X)|T])]\\
        &\le & \E_\ta[\E[L(\ta,\delta(X))|T]]\quad \text{(Jensen's inequality)}\\
        &=&E_\ta[L(\ta,\delta(X))]\quad \text{(Tower property)}\\
        &=&R(\ta,\delta).
    \end{IEEEeqnarray*}
    For (b) we note that this inequality is strict if $P(\eta(T)\neq \delta(X))>0$.
\end{proof}
The take away 
\begin{itemize}
    \item Under convex loss, a deterministic estimator based on sufficient statistics is as good or better than any other estimator.
    \item It our loss is strictly convex, then additional randomness deteriorates performance.
\end{itemize}
\begin{ex}
    $X_1,\ldots, X_n \iid \text{Bern}(\ta)$, $L(\ta,d) = (\ta -d)^2$ and $\delta(X) = X_1$. We know that $T(X) = \sum_{i=1}^n X_i$ is sufficient for this model. Thus Rao-Blackwell states
    \[\eta(T) = \E[X_1|T(X)], \]
    is at least as good as $\delta$. Note that $\eta(T) = \E[X_i|T]$ by the iid assumption. Thus 
    \begin{IEEEeqnarray*}{rCl}
        \eta(T) &=& \frac{1}{n}(n\eta(T))\\
        &=& \frac{1}{n}\left(\sum_{i=1}^n \E[X_i|T]\right)\\
        &=&\frac{1}{n}\E[T|T]\\
        &=& \frac{1}{n}T\\
        &=&\frac{1}{n}\sum_{i=1}^n X_i\\
        &=& \bar{X}.
    \end{IEEEeqnarray*}
    Thus Rao-Blackwell recovers the sample average. The risk of $\delta$ was $\ta(1-\ta)$ and the risk of $\eta$ is $\frac{\ta(1-\ta)}{n}$.

    Consider also the example $\delta_{goofy}(X) = 0.5$. Then $\eta(T) = \E[0.5|T] = 0.5 = \delta_{goofy}(X)$ so Rao-Blackwell does not improve our estimator (it also doesn't make it worse). Conditioning reduces variance but it does not reduce bias.
\end{ex}
\subsection{Optimal unbiased estimators}
\underline{Goal}: Find the uniformly minimum risk unbiased estimator (UMRUE). That is we want for a fixed function $g$
\begin{enumerate}
    \item $\E_\ta \delta(X) = g(\ta)$, for all $\ta \in \Om$.
    \item $R(\ta, \delta) \le R(\ta, \delta')$ for all $\ta \in \Om$ and all decision procedures $\delta$ satisfying (a).
\end{enumerate}
A special case is when $L(\ta,d) = (g(\ta)-d)^2$, then the UMRUE is also called a UMVUE where the V stands for variance. This is because in this case $R(\ta,\delta) = \text{bias}^2+\text{variance}=\text{variance}$.

\begin{thrm}
    \emph{[Lehmann-Scheffe]} If $T$ is complete and sufficient and $\E_\ta h(T) = g(\ta)$ for all $\ta$, then 
    \begin{enumerate}
        \item $h(T)$ is the only function of $T$ that is unbiased for $g(\ta)$.
        \item $h(T)$ is the UMRUE if $L(\ta,\cdot)$ is convex for all $\ta\in \Om$.
        \item $h(T)$ is the unique UMRUE if $L(\ta,\cdot)$ is convex for all $\ta\in \Om$ and $L(\ta_0,\cdot)$ is strictly convex for some $\ta_0 \in \Om$. 
        \item $h(T)$ is the unique UMVUE.
    \end{enumerate}
\end{thrm}
\begin{proof}
    Suppose $\E_\ta \wt{h}(T) = g(\ta)$ for all $\ta \in \Om$, then $E_\ta[(h-\wt{h})(T)] = 0$ for all $\ta \in \Om$. Thus $h - \wt{h}$ is first order ancillary for $T$ and since $T$ is complete, this implies $h-\wt{h}=0$. Thus $\wt{h}=h$ and we have proved part (a).

    Suppose $\delta$ is an unbiased estimator for $g(\ta)$. Define $\eta(T) = \E[\delta(X)|T]$, then $\eta(T)$ is unbiased by the tower property and well defined by sufficiency. Since $\eta(T)$ is a function of $T$ we have $\eta(T) = h(T)$ by part (a). Finally by Rao-Blackwell we have for all $\ta \in \Om$.
    \begin{equation}\label{RB}
        R(\ta, h(t)) = R(\ta, \eta(T)) \le R(\ta, \delta(X)), 
    \end{equation}
    thus $h$ is UMRUE.

    Suppose $L(\ta_0,\cdot)$ is strictly convex, then the inequality \eqref{RB} is strictly convex unless $h(T) = \delta(X)$. This shows that $h(T)$ is the unique UMRUE.

    Finally the case of mean square error is a special case of a strictly convex loss. Thus part (c) implies part (d).
\end{proof}
\subsection{Consequences}
As a consequence of this theorem we can do optimal estimation for full rank exponential families as we know their complete sufficient statistics. This theorem also gives us strategies for finding UMRUEs.
\begin{enumerate}
    \item 
    \begin{enumerate}
        \item Find a complete sufficient statistic $T$.
        \item Find an unbiased estimator $\delta$.
        \item Compute $\E[\delta(X)|T]$.
    \end{enumerate}
    Step (iii) may be hard.
    \item \begin{enumerate}
        \item Find a complete sufficient statistic $T$.
        \item Recall that there is at most one unbiased estimator that is a function of $T$ and that this function is the UMRUE.
        \item Solve for $\delta$ in the equation $\E_\ta[\delta(T)] = g(\ta)$
    \end{enumerate}
    \item Guess the UMRUE.
\end{enumerate}
\end{document}

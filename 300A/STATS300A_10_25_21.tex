\documentclass{article}
\usepackage{ae,aecompl}
\usepackage{todonotes}
\usepackage{chngcntr}
\usepackage{tikz-cd}
\usepackage{graphicx}
\graphicspath{ {./images/}}
\usepackage[all,cmtip]{xy}
\usepackage{amsmath, amscd}
\usepackage{amsthm}
\usepackage{amssymb}
\usepackage{amsfonts}
\usepackage{bm}
\usepackage{qsymbols}
\usepackage{latexsym}
\usepackage{mathrsfs}
\usepackage{mathtools}
\usepackage{cite}
\usepackage{color}
\usepackage{url}
\usepackage{enumerate}
\usepackage{verbatim}
\usepackage[draft=false, colorlinks=true]{hyperref}
\usepackage{pdfpages}
\usepackage[margin=1.2in]{geometry}
\usepackage{IEEEtrantools}

\usepackage{fancyhdr}


\usepackage[nameinlink]{cleveref}


\DeclareMathOperator*{\ac}{accept}
\DeclareMathOperator*{\amax}{argmax}
\DeclareMathOperator*{\amin}{argmin}
\DeclareMathOperator*{\Aut}{Aut}
\newcommand {\al}{{\alpha}}
\newcommand {\abs}[1]{{\left\lvert#1\right\rvert}}
\newcommand {\A}{{\mathcal{A}}}
\newcommand {\AM}{{\mathrm{AM}}}
\newcommand {\AMp}{{\AM_{p}^{X}\!(\Ri_\w)}}
\newcommand {\B}{{\mathcal{B}}}
\DeclareMathOperator*{\Be}{Bern}
\newcommand {\Br}{{\dot{B}}}
\newcommand {\Ba}{{\mathfrak{B}}}
\newcommand {\C}{{\mathbb C}}
\newcommand {\ce}{\mathrm{c}}
\newcommand {\Ce}{\mathrm{C}}
\newcommand {\Cc}{\mathrm{C_{c}}}
\newcommand {\Ccinf}{\mathrm{C_{c}^{\infty}}}
\DeclareMathOperator{\cov}{Cov}
\DeclareMathOperator{\DEV}{DEV}
\newcommand {\Di}{{\mathbb D}}
\newcommand {\dom}{\mathrm{dom}}
\newcommand{\dist}{\stackrel{\mathrm{dist}}{=}}
\newcommand {\ud}{\mathrm{d}}
\newcommand {\ue}{\mathrm{e}}
\newcommand {\eps}{\varepsilon}
\newcommand {\veps}{\varepsilon}
\newcommand {\vrho}{{\varrho}}
\newcommand {\E}{{\mathbb{E}}}
\newcommand {\Ec}{{\mathcal{E}}}
\newcommand {\Ell}{L}
\newcommand {\Ellp}{{L_{p}[0,1]}}
\newcommand {\Ellpprime}{{L_{p'}([0,1])}}
\newcommand {\Ellq}{{L_{q}([0,1])}}
\newcommand {\Ellqprime}{{L_{q'}([0,1])}}
\newcommand {\Ellr}{L^{r}}
\newcommand {\Ellone}{{L_{1}([0,1])}}
\newcommand{\Elltwo}{{L_{2}([0,1])}}
\newcommand{\Ellinfty}{L^{\infty}}
\newcommand{\Ellinftyc}{L_{\mathrm{c}}^{\infty}}
\newcommand{\exb}[1]{\exp\left\{#1\right\}}
\DeclareMathOperator*{\Ext}{Ext}
\newcommand{\F}{{\mathcal{F}}}
\newcommand{\Fe}{{\mathbb{F}}}
\newcommand{\G}{{\mathcal{G}}}
\newcommand{\HF}{\mathcal{H}_{\text{FIO}}^{1}(\Rd)}
\newcommand{\Hr}{H}
\newcommand{\HT}{\mathcal{H}}
\newcommand{\ui}{\mathrm{i}}
\newcommand{\I}{{I}}
\newcommand{\J}{{\mathcal{J}}}
\newcommand{\id}{{\mathrm{id}}}
\newcommand{\iid}{\stackrel{\mathclap{\normalfont\mbox{iid}}}{\sim}}
\newcommand{\im}{{\text{im }}}
\newcommand{\ind}{{\perp\!\!\!\perp}}
\DeclareMathOperator*{\Int}{int}
\newcommand{\intx}{{\overline{\int_{X}}}}
\newcommand{\inte}{{\overline{\int_{\E}}}}
\newcommand{\la}{\lambda}
\newcommand{\rb}{\rangle}
\newcommand{\lb}{{\langle}}
\newcommand{\La}{\Lambda}
\newcommand{\calL}{{\mathcal{L}}}
\newcommand{\lp}{{\mathcal{L}}^{p}}
\newcommand{\lpo}{{\overline{\mathcal{L}}^{p}\!}}
\newcommand{\Lpo}{{\overline{\Ell}^{p}\!}}
\newcommand{\M}{{\mathbf{M}}}
\newcommand{\Ma}{{\mathcal{M}}}
\newcommand{\N}{{{\mathbb N}}}
\newcommand{\Na}{{{\mathcal{N}}}}
\newcommand{\norm}[1]{\left\|#1\right\|}
\newcommand{\normm}[1]{{\left\vert\kern-0.25ex\left\vert\kern-0.25ex\left\vert #1 
    \right\vert\kern-0.25ex\right\vert\kern-0.25ex\right\vert}}
\newcommand{\Om}{{{\Omega}}}
\newcommand{\one}{{{\bf 1}}}
\newcommand{\pic}{\text{Pic }}
\newcommand{\ph}{{\varphi}}
\newcommand{\Pa}{{\mathbb{P}}}
\newcommand{\Po}{{\mathcal{P}}}
\newcommand{\Q}{{\mathbb{Q}}}
\newcommand{\R}{{\mathbb R}}
\newcommand{\Rd}{{\mathbb{R}^{d}}}
\DeclareMathOperator{\rej}{reject }
\newcommand{\Rn}{{\mathbb{R}^{n}}}
\newcommand{\cR}{{\mathcal{R}}}
\newcommand{\Rad}{{\mathrm{Rad}}}
\newcommand{\ran}{{\mathrm{ran}}}
\newcommand{\Ri}{{\mathrm{R}}}
\newcommand{\supp}{{\mathrm{supp}}}
\newcommand{\Se}{\mathrm{S}}
\newcommand{\Sp}{S^{*}(\Rn)}
\newcommand{\St}{{\mathrm{St}}}
\newcommand{\Sw}{\mathcal{S}}
\newcommand{\T}{{\mathcal{T}}}
\newcommand{\ta}{{\theta}}
\newcommand{\Ta}{{\Theta}}
\newcommand{\topp}{\stackrel{p}{\to}}
\newcommand{\todd}{\stackrel{d}{\to}}
\newcommand{\toL}[1]{\stackrel{L^{#1}}{\to}} 
\newcommand{\toas}{\stackrel{a.s.}{\to}}
\DeclareMathOperator{\V}{Var}
\newcommand {\w}{{\omega}}
\newcommand {\W}{{\mathrm{W}}}
\newcommand {\Wnp}{\text{$\mathrm{W}$\textsuperscript{$n,\!p$}}}
\newcommand {\Wnpeq}{\text{$\mathrm{W}$\textsuperscript{$n\!,\!p$}}}
\newcommand {\Wonep}{\text{$\mathrm{W}$\textsuperscript{$1,\!p$}}}
\newcommand {\Wonepeq}{\text{$\mathrm{W}$\textsuperscript{$1\!,\!p$}}}
\newcommand {\X}{{\mathcal{X}}}
\newcommand {\Z}{{{\mathbb Z}}}
\newcommand {\Za}{{\mathcal{Z}}}
\newcommand {\Zd}{{\Z[\sqrt{d}]}}
\newcommand {\vanish}[1]{\relax}

\newcommand {\wh}{\widehat}
\newcommand {\wt}{\widetilde}
\newcommand {\red}{\color{red}}

% Distributions
\newcommand{\normal}{\mathsf{N}}
\newcommand{\poi}{\mathsf{Poisson}}
\newcommand{\bern}{\mathsf{Bernoulli}}
\newcommand{\bin}{\mathsf{Binomal}}
\newcommand{\multi}{\mathsf{Multinomial}}
\newcommand{\Exp}{\mathsf{Exp}}



% put your command and environment definitions here




% some theorem environments
% remove "[theorem]" if you do not want them to use the same number sequence


  \newtheorem{thrm}{Theorem}
  \newtheorem{lemma}{Lemma}
  \newtheorem{prop}{Proposition}
  \newtheorem{cor}{Corollary}

  \newtheorem{conj}{Conjecture}
  \renewcommand{\theconj}{\Alph{conj}}  % numbered A, B, C etc

  \theoremstyle{definition}
  \newtheorem{defn}{Definition}
  \newtheorem{ex}{Example}
  \newtheorem{exs}{Examples}
  \newtheorem{question}{Question}
  \newtheorem{remark}{Remark}
  \newtheorem{notn}{Notation}
  \newtheorem{exer}{Exercise}




\title{STATS300A - Lecture 11}
\author{Dominik Rothenhaeusler\\ Scribed by Michael Howes}
\date{10/25/21}

\pagestyle{fancy}
\fancyhf{}
\rhead{STATS300A - Lecture 11}
\lhead{10/25/21}
\rfoot{Page \thepage}

\begin{document}
\maketitle
\tableofcontents

\section{Announcement}
Please arrive at least 5 minutes early to the exam on Wednesday.

\section{Admissibility $(p=1)$}
We have seen that unique Bayes estimators are admissible. We wish to boost this result to $\bar{X}_n$ which a limit of Bayes estimators.

\begin{ex}
    Suppose $X_1,\ldots, X_n \iid \Na(\mu,\sigma^2)$ where $\sigma^2$ is unknown and we are using squared error loss. To show that $\bar{X}_n$ is admissible we will use a limiting Bayes argument. Suppose without loss of generality that $\sigma^2=1$ and that $\bar{X}_n$ is inadmissible. Note that the risk of $\bar{X}_n$ is $\frac{1}{n}$ constantly. Thus if $\bar{X}_n$ is inadmissible, then there exists an estimator $\delta$ such that $R(\ta,\delta) < 1/n$ for some $\ta$ and $R(\ta,\delta) \le 1/n$ for all $\ta$. 
    
    One can use the dominated convergence theorem to show that $\ta \mapsto R(\ta,\delta)$ is continuous. Thus there exists an interval $(\ta_0,\ta_1)$ such that $\ta_1-\ta_0 > 0$ and $R(\ta,\delta') \le 1/n - \eps$ for all $\ta \in (\ta_0,\ta_1)$. 

    Let $r'_\tau$ be the average risk of $\delta$ with respect to the prior $\ta \sim \Na(0,\tau^2)$. Also let $r_\tau$ be the average risk of the Bayes estimator with respect to the prior $\ta \sim \Na(0,\tau^2)$. We know that $r_\tau$ is the posterior variance of $\tau$ and thus 
    \[r_\tau = \frac{1}{n+1/\tau^2} = \frac{\tau^2}{n\tau^2+1}. \]
    Thus $r_\tau$ approaches $1/n$ as $\tau \nearrow \infty$. We also know that $r_\tau' \le 1/n$ for all $\tau$. We will now look at the ratio $\frac{1/n-r'_\tau}{1/n-r_\tau}$. This is a sort of Taylor's expansion of $r_\tau$ and $r_\tau'$ about $1/n$. Note that
    \begin{align*}
        \frac{1/n-r_\tau'}{1/n-r_\tau} &= \frac{\int_\R (1/n - R(\ta,\delta))\frac{1}{\sqrt{2\pi}\tau}\exp(-1/2\ta^2)d\ta}{\frac{1}{n}-\frac{\tau^2}{n\tau^2+1}}\\
        &=\frac{\int_\R (1/n - R(\ta,\delta))\frac{1}{\sqrt{2\pi}\tau}\exp(-1/2\ta^2)d\ta}{\frac{n\tau^2+1}{n(n\tau^2+1)}-\frac{n\tau^2}{n(n\tau^2+1)}}\\
        &=\frac{\int_\R (1/n - R(\ta,\delta))\frac{1}{\sqrt{2\pi}\tau}\exp(-1/2\ta^2)d\ta}{\frac{1}{n(n\tau^2+1)}}\\
        &=\frac{n(n\tau^2+1)}{\sqrt{2\pi}\tau}\cdot \int_\R (1/n - R(\ta,\delta))\exp(-1/2\ta^2)d\ta\\
        &\ge \frac{n(n\tau^2+1)}{\sqrt{2\pi}\tau}\cdot \int_{\ta_0}^{\ta_1} (1/n - R(\ta,\delta))\exp(-1/2\ta^2)d\ta\\
        &\ge \frac{n(n\tau^2+1)}{\sqrt{2\pi}\tau}\cdot \eps\int_{\ta_0}^{\ta_1}  \exp(-1/2\ta^2)d\ta.
    \end{align*}
    As $\tau \to \infty$, $\frac{n(n\tau^2+1)}{\sqrt{2\pi}\tau} \to \infty$ and by the dominanted convergence theorem $\int_{\ta_0}^{\ta_1}  \exp(-1/2\ta^2)d\ta \to \int_{\ta_0}^{\ta_1} 1d\ta = \ta_1-\ta_0 > 0$. Thus we have 
    \[\lim_{\tau \to \infty}         \frac{1/n-r_\tau'}{1/n-r_\tau'} = \infty.  \]
    In particular there exists $\tau>0$ such that $\frac{1/n-r_\tau'}{1/n-r_\tau} > 1$. This implies that $r_\tau' < r_\tau$ which is a contradiction.
\end{ex}


\section{Inadmissibility $(p \ge 3)$}
We will now look at an example of simultaneous estimation and look at the James-Stein estimator. The take away will be that minimax estimators and UMRUES need not be admissible.

Suppose $X\in \R^p$ and $X \sim \Na(\ta, I_p)$ for some $\ta \in \R^p$. Our goal is to estimate $\ta$ under the loss $L(\ta,d) = \sum_{j=1}^p (\ta_j-d_j)^2 = \norm{\ta-d}_2^2$. The estimator $\delta(X) =X$ is
\begin{itemize}
    \item A minimax estimator for $\ta$.
    \item The UMRUES for $\ta$.
    \item The MLE for $\ta$, that is $X = \arg\max_\ta p(x;\ta)$.
\end{itemize}
From many perspectives $X$ seems like the best estimator but $X$ is admissible for $p \ge 3$. Recall the emperical Bayes estimator for $\ta$
\[\left(1-\frac{p}{\sum_i X_i^2}\right)X. \]
This came up in the setting $\ta_i \iid \Na(0,\tau^2)$ and $X_i|\ta_i \stackrel{ind}{~} \Na(\ta_1,1)$. We will see that a similar estimator will outperform $\delta(X)=X$ uniformly in $\ta$ in a frequentist setting. 

For intuition one may ask what is the problem with $\delta(X)=X$? The problem is that $\norm{X}_2^2$ is normally much larger $\norm{\ta}_2^2$ since $\E[\norm{X}_2^2] =\sum_{i=1}^p \left(\ta_i^2+1\right) = \norm{\ta}_2^2+p >> \norm{\ta}_2^2$.
\begin{thrm}
    \emph{[TPE 5.5.1]} Define the estimator $\delta^0$ by
    \[\delta_j^0(X) = \left(1-\frac{p-2}{\norm{X}_2^2}\right)X_j. \]
    The estimator $\delta^0$ had strictly smaller risk than $\delta(X)=X$ for all $\ta$. Thus $\delta(X)=X$ is inadmissible.
\end{thrm}
We call $\delta^0$ a \emph{James-Stein estimator}.
\begin{proof}
    We know that $R(\ta,\delta) = p$ when $\delta(X)=X$. Now note that
    \begin{align*}
        R(\ta,\delta^0) &= \E_\ta\left[\sum_j \left(\ta_j - \left(1- \frac{p-2}{\norm{X}_2^2}\right)X_j\right)^2\right]\\
        &= \E_\ta\left[\sum_j \left(\ta_j - X_j+ \frac{p-2}{\norm{X}_2^2}X_j\right)^2\right]\\
        &=\sum_j \E_\ta[(\ta_j-X_j)^2]-2\sum_j \E_\ta\left[(X-\ta_j)\left(\frac{p-2}{\norm{X}_2^2}X_j\right)\right]+\sum_j \E_\ta\left[\frac{(p-2)^2X_j^2}{\norm{X}_2^4}\right]\\
        &=p-2\sum_j \E_\ta\left[(X_j-\ta_j)\left(\frac{p-2}{\norm{X}_2^2}X_j\right)\right]+\sum_j \E_\ta\left[\frac{(p-2)^2X_j^2}{\norm{X}_2^4}\right].
    \end{align*}
    Recall Stein's identity if $X\sim \Na(\mu,\sigma^2)$ we have
    \[\E[g(X)(X-\mu)] = \sigma^2\E[g'(X)]. \]
    By conditioning this gives 
    \[ \E_\ta\left[(X_j-\ta_j)\left(\frac{p-2}{\norm{X}_2^2}X_j\right)\right] = \E_\ta\left[(p-2)\frac{\partial}{\partial X_j}\left(\frac{X_j}{\norm{X}_2^2}\right)\right].\]
    If we make this subsitution we have
    \begin{align*}
        R(\ta,\delta^0) &=p-2\sum_j \E_\ta\left[(p-2)\frac{\partial}{\partial X_j}\left(\frac{X_j}{\norm{X}_2^2}\right)\right]+\sum_j \E_\ta\left[\frac{(p-2)^2X_j^2}{\norm{X}_2^4}\right]\\
        &=p-2\sum_j \E_\ta\left[(p-2)\frac{\norm{X}_2^2-2X_j^2}{\norm{X}_2^4}\right]+\sum_j \E_\ta\left[\frac{(p-2)^2X_j^2}{\norm{X}_2^4}\right]\\
        &=p-2(p-2)\E_\ta\left[\sum_j \frac{\norm{X}_2^2-2X_j^2}{\norm{X}_2^4}\right]+(p-2)^2\E_\ta\left[\sum_j \frac{X_j^2}{\norm{X}_2^4}\right]\\
        &=p-2(p-2)\E_\ta\left[\frac{p\norm{X}_2^2-2\norm{X}_2^2}{\norm{X}_2^4}\right]+(p-2)^2\E_\ta\left[\frac{\norm{X}^2_2}{\norm{X}_2^4}\right]\\
        &=p-2(p-2)^2\E_\ta\left[\frac{1}{\norm{X}_2^2}\right]+(p-2)^2\E_\ta\left[\frac{1}{\norm{X}_2^2}\right]\\
        &=p-(p-2)^2\E_\ta\left[\frac{1}{\norm{X}_2^2}\right]\\
        &<p.
    \end{align*}
    Thus $\delta^0$ uniformly outperforms $\delta$.
\end{proof}
\begin{remark}
    \begin{enumerate}
        \item Why only $p > 2$? The random variable $\frac{1}{\norm{X}_2^2}$ is not integrable when $p=2$. Thus the regularity conditions of Stein's identity are not met.
        \item The above theorem is surprising! We can improve the risk by sharing information across different dimensions. The components $X_{\setminus j}$ are used to estimate $\ta_j$ even though $X_j$ are independent and $\ta_j$ do not have any restrictions.
        \item Define $\delta^v_j = \left(1 - \frac{p-2}{\norm{X_v}_2^2}\right)(X_j-v_j)$ for $v\in \R^p$. Then the above proof shows that $\delta^v$ also out performs $\delta$.
        \item If $\sigma \neq 1$, then we can use $\wh{\ta}_{JS} = \left(1-\frac{\sigma^2(p-2)}{\norm{X}_2^2}\right)X$.
        \item $\wh{\ta}_{JS}$ is inadmissible, we can improve by using
        \[\delta_j'(X) = \left(1- \frac{p-2}{\norm{X}_2^2}\right)_+X_j. \]
        \item A warning: For some $j$, $\E[(\delta_j^0(X)-\ta_j)^2]$ may be larger than $\E[(X_j-\ta_j)^2]$. An example is ``Baseball batting averages - James-Stein estimator''
    \end{enumerate}
\end{remark}
\end{document}
\documentclass{article}
\usepackage{ae,aecompl}
\usepackage{todonotes}
\usepackage{chngcntr}
\usepackage{tikz-cd}
\usepackage{graphicx}
\graphicspath{ {./images/}}
\usepackage[all,cmtip]{xy}
\usepackage{amsmath, amscd}
\usepackage{amsthm}
\usepackage{amssymb}
\usepackage{amsfonts}
\usepackage{bm}
\usepackage{qsymbols}
\usepackage{latexsym}
\usepackage{mathrsfs}
\usepackage{mathtools}
\usepackage{cite}
\usepackage{color}
\usepackage{url}
\usepackage{enumerate}
\usepackage{verbatim}
\usepackage[draft=false, colorlinks=true]{hyperref}
\usepackage{pdfpages}
\usepackage[margin=1.2in]{geometry}
\usepackage{IEEEtrantools}

\usepackage{fancyhdr}


\usepackage[nameinlink]{cleveref}


\DeclareMathOperator*{\ac}{accept}
\DeclareMathOperator*{\amax}{argmax}
\DeclareMathOperator*{\amin}{argmin}
\DeclareMathOperator*{\Aut}{Aut}
\newcommand {\al}{{\alpha}}
\newcommand {\abs}[1]{{\left\lvert#1\right\rvert}}
\newcommand {\A}{{\mathcal{A}}}
\newcommand {\AM}{{\mathrm{AM}}}
\newcommand {\AMp}{{\AM_{p}^{X}\!(\Ri_\w)}}
\newcommand {\B}{{\mathcal{B}}}
\DeclareMathOperator*{\Be}{Bern}
\newcommand {\Br}{{\dot{B}}}
\newcommand {\Ba}{{\mathfrak{B}}}
\newcommand {\C}{{\mathbb C}}
\newcommand {\ce}{\mathrm{c}}
\newcommand {\Ce}{\mathrm{C}}
\newcommand {\Cc}{\mathrm{C_{c}}}
\newcommand {\Ccinf}{\mathrm{C_{c}^{\infty}}}
\DeclareMathOperator{\cov}{Cov}
\DeclareMathOperator{\DEV}{DEV}
\newcommand {\Di}{{\mathbb D}}
\newcommand {\dom}{\mathrm{dom}}
\newcommand{\dist}{\stackrel{\mathrm{dist}}{=}}
\newcommand {\ud}{\mathrm{d}}
\newcommand {\ue}{\mathrm{e}}
\newcommand {\eps}{\varepsilon}
\newcommand {\veps}{\varepsilon}
\newcommand {\vrho}{{\varrho}}
\newcommand {\E}{{\mathbb{E}}}
\newcommand {\Ec}{{\mathcal{E}}}
\newcommand {\Ell}{L}
\newcommand {\Ellp}{{L_{p}[0,1]}}
\newcommand {\Ellpprime}{{L_{p'}([0,1])}}
\newcommand {\Ellq}{{L_{q}([0,1])}}
\newcommand {\Ellqprime}{{L_{q'}([0,1])}}
\newcommand {\Ellr}{L^{r}}
\newcommand {\Ellone}{{L_{1}([0,1])}}
\newcommand{\Elltwo}{{L_{2}([0,1])}}
\newcommand{\Ellinfty}{L^{\infty}}
\newcommand{\Ellinftyc}{L_{\mathrm{c}}^{\infty}}
\newcommand{\exb}[1]{\exp\left\{#1\right\}}
\DeclareMathOperator*{\Ext}{Ext}
\newcommand{\F}{{\mathcal{F}}}
\newcommand{\Fe}{{\mathbb{F}}}
\newcommand{\G}{{\mathcal{G}}}
\newcommand{\HF}{\mathcal{H}_{\text{FIO}}^{1}(\Rd)}
\newcommand{\Hr}{H}
\newcommand{\HT}{\mathcal{H}}
\newcommand{\ui}{\mathrm{i}}
\newcommand{\I}{{I}}
\newcommand{\J}{{\mathcal{J}}}
\newcommand{\id}{{\mathrm{id}}}
\newcommand{\iid}{\stackrel{\mathclap{\normalfont\mbox{iid}}}{\sim}}
\newcommand{\im}{{\text{im }}}
\newcommand{\ind}{{\perp\!\!\!\perp}}
\DeclareMathOperator*{\Int}{int}
\newcommand{\intx}{{\overline{\int_{X}}}}
\newcommand{\inte}{{\overline{\int_{\E}}}}
\newcommand{\la}{\lambda}
\newcommand{\rb}{\rangle}
\newcommand{\lb}{{\langle}}
\newcommand{\La}{\Lambda}
\newcommand{\calL}{{\mathcal{L}}}
\newcommand{\lp}{{\mathcal{L}}^{p}}
\newcommand{\lpo}{{\overline{\mathcal{L}}^{p}\!}}
\newcommand{\Lpo}{{\overline{\Ell}^{p}\!}}
\newcommand{\M}{{\mathbf{M}}}
\newcommand{\Ma}{{\mathcal{M}}}
\newcommand{\N}{{{\mathbb N}}}
\newcommand{\Na}{{{\mathcal{N}}}}
\newcommand{\norm}[1]{\left\|#1\right\|}
\newcommand{\normm}[1]{{\left\vert\kern-0.25ex\left\vert\kern-0.25ex\left\vert #1 
    \right\vert\kern-0.25ex\right\vert\kern-0.25ex\right\vert}}
\newcommand{\Om}{{{\Omega}}}
\newcommand{\one}{{{\bf 1}}}
\newcommand{\pic}{\text{Pic }}
\newcommand{\ph}{{\varphi}}
\newcommand{\Pa}{{\mathbb{P}}}
\newcommand{\Po}{{\mathcal{P}}}
\newcommand{\Q}{{\mathbb{Q}}}
\newcommand{\R}{{\mathbb R}}
\newcommand{\Rd}{{\mathbb{R}^{d}}}
\DeclareMathOperator{\rej}{reject }
\newcommand{\Rn}{{\mathbb{R}^{n}}}
\newcommand{\cR}{{\mathcal{R}}}
\newcommand{\Rad}{{\mathrm{Rad}}}
\newcommand{\ran}{{\mathrm{ran}}}
\newcommand{\Ri}{{\mathrm{R}}}
\newcommand{\supp}{{\mathrm{supp}}}
\newcommand{\Se}{\mathrm{S}}
\newcommand{\Sp}{S^{*}(\Rn)}
\newcommand{\St}{{\mathrm{St}}}
\newcommand{\Sw}{\mathcal{S}}
\newcommand{\T}{{\mathcal{T}}}
\newcommand{\ta}{{\theta}}
\newcommand{\Ta}{{\Theta}}
\newcommand{\topp}{\stackrel{p}{\to}}
\newcommand{\todd}{\stackrel{d}{\to}}
\newcommand{\toL}[1]{\stackrel{L^{#1}}{\to}} 
\newcommand{\toas}{\stackrel{a.s.}{\to}}
\DeclareMathOperator{\V}{Var}
\newcommand {\w}{{\omega}}
\newcommand {\W}{{\mathrm{W}}}
\newcommand {\Wnp}{\text{$\mathrm{W}$\textsuperscript{$n,\!p$}}}
\newcommand {\Wnpeq}{\text{$\mathrm{W}$\textsuperscript{$n\!,\!p$}}}
\newcommand {\Wonep}{\text{$\mathrm{W}$\textsuperscript{$1,\!p$}}}
\newcommand {\Wonepeq}{\text{$\mathrm{W}$\textsuperscript{$1\!,\!p$}}}
\newcommand {\X}{{\mathcal{X}}}
\newcommand {\Z}{{{\mathbb Z}}}
\newcommand {\Za}{{\mathcal{Z}}}
\newcommand {\Zd}{{\Z[\sqrt{d}]}}
\newcommand {\vanish}[1]{\relax}

\newcommand {\wh}{\widehat}
\newcommand {\wt}{\widetilde}
\newcommand {\red}{\color{red}}

% Distributions
\newcommand{\normal}{\mathsf{N}}
\newcommand{\poi}{\mathsf{Poisson}}
\newcommand{\bern}{\mathsf{Bernoulli}}
\newcommand{\bin}{\mathsf{Binomal}}
\newcommand{\multi}{\mathsf{Multinomial}}
\newcommand{\Exp}{\mathsf{Exp}}



% put your command and environment definitions here




% some theorem environments
% remove "[theorem]" if you do not want them to use the same number sequence


  \newtheorem{thrm}{Theorem}
  \newtheorem{lemma}{Lemma}
  \newtheorem{prop}{Proposition}
  \newtheorem{cor}{Corollary}

  \newtheorem{conj}{Conjecture}
  \renewcommand{\theconj}{\Alph{conj}}  % numbered A, B, C etc

  \theoremstyle{definition}
  \newtheorem{defn}{Definition}
  \newtheorem{ex}{Example}
  \newtheorem{exs}{Examples}
  \newtheorem{question}{Question}
  \newtheorem{remark}{Remark}
  \newtheorem{notn}{Notation}
  \newtheorem{exer}{Exercise}




\title{STATS305A - Lecture 12}
\author{John Duchi\\ Scribed by Michael Howes}
\date{10/28/21}

\pagestyle{fancy}
\fancyhf{}
\rhead{STATS305A - Lecture 12}
\lhead{10/28/21}
\rfoot{Page \thepage}

\begin{document}
\maketitle
\tableofcontents
\section{Announcements}
\begin{itemize}
    \item Etude 2 due today 5pm.
    \item No class next Tuesday.
\end{itemize}
\section{Model Selection and prediction}
\subsection{Motivation}
Up to this point we've treated the model $Y=X\beta + \eps$ as ``god-give''. This is a bit inaccurate. In real life we will typically have data and no model and have to figure it out and select a model. When selecting a model we have two desiderata:
\begin{itemize}
    \item Identify important features thhat are relating $x$ to our response $y$.
    \item Pure predictive accuracy: how well can we predict $y$ from $x$?
\end{itemize}
These two are intertwinned. We don't always have to choose one over the other.
\subsection{Bias/Variance Decomposition}
Suppose we are in a setting where $y = f(x)+\eps$ and $\E[\eps|x]=0$. This is equivalent to having $f(x) = \E[Y|X=x]$ since if $\eps = y-f(x)$, then \[\E[\eps|x] = \E[y|x] - f(x).\]
Thus $\E[\eps|x]=0$ if and only if $f(x) = \E[y|x]$. Define $\sigma^2(x) = \E[\eps^2|x]$ which is the conditional variance of $\eps$. 

Our goal is to fit a predictor $\wh{f}$ using a sample $\{(x_i,y_i)\}_{i=1}^n$. Note that if we think of the sample of $\{(x_i,y_i)\}_{i=1}^n$ as random, then the predictor $\wh{f}$ is random (like how $\wh{\beta}$ is random in the linear model). Thus we can take the expectation of quantities involving $\wh{f}$ over all samples $\{(x_i,y_i)\}_{i=1}^n$. This idea will be used many times over the course of this lecture. 
\begin{defn}
    If we have a predictor $\wh{f}$ of a model $y = f(x)+\eps$, then we define the \emph{in-sample (MSE) risk} of $\wh{f}$ to be 
    \[R_{in}(\wh{f}) = \E\left[\frac{1}{n}\sum_{i=1}^n (\wh{f}(x_i)-f(x_i))^2\right], \]
    where the above expectation is taken over all samples $\{(x_i,y_i)\}_{i=1}^n$ with $x_i$ fixed. (That is we fix $x$ and calculate $\wh{f}$ using different samples $(x,y)$, we then calculate the quantity $\frac{1}{n}\sum_{i=1}^n (\wh{f}(x_i)-f(x_i))^2$ and take the expectation over all samples $(x,y)$.)
\end{defn}
\begin{aside}
    Sometimes the in-sample risk is called the $L^2(P_n)$ risk. This is because $R_{in}$ is the expectation of the $L^2$ norm error of $\wh{f}-f$ with respect to the distribution
    \[P_n = \frac{1}{n}\sum_{i=1}^n \one_{x_i}.\]
\end{aside}
\begin{defn}
    Sometimes the insample risk is defined with respect to a fresh sample $\{Y_i^*\}_{i=1}^n$ where 
    \[Y_i^* = \text{ a new sample of } Y_i = f(x_i) + \eps_i^*, \]
    where $\eps_i^*$ is an independent copy of $\eps_i$. We then define
    \[R_{in}^*(\wh{f}) = \E\left[\frac{1}{n}\sum_{i=1}^n \left(Y_i^*-\wh{f}(x_i)\right)^2\right], \]
    where here the expectation is over both $Y_1,\ldots, Y_n$ (used to calculate $\wh{f}$) and over $Y_1^*,\ldots, Y_n^*$ (used to calculate $(Y_i^*-\wh{f}(x_i))^2$).
\end{defn}
Note that 
\begin{align}
    R^*_{in}(\wh{f}) &= \E\left[\frac{1}{n} \sum_{i=1}^n (Y_i^*-\wh{f}(x_i))^2\right]\\
    &=\E\left[\frac{1}{n} \sum_{i=1}^n (Y_i^*-f(x_i))^2\right]+\E\left[\frac{1}{n}\sum_{i=1}^n (\wh{f}(x_i)-f(x_i))^2\right]\\
    &=\frac{1}{n}\sum_{i=1}^n \sigma^2(x_i)+R_{in}(\wh{f}).
\end{align}
We call $\frac{1}{n}\sum_{i=1}^n \sigma^2(x_i)$ the irreducible error. 

Now suppose that we have a function $g:\X \to \R$ where $\X$ is the space $X$ lives in. Note that $g$ is different to $\wh{f}$. The predictor $\wh{f}$ is something that the depend on the sample $(x,y)$ used to fit $\wh{f}$. The function $g$ is simply a function. It is a way of taking an $X$ and producing a number. With this in mind we define
\begin{defn}
    Given a function $g : \X \to \R$, the \emph{(MSE) out of sample risk} of $g$ is
    \[R_{out}(g) = \E[(Y-g(X))^2] = \int_\X \E[(Y-g(x))^2|X=x]p(x)dx. \]
    Here the expectation is over both $Y$ and $X$ (hence out of sample - we are allowing $X$ to change).
\end{defn}
Note that 
\begin{align*}
    R_{out}(g) &= \E\left[(Y-f(X)+f(X)+g(X))^2\right]\\
&=\E\left[(Y-f(X))^2\right]+\E[(f(X)-g(X))^2]+2\E\left[(Y-f(X))(f(X)-g(X))\right]\\
&=\E[\sigma^2(X)]+\E[(f(X)-g(X))^2].
\end{align*}
We again call $\E[\sigma^2(X)]$ the irreducible error and we could call $\E[(f(X)-g(X))^2]$ the error in mean prediction (this last term is just a term John used - he said that there isn't really a term in literature for it).

In the out of sample risk we average over all the $X$'s we could possible draw. In the in sample we fix the value $x_i$ and average over all possible $y_i$. Note that if our data if i.i.d., then 
\[R_{out}(g) = \E[(g(X_{n+1})-Y_{n+1})^2]. \]

 


\end{document}
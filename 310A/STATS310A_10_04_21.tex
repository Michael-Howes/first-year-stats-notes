\documentclass{article}
\usepackage{ae,aecompl}
\usepackage{todonotes}
\usepackage{chngcntr}
\usepackage{tikz-cd}
\usepackage{graphicx}
\graphicspath{ {./images/}}
\usepackage[all,cmtip]{xy}
\usepackage{amsmath, amscd}
\usepackage{amsthm}
\usepackage{amssymb}
\usepackage{amsfonts}
\usepackage{bm}
\usepackage{qsymbols}
\usepackage{latexsym}
\usepackage{mathrsfs}
\usepackage{mathtools}
\usepackage{cite}
\usepackage{color}
\usepackage{url}
\usepackage{enumerate}
\usepackage{verbatim}
\usepackage[draft=false, colorlinks=true]{hyperref}
\usepackage{pdfpages}
\usepackage[margin=1.2in]{geometry}
\usepackage{IEEEtrantools}

\usepackage{fancyhdr}


\usepackage[nameinlink]{cleveref}


\DeclareMathOperator*{\ac}{accept}
\DeclareMathOperator*{\amax}{argmax}
\DeclareMathOperator*{\amin}{argmin}
\DeclareMathOperator*{\Aut}{Aut}
\newcommand {\al}{{\alpha}}
\newcommand {\abs}[1]{{\left\lvert#1\right\rvert}}
\newcommand {\A}{{\mathcal{A}}}
\newcommand {\AM}{{\mathrm{AM}}}
\newcommand {\AMp}{{\AM_{p}^{X}\!(\Ri_\w)}}
\newcommand {\B}{{\mathcal{B}}}
\DeclareMathOperator*{\Be}{Bern}
\newcommand {\Br}{{\dot{B}}}
\newcommand {\Ba}{{\mathfrak{B}}}
\newcommand {\C}{{\mathbb C}}
\newcommand {\ce}{\mathrm{c}}
\newcommand {\Ce}{\mathrm{C}}
\newcommand {\Cc}{\mathrm{C_{c}}}
\newcommand {\Ccinf}{\mathrm{C_{c}^{\infty}}}
\DeclareMathOperator{\cov}{Cov}
\DeclareMathOperator{\DEV}{DEV}
\newcommand {\Di}{{\mathbb D}}
\newcommand {\dom}{\mathrm{dom}}
\newcommand{\dist}{\stackrel{\mathrm{dist}}{=}}
\newcommand {\ud}{\mathrm{d}}
\newcommand {\ue}{\mathrm{e}}
\newcommand {\eps}{\varepsilon}
\newcommand {\veps}{\varepsilon}
\newcommand {\vrho}{{\varrho}}
\newcommand {\E}{{\mathbb{E}}}
\newcommand {\Ec}{{\mathcal{E}}}
\newcommand {\Ell}{L}
\newcommand {\Ellp}{{L_{p}[0,1]}}
\newcommand {\Ellpprime}{{L_{p'}([0,1])}}
\newcommand {\Ellq}{{L_{q}([0,1])}}
\newcommand {\Ellqprime}{{L_{q'}([0,1])}}
\newcommand {\Ellr}{L^{r}}
\newcommand {\Ellone}{{L_{1}([0,1])}}
\newcommand{\Elltwo}{{L_{2}([0,1])}}
\newcommand{\Ellinfty}{L^{\infty}}
\newcommand{\Ellinftyc}{L_{\mathrm{c}}^{\infty}}
\newcommand{\exb}[1]{\exp\left\{#1\right\}}
\DeclareMathOperator*{\Ext}{Ext}
\newcommand{\F}{{\mathcal{F}}}
\newcommand{\Fe}{{\mathbb{F}}}
\newcommand{\G}{{\mathcal{G}}}
\newcommand{\HF}{\mathcal{H}_{\text{FIO}}^{1}(\Rd)}
\newcommand{\Hr}{H}
\newcommand{\HT}{\mathcal{H}}
\newcommand{\ui}{\mathrm{i}}
\newcommand{\I}{{I}}
\newcommand{\J}{{\mathcal{J}}}
\newcommand{\id}{{\mathrm{id}}}
\newcommand{\iid}{\stackrel{\mathclap{\normalfont\mbox{iid}}}{\sim}}
\newcommand{\im}{{\text{im }}}
\newcommand{\ind}{{\perp\!\!\!\perp}}
\DeclareMathOperator*{\Int}{int}
\newcommand{\intx}{{\overline{\int_{X}}}}
\newcommand{\inte}{{\overline{\int_{\E}}}}
\newcommand{\la}{\lambda}
\newcommand{\rb}{\rangle}
\newcommand{\lb}{{\langle}}
\newcommand{\La}{\Lambda}
\newcommand{\calL}{{\mathcal{L}}}
\newcommand{\lp}{{\mathcal{L}}^{p}}
\newcommand{\lpo}{{\overline{\mathcal{L}}^{p}\!}}
\newcommand{\Lpo}{{\overline{\Ell}^{p}\!}}
\newcommand{\M}{{\mathbf{M}}}
\newcommand{\Ma}{{\mathcal{M}}}
\newcommand{\N}{{{\mathbb N}}}
\newcommand{\Na}{{{\mathcal{N}}}}
\newcommand{\norm}[1]{\left\|#1\right\|}
\newcommand{\normm}[1]{{\left\vert\kern-0.25ex\left\vert\kern-0.25ex\left\vert #1 
    \right\vert\kern-0.25ex\right\vert\kern-0.25ex\right\vert}}
\newcommand{\Om}{{{\Omega}}}
\newcommand{\one}{{{\bf 1}}}
\newcommand{\pic}{\text{Pic }}
\newcommand{\ph}{{\varphi}}
\newcommand{\Pa}{{\mathbb{P}}}
\newcommand{\Po}{{\mathcal{P}}}
\newcommand{\Q}{{\mathbb{Q}}}
\newcommand{\R}{{\mathbb R}}
\newcommand{\Rd}{{\mathbb{R}^{d}}}
\DeclareMathOperator{\rej}{reject }
\newcommand{\Rn}{{\mathbb{R}^{n}}}
\newcommand{\cR}{{\mathcal{R}}}
\newcommand{\Rad}{{\mathrm{Rad}}}
\newcommand{\ran}{{\mathrm{ran}}}
\newcommand{\Ri}{{\mathrm{R}}}
\newcommand{\supp}{{\mathrm{supp}}}
\newcommand{\Se}{\mathrm{S}}
\newcommand{\Sp}{S^{*}(\Rn)}
\newcommand{\St}{{\mathrm{St}}}
\newcommand{\Sw}{\mathcal{S}}
\newcommand{\T}{{\mathcal{T}}}
\newcommand{\ta}{{\theta}}
\newcommand{\Ta}{{\Theta}}
\newcommand{\topp}{\stackrel{p}{\to}}
\newcommand{\todd}{\stackrel{d}{\to}}
\newcommand{\toL}[1]{\stackrel{L^{#1}}{\to}} 
\newcommand{\toas}{\stackrel{a.s.}{\to}}
\DeclareMathOperator{\V}{Var}
\newcommand {\w}{{\omega}}
\newcommand {\W}{{\mathrm{W}}}
\newcommand {\Wnp}{\text{$\mathrm{W}$\textsuperscript{$n,\!p$}}}
\newcommand {\Wnpeq}{\text{$\mathrm{W}$\textsuperscript{$n\!,\!p$}}}
\newcommand {\Wonep}{\text{$\mathrm{W}$\textsuperscript{$1,\!p$}}}
\newcommand {\Wonepeq}{\text{$\mathrm{W}$\textsuperscript{$1\!,\!p$}}}
\newcommand {\X}{{\mathcal{X}}}
\newcommand {\Z}{{{\mathbb Z}}}
\newcommand {\Za}{{\mathcal{Z}}}
\newcommand {\Zd}{{\Z[\sqrt{d}]}}
\newcommand {\vanish}[1]{\relax}

\newcommand {\wh}{\widehat}
\newcommand {\wt}{\widetilde}
\newcommand {\red}{\color{red}}

% Distributions
\newcommand{\normal}{\mathsf{N}}
\newcommand{\poi}{\mathsf{Poisson}}
\newcommand{\bern}{\mathsf{Bernoulli}}
\newcommand{\bin}{\mathsf{Binomal}}
\newcommand{\multi}{\mathsf{Multinomial}}
\newcommand{\Exp}{\mathsf{Exp}}



% put your command and environment definitions here




% some theorem environments
% remove "[theorem]" if you do not want them to use the same number sequence


  \newtheorem{thrm}{Theorem}
  \newtheorem{lemma}{Lemma}
  \newtheorem{prop}{Proposition}
  \newtheorem{cor}{Corollary}

  \newtheorem{conj}{Conjecture}
  \renewcommand{\theconj}{\Alph{conj}}  % numbered A, B, C etc

  \theoremstyle{definition}
  \newtheorem{defn}{Definition}
  \newtheorem{ex}{Example}
  \newtheorem{exs}{Examples}
  \newtheorem{question}{Question}
  \newtheorem{remark}{Remark}
  \newtheorem{notn}{Notation}
  \newtheorem{exer}{Exercise}




\title{STATS310A - Lecture 5}
\author{Persi Diaconis\\ Scribed by Michael Howes}
\date{10/05/21}

\pagestyle{fancy}
\fancyhf{}
\rhead{STATS310A - Lecture 5}
\lhead{10/05/21}

\begin{document}
\maketitle
\tableofcontents
\section{Applying Borel Cantelli}
Recall, $l_n(\w)=$ length of the head run starting at $n$. That is $l_n(\w)=k$ if and only if $d_n(\w)=d_{n+1}(\w)=\ldots=d_{n+k-1}(\w)=1$ and $d_{n+k}=0$. Last time we saw
\[P(\w: l_n(\w) > (1+\varepsilon)\log_2(n)\; i.o) =0.\]
We will now see
\[P(w :l_n(w) > \log_2(n) \; i.o.) = 1.\]
The problem is that the events $\{l_n \ge r_n\}$ are not independent. The trick is to use subsequences.

\begin{prop}
Let $r_n$ be a weekly increasing subsequence such that $r_n > 0$ and $\sum_{n=1}^\infty \frac{2^{-r_n}}{r_n} = \infty$ (for example $r_n = \log_2(n))$. Then $P(l_n \ge r_n\; i.o.) = 1$.
\end{prop}
\begin{proof}
    Define a sequence $n_k$ by $n_1 = 1$ and $n_{k+1} = n_k + r_{n_k}$ so that $n_{k+1}-n_k = r_{n_k}$. Define $A_k = \{l_{n_k} \ge r_{n_k}\}$. Then $A_k = \{\w : d_i(\w)=1 \text{ for } n_k \le i \le n_{k+1}-1\}$ since $n_k + r_{n_k}=n_{k+1}$. Thus the events $\{A_k\}_{k=1}^\infty$ are independent. Note also that $P(A_k) = \frac{1}{2^{r_{n_k}}}$. The second Borel Cantelli theorem tells us that if
    \[\sum_{k=1}^\infty P(A_k) = \infty, \]
    then $P(A_k \; i.o.) = 1$. Note that
    \begin{IEEEeqnarray*}{rCl+x+}
        \sum_{k=1}^\infty P(A_k)&=&\sum_{k=1}^\infty 2^{-r_{n_k}} \\
        &=&\sum_{k=1}^\infty 2^{-r_{n_k}} \frac{n_{k+1}-n_k}{r_{n_k}}\\
        &=&\sum_{k=1}^\infty \sum_{n=n_k}^{n_{k+1}-1} \frac{2^{-r_{n_k}}}{r_{n_k}}\\
        &\ge&\sum_{k=1}^\infty \sum_{n=n_k}^{n_{k+1}-1} \frac{2^{-{r_n}}}{r_n}\\
        &=&\sum_{n=1}^\infty \frac{2^{-{r_n}}}{r_n}\\
        &=&\infty.&\qedhere
    \end{IEEEeqnarray*}
\end{proof}
\section{Homework}
Read sections 10, 11 and 12.
Do problems 10.3, 10.4, 11.2, 14.5 and two more.
\section{Measure Theory}
Let $\Om$ be a set and $\F$ an algebra of subsets of $\Om$.

\begin{defn}
    A \emph{measure} on $(\Om,\F)$ is a function $\mu : \F \to [0,\infty]$ such that 
    \begin{enumerate}
        \item (Non-trivial) $\mu(\emptyset)=0$.
        \item (Monotone) If $A,B \in \F$ and $A\subseteq B$, then $\mu(A) \le \mu(B)$.
        \item (Countable additivity) If $\{A_i\}_{i=1}^\infty$ is a countable collection of disjoint subsets in $\F$ and $\bigcup_i A_i \in \F$, then
        \[\mu\left(\bigcup_{i=1}^\infty A_i\right)=\sum_{i=1}^\infty \mu(A_i).\]
    \end{enumerate}
\end{defn}
For example with $\Om =\R$ and $\F$ equal to the collection of all subsets, then $\mu(\emptyset) = 0$ and $\mu(A) = \infty$ otherwise is a measure. Likewise $\mu(A) = $ \# of points in $A$, is also a measure of $(\Om,\F)$. 

One might ask why we are doing this?
\begin{enumerate}
    \item We want to define probabilities via densities, that is we want to write
    \[P(A) = \int_A f(\w)d\mu(\w), \]
    where $\mu$ is a measure that is not necessarily a probability measure.
    \item It's free from the hard work we've already done.
\end{enumerate}
A word of caution! For probabilities we had if $A \subseteq B$, then $P(B \setminus A) = P(B) - P(A)$. This is not true in general. For measures we hamy have $P(B)=P(A)=\infty$. For example if $\mu$ is Lesbegue measure, $B = (0,\infty)$ and $A = (x,\infty)$ for some $x \ge 0$, then $\mu(B \setminus A) = \mu((0,x]) = x$ but $\mu(B)=\mu(A) = \infty$.

\begin{defn}
    A measure $\mu$ on $(\Om,\F)$ is \emph{$\sigma$-finite} if there exists a countable collection $B_i \in \F$ s.t. $\bigcup_i B_i = \Om$ and $\mu(B_i) < \infty$ for all $i$.
\end{defn}
For example Lesebgue measure is $\sigma$-finite since we can set $B_i = (-i,i)$. The measure $\mu(A) = \# $ of points in  $A$ is not $\sigma$-finite on $\R$.
\begin{thrm}
    Suppose that we have two measure $\mu_1,\mu_2$ on $\sigma(\Po)$ where $\Po$ is a $\pi$-system. If $\mu_1$ and $\mu_2$ agree on $\Po$ and there exists a countable collection $B_i \in \Po$ such that $\mu_1(B_i)=\mu_2(B_i) < \infty$ and $\Om = \bigcup_i B_i$, then $\mu_1$ and $\mu_2$ agree on $\sigma(\Po)$.
\end{thrm}
\begin{proof}
    For $B \in \Po$ with $\mu_1(B) < \infty$, define $L_B = \{A \in \sigma(\Po): \mu_1(A \cap B) = \mu_2(A \cap B)\}$. Then $L_B$ is a $\la$-system and it contains $\Po$ and thus by the $\pi$-$\lambda$ theorem $\sigma(\Po) = L_B$.

    By assumption we have $\Om = \bigcup_i B_i$ where $B_i \in \Po$ and $\mu_1(B_i) < \infty$. Fix $A \in \sigma(\Po)$ and for $i=1,2$, note that, by the inclusion exclusion formula
    \begin{IEEEeqnarray*}{rCl}
        \mu_i\left(\bigcup_{j=1}^n A \cap B_i\right)&=&\sum_{j=1}^n \mu_i(A \cap B_j)-\sum_{1\le j < k \le n} \mu_i(A \cap B_j \cap B_k)+\ldots.
    \end{IEEEeqnarray*}
    Furthore more, finite interesections of $\{B_i\}_{i=1}^\infty$ are in $\Po$ and have finite measure. Thus since $L_B = \sigma(\Po)$ for all $B \in \Po$ with finite measure we have $\mu_1(A \cap B) = \mu_2(A \cap B)$ where $B$ is any finite intersection of $\{B_i\}_{i=1}^\infty$. Thus 
    \[\mu_1\left(\bigcup_{j=1}^n A \cap B_i\right)=\mu_2\left(\bigcup_{j=1}^n A \cap B_i\right), \]
    for all $n$. Letting $n \to \infty$ we see that $\mu_1(A)=\mu_2(A)$.
\end{proof}
Note that $\sigma$-finiteness is needed. If $\mu_1(A)=\#$ of points in $A$ and $\mu_2(A)=\infty$ if $A \neq \emptyset$, then $\mu_1=\mu_2$ on all intervals but $\mu_1\neq \mu_2$ on the Borel $\sigma$-algebra.
\begin{defn}
    Let $\Om$ be a set. An outer measure $\mu^*$ is a function defined on all subsets of $\Om$ such that
    \begin{enumerate}
        \item $\mu^*(A) \in [0,\infty]$
        \item $\mu^*(\emptyset)=0$,
        \item $\mu^*(A) \le \mu^*(B)$ if $A \subseteq B$, and
        \item $\mu^*\left(\bigcup_{i=1}^\infty A_i\right)\le \sum_{i=1}^\infty \mu^*(A_i)$.
    \end{enumerate}
\end{defn}
For example let $\Om$ be any set and let $\A$ be any collection of subsets of $\Om$ and let $\rho : \A \to [0,\infty]$ be any function such that $\rho(\emptyset)=0$. Then 
\[\mu^*(A)= \inf\left\{\sum_{i=1}^\infty \rho(A_i): A \subseteq \bigcup_{i=1}^\infty A_i, A_i \in \A \right\}, \]
where $\inf \emptyset=\infty$ is an outer measure. This can be seen by using the usual $\varepsilon2^{-i}$ trick to prove countable subadditivity.

A special case of this is Hausdorff measure when $M \subseteq \R^n$ is a crinkly manifold or a fractal or something similar. We define $\A =\{B_\varepsilon(x):x\in \R^n, \varepsilon>0\}$ where $B_\varepsilon(x)$ is a ball of radius $\varepsilon$ centred at $x$. We then define $\rho_{n,\gamma}(B_\varepsilon(x))=(2\varepsilon)^\gamma$ and call the resulting $\mu_{n,\gamma}^*$ the $\gamma-$Hausdorff measure on $\R^n$.

\begin{thrm}
    Let $\mu^*$ be an outer measure on all subsets of $\Om$ and define
    \[\Ma = \{A \subseteq \Om : \mu^*(E) =\mu^*(A \cap E) + \mu^*(A^c \cap E) \text{ for all } E \subseteq \Om\}, \]
    then $\Ma$ is a $\sigma$-algebra and $\mu$ restricted to $\Ma$ is a measure.
\end{thrm}
\begin{proof}
    Symbol for symbol, letter for letter, the same as the proof of the Caratheodory extension theorem.
\end{proof}
\section{Distribution functions}
``People at microsoft research'' were putting probabilities on $\R^n$ using distribution functions in a funny way. 

If $\Pa$ is a probability measure on $\R^n$ we can define a function $F : \R^n \to [0,1]$ by $F(x) = \Pa(A_x)$ where $A_x=\{y : y_i \le x_i, \text{ for } i =1,\ldots, n\}$. We call $F(x)$ the distribution function (DF) generated by $\Pa$. A distribution function satisfies the following
\begin{itemize}
    \item $\lim\limits_{x \to +\infty} F(x)=1$.
    \item $\lim\limits_{x \to -\infty} F(x)=0$.
    \item $F$ is monotone in each coordinate.
    \item $F$ is right continuous. If $x_n \searrow x$, then $F(x_n) \searrow F(x)$.
\end{itemize}
But not every $F$ that satisfies the above is a DF. At microsoft they had the idea to take define on $\R^3$
\[MS(x,y,z) = F(x,y,z)G(x,y)H(x), \]
where $F,G$ and $H$ were all distribution functions. One can ask is $MS$ a distribution function? A function $F: \R^2 \to [0,1]$ is a distribution function if and only if
\begin{enumerate}
    \item $\lim\limits_{x \to +\infty} F(x)=1$.
    \item $\lim\limits_{x \to -\infty} F(x)=0$.
    \item $F$ is monotone in each coordinate.
    \item $F$ is right continuous. If $x_n \searrow x$, then $F(x_n) \searrow F(x)$. 
    \item For all $(x_1,y_1),(x_2,y_2)$ such that $x_1 \le x_2$ and $y_1 \le y_2$, we have
    \[F(x_2,y_2)-F(x_2,y_1)-F(x_1,y_2)+F(x_1,y_1)\ge0. \]
    This property requires that the probability of every rectangle is non-negative.
\end{enumerate}
Furthermore, if $F$ satisfies (a)-(e), then there is a unique probability measure $\Pa$ on $\R^2$ such that $\Pa(A_{(x,y)}) =F(x,y)$.

\end{document}
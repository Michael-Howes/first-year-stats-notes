\documentclass{article}
\usepackage{ae,aecompl}
\usepackage{todonotes}
\usepackage{chngcntr}
\usepackage{tikz-cd}
\usepackage{graphicx}
\graphicspath{ {./images/}}
\usepackage[all,cmtip]{xy}
\usepackage{amsmath, amscd}
\usepackage{amsthm}
\usepackage{amssymb}
\usepackage{amsfonts}
\usepackage{bm}
\usepackage{qsymbols}
\usepackage{latexsym}
\usepackage{mathrsfs}
\usepackage{mathtools}
\usepackage{cite}
\usepackage{color}
\usepackage{url}
\usepackage{enumerate}
\usepackage{verbatim}
\usepackage[draft=false, colorlinks=true]{hyperref}
\usepackage{pdfpages}
\usepackage[margin=1.2in]{geometry}
\usepackage{IEEEtrantools}

\usepackage{fancyhdr}


\usepackage[nameinlink]{cleveref}


\DeclareMathOperator*{\ac}{accept}
\DeclareMathOperator*{\amax}{argmax}
\DeclareMathOperator*{\amin}{argmin}
\DeclareMathOperator*{\Aut}{Aut}
\newcommand {\al}{{\alpha}}
\newcommand {\abs}[1]{{\left\lvert#1\right\rvert}}
\newcommand {\A}{{\mathcal{A}}}
\newcommand {\AM}{{\mathrm{AM}}}
\newcommand {\AMp}{{\AM_{p}^{X}\!(\Ri_\w)}}
\newcommand {\B}{{\mathcal{B}}}
\DeclareMathOperator*{\Be}{Bern}
\newcommand {\Br}{{\dot{B}}}
\newcommand {\Ba}{{\mathfrak{B}}}
\newcommand {\C}{{\mathbb C}}
\newcommand {\ce}{\mathrm{c}}
\newcommand {\Ce}{\mathrm{C}}
\newcommand {\Cc}{\mathrm{C_{c}}}
\newcommand {\Ccinf}{\mathrm{C_{c}^{\infty}}}
\DeclareMathOperator{\cov}{Cov}
\DeclareMathOperator{\DEV}{DEV}
\newcommand {\Di}{{\mathbb D}}
\newcommand {\dom}{\mathrm{dom}}
\newcommand{\dist}{\stackrel{\mathrm{dist}}{=}}
\newcommand {\ud}{\mathrm{d}}
\newcommand {\ue}{\mathrm{e}}
\newcommand {\eps}{\varepsilon}
\newcommand {\veps}{\varepsilon}
\newcommand {\vrho}{{\varrho}}
\newcommand {\E}{{\mathbb{E}}}
\newcommand {\Ec}{{\mathcal{E}}}
\newcommand {\Ell}{L}
\newcommand {\Ellp}{{L_{p}[0,1]}}
\newcommand {\Ellpprime}{{L_{p'}([0,1])}}
\newcommand {\Ellq}{{L_{q}([0,1])}}
\newcommand {\Ellqprime}{{L_{q'}([0,1])}}
\newcommand {\Ellr}{L^{r}}
\newcommand {\Ellone}{{L_{1}([0,1])}}
\newcommand{\Elltwo}{{L_{2}([0,1])}}
\newcommand{\Ellinfty}{L^{\infty}}
\newcommand{\Ellinftyc}{L_{\mathrm{c}}^{\infty}}
\newcommand{\exb}[1]{\exp\left\{#1\right\}}
\DeclareMathOperator*{\Ext}{Ext}
\newcommand{\F}{{\mathcal{F}}}
\newcommand{\Fe}{{\mathbb{F}}}
\newcommand{\G}{{\mathcal{G}}}
\newcommand{\HF}{\mathcal{H}_{\text{FIO}}^{1}(\Rd)}
\newcommand{\Hr}{H}
\newcommand{\HT}{\mathcal{H}}
\newcommand{\ui}{\mathrm{i}}
\newcommand{\I}{{I}}
\newcommand{\J}{{\mathcal{J}}}
\newcommand{\id}{{\mathrm{id}}}
\newcommand{\iid}{\stackrel{\mathclap{\normalfont\mbox{iid}}}{\sim}}
\newcommand{\im}{{\text{im }}}
\newcommand{\ind}{{\perp\!\!\!\perp}}
\DeclareMathOperator*{\Int}{int}
\newcommand{\intx}{{\overline{\int_{X}}}}
\newcommand{\inte}{{\overline{\int_{\E}}}}
\newcommand{\la}{\lambda}
\newcommand{\rb}{\rangle}
\newcommand{\lb}{{\langle}}
\newcommand{\La}{\Lambda}
\newcommand{\calL}{{\mathcal{L}}}
\newcommand{\lp}{{\mathcal{L}}^{p}}
\newcommand{\lpo}{{\overline{\mathcal{L}}^{p}\!}}
\newcommand{\Lpo}{{\overline{\Ell}^{p}\!}}
\newcommand{\M}{{\mathbf{M}}}
\newcommand{\Ma}{{\mathcal{M}}}
\newcommand{\N}{{{\mathbb N}}}
\newcommand{\Na}{{{\mathcal{N}}}}
\newcommand{\norm}[1]{\left\|#1\right\|}
\newcommand{\normm}[1]{{\left\vert\kern-0.25ex\left\vert\kern-0.25ex\left\vert #1 
    \right\vert\kern-0.25ex\right\vert\kern-0.25ex\right\vert}}
\newcommand{\Om}{{{\Omega}}}
\newcommand{\one}{{{\bf 1}}}
\newcommand{\pic}{\text{Pic }}
\newcommand{\ph}{{\varphi}}
\newcommand{\Pa}{{\mathbb{P}}}
\newcommand{\Po}{{\mathcal{P}}}
\newcommand{\Q}{{\mathbb{Q}}}
\newcommand{\R}{{\mathbb R}}
\newcommand{\Rd}{{\mathbb{R}^{d}}}
\DeclareMathOperator{\rej}{reject }
\newcommand{\Rn}{{\mathbb{R}^{n}}}
\newcommand{\cR}{{\mathcal{R}}}
\newcommand{\Rad}{{\mathrm{Rad}}}
\newcommand{\ran}{{\mathrm{ran}}}
\newcommand{\Ri}{{\mathrm{R}}}
\newcommand{\supp}{{\mathrm{supp}}}
\newcommand{\Se}{\mathrm{S}}
\newcommand{\Sp}{S^{*}(\Rn)}
\newcommand{\St}{{\mathrm{St}}}
\newcommand{\Sw}{\mathcal{S}}
\newcommand{\T}{{\mathcal{T}}}
\newcommand{\ta}{{\theta}}
\newcommand{\Ta}{{\Theta}}
\newcommand{\topp}{\stackrel{p}{\to}}
\newcommand{\todd}{\stackrel{d}{\to}}
\newcommand{\toL}[1]{\stackrel{L^{#1}}{\to}} 
\newcommand{\toas}{\stackrel{a.s.}{\to}}
\DeclareMathOperator{\V}{Var}
\newcommand {\w}{{\omega}}
\newcommand {\W}{{\mathrm{W}}}
\newcommand {\Wnp}{\text{$\mathrm{W}$\textsuperscript{$n,\!p$}}}
\newcommand {\Wnpeq}{\text{$\mathrm{W}$\textsuperscript{$n\!,\!p$}}}
\newcommand {\Wonep}{\text{$\mathrm{W}$\textsuperscript{$1,\!p$}}}
\newcommand {\Wonepeq}{\text{$\mathrm{W}$\textsuperscript{$1\!,\!p$}}}
\newcommand {\X}{{\mathcal{X}}}
\newcommand {\Z}{{{\mathbb Z}}}
\newcommand {\Za}{{\mathcal{Z}}}
\newcommand {\Zd}{{\Z[\sqrt{d}]}}
\newcommand {\vanish}[1]{\relax}

\newcommand {\wh}{\widehat}
\newcommand {\wt}{\widetilde}
\newcommand {\red}{\color{red}}

% Distributions
\newcommand{\normal}{\mathsf{N}}
\newcommand{\poi}{\mathsf{Poisson}}
\newcommand{\bern}{\mathsf{Bernoulli}}
\newcommand{\bin}{\mathsf{Binomal}}
\newcommand{\multi}{\mathsf{Multinomial}}
\newcommand{\Exp}{\mathsf{Exp}}



% put your command and environment definitions here




% some theorem environments
% remove "[theorem]" if you do not want them to use the same number sequence


  \newtheorem{thrm}{Theorem}
  \newtheorem{lemma}{Lemma}
  \newtheorem{prop}{Proposition}
  \newtheorem{cor}{Corollary}

  \newtheorem{conj}{Conjecture}
  \renewcommand{\theconj}{\Alph{conj}}  % numbered A, B, C etc

  \theoremstyle{definition}
  \newtheorem{defn}{Definition}
  \newtheorem{ex}{Example}
  \newtheorem{exs}{Examples}
  \newtheorem{question}{Question}
  \newtheorem{remark}{Remark}
  \newtheorem{notn}{Notation}
  \newtheorem{exer}{Exercise}




\title{STATS310A - Lecture 8}
\author{Persi Diaconis\\ Scribed by Michael Howes}
\date{10/14/21}

\pagestyle{fancy}
\fancyhf{}
\rhead{STATS310A - Lecture 8}
\lhead{10/14/21}
\rfoot{Page \thepage}

\begin{document}
\maketitle
\tableofcontents
\section{Integration}
\subsection{Definition}
Let $(\Om, \F, \mu)$ be a measure space and let $f: \Om \to [-\infty, \infty]$ be measurable. We wish to define
\[\int f d\mu = \int_\Om f d\mu = \int_\Om f(\w) \mu(d\w). \]
\begin{defn}
    The function $f$ is a \emph{simple function} if there exists a finite partition of $\Om$ $\{A_i\}_{i=1}^n$ such that $A_i \in \F$ and values $x_i \in [-\infty,\infty]$ such that 
    \[f(\w) = \sum_{i=1}^n x_i \delta_{A_i}(\w), \]
    where we use the conventions $0\cdot \infty = \infty \cdot 0 = 0$ and $x \cdot\infty = \infty \cdot x= \infty$ if $x\in(0,\infty]$.
\end{defn}
\begin{lemma}
    If $f \ge 0$, then there exists a sequence of simple functions $f_n \ge 0$ such that $f_n(\w)\nearrow f(\w)$ for all $\w \in \Om$.
\end{lemma}
\begin{proof}
    Fix $n$ and define
    \[f_n(\w) = \begin{cases}
        \frac{k-1}{2^n} & \text{if } \frac{k-1}{2^n}\le f(\w) < \frac{k}{2^n} \text{ for some } k =1,\ldots, n2^n,\\
        n & \text{if } f(\w) \ge n.
    \end{cases} \qedhere\]
\end{proof}
\begin{defn}
    Suppose $f(\w) \ge 0$ for all $\w$. Define
    \[\int_\Om f(\w)\mu(d\w) = \sup\left(\sum_{i=1}^n \nu_i \mu(A_i)\right), \]
    where the supremum is over all measurable partitions of $\Om$ and $\nu_i = \inf_{\w \in A_i} f(\w)$. For a general $f$ define
    \[f_+(\w) = \max\{f(\w),0\} \text{ and } f_-(\w) = \max\{-f(\w),0\}. \]
    Note that $f_+, f_-\ge 0$ and $\abs{f} = f_+ +f_-$ and $f = f_+ -f_-$. We define $\int f d\mu$ based on cases.
    \begin{itemize}
        \item If $\int f_+ d\mu < \infty$ and,$\int f_- d\mu < \infty$, then define $\int f = \int f_+d\mu - \int f_-d\mu$. 
        \item If $\int f_+ d\mu = \infty$ and $\int f_- d\mu < \infty$, then define $\int fd\mu = \infty$.
        \item If $\int f_+ d\mu < \infty$ and $\int f_- d\mu = \infty$, then define $\int fd\mu = - \infty$.
        \item If $\int f_+ d\mu = \infty$ and $\int f_- d\mu = \infty$, then $\int f d\mu$ is not defined. 
    \end{itemize}

\end{defn}
\subsection{Properties}
\begin{prop}
    Suppose $f,g,f_n \ge 0$.
    \begin{enumerate}
        \item If $f(\w) = \sum_{i=1}^n x_i \delta_{A_i}(\w)$ is simple, then $\int f d\mu = \sum_{i=1}^n x_i \mu(A_i)$.
        \item If $f(\w) \le g(\w)$ for all $\w$, then $\int fd\mu \le \int g d\mu$.
        \item If $f_n(\w) \nearrow f(\w)$ for all $n$ and $\w$, then $\int f_n d\mu \nearrow \int f d\mu$. (Monotone convergence theorem).
        \item If $\al,\beta \ge 0$, then $\int \al f + \beta g d\mu = \al \int f d\mu + \beta \int g d\mu$.
    \end{enumerate}
\end{prop}
\begin{proof}
    \begin{enumerate}
        \item Fix a partition $\{B_i\}_{i=1}^m$ and let $\beta_j = \inf\limits_{\w \in B_j}f(\w)$. If $\w \in A_I \cap B_j$, then $\beta_j \le f(\w) = x_i$. Thus
        \begin{align*}
            \sum_{j=1}^m \beta_j \mu(B_j) &= \sum_{j=1}^m \beta_j \sum_{i=1}^n \mu(A_i \cap B_j)\\
            &= \sum_{i=1}^n \sum_{j=1}^m \beta_j \mu(A_i \cap B_j)\\
            &\le \sum_{i=1}^n \sum_{j=1}^m x_i \mu(A_i \cap B_j)\\
            &=\sum_{i=1}^n x_i \sum_{j=1}^m \mu(A_i \cap B_j)\\
            &= \sum_{i=1}^n x_i\mu(A_i).
        \end{align*}
        Thus $\int f d\mu \le \sum_{i=1}^n x_i \mu(A_i)$. The other direction we get for free since $\{A_i\}_{i=1}^n$ is a partition and $f$ equals $x_i$ on $A_i$.
        \item This follows from the definition since 
        \[\inf\{f(\w): \w \in A_i \} \le \inf \{g(\w) : \w \in A_i\}, \]
        for any measurable set $A_i$.
        \item By (b) we know that $\int f_n d\mu$ is an increasing sequence and that $\int f_n d\mu$ is bounded above by $\int f d\mu$. Thus $\lim_n \int f_n d\mu$ exists and $\lim_n \int f_nd\mu \le \int f d\mu$. It remains to prove that $\int f d\mu \le \lim_n \int f_n d\mu$. Thus we must show that for every partition $\{A_i\}_{i=1}^m$, we have 
        \[\sum_{i=1}^m \nu_i \mu(A_i) \le\lim_n\int f_nd\mu,\]
        where $\nu_i = \inf\{f(\w): \w \in A_i\}$. Let $S = \sum_{i=1}^m \nu_i \mu(A_i)$. We will consider different cases.
        \begin{enumerate}
            \item Supppose that $S$ is finite and $0 < \nu_i < \infty$ and $0 < \mu(A_i)< \infty$ for all $i$. Choose $\varepsilon$ such that $\eps< \mu(A_i)$ for all $i$. Define
            \[ A_{n,i} =\{\w \in A_i : f_n(\w) \ge \nu_i - \eps\}.\]
            Since $f_n \nearrow f$, we know that $A_{n,i} \nearrow A_i$ as $n \to \infty$. Thus $\mu(A_{n,i}) \nearrow \mu(A_i)$. Note that
            \[\int f_n d\mu \ge \sum_{i=1}^m (\nu_i-\eps)\mu(A_{n,i}). \]
            Thus 
            \[\lim_n \int f_n d\mu \ge \sum_{i=1}^n (\nu_i-\eps)\mu(A_i) = \sum_{i=1}^m \nu_i \mu(A_i) - \eps\sum_{i=1}^m\mu(A_i). \]
            Letting $\eps \to 0$, we see that $\lim_n \int f_n d\mu \ge \sum_{i=1}^n \nu_i\mu(A_i)=S$. Thus $\lim_n \int f_n d\mu \ge \int f d\mu$.
            \item Now suppose that $S<\infty$ and $\nu_i$ or $\mu(A_i)$ is equal to 0 or $\infty$ for some $i$. By reording we have $0< \nu_i, \mu(A_i) < \infty$ for $i = 1,\ldots,i_0$ and the rest of the terms are of the form $0\cdot \infty$, $\infty \cdot 0$ or $0\cdot 0$. We can then apply case 1 to the partition $\{A_i\}_{i=1}^{i_0} \cup \left\{\left(\bigcup_{i=1}^{i_0} A_i\right)^c\right\}$.
            \item Suppose that $S=\infty$. Then for some $i_0$ we have $\nu_{i_0}\mu(A_{i_0}) = \infty$. Thus either $\nu_{i_0} = \infty$ and $\mu(A_{i_0})>0$ or $\nu_{i_0} > 0$ and $\mu(A_{i_0}) = \infty$. Choose $x,y >0$ so that $0 < x< \nu_{i_0}$ and $0 < y < \mu(A_{i_0})$. Let $B_n= \{\w: f_n(\w) >x\}$. Then $B_n \nearrow B = \{\w: f(w)\ge \w\}$. So $\mu(B_n) \nearrow \mu(B) \ge \mu(A_{i_0}) \ge y$. Thus by using the parition $\{B_n,B_n^c\}$, we see that 
            \[\int f_n d\mu \ge x\cdot \mu(B_n). \]
            Thus $\lim_n \int f_n d\mu \ge xy$. The product $xy$ can be arbitrary large and thus $\int f_n d\mu \ge \infty$.
        \end{enumerate}
        \item Suppose $f$ and $g$ are both simple functions $f = \sum_{i=1}^n x_i \delta_{A_i}$ and $g = \sum_{i=1}^m y_i \delta_{B_i}$. Then
        \[\al f + \beta g = \sum_{i=1}^n \sum_{j=1}^m (\al x_i + \beta y_j) \delta_{A_i \cap B_j}. \]
        Thus
        \begin{align*}
            \int \al f + \beta g d\mu &= \sum_{i=1}^n \sum_{j=1}^m (\al x_i + \beta y_j) \mu(A_i \cap B_j)\\
        &= \al \sum_{i=1}^n x_i \sum_{j=1}^m \mu(A_i \cap B_j) + \beta \sum_{j=1}^m y_j \sum_{i=1}^n \mu(A_i \cap B_j)\\
        &= \al \sum_{i=1}^n x_i \mu(A_i) + \beta \sum_{j=1}^m y_j \mu(B_j)\\
        &= \al \int f d\mu + \beta \int g d\mu.
        \end{align*}
        Now for general $f$ and $g$, let $f_n \nearrow f$ and $g_n \nearrow g$ be approximating sequences of simple functions. Then $(\al f_n + \beta g_n) \nearrow \al f + \beta g$. Thus 
        \begin{align*}
            \int \al f+\beta g d \mu &= \int \al f_n + \beta g_n d \mu \\
            &= \al \int f_n d \mu + \beta \int g_n d \mu \\
            &= \al\int fd \mu + \beta\int g d \mu. \qedhere
        \end{align*}
    \end{enumerate}
\end{proof}
\subsection{Remarks}
\begin{enumerate}
    \item  If we have a  continuous function $f : [a,b] \to \R$ and $\mu$ is Lebesgue measure on $[a,b]$, then $\int f d\mu$ agrees with the Riemann integral.
    \item In the same setting as (a), the function $f = \delta_{\Q \cap [a,b]}$ is not Riemann integrable but $\int fd\mu$ does exist and $\int f d\mu = 0$.
    \item We still need Riemann integrals for improper integrals. For example $\int_0^\infty \frac{\sin(x)}{x}dx$ is not Lebesgue integrable but it does have a convergent improper Riemann integral. 
    \item Riemann integration is also needed for calculations, for Brownian motion and many other things.
    \item The Henstock integral generalises both the Riemann and Lebesgue integrals.
\end{enumerate}
\begin{ex}
    In general $f_n \to f$, does not imply $\int f_n d \mu \to \int f d \mu$. Consider
    \[f_n(\w) = \begin{cases}
        n^2 & \text{if } \w \in (0, 1/n)\\
        0 & \text{else.}
    \end{cases} \]
    Then $f_n(\w) \to 0$ for all $\w$ but $\int f_n d\mu = n^2 \cdot \frac{1}{n} = n \nearrow \infty$.
\end{ex}
The properties (a)-(d) hold if the hyptheses hold \emph{almost surely}. For example for (b) we can prove
\[\mu(\{\w : f(\w) > g(\w)\}) =0 \Longrightarrow \int f d\mu \le \int g d\mu. \]
\begin{proof}
    Let $G = \{\w : f(\w) \le g(\w)\}$. By hypothesis $\mu(G^c) = 0$. For every partition $\{A_i\}_{i=1}^m$, $\mu(A_i) = \mu(A_i \cap G)$. It follows that
    \begin{align*}
        \sum_{i=1}^n \inf_{A_i} f(\w) \mu(A_i) &= \sum_{i=1}^m \inf_{A_i} f(\w) \mu(A_i\cap G)\\
        &\le \sum_{i=1}^m \inf_{A_i\cap G} f(\w) \mu(A_i\cap G)\\
        &\le \sum_{i=1}^m \inf_{A_i \cap G} g(\w) \mu(A_i\cap G)\\
        &=\sum_{i=1}^m \inf_{A_i \cap G} g(\w) \mu(A_i\cap G) + \inf_{G^c}g(\w) \mu(G^c)\\
        &\le \int g d\mu. \qedhere
    \end{align*}
\end{proof}
\begin{defn}
    If $(\Om, \F, \mu)$ is a probability space and $X:(\Om,F) \to \R$ is a random variable, then we call $\int f d\mu$ \emph{the expectation of $f$} and we write
    \[\E[X] := \int X d\mu. \]
\end{defn}
\end{document}
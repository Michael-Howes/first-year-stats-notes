\documentclass{article}
\usepackage{ae,aecompl}
\usepackage{todonotes}
\usepackage{chngcntr}
\usepackage{tikz-cd}
\usepackage{graphicx}
\graphicspath{ {./images/}}
\usepackage[all,cmtip]{xy}
\usepackage{amsmath, amscd}
\usepackage{amsthm}
\usepackage{amssymb}
\usepackage{amsfonts}
\usepackage{bm}
\usepackage{qsymbols}
\usepackage{latexsym}
\usepackage{mathrsfs}
\usepackage{mathtools}
\usepackage{cite}
\usepackage{color}
\usepackage{url}
\usepackage{enumerate}
\usepackage{verbatim}
\usepackage[draft=false, colorlinks=true]{hyperref}
\usepackage{pdfpages}
\usepackage[margin=1.2in]{geometry}
\usepackage{IEEEtrantools}

\usepackage{fancyhdr}


\usepackage[nameinlink]{cleveref}


\DeclareMathOperator*{\ac}{accept}
\DeclareMathOperator*{\amax}{argmax}
\DeclareMathOperator*{\amin}{argmin}
\DeclareMathOperator*{\Aut}{Aut}
\newcommand {\al}{{\alpha}}
\newcommand {\abs}[1]{{\left\lvert#1\right\rvert}}
\newcommand {\A}{{\mathcal{A}}}
\newcommand {\AM}{{\mathrm{AM}}}
\newcommand {\AMp}{{\AM_{p}^{X}\!(\Ri_\w)}}
\newcommand {\B}{{\mathcal{B}}}
\DeclareMathOperator*{\Be}{Bern}
\newcommand {\Br}{{\dot{B}}}
\newcommand {\Ba}{{\mathfrak{B}}}
\newcommand {\C}{{\mathbb C}}
\newcommand {\ce}{\mathrm{c}}
\newcommand {\Ce}{\mathrm{C}}
\newcommand {\Cc}{\mathrm{C_{c}}}
\newcommand {\Ccinf}{\mathrm{C_{c}^{\infty}}}
\DeclareMathOperator{\cov}{Cov}
\DeclareMathOperator{\DEV}{DEV}
\newcommand {\Di}{{\mathbb D}}
\newcommand {\dom}{\mathrm{dom}}
\newcommand{\dist}{\stackrel{\mathrm{dist}}{=}}
\newcommand {\ud}{\mathrm{d}}
\newcommand {\ue}{\mathrm{e}}
\newcommand {\eps}{\varepsilon}
\newcommand {\veps}{\varepsilon}
\newcommand {\vrho}{{\varrho}}
\newcommand {\E}{{\mathbb{E}}}
\newcommand {\Ec}{{\mathcal{E}}}
\newcommand {\Ell}{L}
\newcommand {\Ellp}{{L_{p}[0,1]}}
\newcommand {\Ellpprime}{{L_{p'}([0,1])}}
\newcommand {\Ellq}{{L_{q}([0,1])}}
\newcommand {\Ellqprime}{{L_{q'}([0,1])}}
\newcommand {\Ellr}{L^{r}}
\newcommand {\Ellone}{{L_{1}([0,1])}}
\newcommand{\Elltwo}{{L_{2}([0,1])}}
\newcommand{\Ellinfty}{L^{\infty}}
\newcommand{\Ellinftyc}{L_{\mathrm{c}}^{\infty}}
\newcommand{\exb}[1]{\exp\left\{#1\right\}}
\DeclareMathOperator*{\Ext}{Ext}
\newcommand{\F}{{\mathcal{F}}}
\newcommand{\Fe}{{\mathbb{F}}}
\newcommand{\G}{{\mathcal{G}}}
\newcommand{\HF}{\mathcal{H}_{\text{FIO}}^{1}(\Rd)}
\newcommand{\Hr}{H}
\newcommand{\HT}{\mathcal{H}}
\newcommand{\ui}{\mathrm{i}}
\newcommand{\I}{{I}}
\newcommand{\J}{{\mathcal{J}}}
\newcommand{\id}{{\mathrm{id}}}
\newcommand{\iid}{\stackrel{\mathclap{\normalfont\mbox{iid}}}{\sim}}
\newcommand{\im}{{\text{im }}}
\newcommand{\ind}{{\perp\!\!\!\perp}}
\DeclareMathOperator*{\Int}{int}
\newcommand{\intx}{{\overline{\int_{X}}}}
\newcommand{\inte}{{\overline{\int_{\E}}}}
\newcommand{\la}{\lambda}
\newcommand{\rb}{\rangle}
\newcommand{\lb}{{\langle}}
\newcommand{\La}{\Lambda}
\newcommand{\calL}{{\mathcal{L}}}
\newcommand{\lp}{{\mathcal{L}}^{p}}
\newcommand{\lpo}{{\overline{\mathcal{L}}^{p}\!}}
\newcommand{\Lpo}{{\overline{\Ell}^{p}\!}}
\newcommand{\M}{{\mathbf{M}}}
\newcommand{\Ma}{{\mathcal{M}}}
\newcommand{\N}{{{\mathbb N}}}
\newcommand{\Na}{{{\mathcal{N}}}}
\newcommand{\norm}[1]{\left\|#1\right\|}
\newcommand{\normm}[1]{{\left\vert\kern-0.25ex\left\vert\kern-0.25ex\left\vert #1 
    \right\vert\kern-0.25ex\right\vert\kern-0.25ex\right\vert}}
\newcommand{\Om}{{{\Omega}}}
\newcommand{\one}{{{\bf 1}}}
\newcommand{\pic}{\text{Pic }}
\newcommand{\ph}{{\varphi}}
\newcommand{\Pa}{{\mathbb{P}}}
\newcommand{\Po}{{\mathcal{P}}}
\newcommand{\Q}{{\mathbb{Q}}}
\newcommand{\R}{{\mathbb R}}
\newcommand{\Rd}{{\mathbb{R}^{d}}}
\DeclareMathOperator{\rej}{reject }
\newcommand{\Rn}{{\mathbb{R}^{n}}}
\newcommand{\cR}{{\mathcal{R}}}
\newcommand{\Rad}{{\mathrm{Rad}}}
\newcommand{\ran}{{\mathrm{ran}}}
\newcommand{\Ri}{{\mathrm{R}}}
\newcommand{\supp}{{\mathrm{supp}}}
\newcommand{\Se}{\mathrm{S}}
\newcommand{\Sp}{S^{*}(\Rn)}
\newcommand{\St}{{\mathrm{St}}}
\newcommand{\Sw}{\mathcal{S}}
\newcommand{\T}{{\mathcal{T}}}
\newcommand{\ta}{{\theta}}
\newcommand{\Ta}{{\Theta}}
\newcommand{\topp}{\stackrel{p}{\to}}
\newcommand{\todd}{\stackrel{d}{\to}}
\newcommand{\toL}[1]{\stackrel{L^{#1}}{\to}} 
\newcommand{\toas}{\stackrel{a.s.}{\to}}
\DeclareMathOperator{\V}{Var}
\newcommand {\w}{{\omega}}
\newcommand {\W}{{\mathrm{W}}}
\newcommand {\Wnp}{\text{$\mathrm{W}$\textsuperscript{$n,\!p$}}}
\newcommand {\Wnpeq}{\text{$\mathrm{W}$\textsuperscript{$n\!,\!p$}}}
\newcommand {\Wonep}{\text{$\mathrm{W}$\textsuperscript{$1,\!p$}}}
\newcommand {\Wonepeq}{\text{$\mathrm{W}$\textsuperscript{$1\!,\!p$}}}
\newcommand {\X}{{\mathcal{X}}}
\newcommand {\Z}{{{\mathbb Z}}}
\newcommand {\Za}{{\mathcal{Z}}}
\newcommand {\Zd}{{\Z[\sqrt{d}]}}
\newcommand {\vanish}[1]{\relax}

\newcommand {\wh}{\widehat}
\newcommand {\wt}{\widetilde}
\newcommand {\red}{\color{red}}

% Distributions
\newcommand{\normal}{\mathsf{N}}
\newcommand{\poi}{\mathsf{Poisson}}
\newcommand{\bern}{\mathsf{Bernoulli}}
\newcommand{\bin}{\mathsf{Binomal}}
\newcommand{\multi}{\mathsf{Multinomial}}
\newcommand{\Exp}{\mathsf{Exp}}



% put your command and environment definitions here




% some theorem environments
% remove "[theorem]" if you do not want them to use the same number sequence


  \newtheorem{thrm}{Theorem}
  \newtheorem{lemma}{Lemma}
  \newtheorem{prop}{Proposition}
  \newtheorem{cor}{Corollary}

  \newtheorem{conj}{Conjecture}
  \renewcommand{\theconj}{\Alph{conj}}  % numbered A, B, C etc

  \theoremstyle{definition}
  \newtheorem{defn}{Definition}
  \newtheorem{ex}{Example}
  \newtheorem{exs}{Examples}
  \newtheorem{question}{Question}
  \newtheorem{remark}{Remark}
  \newtheorem{notn}{Notation}
  \newtheorem{exer}{Exercise}




\title{STATS310A - Lecture 12}
\author{Persi Diaconis\\ Scribed by Michael Howes}
\date{10/28/21}

\pagestyle{fancy}
\fancyhf{}
\rhead{STATS310A - Lecture 12}
\lhead{10/28/21}
\rfoot{Page \thepage}

\begin{document}
\maketitle
\tableofcontents
\section{Kolmogorov's strong law of large numbers}
Today we'll do some serious probability using many of our tools from measure theory.
\begin{thrm}
    \emph{[Kolmogorov]} Let $X_1,X_2,\ldots$ be independent and identically distributed (i.i.d.) with finite mean $\mu = \E[X_1]$. Define 
    \[S_n = \sum_{i=1}^n X_i, \]
    then $\frac{1}{n}S_n \to \mu$ almost surely.
\end{thrm}
\subsection{Proof}
\begin{proof}
    Write $X_i = X_i^+-X_i^-$ and define $\mu^{\pm} = \E[X_1^\pm]$. Then $\mu = \mu^+ - \mu^-$ and so we can assume without loss of generality that $X_i \ge 0$.

    Our next idea is to use \underline{truncation}. Define
    \[Y_i = X_i \delta_{\{X_i \le i\}}, \]
    and let $T_n = \sum_{i=1}^n Y_i$. Our next idea is to look at \underline{subsequences}. Fix $\al >0$ and define $\lfloor \al^n \rfloor$. We now show that for all $\eps > 0$,
    \[\sum_{n=1}^\infty \Pa\left(\abs{\frac{T_{u_n}-\E[T_{u_n}]}{u_n}}>\eps \right)< \infty. \] 
    Note that by the i.i.d. assumption,
    \begin{align*}
        \V(T_{u_n})&=\sum_{i=1}^{u_n}\V(Y_i)\\
        &\le \sum_{i=1}^{u_n}\E[Y_i^2]\\
        &\le \sum_{i=1}^{u_n}\E[X_i^2\delta_{\{X_i \le i\}}]\\
        &=\sum_{i=1}^{u_n}\E[X_1^2\delta_{\{X_1 \le i\}}]\\
        &\le u_n \E[X_1^2 \delta_{\{X_1\le u_n\}}].
    \end{align*}
    By \underline{Chebyshev's inequality} 
    \[\Pa\left(\abs{\frac{T_{u_n}-\E[T_{u_n}]}{u_n}}>\eps \right) \le \frac{\V(T_{u_n})}{u_n^2\eps^2}. \]
    Thus 
    \begin{align*}
        \sum_{n=1}^\infty \Pa\left(\abs{\frac{T_{u_n}-\E[T_{u_n}]}{u_n}}>\eps \right)&\le \sum_{n=1}^\infty  \frac{\V(T_{u_n})}{u_n^2\eps^2}\\
        &\le \frac{1}{\eps^2}\sum_{n=1}^\infty \frac{\E[X_1^2\delta_{\{X_1\le u_n\}}]u_n}{u_n^2}\\
        & =  \frac{1}{\eps^2}\sum_{n=1}^\infty \frac{\E[X_1^2\delta_{\{X_1\le u_n\}}]}{u_n}\\
        &=\frac{1}{\eps^2}\E\left[X_1^2 \sum_{n=1}^\infty \frac{1}{u_n}\delta_{\{X_1\le u_n\}}\right].
    \end{align*}
    We will now show that there exists $k$ such that if $x > 0$, then 
    \[\sum_{u_n \ge x}\frac{1}{u_n} \le \frac{k}{x}. \]
    Let $n_x$ be the smallest integer such that $u_{n_x} \ge x$. One can then show that $\al^{n_x} \ge x$. It follows that
    \begin{align*}
        \sum_{u_n \ge x} \frac{1}{u_n} &\le 2 \sum_{n \ge n_x} \al^{-n}\\
        &= \frac{2\al^{-n_x}}{1-\al}\\
        &= \frac{2}{1-\al} \times \frac{1}{\al^{n_x}}\\
        &\le \frac{2}{1-\al}\times \frac{1}{x}. 
    \end{align*}
    Note that if $X_1 > 0$, then 
    \[\sum_{n=1}^\infty \frac{1}{u_n} \delta_{\{X_1 \le u_n\}} = \sum_{u_n \ge X_1} \frac{1}{u_n} \le \frac{k}{X_1}. \]
    Since we have the convention $0 \cdot \infty = 0$, we can conclude that 
    \[\frac{1}{\eps^2} \E\left[X_1^2 \sum_{n=1}^\infty \frac{1}{u_n}\delta_{\{X_1\le u_n \}}\right] \le \frac{k}{\eps^2}\E[X_1]. \]
    Thus Borel-Cantelli gives 
    \[\Pa\left(\abs{\frac{T_{u_n} -\E[T_{u_n}]}{u_n}} > \eps,~ i.o.\right) =0. \]
    If $(x_i)$ is any sequence of real numbers such that $x_i \to x$, then $\frac{1}{n}\sum_{i=1}^n x_i \to x$. By the monotone convergence theorem: 
    \[\E[Y_n] = \E[X_1\delta_{\{X_1 \ge n\}}] \nearrow \E[X_1] =\mu. \]
    Thus 
    \[\E\left[\frac{T_{u_n}}{u_n}\right] =\frac{1}{u_n}\sum_{i=1}^{u_n}\E[Y_i] \nearrow \mu. \]
    Thus $\frac{T_{u_n}}{u_n} \to \mu$ with probability 1.

    Thus we have proved our result on subsequences of the form $u_n = \lfloor \al^n \rfloor$ for truncated versions of $X_i$. We will first remove the truncation. Note that
    \begin{align*}
        \sum_{i=1}^\infty \Pa(X_i \neq Y_i) &= \sum_{i=1}^\infty \Pa(X_i \ge i)\\
        &= \sum_{i=1}^\infty \Pa(X_1 \ge i)\\
        &\le \int_0^\infty \Pa(X_1 \ge t)dt\\
        &=\E[X_1]\\
        &< \infty.
    \end{align*}
    Thus Borel-Cantelli implies that $X_i = Y_i$ almost surely for sufficiently large $i$. Thus we can conclude $\frac{S_{u_n}}{u_n} \to \mu$ almost surely.

    We will now use \underline{interpolation}. We have $\frac{S_{u_n}}{u_n} \to \mu$ and we wish to conclude $\frac{S_k}{k} \to \mu$. For fixed $k$, let $n$ be such that $u_n \le k \le u_{n+1}$. Then 
    \[\frac{S_{u_n}}{u_n}\cdot \frac{u_n}{u_{n+1}} \le \frac{S_k}{k} \le \frac{S_{u_{n+1}}}{u_{n+1}}\cdot \frac{u_{n+1}}{u_n}. \]
    We also have $\frac{u_n}{u_{n+1}} \to \frac{1}{\al}$ and $\frac{u_{n+1}}{u_n} \to \al$. Thus we have 
    \[\liminf_k \frac{S_k}{k} \ge \frac{1}{\al} \mu ~\text{ almost surely}, \]
    and 
    \[\limsup_k \frac{S_k}{k} \le \al \mu ~\text{ almost surely,} \]
    for all $\al > 1$. Let $(\al_j)$ be a sequence such tht $\al_j \to 1$ and $\al_j > 1$. We then have 
    \[\frac{1}{\al_j}\limsup_k \frac{S_k}{k} \le \mu \le \al_j \liminf_k \frac{S_k}{k} ~\text{ almost surely}. \]
    Thus $\liminf_k \frac{S_k}{k} = \limsup_k \frac{S_k}{k} = \mu$, almost surely.
\end{proof}
\subsection{The four T's}
This was an example of using the four T's:
\begin{itemize}
    \item Truncation,
    \item Tschebyscheff,
    \item Tsubsequences and
    \item InTerpolation.
\end{itemize}
All four of these are very useful ideas.
\section{Remarks and stories}
\subsection{Remarks}
\begin{enumerate}
    \item We showed that if $X_i$ are i.i.d., then 
    \[ \frac{S_n}{n} \to \mu = \E[X_1] < \infty.\]
    If $X_i \sim \text{Cauchy}$ i.e. $\Pa(X_n \le x) = \int_{-\infty}^x \frac{1}{\pi(1+x^2)}dx$, then 
    \[ \frac{X_1+\ldots+X_n}{n} \sim \text{Cauchy}.\]
    Thus we need $\E[X_1]<\infty$ for the strong law of large numbers.
    \item Independence:
    \begin{enumerate}
        \item Our proof works if we weaken the hypothesis from $(X_i)$ independent ot $(X_i)$ pairwise independent.
        \item We really only need that they are not too dependent. For example suppose $Y_i$ are i.i.d. with $\E[\abs{Y_i}]<\infty$. If we define $X_1 = Y_1+Y_2$, $X_2 = Y_2+Y_3, \ldots$, then $(X_i)$ are not independent but they still satisfy a strong law of large numbers because a sum of $X_i$'s is essentially a sum of $Y_i$'s.
        \item Markov chains, martingales, stationary processes are all dependent and all have strong laws.
    \end{enumerate}
    \item Identically distributied? If none of the random variables are too wild, then the strong law is okay. For example we're safe if 
    \[\sum_{i=1}^\infty \frac{\V(X_i)}{i^2} < \infty.\]
    We are not safe if $\V(S_n)$ is dominated by $\V(X_n)$. This might be the case if we are averaging stock prices or some other type of random variables which are sometimes very volatile.  
\end{enumerate}
\subsection{Chebyshev's master thesis}
In Chebyshev's master's thesis he proved that if $X_i \in \{0,1\}$ are independent and $\Pa(X_i = 1)=p_i$, then $\Pa\left(\abs{\frac{S_n}{n}-\frac{\sum_{i=1}^n p_i}{n}}>\eps\right)\to 0$ and he had to prove this without Chebyshev's inequality (that came later).
\subsection{Where's the beef?}
We proved:
\begin{center}
    \fbox{$\frac{S_n}{n}\to \mu ~a.s.$}
\end{center}
which is a very nice statement but it doesn't tell us anything about finite $n$. What we really want is
\[\Pa\left(\abs{\frac{S_n}{n}-\mu}> \eps \text{ for some } n \ge N\right) \le f(N,\eps), \]
for some function $f$. Such results are complicated and rare.
\end{document}
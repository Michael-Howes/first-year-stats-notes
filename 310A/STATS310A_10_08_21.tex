\documentclass{article}
\usepackage{ae,aecompl}
\usepackage{todonotes}
\usepackage{chngcntr}
\usepackage{tikz-cd}
\usepackage{graphicx}
\graphicspath{ {./images/}}
\usepackage[all,cmtip]{xy}
\usepackage{amsmath, amscd}
\usepackage{amsthm}
\usepackage{amssymb}
\usepackage{amsfonts}
\usepackage{bm}
\usepackage{qsymbols}
\usepackage{latexsym}
\usepackage{mathrsfs}
\usepackage{mathtools}
\usepackage{cite}
\usepackage{color}
\usepackage{url}
\usepackage{enumerate}
\usepackage{verbatim}
\usepackage[draft=false, colorlinks=true]{hyperref}
\usepackage{pdfpages}
\usepackage[margin=1.2in]{geometry}
\usepackage{IEEEtrantools}

\usepackage{fancyhdr}


\usepackage[nameinlink]{cleveref}


\DeclareMathOperator*{\ac}{accept}
\DeclareMathOperator*{\amax}{argmax}
\DeclareMathOperator*{\amin}{argmin}
\DeclareMathOperator*{\Aut}{Aut}
\newcommand {\al}{{\alpha}}
\newcommand {\abs}[1]{{\left\lvert#1\right\rvert}}
\newcommand {\A}{{\mathcal{A}}}
\newcommand {\AM}{{\mathrm{AM}}}
\newcommand {\AMp}{{\AM_{p}^{X}\!(\Ri_\w)}}
\newcommand {\B}{{\mathcal{B}}}
\DeclareMathOperator*{\Be}{Bern}
\newcommand {\Br}{{\dot{B}}}
\newcommand {\Ba}{{\mathfrak{B}}}
\newcommand {\C}{{\mathbb C}}
\newcommand {\ce}{\mathrm{c}}
\newcommand {\Ce}{\mathrm{C}}
\newcommand {\Cc}{\mathrm{C_{c}}}
\newcommand {\Ccinf}{\mathrm{C_{c}^{\infty}}}
\DeclareMathOperator{\cov}{Cov}
\DeclareMathOperator{\DEV}{DEV}
\newcommand {\Di}{{\mathbb D}}
\newcommand {\dom}{\mathrm{dom}}
\newcommand{\dist}{\stackrel{\mathrm{dist}}{=}}
\newcommand {\ud}{\mathrm{d}}
\newcommand {\ue}{\mathrm{e}}
\newcommand {\eps}{\varepsilon}
\newcommand {\veps}{\varepsilon}
\newcommand {\vrho}{{\varrho}}
\newcommand {\E}{{\mathbb{E}}}
\newcommand {\Ec}{{\mathcal{E}}}
\newcommand {\Ell}{L}
\newcommand {\Ellp}{{L_{p}[0,1]}}
\newcommand {\Ellpprime}{{L_{p'}([0,1])}}
\newcommand {\Ellq}{{L_{q}([0,1])}}
\newcommand {\Ellqprime}{{L_{q'}([0,1])}}
\newcommand {\Ellr}{L^{r}}
\newcommand {\Ellone}{{L_{1}([0,1])}}
\newcommand{\Elltwo}{{L_{2}([0,1])}}
\newcommand{\Ellinfty}{L^{\infty}}
\newcommand{\Ellinftyc}{L_{\mathrm{c}}^{\infty}}
\newcommand{\exb}[1]{\exp\left\{#1\right\}}
\DeclareMathOperator*{\Ext}{Ext}
\newcommand{\F}{{\mathcal{F}}}
\newcommand{\Fe}{{\mathbb{F}}}
\newcommand{\G}{{\mathcal{G}}}
\newcommand{\HF}{\mathcal{H}_{\text{FIO}}^{1}(\Rd)}
\newcommand{\Hr}{H}
\newcommand{\HT}{\mathcal{H}}
\newcommand{\ui}{\mathrm{i}}
\newcommand{\I}{{I}}
\newcommand{\J}{{\mathcal{J}}}
\newcommand{\id}{{\mathrm{id}}}
\newcommand{\iid}{\stackrel{\mathclap{\normalfont\mbox{iid}}}{\sim}}
\newcommand{\im}{{\text{im }}}
\newcommand{\ind}{{\perp\!\!\!\perp}}
\DeclareMathOperator*{\Int}{int}
\newcommand{\intx}{{\overline{\int_{X}}}}
\newcommand{\inte}{{\overline{\int_{\E}}}}
\newcommand{\la}{\lambda}
\newcommand{\rb}{\rangle}
\newcommand{\lb}{{\langle}}
\newcommand{\La}{\Lambda}
\newcommand{\calL}{{\mathcal{L}}}
\newcommand{\lp}{{\mathcal{L}}^{p}}
\newcommand{\lpo}{{\overline{\mathcal{L}}^{p}\!}}
\newcommand{\Lpo}{{\overline{\Ell}^{p}\!}}
\newcommand{\M}{{\mathbf{M}}}
\newcommand{\Ma}{{\mathcal{M}}}
\newcommand{\N}{{{\mathbb N}}}
\newcommand{\Na}{{{\mathcal{N}}}}
\newcommand{\norm}[1]{\left\|#1\right\|}
\newcommand{\normm}[1]{{\left\vert\kern-0.25ex\left\vert\kern-0.25ex\left\vert #1 
    \right\vert\kern-0.25ex\right\vert\kern-0.25ex\right\vert}}
\newcommand{\Om}{{{\Omega}}}
\newcommand{\one}{{{\bf 1}}}
\newcommand{\pic}{\text{Pic }}
\newcommand{\ph}{{\varphi}}
\newcommand{\Pa}{{\mathbb{P}}}
\newcommand{\Po}{{\mathcal{P}}}
\newcommand{\Q}{{\mathbb{Q}}}
\newcommand{\R}{{\mathbb R}}
\newcommand{\Rd}{{\mathbb{R}^{d}}}
\DeclareMathOperator{\rej}{reject }
\newcommand{\Rn}{{\mathbb{R}^{n}}}
\newcommand{\cR}{{\mathcal{R}}}
\newcommand{\Rad}{{\mathrm{Rad}}}
\newcommand{\ran}{{\mathrm{ran}}}
\newcommand{\Ri}{{\mathrm{R}}}
\newcommand{\supp}{{\mathrm{supp}}}
\newcommand{\Se}{\mathrm{S}}
\newcommand{\Sp}{S^{*}(\Rn)}
\newcommand{\St}{{\mathrm{St}}}
\newcommand{\Sw}{\mathcal{S}}
\newcommand{\T}{{\mathcal{T}}}
\newcommand{\ta}{{\theta}}
\newcommand{\Ta}{{\Theta}}
\newcommand{\topp}{\stackrel{p}{\to}}
\newcommand{\todd}{\stackrel{d}{\to}}
\newcommand{\toL}[1]{\stackrel{L^{#1}}{\to}} 
\newcommand{\toas}{\stackrel{a.s.}{\to}}
\DeclareMathOperator{\V}{Var}
\newcommand {\w}{{\omega}}
\newcommand {\W}{{\mathrm{W}}}
\newcommand {\Wnp}{\text{$\mathrm{W}$\textsuperscript{$n,\!p$}}}
\newcommand {\Wnpeq}{\text{$\mathrm{W}$\textsuperscript{$n\!,\!p$}}}
\newcommand {\Wonep}{\text{$\mathrm{W}$\textsuperscript{$1,\!p$}}}
\newcommand {\Wonepeq}{\text{$\mathrm{W}$\textsuperscript{$1\!,\!p$}}}
\newcommand {\X}{{\mathcal{X}}}
\newcommand {\Z}{{{\mathbb Z}}}
\newcommand {\Za}{{\mathcal{Z}}}
\newcommand {\Zd}{{\Z[\sqrt{d}]}}
\newcommand {\vanish}[1]{\relax}

\newcommand {\wh}{\widehat}
\newcommand {\wt}{\widetilde}
\newcommand {\red}{\color{red}}

% Distributions
\newcommand{\normal}{\mathsf{N}}
\newcommand{\poi}{\mathsf{Poisson}}
\newcommand{\bern}{\mathsf{Bernoulli}}
\newcommand{\bin}{\mathsf{Binomal}}
\newcommand{\multi}{\mathsf{Multinomial}}
\newcommand{\Exp}{\mathsf{Exp}}



% put your command and environment definitions here




% some theorem environments
% remove "[theorem]" if you do not want them to use the same number sequence


  \newtheorem{thrm}{Theorem}
  \newtheorem{lemma}{Lemma}
  \newtheorem{prop}{Proposition}
  \newtheorem{cor}{Corollary}

  \newtheorem{conj}{Conjecture}
  \renewcommand{\theconj}{\Alph{conj}}  % numbered A, B, C etc

  \theoremstyle{definition}
  \newtheorem{defn}{Definition}
  \newtheorem{ex}{Example}
  \newtheorem{exs}{Examples}
  \newtheorem{question}{Question}
  \newtheorem{remark}{Remark}
  \newtheorem{notn}{Notation}
  \newtheorem{exer}{Exercise}




\title{STATS310A - Lecture 6}
\author{Persi Diaconis\\ Scribed by Michael Howes}
\date{10/07/21}

\pagestyle{fancy}
\fancyhf{}
\rhead{STATS310A - Lecture 6}
\lhead{10/07/21}
\rfoot{Page \thepage}

\begin{document}
\maketitle
\tableofcontents
\section{An IOU}
All lot of what we have done has been done assuming that length is countably subadditive. That is if $(a,b] \subseteq \bigcup_{i=1}^\infty (a_i,b_i]$, then $b-a \le \sum_{i=1}^\infty b_i-a_i$. Today we prove this fact.
\begin{proof}
    First we will show that if $(a,b] \subseteq \bigcup_{i=1}^n (a_i,b_i]$, then $b-a \le \sum_{i=1}^n b_i -a_i$. We will proceed by induction, if $(a,b] \subseteq (a_1,b_1]$, then $b-a\le b_1-a\le b_1-a_1$ and we are done. 
    
    Thus suppose the result holds for $n-1$ and that $(a,b] \subseteq \bigcup_{i=1}^n (a_i,b_i]$. Then we must have $b \in (a_i,b_i]$ for some $i$. By relabelling we can assume without loss of generality that $b\in (a_n,b_n]$. If $a_n \le a$, then we are done since then $(a,b] \subseteq (a_n,b_n]$. Now suppose $a_n > a$. We thus have $(a,a_n] \subseteq \bigcup_{i=1}^{n-1}(a_i,b_i]$ and by the inductive hypothesis $a-a_n \le \sum_{i=1}^{n-1} b_i - a_i$. Thus 
    \[b-a \le b_n -a = b_n-a_n + a_n - a \le \sum_{i=1}^n b_i - a_i, \]
    as required. Now consider the infinite case when $(a,b] \subseteq \bigcup_{i=1}^\infty (a_i,b_i]$. Let $\varepsilon > 0$ and note that 
    \[[a+\varepsilon, b] \subseteq \bigcup_{i=1}^\infty (a_i,b_i+2^{-i}\varepsilon). \]
    Since the set $[a+\varepsilon,b]$ is compact, the open cover $\{(a_i,b_i+2^{-i}\varepsilon)\}_{i=1}^\infty$ must contain a finite subcover. Thus
    \[(a+\varepsilon,b] \subseteq [a+\varepsilon, b] \subseteq \bigcup_{i=1}^n (a_i, b_i+2^{-i}\varepsilon) \subseteq \bigcup_{i=1}^n (a_i, b_i+2^{-i}\varepsilon].  \]
    Thus by the finite case we have
    \[b-a-\varepsilon \le \sum_{i=1}^n b_i-a_i+2^{-i}\varepsilon \le \varepsilon+\sum_{i=1}^\infty b_i-a_i. \]
    Thus, by letting $\varepsilon$ go to zero we are done and Persi has paid up.
\end{proof}
\section{Semi-rings}
\begin{defn}
    A class, $\A$, of subsets of a set $\Om$, is a \emph{semi-ring} if
    \begin{enumerate}
        \item $\emptyset \in \A$.
        \item $\A$ is closed under finite intersections.
        \item If $A,B \in \A$ and $A \subseteq B$, then there exist disjoint sets $C_1,\ldots,C_k \in \A$, such that \[ B\setminus A = \bigcup_{i=1}^k C_i.\]
    \end{enumerate}
\end{defn}
Some examples are
\begin{itemize}
    \item Intervals in $\R$.
    \item Rectangles in $\R^2$.
    \item Rectangular prisms in $\R^3$.
    \item Hypercubes in $\R^k$.
\end{itemize}
Note that semi-rings are not necessarily closed under complements and need not contain $\Om$. Semi-rings are $\pi$-systems.
\begin{thrm}
    Let $\Om$ be a set and $\A$ a semi-ring of subsets of $\A$. Let $\mu : \A \to \R \cup \{\infty\}$ be a function such that 
    \begin{enumerate}
        \item $\mu(A) \ge 0$ for all $A \in \A$.
        \item $\mu$ is finitely additive on $\A$.
        \item $\mu$ is countably subadditive on $\A$.
    \end{enumerate}
    Then $\mu$ has an extension to $\sigma(A)$ and the extension is unique if $\mu$ is $\sigma$-finite with respect to $\A$.
\end{thrm}
\begin{proof}
    If $A \subseteq B$ are both in $\A$, then 
    \[\mu(B) = \mu(A)+\sum_{i=1}^k \mu(C_i) \ge \mu(A). \]
    Thus $\mu$ is monotone on $\A$. Define
    \[\mu^*(B) = \inf\left\{\sum_{i=1}^\infty \mu(A_i) : B \subseteq \bigcup_{i=1}^\infty A_i, A_i \in \A\right\}. \]
    We know that $\mu^*$ is a measure on the $\sigma$-algebra $\Ma(\mu^*)$. We need to show that
   \begin{enumerate}
            \item $\A \subseteq \Ma(\mu^*)$, and
            \item $\mu^*(A) = \mu(A)$ if $A \in \A$.
    
   \end{enumerate}
   For (a), suppose that $A \in \A$ and $E \subseteq \Om$. If $\mu^*(E) = \infty$, then $\mu^*(E) \ge \mu^*(A \cap E)+\mu^*(A^c \cap E)$ and we are done. 

   Thus suppose $\mu^*(E) < \infty$ and fix $\varepsilon > 0$. There exists $A_n \in \A$ such that $E \subseteq \bigcup_{n=1}^\infty A_n$ and 
   \[\sum_{n=1}^\infty \mu(A_n) \le \mu^*(E)+\varepsilon. \]
   Define $B_n = A \cap A_n$. Since $\A$ is closed under intersections we have $B_n \in \A$. For each $n$ we have $B_n \subseteq A_n$ and hence by the third semi-ring property there exist disjoint sets $C_{n,i} \in \A$ for $i=1,\ldots, k_n$  such that 
   \[A^c \cap A_n = A_n \setminus B_n = \bigcup_{i=1}^{k_n}C_{n,i}. \]
   We know that $\bigcup_{n=1}^\infty B_n$ covers $A \cap E$ and that $\bigcup_{n=1}^\infty \bigcup_{i=1}^{k_n} C_i$ covers $A^c \cap E$. Thus, since $\mu$ is finitely additive on $\A$ we have
   \begin{IEEEeqnarray*}{rCl}
    \mu^*(A \cap E) + \mu^(A^c \cap E) &\le& \sum_{n=1}^\infty \mu(B_n) + \sum_{n=1}^\infty \sum_{i=1}^{k_n}\mu(C_{n,i})\\
    &=&\sum_{n=1}^\infty \left(\mu(B_n) + \sum_{i=1}^{k_n}\mu(C_{n,i})\right)\\
    &=&\sum_{n=1}^\infty \mu(A_n)\\
    &\le &\mu^*(E) + \varepsilon.
   \end{IEEEeqnarray*}
   Thus we can conclude that $\mu^*(E) \ge \mu^*(A \cap E) + \mu^*(A^c \cap E)$ and so $A \in \Ma(\mu^*)$.

   Now we will prove $(b)$. Since $A \subseteq A$ for all $A \in \A$, we immediately have $\mu^*(A) \le \mu(A)$. Also if $A \subseteq \bigcup_{i=1}^\infty A_i$, then $\mu(A) \le \sum_{i=1}^\infty \mu(A_i)$ by countable subadditivity so $\mu(A) \le \mu^*(A)$. 
\end{proof}
\section{Distribution functions}
Let $\mu$ be a probability on $\R$ equipped with the Borel $\sigma$-algebra $\B$. Define 
\[F(x) = \mu((-\infty, x]), \]
then $F$ satisfies
\begin{enumerate}
    \item $\lim\limits_{x \to \infty} F(x) =1$ and $\lim\limits_{x \to - \infty}F(x) = 0$.
    \item $x \le y$ implies $F(x) \le F(y)$.
    \item If $x_n \searrow x$, then $F(x_n) \searrow F(x)$. Since $\bigcap_{n=1}^\infty (-\infty,x_n] = (-\infty, x]$.
\end{enumerate}
Conversely if $F$ satisfies (a)-(c), then $\mu((a,b]) = F(b)-F(a)$ extends to a measure on $(\R,\B)$ and $\mu((-\infty, x]) = F(x)$ for all $x$. 

For example
\begin{itemize}
    \item  The normal distribution: $F(x) = \frac{1}{\sqrt{2 \pi}}\int_0^x \exp\left(-\frac{1}{2}t^2\right)dt.$
    \item The exponential distribution: \[F(x) = \begin{cases}
        0 & \text{if } x < 0,\\
        1-e^{-x} & \text{if } x \ge 0.
    \end{cases}\] 
    \item A point mass at $x_0$
    \[F(x)= \begin{cases}
        0 & \text{if } x < x_0,\\
        1 & \text{if } x \ge x_0.
    \end{cases}\] 
\end{itemize}
\subsection{Distribution functions in higher dimensions}
If $x \in \R^k$, let $A_x = \{ y\in \R^k : y_i \le x_i, 1 \le i \le k \}$. If $\mu$ is a probability on $\R^k$, define 
\[F(x) := \mu(A_x). \]
The function $F$ satisfies (a)-(c) from before we also have the following extra condition! Suppose $A$ is a rectangle
\[A = \{y \in \R^k : a_i < x \le b_i, i=1,\ldots, k\}, \]
for some $a,b \in \R^k$. Then define
\[\Delta_F(A) = \sum_{v \text{ a vertex of } A} \text{sgn}(v)F(v). \]
where 
\[\text{sgn}(v) = \begin{cases}
    1 & \text{if } v_i = a_i \text{ for an even number of } i's,\\
    -1 & \text{if }v_i = a_i \text{ for an odd number of } i's.
\end{cases} \]
For example if $k=1$, then $A= (a,b]$ and $\Delta_F(A) = F(b)-F(a)$. If $k=2$, then
\[\Delta_F(A) = F(b_1,b_2) - F(a_1,b_2) - F(b_1, a_2) + F(a_1,a_2). \]
Note that by the inclusion exclusion principle 
\[\Delta_F(A) = \mu(A) \ge 0, \]
if $F(x) = \mu(A_x)$.
\begin{thrm}
    If $F$ satisfies (a)-(c) and $\Delta_F(A) \ge 0$ for all rectangles $A$, then there exists a unique probability $\mu$ on $\R^k$ such that $\mu(A_x) = F(x)$ for all $x \in \R^k$
\end{thrm}
\begin{proof}
    The set of all rectangles is a semi-ring and $\mu(A) := \Delta_F(A)$ is a non-negative, finitely additive, countably subadditive function.
\end{proof}
\section{The basic problem in probability}
We can now restate our ``basic problem in probability''. Let $\Om$ be a set, $\A$ a class of subsets of $\Om$ and $\mu$ a probability measure on $\sigma(\A)$ which we can compute on $\A$. Given $B \in \sigma(\A)$, can we compute/approximate $\mu(B)$?
\section{Random variables}
\begin{defn}
    Let $(\Om, \F)$ and $(\Om',\F')$ be measurable spaces (that is, $\F,\F'$ are $\sigma$-algebras). A function $T: \Om \to \Om'$ is \emph{measurable} if for all $A' \in \F'$, $T^{-1}(A') \in \F$.
\end{defn}
For example suppose $\Om=S_n$ the set of all permutations of $\{1,\ldots, n\}$ and $\F$ is the set of all subsets of $\Om$. Suppose also that $\Om'  = \{1,\ldots, n\}$ and again $\F'$ is the collection of all subset of $\Om'$. Define $T(\pi) = \pi(i)$. Then $T^{-1}(\{j\}) = \{\pi \in \Om : \pi(i) = j\}$ and $T^{-1}(A) = \{ \pi \in \Om : \pi(i) \in A\}$. The function $T$ is thus measurable.
\begin{lemma}
    Let $T : \Om \to \Om'$ be a function and let $\{A_i\}_{i\in I}$ be a collection of subsets of $\Om'$, then
    \begin{enumerate}
        \item $T^{-1}\left(\bigcup_{i \in I} A_i\right) = \bigcup_{i\in I} T^{-1}(A_i)$.
        \item $T^{-1}\left(\bigcap_{i \in I} A_i\right) = \bigcap_{i\in I} T^{-1}(A_i)$.
        \item For all $A \subseteq \Om$, $T^{-1}(A^c) = T^{-1}(A)^c$.
    \end{enumerate}
\end{lemma}
\begin{prop}
    Suppose $\F' = \sigma(\A')$ for some collection of subsets $\A'$. Let $T: (\Om,\F) \to (\Om',\F')$ be a function such that $T^{-1}(A') \in \F$ for all $A' \in \A'$, then $T$ is a measurable function.
\end{prop}
\begin{proof}
    Define $\G' = \{A' \in \F' : T^{-1}(A') \in \F\}$, by assumption $\G'$ contains $\A'$ and by the above lemma $\G'$ is a $\sigma$-algebra. Thus $\F' = \sigma(\A') \subseteq \G'$, as required.
\end{proof}
\begin{prop}
    Suppose that $T:(\Om,\F) \to (\Om',\F')$ and $S:(\Om',\F') \to (\Om'', \F'')$ are both measurable functions, then $S \circ T:(\Om, \F) \to (\Om'',\F'')$ is also a measurable function.
\end{prop}
\begin{defn}
    A \emph{random variable} is a measurable function $X : (\Om, \F) \to (\R, \B)$.
\end{defn}
\end{document}
\documentclass{article}
\usepackage{ae,aecompl}
\usepackage{todonotes}
\usepackage{chngcntr}
\usepackage{tikz-cd}
\usepackage{graphicx}
\graphicspath{ {./images/}}
\usepackage[all,cmtip]{xy}
\usepackage{amsmath, amscd}
\usepackage{amsthm}
\usepackage{amssymb}
\usepackage{amsfonts}
\usepackage{bm}
\usepackage{qsymbols}
\usepackage{latexsym}
\usepackage{mathrsfs}
\usepackage{mathtools}
\usepackage{cite}
\usepackage{color}
\usepackage{url}
\usepackage{enumerate}
\usepackage{verbatim}
\usepackage[draft=false, colorlinks=true]{hyperref}
\usepackage{pdfpages}
\usepackage[margin=1.2in]{geometry}
\usepackage{IEEEtrantools}

\usepackage{fancyhdr}


\usepackage[nameinlink]{cleveref}


\DeclareMathOperator*{\ac}{accept}
\DeclareMathOperator*{\amax}{argmax}
\DeclareMathOperator*{\amin}{argmin}
\DeclareMathOperator*{\Aut}{Aut}
\newcommand {\al}{{\alpha}}
\newcommand {\abs}[1]{{\left\lvert#1\right\rvert}}
\newcommand {\A}{{\mathcal{A}}}
\newcommand {\AM}{{\mathrm{AM}}}
\newcommand {\AMp}{{\AM_{p}^{X}\!(\Ri_\w)}}
\newcommand {\B}{{\mathcal{B}}}
\DeclareMathOperator*{\Be}{Bern}
\newcommand {\Br}{{\dot{B}}}
\newcommand {\Ba}{{\mathfrak{B}}}
\newcommand {\C}{{\mathbb C}}
\newcommand {\ce}{\mathrm{c}}
\newcommand {\Ce}{\mathrm{C}}
\newcommand {\Cc}{\mathrm{C_{c}}}
\newcommand {\Ccinf}{\mathrm{C_{c}^{\infty}}}
\DeclareMathOperator{\cov}{Cov}
\DeclareMathOperator{\DEV}{DEV}
\newcommand {\Di}{{\mathbb D}}
\newcommand {\dom}{\mathrm{dom}}
\newcommand{\dist}{\stackrel{\mathrm{dist}}{=}}
\newcommand {\ud}{\mathrm{d}}
\newcommand {\ue}{\mathrm{e}}
\newcommand {\eps}{\varepsilon}
\newcommand {\veps}{\varepsilon}
\newcommand {\vrho}{{\varrho}}
\newcommand {\E}{{\mathbb{E}}}
\newcommand {\Ec}{{\mathcal{E}}}
\newcommand {\Ell}{L}
\newcommand {\Ellp}{{L_{p}[0,1]}}
\newcommand {\Ellpprime}{{L_{p'}([0,1])}}
\newcommand {\Ellq}{{L_{q}([0,1])}}
\newcommand {\Ellqprime}{{L_{q'}([0,1])}}
\newcommand {\Ellr}{L^{r}}
\newcommand {\Ellone}{{L_{1}([0,1])}}
\newcommand{\Elltwo}{{L_{2}([0,1])}}
\newcommand{\Ellinfty}{L^{\infty}}
\newcommand{\Ellinftyc}{L_{\mathrm{c}}^{\infty}}
\newcommand{\exb}[1]{\exp\left\{#1\right\}}
\DeclareMathOperator*{\Ext}{Ext}
\newcommand{\F}{{\mathcal{F}}}
\newcommand{\Fe}{{\mathbb{F}}}
\newcommand{\G}{{\mathcal{G}}}
\newcommand{\HF}{\mathcal{H}_{\text{FIO}}^{1}(\Rd)}
\newcommand{\Hr}{H}
\newcommand{\HT}{\mathcal{H}}
\newcommand{\ui}{\mathrm{i}}
\newcommand{\I}{{I}}
\newcommand{\J}{{\mathcal{J}}}
\newcommand{\id}{{\mathrm{id}}}
\newcommand{\iid}{\stackrel{\mathclap{\normalfont\mbox{iid}}}{\sim}}
\newcommand{\im}{{\text{im }}}
\newcommand{\ind}{{\perp\!\!\!\perp}}
\DeclareMathOperator*{\Int}{int}
\newcommand{\intx}{{\overline{\int_{X}}}}
\newcommand{\inte}{{\overline{\int_{\E}}}}
\newcommand{\la}{\lambda}
\newcommand{\rb}{\rangle}
\newcommand{\lb}{{\langle}}
\newcommand{\La}{\Lambda}
\newcommand{\calL}{{\mathcal{L}}}
\newcommand{\lp}{{\mathcal{L}}^{p}}
\newcommand{\lpo}{{\overline{\mathcal{L}}^{p}\!}}
\newcommand{\Lpo}{{\overline{\Ell}^{p}\!}}
\newcommand{\M}{{\mathbf{M}}}
\newcommand{\Ma}{{\mathcal{M}}}
\newcommand{\N}{{{\mathbb N}}}
\newcommand{\Na}{{{\mathcal{N}}}}
\newcommand{\norm}[1]{\left\|#1\right\|}
\newcommand{\normm}[1]{{\left\vert\kern-0.25ex\left\vert\kern-0.25ex\left\vert #1 
    \right\vert\kern-0.25ex\right\vert\kern-0.25ex\right\vert}}
\newcommand{\Om}{{{\Omega}}}
\newcommand{\one}{{{\bf 1}}}
\newcommand{\pic}{\text{Pic }}
\newcommand{\ph}{{\varphi}}
\newcommand{\Pa}{{\mathbb{P}}}
\newcommand{\Po}{{\mathcal{P}}}
\newcommand{\Q}{{\mathbb{Q}}}
\newcommand{\R}{{\mathbb R}}
\newcommand{\Rd}{{\mathbb{R}^{d}}}
\DeclareMathOperator{\rej}{reject }
\newcommand{\Rn}{{\mathbb{R}^{n}}}
\newcommand{\cR}{{\mathcal{R}}}
\newcommand{\Rad}{{\mathrm{Rad}}}
\newcommand{\ran}{{\mathrm{ran}}}
\newcommand{\Ri}{{\mathrm{R}}}
\newcommand{\supp}{{\mathrm{supp}}}
\newcommand{\Se}{\mathrm{S}}
\newcommand{\Sp}{S^{*}(\Rn)}
\newcommand{\St}{{\mathrm{St}}}
\newcommand{\Sw}{\mathcal{S}}
\newcommand{\T}{{\mathcal{T}}}
\newcommand{\ta}{{\theta}}
\newcommand{\Ta}{{\Theta}}
\newcommand{\topp}{\stackrel{p}{\to}}
\newcommand{\todd}{\stackrel{d}{\to}}
\newcommand{\toL}[1]{\stackrel{L^{#1}}{\to}} 
\newcommand{\toas}{\stackrel{a.s.}{\to}}
\DeclareMathOperator{\V}{Var}
\newcommand {\w}{{\omega}}
\newcommand {\W}{{\mathrm{W}}}
\newcommand {\Wnp}{\text{$\mathrm{W}$\textsuperscript{$n,\!p$}}}
\newcommand {\Wnpeq}{\text{$\mathrm{W}$\textsuperscript{$n\!,\!p$}}}
\newcommand {\Wonep}{\text{$\mathrm{W}$\textsuperscript{$1,\!p$}}}
\newcommand {\Wonepeq}{\text{$\mathrm{W}$\textsuperscript{$1\!,\!p$}}}
\newcommand {\X}{{\mathcal{X}}}
\newcommand {\Z}{{{\mathbb Z}}}
\newcommand {\Za}{{\mathcal{Z}}}
\newcommand {\Zd}{{\Z[\sqrt{d}]}}
\newcommand {\vanish}[1]{\relax}

\newcommand {\wh}{\widehat}
\newcommand {\wt}{\widetilde}
\newcommand {\red}{\color{red}}

% Distributions
\newcommand{\normal}{\mathsf{N}}
\newcommand{\poi}{\mathsf{Poisson}}
\newcommand{\bern}{\mathsf{Bernoulli}}
\newcommand{\bin}{\mathsf{Binomal}}
\newcommand{\multi}{\mathsf{Multinomial}}
\newcommand{\Exp}{\mathsf{Exp}}



% put your command and environment definitions here




% some theorem environments
% remove "[theorem]" if you do not want them to use the same number sequence


  \newtheorem{thrm}{Theorem}
  \newtheorem{lemma}{Lemma}
  \newtheorem{prop}{Proposition}
  \newtheorem{cor}{Corollary}

  \newtheorem{conj}{Conjecture}
  \renewcommand{\theconj}{\Alph{conj}}  % numbered A, B, C etc

  \theoremstyle{definition}
  \newtheorem{defn}{Definition}
  \newtheorem{ex}{Example}
  \newtheorem{exs}{Examples}
  \newtheorem{question}{Question}
  \newtheorem{remark}{Remark}
  \newtheorem{notn}{Notation}
  \newtheorem{exer}{Exercise}




\title{STATS310A - Lecture 7}
\author{Persi Diaconis\\ Scribed by Michael Howes}
\date{10/12/21}

\pagestyle{fancy}
\fancyhf{}
\rhead{STATS310A - Lecture 7}
\lhead{10/12/21}
\rfoot{Page \thepage}

\begin{document}
\maketitle
\tableofcontents
\section{Measurable functions and random variables}
Recall that a function $T:(\Om, \F) \to (\Om', \F')$ is \emph{measurable} if $T^{-1}(A') \in \F$ for all $A' \in \F'$ where $T^{-1}(A') = \{\w \in \Om : T(\w) \in \A\}$.

A \emph{random variable} is a measurable function from $(\Om,\F)$ to $(\R, \B)$ where $\B$ is the set of Borel sets.

A \emph{random vector} is a measurable function from $(\Om, \F)$ to $(\R^k, \B_k)$ where $\B_k$ is the set of Borel subsets of $\R^k$.

\begin{lemma}
    If $Y:(\Om,\F) \to \R^k$ is a function with coordinates $Y_i:(\Om,\F) \to \R$, for $i=1,\ldots, k$, then $Y$ is a random vector if and only if $Y_i$ is a random variable for each $i$.
\end{lemma}
\begin{proof}
    Suppose each $Y_i$ is measurable then 
    \[\{\w \in \Om : Y(\w) \le (x_1,\ldots,x_k)\} = \bigcap_{i=1}^k \{\w \in \Om : Y_i(\w) \le x_i\} \in \F, \]
    since each set $\{\w \in \Om : Y_i(\w) \le x_i \}$ is in $\F$ and $\F$ is closed under finite intersections. Since sets of the form $\{y \in \R^k : y \le x\}$ generate $\B_k$, we have that $Y$ is measurable.
    
    If $Y$ is measurable, then 
    \[\{\w : Y(\w) \le x \} = \bigcup_{n=1}^\infty \{\w : Y \le (n,\ldots, x, \ldots, n)\} \in \F, \]
    since $Y$ is a random vector and $\F$ is closed under countable unions. The intervals $(-\infty, x]$ generate $\B$ and so $Y_i$ is measurable.
\end{proof}
\begin{lemma}
    If $T: \R^k \to \R^j$ is continuous, then $T$ is Borel-measurable.
\end{lemma}
\begin{proof}
    Since $T$ is continuous, $T^{-1}(U)$ is open for all open sets $U \subseteq \R^j$ and thus $T^{-1}(U)$ is Borel for all open sets $U \subseteq \R^j$. Since the open sets generate the Borel $\sigma$-algebra, $T$ is measurable.
\end{proof}

\begin{cor}
    If $X$,$Y$, $(X_n)_{n=1}^\infty$ are random varaibles, then $X+Y$, $XY$, $\max\{X,Y\}$, $\sup\{X_n\}$, $\inf \{X_n\}$, $\limsup \{X_n\}$, $\liminf \{X_n\}$ are all random variables. And the set $\{\w: \lim X_n(\w) \text{ exists }\}$ is measurable.
\end{cor}
\begin{proof}
    We can write $X+Y$ as a composition 
    \begin{IEEEeqnarray*}{RCL}
        \Om \to & \R \times \R & \to \R\\
        \w \mapsto &(X(\w), Y(\w))& \mapsto X(\w)+Y(\w).
    \end{IEEEeqnarray*}
    From the above lemma, $X+Y$ is measurable. The others are similar.
\end{proof}

\section{Push forwards}
\begin{defn}
    Suppose that $(\Om,\F,\mu)$ is a measure space and $T : (\Om,\F,\mu) \to (\Om',\F')$ is measurable. We define the \emph{push forward} of $\mu$ along $T$, to be the measure $\mu^{T^{-1}}$ on $(\Om',\F')$ defined by 
    \[\mu^{T^{-1}}(A) := \mu(T^{-1}(A)) =\mu(\{\w : T(\w) \in A \}). \]
\end{defn}
Note $\mu^{T^{-1}}$ is a measure. It is well defined because $T$ is measurable. And 
\[\mu^{T^{-1}}(\emptyset) = \mu(\emptyset) = 0. \]
If $A \subseteq B$, then $T^{-1}(A) \subseteq T^{-1}(B)$ and so 
\[\mu^{T^{-1}}(A) = \mu(T^{-1}(A))\le \mu(T^{-1}(B)) = \mu^{T^{-1}}(B). \]
If $\{A_i\}_{i=1}^\infty$ are disjoint, then $\{T^{-1}(A_i)\}_{i=1}^\infty$ are disjoint and so
\begin{IEEEeqnarray*}{rCl}
    \mu^{T^{-1}}\left(\bigcup_{i=1}^\infty A_i\right)&=&\mu\left(T^{-1}\left(\bigcup_{i=1}^\infty A_i\right)\right)\\
    &=& \mu\left(\bigcup_{i=1}^\infty T^{-1}(A_i)\right)\\
    &=& \sum_{i=1}^\infty \mu(T^{-1}(A_i))\\
    &=& \sum_{i=1}^\infty \mu^{T^{-1}}(A_i).
\end{IEEEeqnarray*}
Lebesgue's mistake/a warning: If $U \subseteq \R^2$ is a Borel set, then the projections of $U$ are not necessarily Borel sets.

\section{Haar measure}
Let $O_n = \{M \in \R^{n^2} : M^TM = I_n\}$ be the orthogonal group. The group $O_n$ has an invariant probability $\nu$ which we call Haar measure. That is for all measurable $A \subseteq O_n$ and $m \in O_n$, $\nu(m\cdot A) = \nu(A)$. What is this measure?

\subsection{One answer}
We will give a recipe for drawing $M \in O_n$ from $\nu$. To start let $Z_{i,j} \sim N(0,1)$ be independent for $1\le i,j \le n$. Let $Z= (Z_{i,j})_{i,j=1}^n$ and apply Gram-Schmidt to $Z$ to get a matrix $M \in O_n$. 

\subsection{A more mathematical answer}
We know that $\Phi(x) = \int_{-\infty}^x \exp(-t^2/2)dt$ is a distribution. Define on $\R^{n^2}$ 
\[F(x_{1,1}, x_{1,2},\ldots, x_{n,n}) = \prod_{i,j=1}^n \Phi(x_{i,j}). \]
One can check that this defines a probabilities distribution $\mu$ on $\R^{n^2}$. Define a function $T : \R^{n^2} \to O_n$ given by given a matriz $Z$, apply Gram-Schmidt to $Z$ to get $M \in O_n$. Finally define $\nu := \mu^{T^{-1}}$ to be the push forward of $\mu$ along $T$.

\section{Independence}
\begin{defn}
    If $\{X_i\}_{i \in I}$ is a collection of random varaibles, then we define the \emph{$\sigma$-algebra generated by $\{X_i\}_{i \in I}$} to be 
    \[\sigma(X_i, i \in I) := \sigma\left(\{X_i^{-1}((a,b]): i\in I, a,b \in \R\right). \]
\end{defn}
\begin{defn}
    Two random variables $X,Y$ are \emph{independent} if $\sigma(X)$ and $\sigma(Y)$ are independent. That is equivalently, 
    \[\Pa(X \le x, Y \le y) = \Pa(X \le x)\Pa(Y \le y), \]
    for all $x,y \in \R$. Yet another equivalent statement  is that for all $A,B \subseteq \R$ Borel
    \[\Pa(X \in A, Y \in B) = \Pa(X \in A) \Pa(Y \in B). \]
\end{defn}

\section{Constructing random variables}
How do we pick from $F$ where $F$ is a univariate probability distribution? We first pick $U$ which is uniformly distributed on $[0,1]$ and then we define $T:[0,1] \to \R$ by 
\[T(u) = \inf\{x \in \R : T(x) \ge u\}. \]
Then $\Pa(T(U) \le x) = F(x)$. 
\begin{ex}
    Consider the case when 
    \[F(x) = \begin{cases}
        0 & \text{if } x \le 0,\\
        1-e^{-x} & \text{if } x>0.
    \end{cases} \]
    Let $u = 1-e^{-x}$, then $x = - \log(1-u)$. Define $T:(0,1)\to \R$ by $T(u) = -\log(1-u)$ and $X = T(U)$ where $U$ is uniform on $(0,1)$. Then if $x >0$,
    \begin{IEEEeqnarray*}{rCl}
        \Pa(X \le x) &=& \Pa(-\log(1-U) \le x)\\
        &=& \Pa(-x \le \log(1-U))\\
        &=& \Pa(e^{-1} \le 1 - U)\\
        &=& \Pa(U \le 1-e^{-x})\\
        &=& 1-e^{-x}.
    \end{IEEEeqnarray*}
    Another good example if when $X$ is discrete. Say $X=a_i$ with probability $p_i$. Then the above construction divides $[0,1]$ into intervals $A_i$ of length $p_i$. Then if $U$ lies in $A_i$, then we set $T(U)$ to be $a_i$. Thus $T(U)$ and $X$ have the same distribution.  
\end{ex}
\section{Maxima}
Let $X_1,\ldots, X_n$ be independent random variables with distribution
\[\Pa(X_i \le x) = F(x). \]
Define $M_n = \max \{X_i : i =1,\ldots, n\}$. Then 
\[\Pa(M_n \le x) = \Pa\left(\bigcap_{i=1}^n \{X_i\le x\}\right) = \prod_{i=1}^n \Pa(X_i \le x) = F(x)^n. \]
What happens as $n \to \infty$? Suppose that $F(x) = 1-e^{-x}$. Then $\Pa(M_n < x) = (1-e^{-x})^n$. We are interested in what happens when $n \to \infty$. Let $x =\log(n)+y$, then 
\[\Pa(M_n \le x) = \left(1-\frac{e^{-y}}{n}\right)^n \sim e^{-e^{-y}}. \]
Then function $F(y) = e^{-e^{-y}}$, $y \in \R$ is a distribution function and is called the standard Gumble distribution. 

\begin{defn}
    We say that a sequence of distributions $F_n$ \emph{converges in distribution} to a distribution $F$ if 
    \[F_n(x) \to F(x), \]
    for all $x$ such that $F$ is continuous at $x$.
\end{defn}
Why do we only restrict to $x$ at which $F$ is continuous? Consider the following example: $X_n$ is a point mass at $1+\frac{1}{n}$ and $X$ is a point mass at $1$. Then 
\[F_n(x) = \begin{cases}
    0 & \text{if } x < 1 + 1/n,\\
    1 & \text{if } x \ge 1 + 1/n,
\end{cases}\]
and 
\[F(x) = \begin{cases}
    0 & \text{if } x < 1,\\
    1 & \text{if } x \ge 1.
\end{cases}\]
Thus $F_n(x) \to F(x)$ if and only if $x \neq 1$. Thus in the definition of convergence in distribution we do not worry about the points at which $F$ is not continuous.

We can now say that $M_n - \log(n)$ converges in distribution to a Gumble distribution.

Now lets consider the maximum of Gaussians. Let $X_1,\ldots, X_n \sim N(0,1)$. We know that 
\[\Pa(M_n \le x) = (\Phi(x))^n = e^{n\log(\Phi(x))} = e^{n\log(1-(1-\Phi(x)))}. \]
We will use the approximation $\log(1-y) \sim -y$ is $y \to 0$. We also have (homework problem)

\[\frac{x}{1+x^2}\exp^{-x^2/2} \le \int_x^\infty \exp(-t^2/2) \le \frac{1}{x}\exp^{-x^2/2}. \]
Thus we can say $1-\Phi(x) \sim \frac{1}{\sqrt{2\pi}}\frac{e^{-x^2/2}}{x}$. Thus for $n$ large
\[\Pa(M_n \le x) \sim e^{-n \frac{e^{-x^2}}{\sqrt{2\pi}x}}. \]
Let $x = \sqrt{2\log(n)-\log(\log(n))+y}$, so $x \sim \sqrt{2\log(n)+y}$, then 
\[\Pa(M_n \le \sqrt{2\log(n)-\log(\log(n))+y}) \sim e^{-\frac{e^{-y/2}}{\sqrt{2\pi}}}, \]
another Gumble distribution. We can not always perform these sorts of calculations. There are distributions such that $\lim \Pa\left(\frac{M_n-a_n}{b_n} \le x \right)$ does not exist for any choice of $a_n$, $b_n$. Discrete distributions such as the geometric or Poission ditributions tend to show this behaviour. Be careful when looking at the limiting behaviour of the maxima of discrete random variables.
\end{document}
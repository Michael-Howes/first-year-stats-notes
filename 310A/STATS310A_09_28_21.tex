\documentclass{article}
\usepackage{ae,aecompl}
\usepackage{todonotes}
\usepackage{chngcntr}
\usepackage{tikz-cd}
\usepackage{graphicx}
\graphicspath{ {./images/}}
\usepackage[all,cmtip]{xy}
\usepackage{amsmath, amscd}
\usepackage{amsthm}
\usepackage{amssymb}
\usepackage{amsfonts}
\usepackage{bm}
\usepackage{qsymbols}
\usepackage{latexsym}
\usepackage{mathrsfs}
\usepackage{mathtools}
\usepackage{cite}
\usepackage{color}
\usepackage{url}
\usepackage{enumerate}
\usepackage{verbatim}
\usepackage[draft=false, colorlinks=true]{hyperref}
\usepackage{pdfpages}
\usepackage[margin=1.2in]{geometry}
\usepackage{IEEEtrantools}

\usepackage{fancyhdr}


\usepackage[nameinlink]{cleveref}


\DeclareMathOperator*{\ac}{accept}
\DeclareMathOperator*{\amax}{argmax}
\DeclareMathOperator*{\amin}{argmin}
\DeclareMathOperator*{\Aut}{Aut}
\newcommand {\al}{{\alpha}}
\newcommand {\abs}[1]{{\left\lvert#1\right\rvert}}
\newcommand {\A}{{\mathcal{A}}}
\newcommand {\AM}{{\mathrm{AM}}}
\newcommand {\AMp}{{\AM_{p}^{X}\!(\Ri_\w)}}
\newcommand {\B}{{\mathcal{B}}}
\DeclareMathOperator*{\Be}{Bern}
\newcommand {\Br}{{\dot{B}}}
\newcommand {\Ba}{{\mathfrak{B}}}
\newcommand {\C}{{\mathbb C}}
\newcommand {\ce}{\mathrm{c}}
\newcommand {\Ce}{\mathrm{C}}
\newcommand {\Cc}{\mathrm{C_{c}}}
\newcommand {\Ccinf}{\mathrm{C_{c}^{\infty}}}
\DeclareMathOperator{\cov}{Cov}
\DeclareMathOperator{\DEV}{DEV}
\newcommand {\Di}{{\mathbb D}}
\newcommand {\dom}{\mathrm{dom}}
\newcommand{\dist}{\stackrel{\mathrm{dist}}{=}}
\newcommand {\ud}{\mathrm{d}}
\newcommand {\ue}{\mathrm{e}}
\newcommand {\eps}{\varepsilon}
\newcommand {\veps}{\varepsilon}
\newcommand {\vrho}{{\varrho}}
\newcommand {\E}{{\mathbb{E}}}
\newcommand {\Ec}{{\mathcal{E}}}
\newcommand {\Ell}{L}
\newcommand {\Ellp}{{L_{p}[0,1]}}
\newcommand {\Ellpprime}{{L_{p'}([0,1])}}
\newcommand {\Ellq}{{L_{q}([0,1])}}
\newcommand {\Ellqprime}{{L_{q'}([0,1])}}
\newcommand {\Ellr}{L^{r}}
\newcommand {\Ellone}{{L_{1}([0,1])}}
\newcommand{\Elltwo}{{L_{2}([0,1])}}
\newcommand{\Ellinfty}{L^{\infty}}
\newcommand{\Ellinftyc}{L_{\mathrm{c}}^{\infty}}
\newcommand{\exb}[1]{\exp\left\{#1\right\}}
\DeclareMathOperator*{\Ext}{Ext}
\newcommand{\F}{{\mathcal{F}}}
\newcommand{\Fe}{{\mathbb{F}}}
\newcommand{\G}{{\mathcal{G}}}
\newcommand{\HF}{\mathcal{H}_{\text{FIO}}^{1}(\Rd)}
\newcommand{\Hr}{H}
\newcommand{\HT}{\mathcal{H}}
\newcommand{\ui}{\mathrm{i}}
\newcommand{\I}{{I}}
\newcommand{\J}{{\mathcal{J}}}
\newcommand{\id}{{\mathrm{id}}}
\newcommand{\iid}{\stackrel{\mathclap{\normalfont\mbox{iid}}}{\sim}}
\newcommand{\im}{{\text{im }}}
\newcommand{\ind}{{\perp\!\!\!\perp}}
\DeclareMathOperator*{\Int}{int}
\newcommand{\intx}{{\overline{\int_{X}}}}
\newcommand{\inte}{{\overline{\int_{\E}}}}
\newcommand{\la}{\lambda}
\newcommand{\rb}{\rangle}
\newcommand{\lb}{{\langle}}
\newcommand{\La}{\Lambda}
\newcommand{\calL}{{\mathcal{L}}}
\newcommand{\lp}{{\mathcal{L}}^{p}}
\newcommand{\lpo}{{\overline{\mathcal{L}}^{p}\!}}
\newcommand{\Lpo}{{\overline{\Ell}^{p}\!}}
\newcommand{\M}{{\mathbf{M}}}
\newcommand{\Ma}{{\mathcal{M}}}
\newcommand{\N}{{{\mathbb N}}}
\newcommand{\Na}{{{\mathcal{N}}}}
\newcommand{\norm}[1]{\left\|#1\right\|}
\newcommand{\normm}[1]{{\left\vert\kern-0.25ex\left\vert\kern-0.25ex\left\vert #1 
    \right\vert\kern-0.25ex\right\vert\kern-0.25ex\right\vert}}
\newcommand{\Om}{{{\Omega}}}
\newcommand{\one}{{{\bf 1}}}
\newcommand{\pic}{\text{Pic }}
\newcommand{\ph}{{\varphi}}
\newcommand{\Pa}{{\mathbb{P}}}
\newcommand{\Po}{{\mathcal{P}}}
\newcommand{\Q}{{\mathbb{Q}}}
\newcommand{\R}{{\mathbb R}}
\newcommand{\Rd}{{\mathbb{R}^{d}}}
\DeclareMathOperator{\rej}{reject }
\newcommand{\Rn}{{\mathbb{R}^{n}}}
\newcommand{\cR}{{\mathcal{R}}}
\newcommand{\Rad}{{\mathrm{Rad}}}
\newcommand{\ran}{{\mathrm{ran}}}
\newcommand{\Ri}{{\mathrm{R}}}
\newcommand{\supp}{{\mathrm{supp}}}
\newcommand{\Se}{\mathrm{S}}
\newcommand{\Sp}{S^{*}(\Rn)}
\newcommand{\St}{{\mathrm{St}}}
\newcommand{\Sw}{\mathcal{S}}
\newcommand{\T}{{\mathcal{T}}}
\newcommand{\ta}{{\theta}}
\newcommand{\Ta}{{\Theta}}
\newcommand{\topp}{\stackrel{p}{\to}}
\newcommand{\todd}{\stackrel{d}{\to}}
\newcommand{\toL}[1]{\stackrel{L^{#1}}{\to}} 
\newcommand{\toas}{\stackrel{a.s.}{\to}}
\DeclareMathOperator{\V}{Var}
\newcommand {\w}{{\omega}}
\newcommand {\W}{{\mathrm{W}}}
\newcommand {\Wnp}{\text{$\mathrm{W}$\textsuperscript{$n,\!p$}}}
\newcommand {\Wnpeq}{\text{$\mathrm{W}$\textsuperscript{$n\!,\!p$}}}
\newcommand {\Wonep}{\text{$\mathrm{W}$\textsuperscript{$1,\!p$}}}
\newcommand {\Wonepeq}{\text{$\mathrm{W}$\textsuperscript{$1\!,\!p$}}}
\newcommand {\X}{{\mathcal{X}}}
\newcommand {\Z}{{{\mathbb Z}}}
\newcommand {\Za}{{\mathcal{Z}}}
\newcommand {\Zd}{{\Z[\sqrt{d}]}}
\newcommand {\vanish}[1]{\relax}

\newcommand {\wh}{\widehat}
\newcommand {\wt}{\widetilde}
\newcommand {\red}{\color{red}}

% Distributions
\newcommand{\normal}{\mathsf{N}}
\newcommand{\poi}{\mathsf{Poisson}}
\newcommand{\bern}{\mathsf{Bernoulli}}
\newcommand{\bin}{\mathsf{Binomal}}
\newcommand{\multi}{\mathsf{Multinomial}}
\newcommand{\Exp}{\mathsf{Exp}}



% put your command and environment definitions here




% some theorem environments
% remove "[theorem]" if you do not want them to use the same number sequence


  \newtheorem{thrm}{Theorem}
  \newtheorem{lemma}{Lemma}
  \newtheorem{prop}{Proposition}
  \newtheorem{cor}{Corollary}

  \newtheorem{conj}{Conjecture}
  \renewcommand{\theconj}{\Alph{conj}}  % numbered A, B, C etc

  \theoremstyle{definition}
  \newtheorem{defn}{Definition}
  \newtheorem{ex}{Example}
  \newtheorem{exs}{Examples}
  \newtheorem{question}{Question}
  \newtheorem{remark}{Remark}
  \newtheorem{notn}{Notation}
  \newtheorem{exer}{Exercise}




\title{STATS310A - Lecture 3}
\author{Persi Diaconis\\ Scribed by Michael Howes}
\date{9/28/21}

\pagestyle{fancy}
\fancyhf{}
\rhead{STATS310A - Lecture 3}
\lhead{9/28/21}

\rfoot{Page \thepage}

\begin{document}
\maketitle
\tableofcontents
\section{Homework}
Read chapters 3 and 4.
Do book questions 3.6, 3.7, 3.8, 4.5, 4.15 in the books.
If $\xi_n > 0$, $\xi_n \to \infty$ and $\frac{\xi_n}{\sqrt{n}} \to 0$, then 
\[P\{S_n > \sqrt{n}\xi_n\} \ge \exp(-(1+o(1))\xi_n^2/2). \]
See canvas for more details. There are some hints for textbook questions in the back of the textbook.

\section{Extending measures}
Recall our set up: $\Om$ is a set, $\F_0$ is an algebra of subsets of $\Om$, $P$ is a probability measure on $\F_0$.

A set $A \subseteq \Om$ is measurable if
\[P^*(E) = P^*(E \cap A) + P^*(E \cap A^c), \]
for all $E \subseteq \Om$. Where we defined
\[P^*(B) = \inf\left\{\sum_{i=1}^\infty P(A_i) : A_i \in \F_0 \text{ and } B \subseteq \bigcup_{i=1}^\infty A_i\right\}. \]
Let $\Ma$ be the set of measurable subsets of $\Om$. We have been proving the following:
\begin{thrm}
    Let $P^*$ and $\Ma$ be as above, then
    \begin{itemize}
        \item $\Ma$ is $\sigma$-algebra.
        \item $P^*$ is a probability on $\Ma$.
        \item $P^*$ extends $P$ on $\F_0 \subseteq \Ma$.
        \item $P^*$ is unique
    \end{itemize}
\end{thrm}
Last time we showed that $\Ma$ is an algebra. We used the key trick that $A \in \Ma$ if and only if
\[P^*(E) \ge P^*(E \cap A) + P^*(E \cap A^c), \]
since the other inequality always holds by subadditivity.

\underline{Step 1}
If $A_i \in \Ma$ is a countable collection of disjoint sets, then for every $E \subseteq \Om$,
\[P^*\left(E \cap \left(\bigcup_{i=1}^\infty A_i\right)\right) =\sum_{i=1}^\infty P^*(E \cap A_i).\]
\begin{proof}
    We first prove the above for finite unions $A_1,\ldots, A_n$. If $n=1$, then we simply have 
    \[P^*(E \cap A_1) = P^*(E \cap A_1). \]
    Suppose $n=2$. Since $A_1 \in \Ma$ and $A_1 \cap A_2 = \emptyset$, we have
    \begin{IEEEeqnarray*}{rCl}
        P^*(E \cap (A_1 \cup A_2))&=& P^*(E \cap (A_1 \cup A_2)\cap A_1) + P^*(E \cap (A_1 \cup A_2) \cap A_1^c)\\
        &=& P^*(E \cap A_1) + P^*(E \cap A_2).
    \end{IEEEeqnarray*}
    Thus the result holds for $n=2$. Since $\Ma$ is a field we can use induction to conclude the result for general $n \in \N$. Now suppose $A = \bigcup_{i=1}^\infty A_i$, where the sets $A_i$ are disjoint and measurable. Let $F_n = \bigcup_{i=1}^n A_i$, then 
    \begin{IEEEeqnarray*}{rCl}
        P^*(E \cap A) &\ge & P^*(E \cap F_n)\\
        &=& \sum_{i=1}^n P^*(E \cap A_i).
    \end{IEEEeqnarray*}
    Thus $P^*(E \cap A) \ge \sum_{i=1}^\infty P^*(E \cap A)$. Also $P^*(E \cap A)\le \sum_{i=1}^\infty P^*(E \cap A_i)$ by countable subadditivity.
\end{proof}

\underline{Step 2} The collection $\Ma$ is a $\sigma$-algebra and $P^*$ is countably additive on $\Ma$. 
\begin{proof}
    If $E = \Om$ in step 1, then we can immediately see that $P^*$ is countably additive on $\Ma$. To show $\Ma$ is a $\sigma$-algebra, we need to show that $\Ma$ is closed under countable unions.

    Let $A_i \in \Ma$ for each $i \in \N$. If we define $A_i' = A_i \cap \left(\bigcup_{j=1}^{i-1} A_j^c\right)$, then $A_i' \in \Ma$, the sets $A_i'$ are disjoint and $\bigcup_{i=1}^\infty A_i = \bigcup_{i=1}^\infty A_i'$. Thus we may assume that the sets $A_i$ are disjoint.

    As before, define
    \[F_n = \bigcup_{i=1}^nA_i \text{ and } A = \bigcup_{i=1}^\infty A_i. \]
    Note that $A^c \subseteq F_n^c$. We know that $F_n \in \Ma$ and thus for any $E \subseteq \Om$,
    \begin{IEEEeqnarray*}{rCl}
        P^*(E) &=& P^*(E \cap F_n) + P^*(E \cap F_n^c)\\
        &=& \sum_{i=1}^n P^*(E \cap A_i) + P^*(E \cap F_n^c)\\
        &\ge& \sum_{i=1}^nP^*(E \cap A_i) + P^*(E \cap A^c). 
    \end{IEEEeqnarray*}
    Letting $n$ go to infinity we can conclude 
    \[P^*(E) \ge \sum_{i=1}^\infty P^*(E \cap A_i) + P^*(E \cap A^c) = P^*(E \cap A) + P^*(E \cap A^c), \]
    thus $A \in \Ma$.
\end{proof}
\underline{Step 3} $\F_0 \subseteq \Ma$.
\begin{proof}
    Pick $A \in \F_0$, $E \subseteq \Om$ and $\varepsilon > 0$. Find a collection $(A_i)_{i=1}^\infty$ such that $E \subseteq \bigcup_{i=1}^\infty A_i$, $A_i \in \F_0$ and $\sum_{i=1}^\infty P(A_i) \le P^*(A)+\varepsilon$.
    Let $B_n = A_n\cap A$ and $C_n = A_n \cap A^c$. Then $E \cap A \subseteq \bigcup_{n=1}^\infty B_n$ and $E \cap A^c \subseteq \bigcup_{n=1}^\infty C_n$ and $B_n, C_n \in \F_0$. Thus
    \[P^*(E \cap A) \le \sum_{n=1}^\infty P(B_n) \text{ and } P^*(E \cap A^c) \le \sum_{n=1}^\infty P(C_n). \]
    Thus
    \begin{IEEEeqnarray*}{rCl}
        P^*(E) &\ge&\sum_{n=1}^\infty P(A_n))-\varepsilon\\
        &=&\left(\sum_{n=1}^\infty P(B_n)+P(C_n)\right)-\varepsilon\\
        &=&\left(\sum_{n=1}^\infty P(B_n) +\sum_{n=1}^\infty P(C_n)\right)-\varepsilon\\
        &\ge & P^*(E \cap A) + P^*(E \cap A^c) -\varepsilon.
    \end{IEEEeqnarray*} 
    Letting $\varepsilon$ go to 0, we see that $P^*(E) \ge P^*(E \cap A) + P^*(E \cap A^c)$ so $A \in \Ma$.
\end{proof}
\underline{Step 4} If $A \in \F_0$, then $P^*(A) = P(A)$. (see proof in the textbook).

\section{Uniqueness of the extension}
We will use the $\pi-\lambda$ theorem.
\begin{defn}
    A collection of sets $\Po$ is a \emph{$\pi$-system} if $\Po$ is closed under finite intersection.
\end{defn}

\begin{defn}
    A collection of sets $L$ is a \emph{$\lambda$-system} if
    \begin{enumerate}
        \item $\Om \in L$,
        \item $A \in L$ implies $A^c \in L$,
        \item $L$ is closed under countable disjoint unions.
    \end{enumerate}
\end{defn}
\begin{ex}
    If $\Om = \{1,2,3,4\}$, then 
    \[S = \{\emptyset, \Om, \{1,2\}, \{1,3\}, \{1,4\}, \{2,3\}, \{2,4\}, \{3,4\} \},\]
    is a $\la$-system but not a $\pi$-system and not an algebra.
\end{ex}
The following is due to Dynkin and is a clean subsititute for monotone class arguments.
\begin{thrm}
    If $\Po$ is a $\pi$-system and $L$ is a $\la$-system and $\Po \subseteq L$, then $\sigma(\Po) \subseteq L$.
\end{thrm}
As an immediate application we have the following.
\begin{prop}
    Suppose $\F_0$ is an algebra of subsets of $\Om$ and $P$ is a probability on $\F_0$. If $P'$ and $P''$ are two probability measures on $\sigma(\F_0)$ that extend $P$, then $P' = P''$.
\end{prop}
\begin{proof}
    Let $L = \{A \in \sigma(\F_0) : P'(A) = P''(A)\}$. Then $L$ is a $\lambda$-system and since $\F_0$ is a $\pi$-system contained in $L$, we have $\sigma(\F_0) \subseteq L$ and thus $P'$ and $P''$ agree on $\sigma(\F_0)$.
\end{proof}
Note that in the case of section 2 we know that $\sigma(\F_0) \subseteq \Ma$. 

\begin{cor}
    Lebesgue measure ($P^*$ on $(0,1]$) is the unique extension of length to the Borel sets.
\end{cor}

\underline{Important note}: When extending from $\F_0$ to $\Ma$, we started with a probability on $\F_0$. This means that if $A \in \F_0$ and $(A_i)_{i=1}^\infty$ is a countable disjoint collection of sets in $\F_0$ such that $A = \bigcup_{i=1}^\infty A_i$, then $P(A) =\sum_{i=1}^\infty P(A_i)$. Proving this in the Lebesgue measure case requires the Heine-Borel theorem. We need to use compactness of closed, bounded intervals at some point.
\section{Comments on homework}
\begin{itemize}
    \item 3.6-3.8 are similar to what we have done in class but for finitely additive measures.
    
    \item 4.5 is about limit superiors and limit inferiors. 
    
    \item 4.15 is about showing that the square free numbers have density $6\pi^2$!
\end{itemize}
\end{document}
\documentclass{article}
\usepackage{ae,aecompl}
\usepackage{todonotes}
\usepackage{chngcntr}
\usepackage{tikz-cd}
\usepackage{graphicx}
\graphicspath{ {./images/}}
\usepackage[all,cmtip]{xy}
\usepackage{amsmath, amscd}
\usepackage{amsthm}
\usepackage{amssymb}
\usepackage{amsfonts}
\usepackage{bm}
\usepackage{qsymbols}
\usepackage{latexsym}
\usepackage{mathrsfs}
\usepackage{mathtools}
\usepackage{cite}
\usepackage{color}
\usepackage{url}
\usepackage{enumerate}
\usepackage{verbatim}
\usepackage[draft=false, colorlinks=true]{hyperref}
\usepackage{pdfpages}
\usepackage[margin=1.2in]{geometry}
\usepackage{IEEEtrantools}

\usepackage{fancyhdr}


\usepackage[nameinlink]{cleveref}


\DeclareMathOperator*{\ac}{accept}
\DeclareMathOperator*{\amax}{argmax}
\DeclareMathOperator*{\amin}{argmin}
\DeclareMathOperator*{\Aut}{Aut}
\newcommand {\al}{{\alpha}}
\newcommand {\abs}[1]{{\left\lvert#1\right\rvert}}
\newcommand {\A}{{\mathcal{A}}}
\newcommand {\AM}{{\mathrm{AM}}}
\newcommand {\AMp}{{\AM_{p}^{X}\!(\Ri_\w)}}
\newcommand {\B}{{\mathcal{B}}}
\DeclareMathOperator*{\Be}{Bern}
\newcommand {\Br}{{\dot{B}}}
\newcommand {\Ba}{{\mathfrak{B}}}
\newcommand {\C}{{\mathbb C}}
\newcommand {\ce}{\mathrm{c}}
\newcommand {\Ce}{\mathrm{C}}
\newcommand {\Cc}{\mathrm{C_{c}}}
\newcommand {\Ccinf}{\mathrm{C_{c}^{\infty}}}
\DeclareMathOperator{\cov}{Cov}
\DeclareMathOperator{\DEV}{DEV}
\newcommand {\Di}{{\mathbb D}}
\newcommand {\dom}{\mathrm{dom}}
\newcommand{\dist}{\stackrel{\mathrm{dist}}{=}}
\newcommand {\ud}{\mathrm{d}}
\newcommand {\ue}{\mathrm{e}}
\newcommand {\eps}{\varepsilon}
\newcommand {\veps}{\varepsilon}
\newcommand {\vrho}{{\varrho}}
\newcommand {\E}{{\mathbb{E}}}
\newcommand {\Ec}{{\mathcal{E}}}
\newcommand {\Ell}{L}
\newcommand {\Ellp}{{L_{p}[0,1]}}
\newcommand {\Ellpprime}{{L_{p'}([0,1])}}
\newcommand {\Ellq}{{L_{q}([0,1])}}
\newcommand {\Ellqprime}{{L_{q'}([0,1])}}
\newcommand {\Ellr}{L^{r}}
\newcommand {\Ellone}{{L_{1}([0,1])}}
\newcommand{\Elltwo}{{L_{2}([0,1])}}
\newcommand{\Ellinfty}{L^{\infty}}
\newcommand{\Ellinftyc}{L_{\mathrm{c}}^{\infty}}
\newcommand{\exb}[1]{\exp\left\{#1\right\}}
\DeclareMathOperator*{\Ext}{Ext}
\newcommand{\F}{{\mathcal{F}}}
\newcommand{\Fe}{{\mathbb{F}}}
\newcommand{\G}{{\mathcal{G}}}
\newcommand{\HF}{\mathcal{H}_{\text{FIO}}^{1}(\Rd)}
\newcommand{\Hr}{H}
\newcommand{\HT}{\mathcal{H}}
\newcommand{\ui}{\mathrm{i}}
\newcommand{\I}{{I}}
\newcommand{\J}{{\mathcal{J}}}
\newcommand{\id}{{\mathrm{id}}}
\newcommand{\iid}{\stackrel{\mathclap{\normalfont\mbox{iid}}}{\sim}}
\newcommand{\im}{{\text{im }}}
\newcommand{\ind}{{\perp\!\!\!\perp}}
\DeclareMathOperator*{\Int}{int}
\newcommand{\intx}{{\overline{\int_{X}}}}
\newcommand{\inte}{{\overline{\int_{\E}}}}
\newcommand{\la}{\lambda}
\newcommand{\rb}{\rangle}
\newcommand{\lb}{{\langle}}
\newcommand{\La}{\Lambda}
\newcommand{\calL}{{\mathcal{L}}}
\newcommand{\lp}{{\mathcal{L}}^{p}}
\newcommand{\lpo}{{\overline{\mathcal{L}}^{p}\!}}
\newcommand{\Lpo}{{\overline{\Ell}^{p}\!}}
\newcommand{\M}{{\mathbf{M}}}
\newcommand{\Ma}{{\mathcal{M}}}
\newcommand{\N}{{{\mathbb N}}}
\newcommand{\Na}{{{\mathcal{N}}}}
\newcommand{\norm}[1]{\left\|#1\right\|}
\newcommand{\normm}[1]{{\left\vert\kern-0.25ex\left\vert\kern-0.25ex\left\vert #1 
    \right\vert\kern-0.25ex\right\vert\kern-0.25ex\right\vert}}
\newcommand{\Om}{{{\Omega}}}
\newcommand{\one}{{{\bf 1}}}
\newcommand{\pic}{\text{Pic }}
\newcommand{\ph}{{\varphi}}
\newcommand{\Pa}{{\mathbb{P}}}
\newcommand{\Po}{{\mathcal{P}}}
\newcommand{\Q}{{\mathbb{Q}}}
\newcommand{\R}{{\mathbb R}}
\newcommand{\Rd}{{\mathbb{R}^{d}}}
\DeclareMathOperator{\rej}{reject }
\newcommand{\Rn}{{\mathbb{R}^{n}}}
\newcommand{\cR}{{\mathcal{R}}}
\newcommand{\Rad}{{\mathrm{Rad}}}
\newcommand{\ran}{{\mathrm{ran}}}
\newcommand{\Ri}{{\mathrm{R}}}
\newcommand{\supp}{{\mathrm{supp}}}
\newcommand{\Se}{\mathrm{S}}
\newcommand{\Sp}{S^{*}(\Rn)}
\newcommand{\St}{{\mathrm{St}}}
\newcommand{\Sw}{\mathcal{S}}
\newcommand{\T}{{\mathcal{T}}}
\newcommand{\ta}{{\theta}}
\newcommand{\Ta}{{\Theta}}
\newcommand{\topp}{\stackrel{p}{\to}}
\newcommand{\todd}{\stackrel{d}{\to}}
\newcommand{\toL}[1]{\stackrel{L^{#1}}{\to}} 
\newcommand{\toas}{\stackrel{a.s.}{\to}}
\DeclareMathOperator{\V}{Var}
\newcommand {\w}{{\omega}}
\newcommand {\W}{{\mathrm{W}}}
\newcommand {\Wnp}{\text{$\mathrm{W}$\textsuperscript{$n,\!p$}}}
\newcommand {\Wnpeq}{\text{$\mathrm{W}$\textsuperscript{$n\!,\!p$}}}
\newcommand {\Wonep}{\text{$\mathrm{W}$\textsuperscript{$1,\!p$}}}
\newcommand {\Wonepeq}{\text{$\mathrm{W}$\textsuperscript{$1\!,\!p$}}}
\newcommand {\X}{{\mathcal{X}}}
\newcommand {\Z}{{{\mathbb Z}}}
\newcommand {\Za}{{\mathcal{Z}}}
\newcommand {\Zd}{{\Z[\sqrt{d}]}}
\newcommand {\vanish}[1]{\relax}

\newcommand {\wh}{\widehat}
\newcommand {\wt}{\widetilde}
\newcommand {\red}{\color{red}}

% Distributions
\newcommand{\normal}{\mathsf{N}}
\newcommand{\poi}{\mathsf{Poisson}}
\newcommand{\bern}{\mathsf{Bernoulli}}
\newcommand{\bin}{\mathsf{Binomal}}
\newcommand{\multi}{\mathsf{Multinomial}}
\newcommand{\Exp}{\mathsf{Exp}}



% put your command and environment definitions here




% some theorem environments
% remove "[theorem]" if you do not want them to use the same number sequence


  \newtheorem{thrm}{Theorem}
  \newtheorem{lemma}{Lemma}
  \newtheorem{prop}{Proposition}
  \newtheorem{cor}{Corollary}

  \newtheorem{conj}{Conjecture}
  \renewcommand{\theconj}{\Alph{conj}}  % numbered A, B, C etc

  \theoremstyle{definition}
  \newtheorem{defn}{Definition}
  \newtheorem{ex}{Example}
  \newtheorem{exs}{Examples}
  \newtheorem{question}{Question}
  \newtheorem{remark}{Remark}
  \newtheorem{notn}{Notation}
  \newtheorem{exer}{Exercise}




\title{STATS310A - Lecture 14}
\author{Persi Diaconis\\ Scribed by Michael Howes}
\date{11/09/21}

\pagestyle{fancy}
\fancyhf{}
\rhead{STATS310A - Lecture 14}
\lhead{11/09/21}
\rfoot{Page \thepage}

\begin{document}
\maketitle
\tableofcontents
\section{Recap}
Today we will continue with Stien's method and Poisson approximation. As before $\Po_\la$ will be used to denote the Poisson distribution with parameter $\la > 0$. That is, for $j \in \N = \{0,1,2,\ldots\}$, we have $\Po_\la(\{j\}) = \frac{e^{-\la}\la^j}{j!}$. As before we will say that a random variable $Z$ is Poisson$(\la)$ to mean that $Z$ has distribution $\Po_\la$ and so  
\[\Pa(Z \in A) = \Po_\la(A). \]


Suppose we have a finite index set $I$ and random variables $\{X_i\}_{i \in I}$ such that $X_i$ takes values 0, 1. Suppose $\Pa(X_i=1)=\E[X_i]=p_i$ and that $\Pa(X_i=1,X_j=1) = \E[X_iX_j] = p_{ij}$. Let 
\[W = \sum_{i\in I}X_i, \]
and
\[\la = \sum_{i \in I}p_i = \E[W]. \]
Also define $\Pa_W$ to be the probability measure $\Pa_W(A) = \Pa(W \in A)$ for $A \subseteq \N$. Suppose that we have a dependency graph $\Gamma$ for $\{X_i\}_{i \in I}$. That is, for all subsets $A,B \subseteq I$, if $A$ and $B$ are disjoint and there are no edges between $A$ and $B$ in $\Gamma$, then 
\[\{X_i\}_{i \in A} ~~ \text{and}~~ \{X_j\}_{j \in B}, \]
are independent. For $i \in I$ we define $N_i$ to be the neighbourhood of $i$ in $\Gamma$. That is,
\[N_i = \{j \in I : \text{there is an edge from $i$ to $j$ in $\Gamma$}\} \cup \{i\}. \]
We wish to prove
\begin{thrm}
    With notation as above
    \[\norm{\Po_\la-\Pa_W}_{TV} \le \min(3,\la^{-1})\left(\sum_{i \in I}\sum_{i \in N_i \setminus \{i\}}p_{ij}+\sum_{i \in I}\sum_{j \in N_i}p_{ij}\right),\] 
    where $\norm{\cdot}_{TV}$ is the total variation distance.
\end{thrm}
\section{Stein's equation}
The key idea is the following proposition:
\begin{prop}\label{stien}
    A random variable is Poisson$(\la)$ if and only if for all bounded $f : \N \to \R$,
    \[\E[Zf(Z)]-\la \E[f(Z+1)]=0. \]
\end{prop}
We'll need the following analytic lemma.
\begin{lemma}[$**$]\label{**}
    For all $A \subseteq \N$ and $\la > 0$, there exists a unique function $f : \N \to \R$ such that 
    \begin{enumerate}
        \item[i.] $f(0)=0$,
        \item[ii.] For all $j \in \N$, $\la f(j+1)-jf(j) = \delta_A(j)-\Po_\la(A)$
        \item[iii.] For all $j \in \N$, $\abs{f(j)} \le 1.25$.
        \item[iv.] For all $j \in \N$, $\abs{f(j+1)-f(j)} \le \min(3,\la^{-1})$. 
    \end{enumerate} 
\end{lemma}
\begin{proof}
    Starting at $j=0$ we can set 
    \[f(j+1) = \frac{1}{\la} \left(j f(j) +\delta_A(j)-\Po_\la(A)\right). \]
    The function $f$ is well-defined by recursion and unique by induction. Thus there exists a unique function $f$ satisfying items \emph{i.} and \emph{ii.}. We wish to show that $f$ satisfies items \emph{iii.} and \emph{iv.}. If we multiply the equation 
    \[ \la f(j+1)-jf(j) = \delta_A(j)-\Po_\la(A),\]
    by $\frac{\la^j}{j!}$, we get the equation 
    \[\frac{\la^{j+1}}{j!}f(j+1) - \frac{\la^j}{(j-1)!}f(j) = \frac{\la^j}{j!}\left(\delta_A(j)-\Po_\la(A)\right), \]
    for $j \ge 1$ and for $j=0$ we have
    \[\la f(1)=\la \left(\delta_A(0)-\Po_\la(A)\right). \]
    Thus 
    \begin{align*}
        \frac{\la^k}{(k-1)!}f(k)&=\la f(1)+\sum_{j=1}^{k-1} \left(\frac{\la^{j+1}}{j!}f(j+1)-\frac{\la^j}{(j-1)!} f(j)\right)\\
        &=\sum_{j=0}^{k-1} \frac{\la^j}{j!}\left(\delta_A(j)-\Po_\la(A)\right)\\
        &=-\sum_{j=k}^\infty \frac{\la^j}{j!}\left(\delta_A(j)-\Po_\la(A)\right).
    \end{align*}
    The last equality hold because  
    \begin{align*}
        \sum_{j=0}^\infty \frac{\la^j}{j!}\left(\delta_A(j)-\Po_\la(A)\right)&=e^\la \sum_{j=0}^\infty \delta_A(j)\Po_\la(\{j\}) - \Po_\la(A)\sum_{j=0}^\infty \frac{\la^j}{j!}\\
        &=e^\la \Po_\la(A) - \Po_\la(A)e^{\la}\\
        &=0.
    \end{align*}
    Taking absolute values we get
    \begin{align*}
        \abs{f(k)} &= \frac{(k-1)!}{\la^k}\abs{\sum_{j=0}^{k-1}\frac{\la^j}{j!}\left(\delta_A(j)-\Po_\la(A)\right)}\\
        &\le\frac{(k-1)!}{\la^k}\sum_{j=0}^{k-1}\frac{\la^j}{j!}\abs{\delta_A(j)-\Po_\la(A)}\\
        &=\frac{1}{\la}\sum_{j=0}^{k-1}\frac{(k-1)!}{\la^{k-j-1}j!}\abs{\delta_A(j)-\Po_\la(A)}\\
        &\le \frac{1}{\la}\sum_{j=0}^{k-1}\frac{(k-1)!}{\la^{k-j-1}j!}.
    \end{align*}
    The last equality holds since $\delta_A(j),\Po_\la(A) \in [0,1]$. We will now perform a change of variables and sum over $j' = k-j-1$ so that $j =k-j'-1$. We thus have
    \begin{align*}
        \abs{f(k)}&\le \frac{1}{\la}\sum_{j=0}^{k-1}\frac{(k-1)!}{\la^{k-j-1}j!}\\
        &=\frac{1}{\la}\sum_{j'=0}^{k-1} \frac{(k-1)!}{\la^{j'}(k-1-j')!}\\
        &=\frac{1}{\la}\sum_{j'=0}^{k-1} \frac{(k-1)(k-2)\ldots(k-j')}{\la^{j'}}\\
        &\le \frac{1}{\la}\sum_{j'=0}^{k-1}\left(\frac{k-1}{\la}\right)^{j'}\\
        &\le \frac{1}{\la}\sum_{j'=0}^{\infty}\left(\frac{k-1}{\la}\right)^{j'}.
    \end{align*}
    If $k < \la+1$, then above series is convergent and we have 
    \begin{align*}
        \abs{f(k)}&\le\frac{1}{\la}\left(\frac{1}{1-\frac{k-1}{\la}}\right)\\
        &=\frac{1}{\la-k+1}. 
    \end{align*}
    In particular when $k \le \la + \frac{1}{5}$,
    \[\abs{f(k)} \le \frac{1}{4/5} = 1.25. \]
    Using 
    \[\frac{\la^k}{(k-1)!}f(k) = -\sum_{j=k}^\infty \frac{\la^j}{j!}\left(\delta_A(j)-\Po_\la(A)\right), \]
    we also have
    \begin{align*}
        \abs{f(k)} &\le \frac{(k-1)!}{\la^k} \sum_{j=k}^\infty \frac{\la^j}{j!}\abs{\delta_A(j)-\Po_\la(A)}\\
        &\le \frac{(k-1)!}{\la^k}\sum_{j=k}^\infty \frac{\la^j}{j!}\\
        &=\frac{1}{k}\sum_{j=k}^\infty \frac{\la^{j-k}k!}{j!}\\
        &=\frac{1}{k}\sum_{m=0}^\infty \frac{\la^{m}k!}{(m+k)!}\\
        &=\frac{1}{k}\sum_{m=0}^\infty \frac{\la^{m}}{(m+k)(m+k-1)\ldots (k+1)}\\
        &\le \frac{1}{k}\sum_{m=0}^\infty \left(\frac{\la}{k+1}\right)^m
    \end{align*}
    If $k > \la -1$, then the above series is convergent and so
    \[\abs{f(k)} \le \frac{1}{k}\left(\frac{1}{1-\frac{\la}{k+1}}\right) = \frac{k+1}{k(k+1-\la)}. \] 
    In particular if $k > \la + 1/5$ and $k \ge 2$, then $\frac{k+1}{k} \le \frac{3}{2}$ and so  
    \[\abs{f(k)} \le \frac{k+1}{k(1+1/5)} = \frac{5(k+1)}{6k} \le 1.25. \]
    For $k < 2$, we have $f(0)=0$ and 
    \[\abs{f(1)} = \frac{1}{\la}\abs{\delta_A(1)-\Po_\la(A)}, \]
    which is maximized when $A = \{0\}$ or $A = \N \setminus \{0\}$. In both these cases,
    \[\abs{\delta_A(1)-\Po_\la(A)} = 1-e^{-\la}, \]
    and thus 
    \[\abs{f(1)} \le \frac{1}{\la}\left(
        1-e^{-\la}\right) \le 1. \]
        Thus we have shown \emph{iii.}. To show \emph{iv.}  we need to bound $\abs{f(j+1)-f(j)}$. 
        By the triangle inequality
        \[\abs{f(j+1)-f(j)} \le \abs{f(j+1)}+\abs{f(j)} \le 2\times 1.25 \le 3. \]
        For homework show that 
        \[\abs{f(j+1)-f(j)} \le \la^{-1}, \]
        for $\la \ge \frac{1}{3}$.
\end{proof}
We are now ready to prove proposition \ref{stien}.
\begin{proof}
    First suppose that $Z$ is Poisson$(\la)$ and $f:\N \to \R$ is bounded, then 
    \begin{align*}
        \E[Zf(Z)]&=\sum_{j=0}^\infty j f(j) \Po_\la(\{j\})\\
        &=\sum_{j=0}^\infty jf(j) \frac{e^{-\la}\la^j}{j!}\\
        &=\sum_{j=1}^\infty jf(j) \frac{e^{-\la}\la^j}{j!}\\
        &=\sum_{j=1}^\infty f(j) \frac{e^{-\la}\la^j}{(j-1)!}\\
        &=\sum_{k=0}^\infty f(k+1) \frac{e^{-\la}\la^{k+1}}{k!}\\
        &=\la \sum_{k=0}^\infty f(k+1)\Po_\la(\{k\})\\
        &=\la\E[f(Z+1)]. 
    \end{align*}
    Alternative one can note that the equation 
    \[\E[Zf(Z)] = \la \E[f(Z+1)],\]
    is linear in $f$ and thus reduce the result to the case when $f$ is an indicator function (that is, use a (1),(2),(3) argument).

    Now conversely suppose that for all bound $f:\N \to \R$,
    \[ \E[Zf(Z)] = \la \E[f(Z+1)].\]
    Let $A$ be a subset of $\N$ and let $f$ be as in Lemma $(**)$. We then have
    \begin{align*}
        \Pa(Z \in A) - \Po_\la(A) &= \sum_{j=0}^\infty \left(\delta_A(j)-\Po_\la(A)\right)\Pa(Z=j)\\
        &=\sum_{j=0}^\infty \left(jf(j)-\la f(j+1)\right)\Pa(Z=j)\\
        &=\E[Zf(Z)-\la f(Z+1)]\\
        &=0.
    \end{align*}
    So $\Pa(Z \in A) = \Po_\la(A)$ and $Z$ is Poisson$(\la)$.
\end{proof}
\section{Proof of the Poisson approximation}
We are now ready to prove 
\[\norm{\Pa_W - \Po_\la}_{TV} \le \min(3,\la^{-1})\left(\sum_{i \in I}\sum_{i \in N_i \setminus \{i\}}p_{ij}+\sum_{i \in I}\sum_{j \in N_i}p_{ij}\right), \]
where the notation is as at the start of this lecture.
\begin{proof}
    Since \[\norm{\Pa_W-\Po_\la}_{TV} = \sup_{A \subseteq \N}\abs{\Pa_W(A)-\Po_\la(A)},\]
    it suffices to show that 
    \[\abs{\Pa_W(A)-\Po_\la(A)}\le \min(3,\la^{-1})\left(\sum_{i \in I}\sum_{i \in N_i \setminus \{i\}}p_{ij}+\sum_{i \in I}\sum_{j \in N_i}p_{ij}\right),\]
    for all $A \subseteq \N$. Fix such an $A$ and define $\Delta = \Pa_W(A)-\Po_\la(A)$. Let $f$ be as in Lemma $(**)$. Then, as seen in the previous proof,
    \begin{align*}
        \Delta &= \Pa(W \in A) - \Po_\la(A) \\
        &= \E[Wf(W)-\la f(W+1)]\\
        &=\E\left[\sum_{i \in I}X_i f(W)-p_i f(W+1)
            \right]\\
        &=\sum_{i \in I}\E\left[X_if(W)-p_i f(W+1)\right]
    \end{align*}
    For every $i$, let $W_i = W-X_i$ and $V_i = \sum_{j \in N_i^c} X_j$. Note that by the definition of a dependency graph, $V_i$ is independent of $X_i$. Note also that 
    \begin{align*}
        X_if(W) &= \begin{cases}
            0 & \text{if } X_i=0,\\
            f(W_i+1)&\text{if } X_i = 1.
        \end{cases}\\
        &=X_i f(W_i+1).
    \end{align*}
    Thus
    \begin{align}\label{1and2}
        \Delta &= \sum_{i \in I}\E\left[(X_i-p_i)f(W_i+1)+p_i(f(W_i+1)-f(W+1))\right]\nonumber\\
        &=\sum_{i \in I}\E\left[(X_i-p_i)(f(W_i+1)-f(V_i+1))\right]+\E\left[p_i(f(W_i+1)-f(W+1))\right].\\
        &=\sum_{i \in I}\E\left[(X_i-p_i)(f(W_i+1)-f(V_i+1))\right]+\sum_{i\in I}\E\left[p_i(f(W_i+1)-f(W+1))\right]\nonumber \\
        &=(I)+(II).\nonumber
    \end{align}
   The equality in \eqref{1and2} holds because $V_i$ and $X_i$ are independent and thus for each $i$,
    \begin{align*}
        E\left[(X_i-p_i)f(W_i+1)\right] &= \E\left[(X_i-p_i)(f(W_i+1)-f(V_i+1))\right]+\E[(X_i-p_i)f(V_i+1)]\\
        &=\E\left[(X_i-p_i)(f(W_i+1)-f(V_i+1))\right]+\E[(X_i-p_i)]\E[f(V_i+1)]\\
        &=\E\left[(X_i-p_i)(f(W_i+1)-f(V_i+1))\right].
    \end{align*}
    We will now bound the absolute value of the sums $(I)$ and $(II)$. We'll start with $(II)$ which is simplier. For each $i$, $f(W_i+1)=f(W+1)$ if $X_i=0$ and otherwise $W_i+1$ and $W+1$ differ by 1. Thus,
    \[\abs{f(W_i+1)-f(W+1)} \le X_i \min(3,\la^{-1}). \]
    And so we have
    \begin{align*}
        \abs{\sum_{i \in I}\E\left[p_i(f(W_i+1)-f(W+1))\right]}&\le \sum_{i\in I} p_i\E\left[\abs{f(W_i+1)-f(W+1)}\right]\\
        &\le \min(3,\la^{-1})\sum_{i \in I}p_i \E[X_i]\\
        &=\min(3,\la^{-1})\sum_{i \in I}p_ip_i.
    \end{align*}
    The sum $(I)$ is trickier but similar ideas can be used to bound it. For a fixed $i$, let $X_1',\ldots, X_m'$ be an enumeration of the variables in $N_i \setminus \{i\}$. We then have
     \begin{align*}
         \abs{f(W_i+1)-f(V_i+1)}&=\abs{f\left(1+V_i+\sum_{k=1}^m X_k'\right)-f(1+V_i)}\\
         &=\abs{\sum_{j=1}^m X_j'\left(f\left(1+V_i+\sum_{k=1}^j X_k'\right)-f\left(1+V_i+\sum_{k=1}^{j-1} X_k'\right)\right)}\\
         &\le \sum_{j=1}^m X_j'\abs{f\left(1+V_i+\sum_{k=1}^j X_k'\right)-f\left(1+V_i+\sum_{k=1}^{j-1} X_k'\right)}\\
         &\le \min(3,\la^{-1})\sum_{j=1}^m X_j'\\
         &=\min(3,\la^{-1})\sum_{j \in N_i \setminus \{i\}} X_j.
     \end{align*}
     Thus we have that 
     \begin{align*}
         \abs{\sum_{i \in I}\E\left[(X_i-p_i)(f(W_i+1)-f(V_i+1))\right]}&\le \min(3,\la^{-1})\sum_{i\in I}\sum_{j \in N_i \setminus \{i\}} \E[\abs{X_i-p_i}X_j]\\
         &=\min(3,\la^{-1})\sum_{i\in I}\sum_{j \in N_i \setminus \{i\}} \E[X_iX_j]+p_i\E[X_j]\\
         &=\min(3,\la^{-1})\sum_{i\in I}\sum_{j \in N_i \setminus \{i\}} p_{ij}+p_ip_j.\\
     \end{align*}
     Thus combining our bounds on $(I)$ and $(II)$ we have 
     \begin{align*}
         \abs{\Delta} &\le \min(3,\la^{-1})\left(\sum_{i\in I}\sum_{j \in N_i \setminus \{i\}} p_{ij}+p_ip_j + p_{i}p_i\right)\\
         &= \min(3,\la^{-1})\left(\sum_{i \in I}\sum_{i \in N_i \setminus \{i\}}p_{ij}+\sum_{i \in I}\sum_{j \in N_i}p_{ij}\right)\qedhere 
     \end{align*}
\end{proof}
\newpage
\section{References}
Three references are 
\begin{itemize}
    \item ``\href{https://projecteuclid.org/journals/statistical-science/volume-5/issue-4/Poisson-Approximation-and-the-Chen-Stein-Method/10.1214/ss/1177012015.full}{Poisson Approximation and the Chen-Stein Method}'' by Arratia, Goldstein and Gord.
    \item ``\href{https://arxiv.org/abs/math/0411525}{Exchangeable pairs and Poisson approximation}'' by Chatterjee, Diaconis and Meckes.
    \item ``\href{https://searchworks.stanford.edu/view/12910822}{An Introduction to Stein's Method}'' by Barbour and Chen. This is a textbook which is available online through Stanford Libraries.
\end{itemize}

\end{document}
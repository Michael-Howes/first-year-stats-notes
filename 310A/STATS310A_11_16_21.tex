\documentclass{article}
\usepackage{ae,aecompl}
\usepackage{todonotes}
\usepackage{chngcntr}
\usepackage{tikz-cd}
\usepackage{graphicx}
\graphicspath{ {./images/}}
\usepackage[all,cmtip]{xy}
\usepackage{amsmath, amscd}
\usepackage{amsthm}
\usepackage{amssymb}
\usepackage{amsfonts}
\usepackage{bm}
\usepackage{qsymbols}
\usepackage{latexsym}
\usepackage{mathrsfs}
\usepackage{mathtools}
\usepackage{cite}
\usepackage{color}
\usepackage{url}
\usepackage{enumerate}
\usepackage{verbatim}
\usepackage[draft=false, colorlinks=true]{hyperref}
\usepackage{pdfpages}
\usepackage[margin=1.2in]{geometry}
\usepackage{IEEEtrantools}

\usepackage{fancyhdr}


\usepackage[nameinlink]{cleveref}


\DeclareMathOperator*{\ac}{accept}
\DeclareMathOperator*{\amax}{argmax}
\DeclareMathOperator*{\amin}{argmin}
\DeclareMathOperator*{\Aut}{Aut}
\newcommand {\al}{{\alpha}}
\newcommand {\abs}[1]{{\left\lvert#1\right\rvert}}
\newcommand {\A}{{\mathcal{A}}}
\newcommand {\AM}{{\mathrm{AM}}}
\newcommand {\AMp}{{\AM_{p}^{X}\!(\Ri_\w)}}
\newcommand {\B}{{\mathcal{B}}}
\DeclareMathOperator*{\Be}{Bern}
\newcommand {\Br}{{\dot{B}}}
\newcommand {\Ba}{{\mathfrak{B}}}
\newcommand {\C}{{\mathbb C}}
\newcommand {\ce}{\mathrm{c}}
\newcommand {\Ce}{\mathrm{C}}
\newcommand {\Cc}{\mathrm{C_{c}}}
\newcommand {\Ccinf}{\mathrm{C_{c}^{\infty}}}
\DeclareMathOperator{\cov}{Cov}
\DeclareMathOperator{\DEV}{DEV}
\newcommand {\Di}{{\mathbb D}}
\newcommand {\dom}{\mathrm{dom}}
\newcommand{\dist}{\stackrel{\mathrm{dist}}{=}}
\newcommand {\ud}{\mathrm{d}}
\newcommand {\ue}{\mathrm{e}}
\newcommand {\eps}{\varepsilon}
\newcommand {\veps}{\varepsilon}
\newcommand {\vrho}{{\varrho}}
\newcommand {\E}{{\mathbb{E}}}
\newcommand {\Ec}{{\mathcal{E}}}
\newcommand {\Ell}{L}
\newcommand {\Ellp}{{L_{p}[0,1]}}
\newcommand {\Ellpprime}{{L_{p'}([0,1])}}
\newcommand {\Ellq}{{L_{q}([0,1])}}
\newcommand {\Ellqprime}{{L_{q'}([0,1])}}
\newcommand {\Ellr}{L^{r}}
\newcommand {\Ellone}{{L_{1}([0,1])}}
\newcommand{\Elltwo}{{L_{2}([0,1])}}
\newcommand{\Ellinfty}{L^{\infty}}
\newcommand{\Ellinftyc}{L_{\mathrm{c}}^{\infty}}
\newcommand{\exb}[1]{\exp\left\{#1\right\}}
\DeclareMathOperator*{\Ext}{Ext}
\newcommand{\F}{{\mathcal{F}}}
\newcommand{\Fe}{{\mathbb{F}}}
\newcommand{\G}{{\mathcal{G}}}
\newcommand{\HF}{\mathcal{H}_{\text{FIO}}^{1}(\Rd)}
\newcommand{\Hr}{H}
\newcommand{\HT}{\mathcal{H}}
\newcommand{\ui}{\mathrm{i}}
\newcommand{\I}{{I}}
\newcommand{\J}{{\mathcal{J}}}
\newcommand{\id}{{\mathrm{id}}}
\newcommand{\iid}{\stackrel{\mathclap{\normalfont\mbox{iid}}}{\sim}}
\newcommand{\im}{{\text{im }}}
\newcommand{\ind}{{\perp\!\!\!\perp}}
\DeclareMathOperator*{\Int}{int}
\newcommand{\intx}{{\overline{\int_{X}}}}
\newcommand{\inte}{{\overline{\int_{\E}}}}
\newcommand{\la}{\lambda}
\newcommand{\rb}{\rangle}
\newcommand{\lb}{{\langle}}
\newcommand{\La}{\Lambda}
\newcommand{\calL}{{\mathcal{L}}}
\newcommand{\lp}{{\mathcal{L}}^{p}}
\newcommand{\lpo}{{\overline{\mathcal{L}}^{p}\!}}
\newcommand{\Lpo}{{\overline{\Ell}^{p}\!}}
\newcommand{\M}{{\mathbf{M}}}
\newcommand{\Ma}{{\mathcal{M}}}
\newcommand{\N}{{{\mathbb N}}}
\newcommand{\Na}{{{\mathcal{N}}}}
\newcommand{\norm}[1]{\left\|#1\right\|}
\newcommand{\normm}[1]{{\left\vert\kern-0.25ex\left\vert\kern-0.25ex\left\vert #1 
    \right\vert\kern-0.25ex\right\vert\kern-0.25ex\right\vert}}
\newcommand{\Om}{{{\Omega}}}
\newcommand{\one}{{{\bf 1}}}
\newcommand{\pic}{\text{Pic }}
\newcommand{\ph}{{\varphi}}
\newcommand{\Pa}{{\mathbb{P}}}
\newcommand{\Po}{{\mathcal{P}}}
\newcommand{\Q}{{\mathbb{Q}}}
\newcommand{\R}{{\mathbb R}}
\newcommand{\Rd}{{\mathbb{R}^{d}}}
\DeclareMathOperator{\rej}{reject }
\newcommand{\Rn}{{\mathbb{R}^{n}}}
\newcommand{\cR}{{\mathcal{R}}}
\newcommand{\Rad}{{\mathrm{Rad}}}
\newcommand{\ran}{{\mathrm{ran}}}
\newcommand{\Ri}{{\mathrm{R}}}
\newcommand{\supp}{{\mathrm{supp}}}
\newcommand{\Se}{\mathrm{S}}
\newcommand{\Sp}{S^{*}(\Rn)}
\newcommand{\St}{{\mathrm{St}}}
\newcommand{\Sw}{\mathcal{S}}
\newcommand{\T}{{\mathcal{T}}}
\newcommand{\ta}{{\theta}}
\newcommand{\Ta}{{\Theta}}
\newcommand{\topp}{\stackrel{p}{\to}}
\newcommand{\todd}{\stackrel{d}{\to}}
\newcommand{\toL}[1]{\stackrel{L^{#1}}{\to}} 
\newcommand{\toas}{\stackrel{a.s.}{\to}}
\DeclareMathOperator{\V}{Var}
\newcommand {\w}{{\omega}}
\newcommand {\W}{{\mathrm{W}}}
\newcommand {\Wnp}{\text{$\mathrm{W}$\textsuperscript{$n,\!p$}}}
\newcommand {\Wnpeq}{\text{$\mathrm{W}$\textsuperscript{$n\!,\!p$}}}
\newcommand {\Wonep}{\text{$\mathrm{W}$\textsuperscript{$1,\!p$}}}
\newcommand {\Wonepeq}{\text{$\mathrm{W}$\textsuperscript{$1\!,\!p$}}}
\newcommand {\X}{{\mathcal{X}}}
\newcommand {\Z}{{{\mathbb Z}}}
\newcommand {\Za}{{\mathcal{Z}}}
\newcommand {\Zd}{{\Z[\sqrt{d}]}}
\newcommand {\vanish}[1]{\relax}

\newcommand {\wh}{\widehat}
\newcommand {\wt}{\widetilde}
\newcommand {\red}{\color{red}}

% Distributions
\newcommand{\normal}{\mathsf{N}}
\newcommand{\poi}{\mathsf{Poisson}}
\newcommand{\bern}{\mathsf{Bernoulli}}
\newcommand{\bin}{\mathsf{Binomal}}
\newcommand{\multi}{\mathsf{Multinomial}}
\newcommand{\Exp}{\mathsf{Exp}}



% put your command and environment definitions here




% some theorem environments
% remove "[theorem]" if you do not want them to use the same number sequence


  \newtheorem{thrm}{Theorem}
  \newtheorem{lemma}{Lemma}
  \newtheorem{prop}{Proposition}
  \newtheorem{cor}{Corollary}

  \newtheorem{conj}{Conjecture}
  \renewcommand{\theconj}{\Alph{conj}}  % numbered A, B, C etc

  \theoremstyle{definition}
  \newtheorem{defn}{Definition}
  \newtheorem{ex}{Example}
  \newtheorem{exs}{Examples}
  \newtheorem{question}{Question}
  \newtheorem{remark}{Remark}
  \newtheorem{notn}{Notation}
  \newtheorem{exer}{Exercise}




\title{STATS310A - Lecture 16}
\author{Persi Diaconis\\ Scribed by Michael Howes}
\date{11/16/21}

\pagestyle{fancy}
\fancyhf{}
\rhead{STATS310A - Lecture 16}
\lhead{11/16/21}
\rfoot{Page \thepage}

\begin{document}
\maketitle
\tableofcontents
\section{Announcements}
Final homeowork uploaded to Canvas and due November 30.
\begin{itemize}
    \item Read chpaters 25,26.
    \item Do 25 / 1,3 and 26 1,3,12-14.
    \item The hints contain surprises.
\end{itemize}
\section{The central limit theorem}
We are in the process of proving Lindeberg's version of the central limit theorem. Recall that we have a triangular array of random variables $\{X_{n,i}\}$ where $i=1,\ldots, k_n$ and $n=1,2,\ldots$. We assume that the array has independent rows. That is, for each $n$, $\{X_{n,i}\}_{i=1}^{k_n}$ are independent. Assume also that 
\[\E[X_{n,i}]=0 ~~~\text{and}~~~\sigma_{i,n}^2 = \V(X_{n,i})< \infty. \]
Define $S_n = \sum_{i=1}^{k_n}X_{n,i}$ and $s_n^2=\sum_{i=1}^{k_n} \sigma_{i,n}^2 = \V(S_n)$. 
\begin{defn}
    A triangular array  with independent rows $\{X_{n,i}\}$ is said to satisfy \emph{Lindeberg's condtion} if for all $\eps > 0$
    \[\frac{1}{s_n^2} \sum_{i=1}^{k_n} \int_{\{\abs{X_{n,i}}>\eps s_n\}}\abs{X_{n,i}}^2d\Pa \stackrel{n}{\longrightarrow} 0. \]
\end{defn}
Lindeberge's version of the central limit theorem is:
\begin{thrm}[Lindeberg]
    Let $\{X_{n,i}\}$ be a triangular array with independent rows. If $\{X_{n,i}\}$ satisfy Lindeberg's condtion, then for all $x \in \R$,
    \[\Pa\left(\frac{S_n}{s_n} \le x\right) \to \Phi(x),\]
    where $\Phi(x) = \Pa(Z \le x)$ for $Z \sim \Na(0,1)$.
\end{thrm}
\begin{proof}
To prove this we will use the portmanteau theorem. Let $C^\infty_c(\R)$ be the class of infinitely differentiable functions on $\R$ with compact support. By the portmanteau theorem it suffices to show that for all $f \in C^\infty_c(\R)$, $\E[f(S_n/s_n)] \stackrel{n}{\to} \E[f(Z)]$ where $Z \sim \Na(0,1)$. Thus fix such an $f$. Define $Z_{n,i}$ is be independent random variables such that $Z_{n,i} \sim \Na(0,\sigma_{n,i}^2)$. Let $Z_n = \sum_{i=1}^{k_n}Z_{n,i}$, Then
\[Z=\frac{1}{s_n}Z_n \sim \Na(0,1). \]
The idea behind the proof is to swap out $X_{n,i}$ for $Z_{n,i}$ one at a time. With this in mind, define
\[T_{n,i}= X_{n,1}+\ldots+X_{n,i-1}+Z_{n,i}+\ldots+Z_{n,k_n}. \]
Note that $X_i,Z_i$ are independent of $T_{n,i}$ for each $i$. Furthermore we have
\[S_n = T_{n,k_n}+Z_{n,k_n}~~~\text{and}~~~Z_n = T_{n,1}+Z_{n,1}. \]
And also
\[T_{n,i}+Z_{n,i} = T_{n,i-1}+X_{n,i-1}, \]
for $i=2,\ldots, k_n$. Thus by telescoping we have
\[f\left(\frac{S_n}{s_n}\right)-f\left(\frac{Z_n}{s_n}\right)=\sum_{i=1}^{k_n} f\left(\frac{T_{n,i}+X_{n,i}}{s_n}\right)-f\left(\frac{T_{n,i}+Z_{n,i}}{s_n}\right). \]
And so 
\begin{equation}\label{telescope}
    \abs{\E\left[f\left(\frac{S_n}{s_n}\right)\right]-\E\left[f(Z)\right]}\le \sum_{i=1}^{k_n}\abs{\E\left[f\left(\frac{T_{n,i}+X_{n,i}}{s_n}\right)\right]-\E\left[f\left(\frac{T_{n,i}+Z_{n,i}}{s_n}\right)\right]}
\end{equation}
We will now use Taylor's approximation to bounded each of the terms in the above sum. For $h \in \R$, define
\[g(h) = \abs{f(x+h)-f(x)-hf'(x)-\frac{h^2}{2}f''(x)}.\]
Since all deriviates of $f$ are bounded, Taylor's approximation with remainder says that there exists $k > 0$ such that for all $h$ and $x$
\[g(h) \le k \min\{\abs{h}^3,\abs{h}^2\}. \]
Thus for all $x,h_1,h_2$ we have
\[\abs{f(x+h_1)-f(x+h_2)-f'(x)(h_1-h_2)-\frac{1}{2}f''(x)(h_1^2-h_2^2)}=\abs{g(h_1)-g(h_2)} \le \abs{g(h_1)}+\abs{g(h_2)}. \] 
We wish to apply this to equation \eqref{telescope} with $x=\frac{T_{n,i}}{s_n}$, $h_1=\frac{X_{n,i}}{s_n}$ and $h_2 = \frac{Z_{n,i}}{s_n}$. Thus we need to add the high order terms $f'(x)(h_1-h_2)$ and $\frac{1}{2}f''(x)(h_1^2-h_2^2)$. Since $X_{n,i}$ and  $Z_{n,i}$ have the same mean and variance and $X_{n,i},Z_{n,i}$ are independent of $T_{n,i}$, we have
\begin{align*}
    & \E\left[\left(\frac{T_{n,i}+X_{n,i}}{s_n}\right)\right]-\E\left[f\left(\frac{T_{n,i}+Z_{n,i}}{s_n}\right)\right]\\
    =&
\end{align*}
\end{proof}
\end{document}
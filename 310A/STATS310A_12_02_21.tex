\documentclass{article}
\usepackage{ae,aecompl}
\usepackage{todonotes}
\usepackage{chngcntr}
\usepackage{tikz-cd}
\usepackage{graphicx}
\graphicspath{ {./images/}}
\usepackage[all,cmtip]{xy}
\usepackage{amsmath, amscd}
\usepackage{amsthm}
\usepackage{amssymb}
\usepackage{amsfonts}
\usepackage{bm}
\usepackage{qsymbols}
\usepackage{latexsym}
\usepackage{mathrsfs}
\usepackage{mathtools}
\usepackage{cite}
\usepackage{color}
\usepackage{url}
\usepackage{enumerate}
\usepackage{verbatim}
\usepackage[draft=false, colorlinks=true]{hyperref}
\usepackage{pdfpages}
\usepackage[margin=1.2in]{geometry}
\usepackage{IEEEtrantools}

\usepackage{fancyhdr}


\usepackage[nameinlink]{cleveref}


\DeclareMathOperator*{\ac}{accept}
\DeclareMathOperator*{\amax}{argmax}
\DeclareMathOperator*{\amin}{argmin}
\DeclareMathOperator*{\Aut}{Aut}
\newcommand {\al}{{\alpha}}
\newcommand {\abs}[1]{{\left\lvert#1\right\rvert}}
\newcommand {\A}{{\mathcal{A}}}
\newcommand {\AM}{{\mathrm{AM}}}
\newcommand {\AMp}{{\AM_{p}^{X}\!(\Ri_\w)}}
\newcommand {\B}{{\mathcal{B}}}
\DeclareMathOperator*{\Be}{Bern}
\newcommand {\Br}{{\dot{B}}}
\newcommand {\Ba}{{\mathfrak{B}}}
\newcommand {\C}{{\mathbb C}}
\newcommand {\ce}{\mathrm{c}}
\newcommand {\Ce}{\mathrm{C}}
\newcommand {\Cc}{\mathrm{C_{c}}}
\newcommand {\Ccinf}{\mathrm{C_{c}^{\infty}}}
\DeclareMathOperator{\cov}{Cov}
\DeclareMathOperator{\DEV}{DEV}
\newcommand {\Di}{{\mathbb D}}
\newcommand {\dom}{\mathrm{dom}}
\newcommand{\dist}{\stackrel{\mathrm{dist}}{=}}
\newcommand {\ud}{\mathrm{d}}
\newcommand {\ue}{\mathrm{e}}
\newcommand {\eps}{\varepsilon}
\newcommand {\veps}{\varepsilon}
\newcommand {\vrho}{{\varrho}}
\newcommand {\E}{{\mathbb{E}}}
\newcommand {\Ec}{{\mathcal{E}}}
\newcommand {\Ell}{L}
\newcommand {\Ellp}{{L_{p}[0,1]}}
\newcommand {\Ellpprime}{{L_{p'}([0,1])}}
\newcommand {\Ellq}{{L_{q}([0,1])}}
\newcommand {\Ellqprime}{{L_{q'}([0,1])}}
\newcommand {\Ellr}{L^{r}}
\newcommand {\Ellone}{{L_{1}([0,1])}}
\newcommand{\Elltwo}{{L_{2}([0,1])}}
\newcommand{\Ellinfty}{L^{\infty}}
\newcommand{\Ellinftyc}{L_{\mathrm{c}}^{\infty}}
\newcommand{\exb}[1]{\exp\left\{#1\right\}}
\DeclareMathOperator*{\Ext}{Ext}
\newcommand{\F}{{\mathcal{F}}}
\newcommand{\Fe}{{\mathbb{F}}}
\newcommand{\G}{{\mathcal{G}}}
\newcommand{\HF}{\mathcal{H}_{\text{FIO}}^{1}(\Rd)}
\newcommand{\Hr}{H}
\newcommand{\HT}{\mathcal{H}}
\newcommand{\ui}{\mathrm{i}}
\newcommand{\I}{{I}}
\newcommand{\J}{{\mathcal{J}}}
\newcommand{\id}{{\mathrm{id}}}
\newcommand{\iid}{\stackrel{\mathclap{\normalfont\mbox{iid}}}{\sim}}
\newcommand{\im}{{\text{im }}}
\newcommand{\ind}{{\perp\!\!\!\perp}}
\DeclareMathOperator*{\Int}{int}
\newcommand{\intx}{{\overline{\int_{X}}}}
\newcommand{\inte}{{\overline{\int_{\E}}}}
\newcommand{\la}{\lambda}
\newcommand{\rb}{\rangle}
\newcommand{\lb}{{\langle}}
\newcommand{\La}{\Lambda}
\newcommand{\calL}{{\mathcal{L}}}
\newcommand{\lp}{{\mathcal{L}}^{p}}
\newcommand{\lpo}{{\overline{\mathcal{L}}^{p}\!}}
\newcommand{\Lpo}{{\overline{\Ell}^{p}\!}}
\newcommand{\M}{{\mathbf{M}}}
\newcommand{\Ma}{{\mathcal{M}}}
\newcommand{\N}{{{\mathbb N}}}
\newcommand{\Na}{{{\mathcal{N}}}}
\newcommand{\norm}[1]{\left\|#1\right\|}
\newcommand{\normm}[1]{{\left\vert\kern-0.25ex\left\vert\kern-0.25ex\left\vert #1 
    \right\vert\kern-0.25ex\right\vert\kern-0.25ex\right\vert}}
\newcommand{\Om}{{{\Omega}}}
\newcommand{\one}{{{\bf 1}}}
\newcommand{\pic}{\text{Pic }}
\newcommand{\ph}{{\varphi}}
\newcommand{\Pa}{{\mathbb{P}}}
\newcommand{\Po}{{\mathcal{P}}}
\newcommand{\Q}{{\mathbb{Q}}}
\newcommand{\R}{{\mathbb R}}
\newcommand{\Rd}{{\mathbb{R}^{d}}}
\DeclareMathOperator{\rej}{reject }
\newcommand{\Rn}{{\mathbb{R}^{n}}}
\newcommand{\cR}{{\mathcal{R}}}
\newcommand{\Rad}{{\mathrm{Rad}}}
\newcommand{\ran}{{\mathrm{ran}}}
\newcommand{\Ri}{{\mathrm{R}}}
\newcommand{\supp}{{\mathrm{supp}}}
\newcommand{\Se}{\mathrm{S}}
\newcommand{\Sp}{S^{*}(\Rn)}
\newcommand{\St}{{\mathrm{St}}}
\newcommand{\Sw}{\mathcal{S}}
\newcommand{\T}{{\mathcal{T}}}
\newcommand{\ta}{{\theta}}
\newcommand{\Ta}{{\Theta}}
\newcommand{\topp}{\stackrel{p}{\to}}
\newcommand{\todd}{\stackrel{d}{\to}}
\newcommand{\toL}[1]{\stackrel{L^{#1}}{\to}} 
\newcommand{\toas}{\stackrel{a.s.}{\to}}
\DeclareMathOperator{\V}{Var}
\newcommand {\w}{{\omega}}
\newcommand {\W}{{\mathrm{W}}}
\newcommand {\Wnp}{\text{$\mathrm{W}$\textsuperscript{$n,\!p$}}}
\newcommand {\Wnpeq}{\text{$\mathrm{W}$\textsuperscript{$n\!,\!p$}}}
\newcommand {\Wonep}{\text{$\mathrm{W}$\textsuperscript{$1,\!p$}}}
\newcommand {\Wonepeq}{\text{$\mathrm{W}$\textsuperscript{$1\!,\!p$}}}
\newcommand {\X}{{\mathcal{X}}}
\newcommand {\Z}{{{\mathbb Z}}}
\newcommand {\Za}{{\mathcal{Z}}}
\newcommand {\Zd}{{\Z[\sqrt{d}]}}
\newcommand {\vanish}[1]{\relax}

\newcommand {\wh}{\widehat}
\newcommand {\wt}{\widetilde}
\newcommand {\red}{\color{red}}

% Distributions
\newcommand{\normal}{\mathsf{N}}
\newcommand{\poi}{\mathsf{Poisson}}
\newcommand{\bern}{\mathsf{Bernoulli}}
\newcommand{\bin}{\mathsf{Binomal}}
\newcommand{\multi}{\mathsf{Multinomial}}
\newcommand{\Exp}{\mathsf{Exp}}



% put your command and environment definitions here




% some theorem environments
% remove "[theorem]" if you do not want them to use the same number sequence


  \newtheorem{thrm}{Theorem}
  \newtheorem{lemma}{Lemma}
  \newtheorem{prop}{Proposition}
  \newtheorem{cor}{Corollary}

  \newtheorem{conj}{Conjecture}
  \renewcommand{\theconj}{\Alph{conj}}  % numbered A, B, C etc

  \theoremstyle{definition}
  \newtheorem{defn}{Definition}
  \newtheorem{ex}{Example}
  \newtheorem{exs}{Examples}
  \newtheorem{question}{Question}
  \newtheorem{remark}{Remark}
  \newtheorem{notn}{Notation}
  \newtheorem{exer}{Exercise}

\DeclareMathOperator*{\sgn}{sgn}


\title{STATS310A - Lecture 19}
\author{Persi Diaconis\\ Scribed by Michael Howes}
\date{12/02/21}

\pagestyle{fancy}
\fancyhf{}
\rhead{STATS310A - Lecture 19}
\lhead{12/02/21}
\rfoot{Page \thepage}

\begin{document}
\maketitle
\tableofcontents
\section{Fourier inversion}
\begin{thrm}
    Suppose that $\mu$ is a probability on $\R$ with charateristic function $\phi$. If $\mu(\{a\})=\mu(\{b\})=0$, then 
    \begin{equation}\label{inv}\mu((a,b)) = \lim_{T \to \infty} \frac{1}{2\pi}\int_{-T}^T \frac{e^{-ita}-e^{-itb}}{it}\phi(t)dt.
    \end{equation}
\end{thrm}
To prove this we will need several classic calculus facts. We won't prove them here but their derivations are in section 26 of Billingsley. 
\begin{enumerate}
    \item Let $S(T) = \int_0^T \frac{\sin(t)}{t}dt$ ($S(T)$ is also known as $\text{sinc}(T)$). While $\int_0^\infty \frac{\sin(t)}{t}dt$ does not exist as a Lebesgue integral, a classical calculation shows that 
    \[\lim_{T \to \infty} S(T)=\frac{\pi}{2}.  \]
    Thus $S(T)$ is bounded on $[0,\infty)$. Let $\sgn$ be the function 
    \[\sgn(\ta) =\begin{cases}
        -1 & \text{if } \ta < 0,\\
        0 & \text{if } \ta=0,\\
        1 & \text{if } \ta > 0.
    \end{cases} \]
    By using a change of variables, one can show that 
    \begin{equation}\label{sinc}\int_0^T \frac{\sin(t\ta)}{t}dt = \sgn(\ta)S(T\abs{\ta}).\end{equation}
    \item Using Taylor's theorem we have for all real $t$.
    \begin{enumerate}
        \item $\abs{e^{ix}-1}\le\min\{\abs{x},2\}$.
        \item $\abs{e^{ix}-(1+ix)}\le \min\left\{\frac{1}{2}x^2,2\abs{x}\right\}$.
    \end{enumerate}
    Using (ii.) we see that the integrand in equation \eqref{inv} converges to $b-a$ as $t \to 0$. Thus the integrand in \eqref{inv} is continuous and by (i.) it is bounded. Thus the integral in \eqref{inv} exists for every $T$.
\end{enumerate}
With these calculus facts we are ready to prove Theorem 1.
\begin{proof}
    For $T>0$, let
    \[I_T =\frac{1}{2\pi}\int_{-T}^T \frac{e^{-ita}-e^{-itb}}{it}\phi(t)dt. \]
    Note that 
    \[I_T =\frac{1}{2\pi}\int_{-T}^T \int_\R \frac{e^{-ita}-e^{-itb}}{it}e^{itx} \mu(dx)dt. \]
    The above integrand is a bounded function and the space $[-T,T]\times \R$ has finite measure with respect to $dt\times \mu(dx)$. Thus we can apply Fubinni's theorem which gives
    \[I_T = \frac{1}{2\pi} \int_\R \int_{-T}^T \frac{e^{it(x-a)}-e^{it(x-b)}}{it}dt\mu(dx). \]
    We will now use equation \eqref{sinc} and the fact $\sin$ and $\cos$ are odd and even respectively. This gives us 
    \begin{align*}
        I_T& = \frac{1}{2\pi} \int_\R 2\int_0^T \frac{\sin(t(x-a))}{t}-\frac{\sin(t(x-b))}{t}dt \\
        &= \int_\R \left[\frac{\sgn(x-a)}{\pi}S(T\abs{x-a})-\frac{\sgn(x-b)}{\pi}S(T\abs{x-b})\right]\mu(dx).
    \end{align*}
    The function $f_T(x)=\frac{\sgn(x-a)}{\pi}S(T\abs{x-a})-\frac{\sgn(x-b)}{\pi}S(T\abs{x-b})$ is uniformly bounded over $x$ and $T$. We know that for $x \neq a,b$, \[\frac{1}{\pi}S(T\abs{x-a}), \frac{1}{\pi}S(T\abs{x-b}) \to \frac{1}{\pi}\cdot \frac{\pi}{2}=\frac{1}{2}.\]
    Thus if $\sgn(x-a)=\sgn(x-b)$ and $x\neq a,b$, then $f_T(x) \to 0$. Also if $\sgn(x-a)\neq \sgn(x-b)$ and $x \neq a,b$, then $f_T(x) \to 1$. If $x=a$, then $\sgn(x-b)=-1$ and $f_T(x) \to \frac{1}{2}$. Lastly, if $x = b$, then $\sgn(x-a)=1$ and $f_T(x) \to \frac{1}{2}$. Summarising this, we can conclude that $f_T$ converges pointwise to the function $\psi_{a,b}$ where 
    \[\psi_{a,b}(x) = \begin{cases}
        0 & \text{if } x < a,\\
        \frac{1}{2} & \text{if } x = a,\\
        1 & \text{if } a<x<b,\\
        \frac{1}{2} & \text{if } x = b,\\
        0 & \text{if } x > b.
    \end{cases} \]
    Since $\mu(\{a,b\})=0$, we can conclude that $f_T$ converges to $\delta_{(a,b)}$ $\mu$-almost surely. Thus, by the dominated convergence theorem we have
    \[\lim_{T\to \infty} I_T = \lim_{T \to \infty} \int_\R f_T(x)\mu(dx)=\int_\R \delta_{(a,b)}(x)\mu(dx)=\mu((a,b)). \qedhere  \]
\end{proof}
Note that this proof also shows that 
\[\mu((a,b))+\frac{1}{2}\mu(\{a,b\}) = \lim_{T \to \infty} \frac{1}{2\pi}\int_{-T}^T \frac{e^{-ita}-e^{-itb}}{it}\phi(t)dt, \]
which was the version of the Fourier inversion theorem stated on Tuesday.
\begin{remark}
    To be honest, while careful, this proof sort of ``stink''. It doesn't give any feeling for what's going on. We'll come back to this later when we look at a discrete version of characteristic functions. This ``lack of transparency'' is what led Stein to develop Stein's method. It's also why we studied Lindeberg's proof of the central limit theorem in class.
\end{remark}
\begin{remark}
    Note that if $F$ is the cumulative distribution function of $\mu$, then equation \eqref{inv} can be written as 
    \[F(b)-F(a) = \lim_{T \to \infty} \frac{1}{2\pi}\int_{-T}^T \frac{e^{-ita}-e^{-itb}}{it}\phi(t)dt. \]
    In particular for $c \in \R$ and $h>0$,
    \[\frac{F(c+h)-F(c)}{h} =\lim_{T \to \infty} \frac{1}{2\pi}\int_{-T}^T \frac{e^{-itc}(1-e^{-ith})}{ith}\phi(t)dt. \]
    If we assume that $\phi$ is integrable (so that $\int_\R \abs{\phi(t)}dt < \infty$), then an analysis of the aboe equation can be used to show that $\mu$ has a density $f$ given by
    \[f(c) = F'(c) = \int_\R e^{-itc}\phi(t)dt. \]
\end{remark}
\begin{ex}[The usual CLT]
    The following is a typical application. Let $X_1,X_2,\ldots$ be i.i.d. with $\E[X_1]=0$, $\E[X_1^2]=1$ and characteristic function $\phi$. Let $S_n = \frac{1}{\sqrt{n}}\sum_{j=1}^n X_j$. Since the first two moments of $X_j$ exist $\phi$ is twice continuously differentiable with $\phi'(0)=1$ and $\phi''(0)=-1$. 

    Since $X_1,X_2,\ldots$, are independent, we know the that
    \[\phi_{S_n}(t)=\phi(t/\sqrt{n})^n = \left(1-\frac{t^2}{2n}+o\left(1/n\right)\right)^n \to e^{-\frac{t^2}{2}} =\phi_{\Na}(t), \]
    where $\phi_\Na$ denotes the characteristic function of the standard normal distribution.
\end{ex}
\end{document}